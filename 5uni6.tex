% !TEX root = main.tex

\begin{lemma}\label{wq}
  For $\alpha\in G\cup N,w$ has a unique zero $r_0\in(0,z(\alpha)).$
  Furthermore, $w(z(\alpha))<0$ if $\alpha\in N$
  and $\underset{r\to\infty}{\lim}w(r)=-\infty$ if $\alpha\in G$.

  \seperate

  \emph{Some remarks: the functions $u(r)$ and $w(r)$ will be compared `'in the rest of the uniqueness proof'. This lemma derives that $w(r)$ has a \textbf{unique} zero $r_0$ to the left of $z(\alpha)$. Remember $z(\alpha)$ is the zero of $u(r)$. On the other hand, if $u(r)$ is positive and decreasing everywhere--has no finite zero $z(\alpha)$ such that $u(z(\alpha))=0$--then still $w(r)$ has a finite zero $r_0$ and $w(r)$remains negative to the right of the zero. Even more, the function $w(r)$ tends to negative infinity for $r\to\infty$, $\underset{r\to\infty}{\lim}w(r)=-\infty$.}

\begin{proof}
The solution $\nu(r)$ changes sign in $\rho_0\in[0,z(\alpha)]$.
As $\nu(z(\alpha))z(\alpha)u'(z(\alpha))<0$, the solution $\nu(r)$ changes sign once.
Let $\tau$ be the first zero of $w(r)$.
Then by Sturm comparison, $\rho_0\in(0,\tau)$, i.e. $\nu$ oscillates faster than $w$.

Also by Sturm comparison, $w$ can have no further zero and $\tau$ is the unique zero of $w$.

\seperate

\underline{Sign of $v$ changes in $\rho_0$}
`'By previous lemmata' the sign of $\nu$ changes in $\rho_0$.
Refer to lemma ... and/or figure ...
Also, remember ...\\ \\

\underline{$\nu$ changes sign only once}
The argument that $\nu$ changes sign only once is given in lemma ...
Remember that ...\\ \\

\underline{$w$ has a first zero}
Suppose $w(r)$ does not have a zero.
Then ...\\ \\

\underline{Comparing $w$ and $v$, conclusion: $w$ has a unique zero}
One can compare $\nu$ and $w$ with Sturm's comparison theorem,
which is given as ...
In this theorem, take ... as ... and ... as ... to see that ...\\ \\

\seperate

\underline{Also if $\alpha\in G$}
When $\alpha\in G$ then $u(r)$ has no zero.
The solution depends on $\alpha$ and the change of $u(r)$ with respect to $\alpha$ is given by
the function $w(r)=\frac{\partial}{\partial\alpha}u(r)$.
`'Since there are solutions that have a zero and some that don't',
the function $w(r)$ must have a zero.
Even more, the zero of $w$ will be to the left of the zero of $u(r)$.
CIRKELREDENERING?
Then $u(r)$ does not have a zero, the limit of $w(r)$ for $r$ to infinity is negative infinity.
To see this, note that for increasing $\alpha$, one will obtain a solution that \emph{has} a zero.
Hence in that $r$-value, the function $u(r)$ decreases with increasing $\alpha$, ergo,
the function $w(r)$ is increasingly ?? negative for $r$-values where $u(r)$ is positive.

\seperate

\end{proof}
\end{lemma}
