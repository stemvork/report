\begin{lemma}\label{wq}For $\alpha\in G\cup N,w$ has a unique zero $r_0\in(0,z(\alpha)).$ Furthermore, $w(z(\alpha))<0$ if $\alpha\in N$ and $\underset{r\to\infty}{\lim}w(r)=-\infty$ if $\alpha\in G$.

\emph{Some remarks: the functions $u(r)$ and $w(r)$ will be compared `'in the rest of the uniqueness proof'. This lemma derives that $w(r)$ has a \textbf{unique} zero $r_0$ to the left of $z(\alpha)$. Remember $z(\alpha)$ is the zero of $u(r)$. On the other hand, if $u(r)$ is positive and decreasing everywhere--has no finite zero $z(\alpha)$ such that $u(z(\alpha))=0$--then still $w(r)$ has a finite zero $r_0$ and $w(r)$remains negative to the right of the zero. Even more, the function $w(r)$ tends to negative infinity for $r\to\infty$, $\underset{r\to\infty}{\lim}w(r)=-\infty$.}\\[2cm]
\begin{proof}
The solution $\nu(r)$ changes sign in $\rho_0\in[0,z(\alpha)]$. As $\nu(z(\alpha))z(\alpha)u'(z(\alpha))<0$, the solution $\nu(r)$ changes sign once. Let $\tau$ be the first zero of $w(r)$. Then by Sturm comparison, $\rho_0\in(0,\tau)$, i.e. $\nu$ oscillates faster than $w$.

Also by Sturm comparison, $w$ can have no further zero and $\tau$ is the unique zero of $w$.

\underline{Sign of $v$ changes in $\rho_0$} \\ \\

\underline{$v$ changes sign only once} \\ \\

\underline{$w$ has a first zero} \\ \\

\underline{Comparing $w$ and $v$, conclusion: $w$ has a unique zero} \\ \\

\underline{Also if $\alpha\in G$}


\end{proof}
\end{lemma}
