\newpage
\begin{lemma}Let $\alpha\in G\cup N$. There exist $\beta_0>0$ and a unique function $\sigma:[0,\bar\beta]\to[0,\infty)$ with the following properties: \begin{enumerate}[(a)]
	\item  $\sigma$ is continuous and decreasing, $\sigma(0)>0$ and $\sigma(\bar\beta)=0$;
    \item for all $\beta>0$ we have: $\phi_\beta(r)<0$ if $r<\sigma(\beta)$, and $\phi_\beta>0$ if $r>\sigma(\beta)$.
\end{enumerate}
\begin{proof} 
Let $\beta>0$ and $r\in[0,z(\alpha))$. Then \begin{align*}
\phi_{\beta}(r)&=\left[\beta(p-1)-2\right]V(r)u(r)^p-rV'(r)u(r)^p+2\lambda u(r) %\\ &=V(r)u(r)^p\left[\beta(p-1)-2-\frac{rV'(r)u(r)^p}{V(r)u(r)^p}+\frac{2\lambda u(r)}{V(r)u(r)^p}\right] 
\\ &=V(r)u(r)^p\left[\beta(p-1)-2-r\frac{V'(r)}{V(r)}+\frac{2\lambda}{V(r)u(r)^{p-1}}\right] 
\\ &= V(r)u(r)^p\left[\beta(p-1)-2-\xi(r)\right]
\\\text{where }\xi(r)&=r\frac{V'(r)}{V(r)}-\frac{2\lambda}{V(r)u(r)^{p-1}}.
\end{align*}
\\
To conclude about the sign of $\phi(r)$, note that $V(r)\geq0$ and $u(r)\geq0$. Hence the sign of $\phi(r)$ will vary with $\beta$ and $r$ as dictated by the term in brackets. Write $\beta(p-1)-2-\xi(r)>0\iff\beta>\frac{2+\xi(r)}{p-1}\coloneqq\Xi(r)$.

The function $\xi(r)\leq0$ is strictly decreasing on $(0,z(\alpha))$ with $\lim_{r\to z(\alpha)}\xi(r)=-\infty$. To see this, note $h(r)=r\frac{V'(r)}{V(r)}$ is nonincreasing, $V'(r)\leq0$ and $u'(r)<0$ hence the second term of $\xi(r)$ is strictly decreasing. Thus $\xi(r)$ is strictly decreasing. Since $u(z(\alpha))=0$, the limit is $-\infty$.

With this information, $\Xi(r)$ is also continuous and strictly decreasing.

It remains to show $\Xi(r)=\left[2+\xi(r)\right]/(p-1)$ satisfies $\Xi(0)>0$. Remember that  Indeed, let $\Xi(0)=\bar\beta$, LLLLLL, for all $\beta\in[0,\bar\beta]$. Note $\Xi(r)$ is continuous and decreasing, so $\sigma(\beta)\coloneqq\left.\Xi(r)^{-1}\right|_{[0,\bar\beta]}$ has properties (a) and (b). SHOW. \\

Also, note $\Xi(0)>0\iff\xi(0)>-2$. SHOW.\\

Consider the following cases ......$V(0)<\infty$ .... $V(0)=\infty$ ... hence $\xi(0)>-2$ for $\alpha\in(G\cup N)$.

\end{proof}
\end{lemma}