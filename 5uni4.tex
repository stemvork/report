% !TEX root = main.tex

\newpage
\begin{lemma}
	Let $\alpha\in G\cup N$.
	There exist $\beta_0>0$ and a unique function
	$\sigma:[0,\bar\beta]\to[0,\infty)$ with the following properties:
	\begin{enumerate}[(a)]
		\item  $\sigma$ is continuous and decreasing,
		$\sigma(0)>0$ and $\sigma(\bar\beta)=0$;
    \item for all $\beta>0$ we have: $\phi_\beta(r)<0$ if $r<\sigma(\beta)$,
		and $\phi_\beta>0$ if $r>\sigma(\beta)$.
	\end{enumerate}

	\seperate

	\emph{Some remarks: Why is this $\sigma$ unique? There are infinitely many functions that agree with $\sigma(0)>0$ and $\sigma(\bar{\beta})=0$. Clearly, the other property of $\sigma$ uniquely defines the function. Draw the following: $x$-axis is the $\beta$-axis and goes from 0 to $\bar{\beta}$. The $y$-axis is the $r$-axis and goes from 0 (in $\bar{\beta}$) to some finite number $\sigma(0)>0$. For any $\beta$ between 0 and $\bar{\beta}$ there is a corresponding $\sigma(\beta)$. As this value needs to agree with the property of $\phi_{\beta}(r)$--$\phi_{\beta}(r)<0$ for $r<\sigma(\beta)$ and $\phi_{\beta}(r)>0$ for $r>\sigma{\beta}$-- for all $\beta$ then $\sigma$ is uniquely defined!}\\[11pt]

% TODO finish the proof that there exists a unique function sigma that satisfies the properties stated!

\begin{proof}
	\begin{outline}
		\1
		\1
		\1
		\1
	\end{outline}

\seperate

Let $\beta>0$ and $r\in[0,z(\alpha))$.
Then \begin{align*}
	\phi_{\beta}(r)&=\left[\beta(p-1)-2\right]V(r)u(r)^p-rV'(r)u(r)^p+2\lambda u(r) %\\ &=V(r)u(r)^p\left[\beta(p-1)-2-\frac{rV'(r)u(r)^p}{V(r)u(r)^p}+\frac{2\lambda u(r)}{V(r)u(r)^p}\right]
\\ &=V(r)u(r)^p\left[\beta(p-1)-2-r\frac{V'(r)}{V(r)}+\frac{2\lambda}{V(r)u(r)^{p-1}}\right]
\\ &= V(r)u(r)^p\left[\beta(p-1)-2-\xi(r)\right]
\\\text{where }\xi(r)&=r\frac{V'(r)}{V(r)}-\frac{2\lambda}{V(r)u(r)^{p-1}}.
\end{align*}

\seperate

To conclude about the sign of $\phi(r)$, note that $V(r)\geq0$ and $u(r)\geq0$.
Hence the sign of $\phi(r)$ will vary with $\beta$ and $r$ as dictated by the term in brackets.
Write $\beta(p-1)-2-\xi(r)>0\iff\beta>\frac{2+\xi(r)}{p-1}\coloneqq\Xi(r)$.

The function $\xi(r)\leq0$ is strictly decreasing on $(0,z(\alpha))$
with $\lim_{r\to z(\alpha)}\xi(r)=-\infty$. {\color{red}
To see this, note $h(r)=r\frac{V'(r)}{V(r)}$ is nonincreasing,
$V'(r)\leq0$ and $u'(r)<0$ hence the second term of $\xi(r)$ is strictly decreasing.
Thus $\xi(r)$ is strictly decreasing.
Since $u(z(\alpha))=0$, the limit is $-\infty$.}

It remains to show $\Xi(r)=\left[2+\xi(r)\right]/(p-1)$ satisfies $\Xi(0)>0$.
Remember that $\xi(r)$ is continuous and decreasing, so $\Xi(r)$ is too. \\

\seperate

With this information, $\Xi(r)$ is also continuous and strictly decreasing.

Indeed, let $\Xi(0)=\bar\beta$, LLLLLL, for all $\beta\in[0,\bar\beta]$.
Note $\Xi(r)$ is continuous and decreasing, so
$\sigma(\beta)\coloneqq\left.\Xi(r)^{-1}\right|_{[0,\bar\beta]}$
has properties (a) and (b). SHOW. \\

Also, note $\Xi(0)>0\iff\xi(0)>-2$. SHOW.\\

\seperate

\emph{}\\[11pt]\emph{TODO: Show that there is an implication from the property of $V(0)$ to the property of $\xi(0)$. By definition? $\xi(0) = 0 * V'(0)/V(0) + 2\lambda / (V(0)*u(0)^{p-1}$... What do I know about $V'(0)?$ And about $V(0)?$ By lemma 5.1 ?? Also, as $u(0)=\alpha>0$, we have $u(0)^{p-1}>0$ and $V(0)>0$ (finite or infinite) implies the second term of $\xi > 0$... Yes, definitely some Lyapunov/lemma 5.1 property involved here.}\\[11pt]
Consider the following cases ......$V(0)<\infty$ .... $V(0)=\infty$ ... hence $\xi(0)>-2$ for $\alpha\in(G\cup N)$.

\end{proof}
\end{lemma}
