% !TEX root = main.tex

\begin{lemma}
  Let $\alpha^*\in N$.
  Then $[\alpha^*,\infty)\subset N\text{ and }z:[\alpha^*,\infty)\to(0,\infty)$ is monotone decreasing.
\begin{proof}
% \begin{outline}
%   \1 First of all, $N$ is open.
%     \2 To see this, let $\hat\a\in N$.
%     \2 There exists $r_0$ such that $u(r_0;\hat\a)=0$.
%     \2 Let $\hat r> r_0$ then $u(\hat r;\hat\a)<0$.
%     \2 By continuous dependence of $u$ on $\a$, for all $\a$ sufficiently close to $\hat\a$ one has $u(\hat r;\a)<0$.
%     \2 Thus $N$ is open.
%   \1 Secondly, $z:N\to(0,\infty)$ is continuous.
%     \2 For all $\epsilon>0$ there exists a $\delta>0$ such that
% \end{outline}

% \seperate

% \underline{$N$ is open}
Let $\hat\alpha\in N$.
By definition of $N$, there exists a $\hat r>0$ such that $u(\hat r;\hat\alpha)<0$.
By continuous dependence on the initial data [[Cod. Lev.]],
for all $\alpha$ sufficiently close to $\hat\alpha$, $u(\hat r;\alpha)<0$ as well.

% \underline{$z$ is continuous}
By definition of $N$ and continuous dependence on the initial data,
none of the solutions in $N$ can be tangent to the $r$-axis.
Thus $z:N\to(0,\infty)$ is continuous.
% None of these solutions can be tangent to the $r$-axis, \_SHOW
% hence the function $z:N\to(0,\infty)$ is continuous.
% That is...

% \seperate

% \underline{$z$ is decreasing}
Let $\alpha^*\in N$.
Then by \ref{}, $w(z(\alpha^*))<0$ and for
$\epsilon>0$ sufficiently small,
$(\alpha^*,\alpha^*+\epsilon)\subset N$
and $u(z(\alpha^*),\alpha)<0$
for all $\alpha\in(\alpha^*,\alpha^*+\epsilon)$.

Remember that $w$ is the derivative of $u$ with respect to the initial condition.
Since $w(z(\alpha^*))<0$, for initial conditions upward of $\alpha^*$ ($\alpha\in(\alpha^*,\alpha^*+\epsilon)$):
$u(\alpha,z(\alpha^*))<u(\alpha^*,z(\alpha^*))=0$.
By the intermediate value theorem [[Cod. Lev.]], there exists a $r\in(0,z(\alpha^*))$ such that $u(\alpha,r)=0$.
Then $z(\alpha)\leq r\leq z(\alpha^*)$ for all $\alpha\in(\alpha^*,\alpha^*+\epsilon)$.
Conclusion: $z$ is decreasing on $(\alpha^*,\alpha^*+\epsilon)$.

\seperate

\underline{Domain of $z$ extends to infinity}
In fact, $z$ is decreasing on $[\alpha^*,\infty)$.
That is, let $$\bar\alpha\coloneqq\sup\{\alpha>\alpha^*\subset N\text{
and }z:[\alpha^*,\alpha)\to(0,\infty)\text{ is decreasing}\}.$$
Then the lemma requires $\bar\alpha=\infty$.
Suppose by contradiction $\bar\alpha<\infty$.
Then there exists $z(\bar\alpha)\coloneqq\lim_{\alpha\to\bar\alpha}z(\alpha)\in[0,\infty).$
Clearly, $\bar\alpha\in N$, since $u(\bar\alpha,z(\bar\alpha)=0$ by continuity of $z$.
But then $[\bar\alpha,\bar\alpha+\epsilon)\in N$ for $\epsilon>0$ sufficiently small.
This contradicts the definition of $\bar\alpha$ as the supremum.
Then $\bar\alpha=\infty$.
Conclusion: for $\alpha^*\in N$, $[\alpha^*,\infty)\subset N$
and $z:[\alpha^*,\infty)\to(0,\infty)$ is decreasing.

\end{proof}
\end{lemma}
