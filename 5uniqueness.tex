\newpage \section{Uniqueness of ground state}
\comm{Uniqueness is proven by showing solution set $G$ contains at most one point: since $G$ is non-empty by chapter \ref{}, then $G$ contains precisely one point. The first lemma states that any points near a ground state initial condition ($\alpha\in G$) are not ground state. [[how about P?]]}

% !TEX root = main.tex
\newpage
\begin{lemma}\label{lya}
  Suppose that $V(0):=\underset{r\to0}{\lim}~V(r)$ exists and is finite.
  Then $$0<\alpha<\left[\left(\frac{p+1}{2}\right)\frac{\lambda}{V(0)} \right]^{1/(p-1)}\implies \alpha\in P.$$

% \emph{\color{teal}Some remarks: as mentioned, the function $V(r)$ possibly has a singularity in $r=0$. In this lemma, we suppose the limit of $V(r)$ for $r\to0$ exists and is finite. Then a set of initial conditions is implied for which the solution is positive everywhere. This set is bound below by 0 and the upper bound is given by analysis of the Lyapunov function for this problem. Then, as the Lyapunov function is nonincreasing, for negative Lyapunov initial value ($E(0)<0$) the Lyapunov function remains negative. However, evaluating the Lyapunov function in $z(\alpha)$ or for $r\to\infty$ respectively, it follows that certain initial conditions yield nonnegative Lyapunov function values! From these contradictions, the set of initial conditions for which the solution belongs to $P$ is concluded.}

% \emph{\color{red}I would like to derive the Lyapunov function and explain/show how it is analogous to the classical dynamical potential. PHYSICS :). The argument that these initial conditions lead to negative $E(r)$ is given, but needs restructuring. Now, are these all initial conditions yielding solutions of $P$? Is it possible for $V(0)$ to be infinite and still yield $u\in P$?}

\begin{proof}
% TODO: Actually, don't need this r_0 business.. $\a\in P\implies\text{ PD for all }r>0$.
% Remember solutions $u(r;\a)$ to \ref{ivp} with $u(0)=\alpha\in P$ are positive definite on $(0,r_0)$ for some $r_0$ that satisfies $u'(r_0)=0$. Do such initial conditions exist? What interval of initial conditions belongs to $P$?
Remember solutions $u(r;\a)$ to \ref{ivp} with $u(0)=\alpha\in P$ are positive everywhere. Do such initial conditions exist? (See Chapter 4: $P$ is nonempty.) What interval of initial conditions belongs to $P$?
% TODO: define _the_ Lyapunov function $E(r)$ as
% Is it the Lyapunov function or some Lyapunov function?
% When is a function a Lyapunov function? Does this function satisfy those conditions?
To determine such an interval (of initial conditions), define (the Lyapunov or the energy) function $E(r)$ on $(0,z(\a))$ as:
$$E(r)\coloneqq\frac{1}{2}u'(r)^2-\frac{\lambda}{2}u(r)^2+\frac{1}{p+1}V(r)u(r)^{p+1},$$
% for $r\in[0,\infty)$ if $\alpha\in P\cup G$ and $r\in[0,z(\alpha)]$ if $\alpha\in N$.
% rewrite \ref{ivp} such that $u'(r)$ in $E(r)$ and
Then, calculate $E'(r)$, where the IVP can be used to simplify the expression:
\begin{align*}
    % &u''(r)+\frac{1}{r}u'(r)-\lambda u(r)+V(r)u(r)^p=0\\
    % &\left[u''(r)-\lambda u(r)+V(r)u(r)^p\right]=-\frac{1}{r}u'(r)\\
    E'(r)&=u''(r)u'(r)-\lambda u(r)u'(r)+V(r)u(r)^pu'(r)+\frac{1}{p+1}V'(r)u(r)^{p+1}\\
    &=\left[u''(r)-\lambda u(r)+V(r)u(r)^p\right]u'(r)+\frac{1}{p+1}V'(r)u(r)^{p+1}\\
    &=-\frac{u'(r)^2}{r}+\frac{1}{p+1}V'(r)u(r)^{p+1}\leq0\text{ for }r>0.\\
    \text{as }&\left[u''(r)-\lambda u(r)+V(r)u(r)^p\right]=-\frac{1}{r}u'(r)
\end{align*}
% {\color{red}Layout?}
% TODO: u(r) is not positive everywhere if $\a\in N$...?
Each of the terms of $E'(r)$ is non-negative: (i) by hypothesis H2, $V'(r)\leq0$; (ii) $u(r)^{p+1}\geq0$ because $u(r)$ is positive on $(0,z(\a))$ for any initial condition; (iii) $r\geq0$ and (iv) $u'(r)^2\geq0$. \underline{Result:} $E(r)$ is non-increasing $(0,z(\a))$.
% \seperate

% TODO: Choose order of argument, for now: first explain consequence of initial condition.
To conclude about behaviour of solutions by type of initial condition, regard $E(r)$ for large $r>0$.
Suppose $\a\in N$, then $u(z(\a))=0$.
Now evaluate:
\begin{empheq}{align*}
  \underset{r\to z(\alpha)}{\lim}E(r)&=
  \underset{r\to z(\alpha)}{\lim}\bigg[
    \frac{1}{2}u'(r)^2-\frac{\lambda}{2}u(r)^2+\frac{1}{p+1}V(r)u(r)^{p+1}
  \bigg] \\
  &=u'(z(\alpha))^2\geq0.
\end{empheq}
% $E(z(\a))=\frac{1}{2}u'(z(\alpha))^2\geq0$.
As $E(r)$ is non-increasing, $E(0)\geq0$.
Alternatively, suppose $\a\in G$.
Then $u(r)\to0$ and $u'(r)\to0$ as $\rtinf$.
% TODO: $\limrtoinf V'(r)$?
Then $E(r)=0$ for $\rtinf$, so again, $E(0)\geq0$.
\underline{Results:} $E(0)\geq0$ and $E(r)$ well-defined on $[0,z(\a)]$ for $\a\in G\cup N$.

For $\a\in P$, require $E(0)<0$.
Now evaluate $E(0)$ and solve for $\a$:
\begin{align*}
  E(0)=&\frac{1}{2}u'(0)^2-\frac{\lambda}{2}u(0)^2+\frac{1}{p+1}V(0)u(0)^{p+1}<0\\
  &\iff -\frac{\lambda}{2}\alpha^2+\frac{1}{p+1}V(0)\alpha^{p+1}<0\\
  &\iff\alpha^{p-1}<\left(\frac{p+1}{2}\right)\frac{\lambda}{V(0)}\\
  &\iff\alpha<\left[\left(\frac{p+1}{2}\right)\frac{\lambda}{V(0)} \right]^{1/(p-1)}
  % \iff\alpha^{p-1}<\left(\frac{p+1}{2}\right)\frac{\lambda}{V(0)}\iff
%\frac{1}{p+1}V(0)\alpha^{p-1}<\frac{\lambda}{2}\\
  % -\frac{\lambda}{2}\alpha^2+\frac{1}{p+1}V(0)\alpha^{p+1}<0\\
  % \implies E(0)=\frac{1}{2}u'(0)^2-\frac{\lambda}{2}u(0)^2+\frac{1}{p+1}V(0)u(0)^{p+1}<0.
\end{align*}
\underline{Conclusion:} $\a\in P$ whenever $0<\a<\left[\left(\frac{p+1}{2}\right)\frac{\lambda}{V(0)}\right]^{1/(p-1)}$.
% \seperate
% To prove $\alpha\in P$, consider the contradictory cases:
% (i) $\alpha\in N$, (ii) $\alpha\in G$.
% Suppose $\alpha\in N$.
% Then $u(z(\alpha))=0$ and $E(z(\alpha))=\frac{1}{2}u'(z(\alpha))^2\geq0$.
% This contradicts $E<0$.
% Suppose $\alpha\in G$.
% Then $u(r)\to0,u'(r)\to0$ as $r\to\infty$.
% Then $E(r)\to0$ as $r\to\infty$, contradicting $E<0$.
% \underline{Conclusion:} $\alpha\in P$.

% By these properties of $E(r)$ the implication will follow.
% What about the initial value of $E(r)$? Evaluate
% Let $0<\alpha<\left[\left(\frac{p+1}{2}\right)\frac{\lambda}{V(0)} \right]^{1/(p-1)}.$
% Rewrite this assumption on $\alpha$ and evaluate $E(0)$:
% \begin{gather*}
%   \alpha<\left[\left(\frac{p+1}{2}\right)\frac{\lambda}{V(0)} \right]^{1/(p-1)}
%   \iff\alpha^{p-1}<\left(\frac{p+1}{2}\right)\frac{\lambda}{V(0)}\iff
% %\frac{1}{p+1}V(0)\alpha^{p-1}<\frac{\lambda}{2}\\
%   -\frac{\lambda}{2}\alpha^2+\frac{1}{p+1}V(0)\alpha^{p+1}<0\\
%   \implies E(0)=\frac{1}{2}u'(0)^2-\frac{\lambda}{2}u(0)^2+\frac{1}{p+1}V(0)u(0)^{p+1}<0.
% \end{gather*}
% Remember $E(r)$ is nonincreasing, so $E(r)<0$ for $r>0$.
% \seperate
% To prove $\alpha\in P$, consider the contradictory cases:
% (i) $\alpha\in N$, (ii) $\alpha\in G$.
% Suppose $\alpha\in N$.
% Then $u(z(\alpha))=0$ and $E(z(\alpha))=\frac{1}{2}u'(z(\alpha))^2\geq0$.
% This contradicts $E<0$.
% Suppose $\alpha\in G$.
% Then $u(r)\to0,u'(r)\to0$ as $r\to\infty$.
% Then $E(r)\to0$ as $r\to\infty$, contradicting $E<0$.
% \underline{Conclusion:} $\alpha\in P$.

\end{proof}
\end{lemma}

% \subsection{Discussion} Consider the Lyapunov function $E(r)$, an analogue to the potential function of classical dynamics. Lyapunov theory treats the stability of a solution near an equilibrium point. For more on Lyapunov theory, see \cite{}.


\newpage
\begin{lemma}Let $\alpha\in G\cup N$, and $u=u(\alpha,r)$. Then $u'(r)<0$ for all $r\in(0,z(\alpha))$ and $u'(z(\alpha))<0$ if $\alpha\in N$.
\begin{proof} 
Write $z(\alpha)=\infty$ when $\alpha\in G$, since: $u(\alpha,r)\to0$ as $r\to\infty$. %Thus $z(\alpha)=\infty$. 
Let $\alpha\in G\cup N$. By lemma \ref{lya} then, $E(r)$ is well-defined for $r\in[0,z(\alpha))$ at least and $\lim_{r\to z(\alpha)}E(r)\geq0.$ Evaluate: \begin{empheq}{align*}\underset{r\to z(\alpha)}{\lim}E(r)&=\underset{r\to z(\alpha)}{\lim}\bigg[\frac{1}{2}u'(r)^2-\frac{\lambda}{2}u(r)^2+\frac{1}{p+1}V(r)u(r)^{p+1}\bigg] \\ &=u'(z(\alpha))^2\geq0.\end{empheq} 
Since $E(r)$ is non-increasing, $E(r)\geq0$ for all $r\in[0,z(\alpha)]$.

Suppose $u''(0)=0$ and note $u'(0)=0$, so $u\equiv\alpha$ \Lightning. Suppose $u''(0)>0$ then $u'(r)>0$ and $u(r)>u(0)=\alpha$ for $r>0$ and small. Now $\alpha\in(G\cup N)\iff -\frac{\lambda}{2}\alpha^2+\frac{1}{p+1}V(0)\alpha^{p+1}\geq0$, hence $$E(r)-E(0)=\frac{1}{2}\left[u'(r)^2-u'(0)^2\right]-\frac{\lambda}{2}\left[u(r)^2-u(0)^2\right]+\frac{1}{p+1}\left[V(r)u(r)^{p+1}-V(0)u(0)^{p+1}\right]\geq0.$$ This contradicts $E(r)$ nonincreasing. Hence $u''(0)<0$ and $u'(r)<0$ for $r>0$ and small.

Claim: $u'<0$ on $(0,z(\alpha))$. Suppose by contradiction $r_0=\inf(r>0,u'(r)=0)<z(\alpha)$. Note how $u''(r_0)<0\implies u'(r)>0$ on $(0,r_0)$ \Lightning. Hence $u''(r_0)\geq0$. Invoke \ref{ivp}:
%Note $u'(r_0)=0$ and 
\begin{gather*}u''(r_0)=\lambda u(r_0)-V(r_0)u(r_0)^p\geq0\\u(r_0)\leq\left[\frac{\lambda}{V(r_0)}\right]^{1/(p-1)}<\left[\left(\frac{p+1}{2}\right)\frac{\lambda}{V(r_0)}\right]^{1/(p-1)}\\
\implies-\frac{\lambda}{2}u(r_0)^2+\frac{1}{p+1}V(r_0)u(r_0)^{p+1}<0,\end{gather*}Then using $u'(r_0)=0$, evaluate $E(r_0)$: $$E(r_0)=-\frac{\lambda}{2}u(r_0)^2+\frac{1}{p+1}V(r_0)u(r_0)^{p+1}<0.$$ %\begin{empheq}{align*} \eqref{ivp}:\quad&u''(r)+\frac{N-1}{r}u'(r)+g(u(r))=0 \\ r=r_0, u'(r_0)=0:\quad&u''(r_0)+g(u(r))=0 \\ &-g(u(r_0))=u''(r_0) \\ &\lambda u(r_0)-V(r_0)u(r_0)^p=u''(r_0)\geq 0 \\ &\lambda u(r_0)-V(r_0)u(r_0)^p\geq0\quad[[]]\\ \implies &u(r_0)\leq\Bigg[\frac{\lambda}{V(r_0)}\Big]^{1/(p-1)}<\Bigg[\Bigg(\frac{p+1}{2}\Bigg)\frac{\lambda}{V(r_0)}\Big]^{1/(p-1)}\Bigg]\end{empheq}
But $E(r_0)<0$ contradicts $E(r)\geq0$, so $u'<0$ on $(0,z(\alpha))$.

It remains to show $u'(z(\alpha))<0$ whenever $\alpha\in N$. Suppose $u'(z(\alpha))=0$. Then $u\equiv0$, since $u(z(\alpha))=0$. Hence $u'(z(\alpha))<0$.
%The following claim contradicts $\alpha\in(G\cup N)$. \underline{Claim:} $E(r_0)<0$. To see this, evaluate $E(r)$ in $r_0$ as follows: \begin{empheq}{align*} E(r_0)=~&\frac{1}{2}u'(r_0)^2-\frac{\lambda}{2}u(r_0)^2+\frac{1}{p+1}V(r_0)u(r_0)^{p+1} \\ =&-\frac{\lambda}{2}u(r_0)^2+\frac{1}{p+1}V(r_0)u(r_0)^{p+1} \end{empheq}

%\underline{$u'(z(\alpha))<0$} Now if $\alpha\in N$, note that by the above $u'(r)<0$ on $(0,z(\alpha))$. If $u'(z(\alpha))=0$ then since $u(z(\alpha))=0$, the solution would be trivial, $u\equiv0$. Hence $u'(z(\alpha))<0$ and the proof is complete.
\end{proof}
\end{lemma}
\newpage
\begin{lemma}Let $\alpha\in (G\cup N),\text{ then }w$ has at least one zero in $(0,z(\alpha))$.
\begin{proof}
The Lagrange identity for \ref{ivp} and \ref{} will yield information about the zeroes of $w(r)$. Observe the following identities arising from the differential equations:\begin{gather*}
(ru'(r))'+r\left[-\lambda u(r)+V(r)u(r)^p\right]=0\\
(rw'(r))'+r\left[-\lambda w(r)+pV(r)u(r)^{p-1}w(r)\right]=0.
\end{gather*} Now multiply by $w(r)$ and $u(r)$ respectively, subtract them and integrate from 0 to $z(\alpha)$, \begin{gather*}
\int_0^{z(\alpha)}w(r)(ru'(r))'-u(r)(rw'(r))'dr=%
\int_0^{z(\alpha)}r\left\{pV(r)u(r)^pw(r)-V(r)u(r)^pw(r)\right\}dr,
\end{gather*} and perform partial integration:\begin{gather*}
rw(r)u'(r)\at_0^{z(\alpha)}-ru(r)w'(r)\at_0^{z(\alpha)}-\int_0^{z(\alpha)}ru'(r)w'(r)-ru'(r)w'(r)dr\\=(p-1)\int_0^{z(\alpha)}rV(r)u(r)^pw(r)dr\\z(\alpha)w(z(\alpha))u'(z(\alpha))=(p-1)\int_0^{z(\alpha)}rV(r)u(r)^pw(r)dr.
\end{gather*} Note that $u(z(\alpha))=0$.   In evaluating the integral term, {\color{gray}note $r>0$, $V(r)>0$, and $u(r)^p>0$. Suppose $w>0$ on $(0,z(\alpha))$. Then $z(\alpha)w(z(\alpha))u'(z(\alpha))\leq0$ contradicts $(p-1)\int_0^{z(\alpha)}rV(r)u(r)^pw(r)dr>0$ \Lightning. Hence $w$ has at least one zero in $(0,z(\alpha))$.}

\underline{$\alpha\in G$} Suppose by contradiction that $w>0$ on $(0,\infty)$. {\color{gray} Then rewrite ... and note how the integral is still positive by assumption. Then $\left(\frac{u}{w}\right)'$ is positive. So $\frac{u}{w}$ is increasing. By RRR there exists two independent solutions that satisfy TTT. So TTT for some constants $\alpha_1,\alpha_0$. Since $w>0$ by hypothesis, $\alpha_1\geq0$. Suppose $\alpha_1=0$, then $w(r)\to0$ exponentially as $r\to\infty$. So $w$ changes sign by RRR, a contradiction. On the other hand, suppose $\alpha_1>0$, then by RRR there exists a $C$ such that LLL. To see this, note how RRR implies $u(r)\sim r^{-1/2}\exp^{-\sqrt{\lambda}r}$ as $r\to\infty$. These contradictions yield $w$ changes sign at least once on $(0,z(\alpha))$ for $\alpha\in(G\cup N)$.} the Lagrange identity for and the integral term is positive by the same reasoning as above. Claim: the function $u/w$ is positive and increasing. Firstly, it is positive in the origin, because $u(0),w(0)>0$. Secondly, it is increasing since $r$, $w(r)^2$ are positive, so is the derivative of $u/w$. Also, since $u(0),w(0)>0$ the function \\ \\
\underline{}\\ \\
\underline{}
\end{proof}
\end{lemma}
{\color{gray} Intermezzo: some definitions. Let $\theta(r)\coloneqq-ru'(r)/u(r)$ for $r\in[0,z(\alpha)$ and $\rho\coloneqq\theta^{-1}$. Then PPProperties. Many auxiliary functions will be introduced now. These functions construct the information needed to prove $w$ has a unique zero. Bear with me as the following functions and variables are introduced: $\theta(r),\beta,\rho(\beta),\phi_{\beta}(r),\nu_{\beta}(r)$. In the lemma that follows, even more functions and variables are introduced: $\bar\beta,\sigma(\beta),\xi(r), \Xi(r)\coloneqq\sigma(\beta)^{-1},\beta_0$.}

{\color{teal}Define $\theta(r)\coloneqq-ru'(r)/u(r)$ on $r\in[0,z(\alpha))$. Note $\theta(r)$ has the following properties: \begin{enumerate}[(i)]\item $\theta(0)=0$, \item $\theta'(r)>0$ for all $r\in(0,z(\alpha))$, and \item $\underset{r\to z(\alpha)}{\lim}\theta(r)=\infty$. \end{enumerate} SHOW THETA(R) HAS THESE PROPERTIES.

Define $\rho\coloneqq\theta^{-1}$. Since $\theta(r)$ is continuous and increasing SHOW.., there exists a unique $r=\rho(\beta)>0$ such that $\theta(r)=\beta$. SHOW. Then $\rho(r)$ is continuous and increasing, with $\rho(0)=0$ and $\underset{\beta\to\infty}{\lim}\rho(\beta)=z(\alpha).$ SHOW.

Define $\nu_{\beta}(r)\coloneqq ru'(r)+\beta u(r)=-u(r){\theta(r)-\beta}.$ Let $\beta>0,$ then $\nu_{\beta}(r)>0$ if $r<\rho(\beta)$ and $\nu_{\beta}(r)<0$ if $r>\rho(\beta)$. SHOW.

Define $\phi_{\beta}(r)\coloneqq\left[\beta(p-1)-2\right]V(r)u(r)^p-rV'(r)u(r)^p+2\lambda u(r).$ Now observe $\nu_{\beta}(r)$ satisfies the differential equation $$\nu_{\beta}''(r)+\frac{1}{r}\nu_{\beta}(r)'-\lambda\nu_{\beta}+pV(r)u(r)^{p-1}\nu_{\beta}=\phi_{\beta}(r)$$} SHOW.

\begin{outline}
  \1 After these three lemmata, the following results are obtained:
    \2 $\a\in P$ for $\a\in(0,\a_0)$ with $\a_0=
      \left[\frac{\lambda}{V(0)}\frac{p+1}{2}\right]^{\frac{1}{p-1}}$, we have $\a\in P$.
    \2 For $\a\in G$ we have $u'(r)<0$ on $(0,\za)$ and for $\a\in N$ we have $u'(r)<0$ on $(0,\za]$.
    \2 For $\a\in G\cup N$ at least one zero for $w(r)$ in $(0,\za)$. In other words, $w$ oscillates faster.
  \1 To obtain results about the zeroes of $u$ we will need more information about the zeroes of $w$.
  \1 In fact, it will be shown that $w$ has a unique zero in $(0,\za)$.
  \1 Then, by the difference function $z\coloneqq u-\tilde u$ and Sturm theory:
    \2 $\a\in N\implies \tilde\a\in N\text{ for }\tilde\a\in(\a,\a+\epsilon)$.
  \1 Lastly, $\a\in N\implies\tilde\a\in N\text{ for }\tilde\a\in[\a,\infty)$.
    \2 So all initial conditions greater than or equal to initial conditions with solution in $N$ are also in $N$.
    \2 Furthermore, the difference between $u$ and $\tilde u$ is decreasing with respect to increasing initial condition.
  \1 This is sufficient to show that any initial condition below $\a_0\in G$ cannot be in $N$ as that would contradict $\a_0\in G$ by this property of initial conditions in $N$. As $G$ is nonempty by chapter 4 and $G$ and $N$ are now disjoint... also noting $P$ is disjoint with $G$ and $N$ and nonempty by lemma 5.1... $\a_0\in G$ is unique!
    \2 Most work remains in explaining using lemmata 7 and 8 that $G$ contains at most one point.
\end{outline}
\noindent\hrulefill
\begin{outline}
  \1 In the lemmata following, many related properties are used. This section is dedicated to properly introducing those functions and their properties.
  \1 First, let $\theta(r)\coloneqq-r\frac{u'(r)}{u(r)}$ for $r\in[0,\za]$.
    \2 Note that by IVP: $\theta(0)=\underset{r\downarrow0}{\lim}\theta(r)=\underset{r\downarrow0}{\lim}-r\frac{u'(r)}{u(r)}=\underset{r\downarrow0}{\lim}\frac{-ru'(r)}{\a}=0$
    \2 By properties $r>0$, $u'(r)<0$ on $(0,\za)$ and $u(r)>0$ on $(0,\za)$: $\theta(r)>0$ on $(0,\za)$.
    \2 Also, $\theta(r)$ is increasing, that is $\theta(r)'>0$.
      \3 To see this... SHOW
    \2 The limit $\underset{r\to\za}{\lim}\theta(r)=\infty$.
      \3 To see this for $\a\in N$ SHOW
      \3 To see this for $\a\in G$ SHOW
  \1 As $\theta(r)$ is continuous and increasing, there exists $\rho\coloneqq\theta^{-1}$ as there exists a unique $r=\rho(\beta)>0\text{ s.t. } \theta(r)=\beta$.
    \2 Note that $\rho(0)=0$ and $\underset{\beta\to\infty}{\lim}\rho(\beta)=\za$.
    \2 Note this $\rho(\beta)$ is continuous and increasing with respect to $\beta$.
    \2 Note that $\rho(r)>0$ on $(0,\infty)$.
  \1 Now define $\nub\coloneqq ru'(r)+\beta u(r)=-u(r)\left[\theta(r)-\beta\right]$
  \1 Let $\beta>0$ then $\nub(r)>0$ if $r<\rho(\beta)$ and $\nub<0$ if $r>\rho(\beta)$.
    \2 Note that $u(r)>0$ so $\nub>0$ if $\theta(r)-\beta<0$. That is, $0>\theta(r)-\beta\iff\beta>\theta(r)\iff\rho(\beta)>r$. Similarly, $\nub<0$ if $\beta<\theta(r)\iff\rho(\beta)<r$.
  \1 All of these properties can also be illustrated by beautiful figures. SHOW
  \1 \emph{Complicated:} $\nub(r)$ satisfies the following differential equation:
    \2 $\nub''(r)+\frac{1}{r}\nub'(r)-\lambda\nub(r)+pV(r)u(r)^{p-1}\nub(r)=\phib(r)$
    \2 Where $\phib(r)=\left[\beta(p-1)-2\right]V(r)u(r)^p-V'(r)u(r)^p+2\lambda u(r)$.
    \2 SHOWWWWWWW
  \1 In lemma 4 this will be used to show that the sign of $\phib$ is related to a continuous decreasing function $\sigma(\beta)$ and this function crosses $\rho(\beta)$ in a unique $\beta_0$ (lemma 5.5). The specific $\nu_{\beta_0}$ and $\rho_0=\rho(\beta_0)$ then are used in strong version of Sturm comparison theorem in lemma 6 to show that $w$ has a unique zero on $(0,\za)$! COOL! Hooray! Useful.
\end{outline}

\newpage
\begin{lemma}Let $\alpha\in G\cup N$. There exist $\beta_0>0$ and a unique function $\sigma:[0,\bar\beta]\to[0,\infty)$ with the following properties: \begin{enumerate}[(a)]
	\item  $\sigma$ is continuous and decreasing, $\sigma(0)>0$ and $\sigma(\bar\beta)=0$;
    \item for all $\beta>0$ we have: $\phi_\beta(r)<0$ if $r<\sigma(\beta)$, and $\phi_\beta>0$ if $r>\sigma(\beta)$.
\end{enumerate}
\begin{proof} 
Let $\beta>0$ and $r\in[0,z(\alpha))$. Then \begin{align*}
\phi_{\beta}(r)&=\left[\beta(p-1)-2\right]V(r)u(r)^p-rV'(r)u(r)^p+2\lambda u(r) %\\ &=V(r)u(r)^p\left[\beta(p-1)-2-\frac{rV'(r)u(r)^p}{V(r)u(r)^p}+\frac{2\lambda u(r)}{V(r)u(r)^p}\right] 
\\ &=V(r)u(r)^p\left[\beta(p-1)-2-r\frac{V'(r)}{V(r)}+\frac{2\lambda}{V(r)u(r)^{p-1}}\right] 
\\ &= V(r)u(r)^p\left[\beta(p-1)-2-\xi(r)\right]
\\\text{where }\xi(r)&=r\frac{V'(r)}{V(r)}-\frac{2\lambda}{V(r)u(r)^{p-1}}.
\end{align*}

To conclude about the sign of $\phi(r)$, note that $V(r)\geq0$ and $u(r)\geq0$. Hence the sign of $\phi(r)$ will vary with $\beta$ and $r$ as dictated by the term in brackets. Write $\beta(p-1)-2-\xi(r)>0\iff\beta>\frac{2+\xi(r)}{p-1}\coloneqq\Xi(r)$.

Note that $\xi(r)\leq0$ and strictly decreasing on $(0,z(\alpha))$ with $\lim_{r\to z(\alpha)}\xi(r)=-\infty$. To see this, note $h(r)=r\frac{V'(r)}{V(r)}$ is nonincreasing, $V'(r)\leq0$ and $u'(r)<0$ hence the second term of $\xi(r)$ is strictly decreasing. Thus $\xi(r)$ is strictly decreasing.

It remains to show $\Xi(r)=\left[2+\xi(r)\right]/(p-1)$ satisfies $\Xi(0)>0$. Remember that  Indeed, let $\Xi(0)=\bar\beta$, LLLLLL, for all $\beta\in[0,\bar\beta]$. Note $\Xi(r)$ is continuous and decreasing, so $\sigma(\beta)\coloneqq\left.\Xi(r)^{-1}\right|_{[0,\bar\beta]}$ has properties (a) and (b). SHOW. \\

Also, note $\Xi(0)>0\iff\xi(0)>-2$. SHOW.\\

Consider the following cases ......$V(0)<\infty$ .... $V(0)=\infty$ ... hence $\xi(0)>-2$ for $\alpha\in(G\cup N)$.

\end{proof}
\end{lemma}
% !TEX root = main.tex

\begin{lemma}
  Let $\alpha\in G\cup N$.
  There exists a unique $\beta_0>0$ such that $\rho(\beta_0)=\sigma(\beta_0)$.
  This follows immediately from the aforementioned properties of $\rho(\beta)$ and $\sigma(\beta)$.
  Sketching the two graphs, there is a unique intersection.
  % TODO: actually make the graph

%\begin{proof} \end{proof}
\end{lemma}

\begin{lemma}\label{wq}For $\alpha\in G\cup N,w$ has a unique zero $r_0\in(0,z(\alpha)).$ Furthermore, $w(z(\alpha))<0$ if $\alpha\in N$ and $\underset{r\to\infty}{\lim}w(r)=-\infty$ if $\alpha\in G$.

\emph{}\\[11pt]\emph{Some remarks: the functions $u(r)$ and $w(r)$ will be compared `'in the rest of the uniqueness proof'. This lemma derives that $w(r)$ has a \textbf{unique} zero $r_0$ to the left of $z(\alpha)$. Remember $z(\alpha)$ is the zero of $u(r)$. On the other hand, if $u(r)$ is positive and decreasing everywhere--has no finite zero $z(\alpha)$ such that $u(z(\alpha))=0$--then still $w(r)$ has a finite zero $r_0$ and $w(r)$remains negative to the right of the zero. Even more, the function $w(r)$ tends to negative infinity for $r\to\infty$, $\underset{r\to\infty}{\lim}w(r)=-\infty$.}\\[11pt]
\begin{proof}
The solution $\nu(r)$ changes sign in $\rho_0\in[0,z(\alpha)]$. As $\nu(z(\alpha))z(\alpha)u'(z(\alpha))<0$, the solution $\nu(r)$ changes sign once. Let $\tau$ be the first zero of $w(r)$. Then by Sturm comparison, $\rho_0\in(0,\tau)$, i.e. $\nu$ oscillates faster than $w$.

Also by Sturm comparison, $w$ can have no further zero and $\tau$ is the unique zero of $w$.

\underline{Sign of $v$ changes in $\rho_0$} \\ \\

\underline{$v$ changes sign only once} \\ \\

\underline{$w$ has a first zero} \\ \\

\underline{Comparing $w$ and $v$, conclusion: $w$ has a unique zero} \\ \\

\underline{Also if $\alpha\in G$}


\end{proof}
\end{lemma}

% !TEX root = main.tex

\begin{lemma}
  Let $\alpha^*\in N$.
  Then $[\alpha^*,\infty)\subset N\text{ and }z:[\alpha^*,\infty)\to(0,\infty)$ is monotone decreasing.
\begin{proof}
\begin{outline}
  \1 First of all, $N$ is open.
    \2 To see this, let $\hat\a\in N$.
    \2 There exists $r_0$ such that $u(r_0;\hat\a)=0$.
    \2 Let $\hat r> r_0$ then $u(\hat r;\hat\a)<0$.
    \2 By continuous dependence of $u$ on $\a$, for all $\a$ sufficiently close to $\hat\a$ one has $u(\hat r;\a)<0$.
    \2 Thus $N$ is open.
  \1 Secondly, $z:N\to(0,\infty)$ is continuous.
    \2 For all $\epsilon>0$ there exists a $\delta>0$ such that
\end{outline}

\seperate

\underline{$N$ is open}
Let $\hat\alpha\in N$.
Then there exists a $\hat r>0$ such that $u(\hat\alpha,\hat r)<0$.
By continuous dependence on the initial data [[Cod. Lev.]],
$u(\alpha,\hat r)<0$ for all $\alpha$ sufficiently close to $\hat\alpha$.

\underline{$z$ is continuous}
None of these solutions can be tangent to the $r$-axis, \_SHOW
hence the function $z:N\to(0,\infty)$ is continuous.
That is...

\seperate

\underline{$z$ is decreasing}
Let $\alpha^*\in N$.
Then by \ref{}, $w(z(\alpha^*))<0$ and for
$\epsilon>0$ sufficiently small,
$(\alpha^*,\alpha^*+\epsilon)\subset N$
and $u(z(\alpha^*),\alpha)<0$
for all $\alpha\in(\alpha^*,\alpha^*+\epsilon)$.

Remember that $w$ is the derivative of $u$ with respect to the initial condition.
Since $w(z(\alpha^*))<0$, for initial conditions upward of $\alpha^*$ ($\alpha\in(\alpha^*,\alpha^*+\epsilon)$):
$u(\alpha,z(\alpha^*))<u(\alpha^*,z(\alpha^*))=0$.
By the intermediate value theorem [[Cod. Lev.]], there exists a $r\in(0,z(\alpha^*))$ such that $u(\alpha,r)=0$.
Then $z(\alpha)\leq r\leq z(\alpha^*)$ for all $\alpha\in(\alpha^*,\alpha^*+\epsilon)$.
Conclusion: $z$ is decreasing on $(\alpha^*,\alpha^*+\epsilon)$.

\seperate

\underline{Domain of $z$ extends to infinity}
In fact, $z$ is decreasing on $[\alpha^*,\infty)$.
That is, let $$\bar\alpha\coloneqq\sup\{\alpha>\alpha^*\subset N\text{
and }z:[\alpha^*,\alpha)\to(0,\infty)\text{ is decreasing}\}.$$
Then the lemma requires $\bar\alpha=\infty$.
Suppose by contradiction $\bar\alpha<\infty$.
Then there exists $z(\bar\alpha)\coloneqq\lim_{\alpha\to\bar\alpha}z(\alpha)\in[0,\infty).$
Clearly, $\bar\alpha\in N$, since $u(\bar\alpha,z(\bar\alpha)=0$ by continuity of $z$.
But then $[\bar\alpha,\bar\alpha+\epsilon)\in N$ for $\epsilon>0$ sufficiently small.
This contradicts the definition of $\bar\alpha$ as the supremum.
Then $\bar\alpha=\infty$.
Conclusion: for $\alpha^*\in N$, $[\alpha^*,\infty)\subset N$
and $z:[\alpha^*,\infty)\to(0,\infty)$ is decreasing.

\end{proof}
\end{lemma}

\begin{lemma} Let $\alpha\in G$. There exists $\epsilon>0$ such that $(\alpha,\alpha+\epsilon)\subset N$.
\begin{proof} By chapter 4, the solution set $G$ is non-empty. Let $\alpha\in G$ and let $u(\alpha,r)$ be the corresponding solution.  The function $w(\alpha,r)=\frac{\partial}{\partial\alpha}u(\alpha;r)$ satisfies

By lemma \ref{wq}, the function $w$ is unbounded and by lemma \ref{kwong6}, $w$ has a unique zero $r_0\in(d,\infty)$. Let $r_1,r_2$ be such that $d<r_1<r_0<r_2$ and note that $w$ is positive in $r_1$ and negative in $r_2$. There exists $\epsilon>0$ such that, for all $\tilde\alpha\in(\alpha,\alpha+\epsilon)$,$$\tilde u(r_1)>u(r_1)\text{ and }\tilde u(r_2)<u(r_2),$$ where $\tilde u=u(\tilde\alpha,r)$. Hence there exists $r_3\in(r_1,r_2)$ such that the graphs of $u$ and $\tilde u$ intersect, i.e. $\tilde u(r_3)=u(r_3)$. See also figure \ref{}.

To conclude $\tilde\alpha\in N$ requires existence of a $\tilde r$ such that $\tilde u(\tilde r)=0$. By contradiction, suppose that $\tilde u(r)>0$ for all $r>r_3$. (Note that $\tilde u(r)\geq u(r)>0$ on $r\leq r_3$, since $u(r)>0$ for all $r$.)

Define the difference $z\coloneqq u-\tilde u$ and note that proving $z(r)>0$ on $(r_3,\infty)$ implies $\tilde u(r)<u(r)$ on that interval. This function satisfies the differential equation $$\label{zivp} BLA $$.
\\
\\
Also by Sturm comparison of $z$ and $w$, the latter oscillates faster. Let $\tilde w$ be a solution of \eqref{wivp} such that $\tilde w(r_3)=0$. 
\\
\\
By integration of the strong version of Sturm, $z(r)\to\infty$ as $r\to\infty$, but this is impossible as $0<\tilde u(r)<u(r)$ on $(r_3,\infty)$ and $u(r)\to0$ as $r\to\infty$. Therefore, $\tilde u$ vanishes at some point $\tilde r\in(r_3,\infty)$ and the proof is complete.
\\
\\ \comm{
for any $\epsilon>0$, all solutions $\tilde u=u(\tilde\alpha,r)$ with $\tilde\alpha\in(\alpha,\alpha+\epsilon)$ are non-vanishing. Then  Remember that $w$ is the derivative of \eqref{ivp} with respect to the initial condition. And by \ref{} $w$ has a unique zero that is contained in the disconjugacy interval. [[]] Let $r_1,r_2$ be such that $d<r_1<r_0<r_2$ and note $w(r_1)=\frac{\partial}{\partial\alpha}u(\alpha;r_1)>0$ and $w(r_2)=\frac{\partial}{\partial\alpha}u(\alpha;r_2)<0$. Let $\epsilon>0$ such that for all $\tilde\alpha\in(\alpha,\alpha+\epsilon)$, $$\tilde u(r_1)>u(r_1)\text{ and }\tilde u(r_2)<u(r_2),$$ where $\tilde u=u(\tilde\alpha,r)$ and $\tilde\alpha\in(\alpha,\alpha+\epsilon)$. The solutions $u$ and $\tilde u$ intersect in some point $r_3$. THEY STURM COMPARE. RELATION Z=U-UT.} \end{proof}
\end{lemma}

\newpage
\begin{lemma}\label{kwong6} (Lemma 6 in \cite{kwong}) Let $(c,\infty)$ be the disconjugacy interval of \eqref{}. Every solution of \eqref{} with a zero in $(c,\infty)$ is unbounded.

Conversely, if the last zero of an unbounded solution of \eqref{} is $\rho$, then $\rho$ is an interior point of the disconjugacy interval. In the other words, $\rho>c$.
\begin{proof} \end{proof}
\end{lemma}