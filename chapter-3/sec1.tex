\section{Introduction}
% In this chapter, we study the paper "Uniqueness of the Ground State Solution for
% $\Delta u - u + u^3 = 0$ and a Variational Characterization of Other Solutions"
% written in 1972 by Charles V. Coffman. 

% \todo{Make connection to chapter 2 and discuss similarities and differences.}
In \Cref{existence} we studied the existence result \cite{ber81} for
\be\label{updeber}
\Delta u - u = f(u)\quad\text{in }\R^3,
\ee
where $f(u)$ satisfies certain general conditions. Motivated by \Cref{physics},
we are interested in the specific case $f(u) = u - u^3$.  In the present
chapter, we study the uniqueness result \cite{coffm} for the problem 
\be\label{upde}
\Delta u - u + u^3 = 0\quad\text{in }\R^3.
\ee

In particular, \cite{coffm}  proves uniqueness of the positive radially
symmetric (ground state) solution $u=\phi_1\in C^2\cap L^4$ for \eqref{upde}.
All function spaces considered in this chapter consist of real valued functions
on $\R^3$. Furthermore, radial symmetry is with respect to the origin only. 

The existence of such a function $\phi_1$ was shown in \cite{nehari},
where $\phi_1=v_1(|x|)$ solves \eqref{upde} as a specific case of $f(u) = u -
u^k$ with $1<k\leq 4$. 

In addition, \cite{coffm} refers to the existence of functions $v_n(|x|)\in
C^2([0, \infty))$, $n=1, 2, \ldots$, such that for each $n$, $v_n$ has exacly
$n-1$ isolated zeroes in $[0, \infty)$ and decays exponentially as $r\to\infty$.
This was shown in \cite{ryder}, which considered $f(u) = u - ug(u^2)$,
as well as \cite{berger}, which considered
$$i\frac{\partial u}{\partial t} = \Delta u + f(|x|, |u|^2)u$$ 
under the assumption that $f(|x|, |u|^2)=k(|x|)|u|^\sigma$ for $0<\sigma<4$ and
$k(|x|)$ a Lipschitz continuous positive bounded function.

Furthermore, Theorem 3.1 of \cite{coffm} improves the result of \cite{robinson},
which also studied \eqref{upde} (in the context of variational calculus).  In
\cite{robinson}, they show that the Lagrangian associated with \eqref{upde} is
zero in its first variation and the second variation is positive if $\lambda_1 >
1$. The latter is shown only through approximations. Theorem 3.1 of \cite{coffm}
shows that the Rayleigh quotient $J$ associated with \eqref{upde}
\be\label{rayleigh} 
J(u) = \dfrac{\left(\int\left|\nabla u\right|^2+u^2~dx\right)^2}{\int u^4~dx}
\ee

is indeed minimal for %$u=\phi_1$ and for 
\be\label{uts}
u(x)=k\phi_1(x+x_0),\quad\text{where}~k\neq0~\text{and}~x_0\in\R^3
\ee
% for any $k\neq 0$ and $x_0\in \R^3$. 
The right hand side of \eqref{rayleigh} is
meaningful for admissible functions.

\begin{definition} 
    A function $u$ is \emph{admissible} if $u\in H^1$ and $u\neq 0$.
\end{definition}
