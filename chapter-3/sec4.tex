% 4.1 is qlem en 4.2 is qneglem
\section{Uniqueness theorem}
The radially symmetric solutions of \eqref{upde} are of the form
$$u(x)=|x|^{-1}w(|x|),$$ 
where $w(|x|)=w(r)$ solves 
\be\label{wvp}
w''-w+r^{-2}w^3=0.
\ee
The derivation of \eqref{wvp} is included in \Cref{weqv} below.
%
% \todo{Rephrase this paragraph}
To prove the uniqueness of the ground state solution $\phi_1$ for \eqref{upde},
it suffices to prove that \eqref{wvp} has at most one positive solution
satisfying the boundary conditions
\be\label{wbc}
% \underset{r\to0}{\lim}~r^{-1}w(r) in some sense is u(0) \in (0, \infty)
0 < \underset{r\to0}{\lim}~r^{-1}w(r) < \infty,\quad
\underset{r\to\infty}{\lim} w(r)=0.
\ee

% The problem \eqref{wvp},\eqref{wbc} is transformed to an initial value problem where 
Similar to the shooting method in \Cref{existence}, we consider the initial
condition 
\be\label{wic}
\underset{r\to0}{\lim}~r^{-1}w(r)=\a>0,
\ee
rather than the boundary conditions \eqref{wbc}. We write $w=w(r,\a)$ to denote
the solution $w(r)$ with initial condition $\a>0$. The notation in this chapter
differs from the source \cite{coffm} to agree with \Cref{existence}. The
initial conditions $\a>0$ are categorised into three sets $P$, $G$ and $N$,
similar to \eqref{pset}, \eqref{gset} and \eqref{nset} in \Cref{setsdef}.

Furthermore, $\za<\infty$ is the first zero of the solution $u(r)$ if $\a\in N$
and we write $\za=\infty$ if $\a\in G\cup N$.  Formally, let
\be\label{zdef} z(\a)\coloneqq \sup 
\left\{z^\ast>0\;\middle|\; w(r,\a) > 0~
\text{for all}~r\in [0,z^\ast)\right\}.\ee
We formulate the uniqueness of the ground state solution in terms of this
initial condition $\a>0$ in \Cref{uniqthm} below.  The proof method is to show
that $\za$ is monotonically decreasing and that an initial condition $\a_2$ in a
right neighborhood of $\a_1\in G$ must belong to $N$.  The basic facts regarding
the problem \eqref{wvp}, \eqref{wic} are summarised in \Cref{qlem}. The proof of
\Cref{qlem} is omitted in the original paper \cite{coffm}.

\newcommand{\was}{w_\a}
\newcommand{\wa}{w_\a(r,\a)}
\newcommand{\wap}{w_\a'(r,\a)}
\newcommand{\qa}{q_\a(r,\a)}
\newcommand{\qap}{q_\a'(r,\a)}
\begin{lemma}\label{qlem} 
    For each $\a>0$ the equation \eqref{wvp} has a unique solution $w=w(r,\a)$
    which is of class $C^2\left((0, \infty)\right)$ and satisfies
    \eqref{wic}.  The partial derivatives $\wa = \partial_\a w(r, \a) =
    \frac{\partial w(r,\a)}{\partial \a}$ and $\wap = \partial_\a w'(r,
    \a) = \frac{\partial w'(r, \a)}{\partial \a}$ exist for all $r>0$ and $\a>0$.
    %Let $q(r, a)$ denote $w_a(r, a)$. 
    % Furthermore, $\wa$ coincides on $(0, \infty)$ with the solution
    % $\qa$ of the regular initial value problem 
    % \be\label{qvp} \begin{cases}
    % q'' - q + 3r^{-2}w^2q=0,\\ 
    % q(0)=0,\quad q'(0)=1,
    % \end{cases}\ee
    % with $w=w(r, a)$; $q'(r, a)=\wap$; $q''(r, a)=w_a''(r,
    % a)=\frac{\partial w''(r, a)}{\partial a}$.  
    Furthermore, $\was=\wa$ solves the regular initial value problem
    \be\label{qvp} \begin{cases}
        \was'' - \was + 3r^{-2}w^2\was=0,\quad\text{on}~(0,\infty)\\ 
    \was(0,\a)=0,\quad \was'(0,\a)=1,
    \end{cases}\ee
\end{lemma}

% \todo{Rephrase result.}
% \rd{Move up, remove repeat.} In the following lemmata, we will write
% $w(r)=w(r,\a)$ since the solution $w(r)$ depends on the initial condition
% $\a>0$. The uniqueness result is formulated in \Cref{uniqthm} below in terms of
% $\a>0$.
% The solution $w(r)$ to problem \eqref{wvp}, \eqref{wic} depends on the parameter
% $a>0$ (initial condition). This motivates us to write $w=w(r, a)$ and formulate
% the uniqueness result in terms of $a>0$.

\begin{theorem}\label{uniqthm}
There is at most one $\a\in G$. That is, at most one $\a>0$ for which
\be\label{wpos}w(r,\a)>0\quad\text{on}~(0,\infty)\ee% 0<r<\infty\ee
and
\be\label{wlim}\underset{r\to\infty}{\lim} w(r, \a)=0.\ee
\end{theorem}

% \todo{Expand explanation}
% \rd{Move up, remove repeat.} Similar to the approach in \Cref{existence}, the
% solutions $w(r,a)$ are of one of three types: (i) solutions that have a finite
% zero, (ii) solutions that are positive everywhere and (iii) solutions that are
% positive everywhere and vanish at infinity. We refer to definitions
% \eqref{nset}, \eqref{pset} and \eqref{gset} respectively. {\red Revise notation
% of this chapter to match/contract the notation in \Cref{existence} where
% necessary.} Let \rd{sup over $w>0$ rather than inf}
% \be\label{z1def}
% z_1 \coloneqq \inf\left\{\; r>0 \;\middle|\; w(r, a)=0\;\right\}.
% \ee
% We note that $z_1<\infty$ for $a\in N$ and we write $z_1=\infty$ for $a\in G\cup
% N$. 
\rd{Expand introduction of lemma} \Cref{uniqthm} is implied by the following lemma.

\begin{lemma}\label{qneglem}
    \begin{enumerate}[(i)]
        \item If $\a>0$ and $w(r, \a)>0$ on $(0, \za)$ with $w(\za, \a)=0$, then
            $\was(\za,\a)<0$.
        \item If $\a>0$ and $w(r, \a)$ satisfies \eqref{wpos} and \eqref{wlim} then
            \be\label{qexplim}
                \underset{r\to\infty}{\lim} e^{-r}\was(r, \a)<0.
            \ee
    \end{enumerate}
\end{lemma}

\begin{proof}[Proof of Theorem \ref{uniqthm}]
We assume \Cref{qneglem}. The proof will be in two steps. We first show that
$\za$ is monotonically decreasing in $\a>0$. Then, we show that if $\a_1>0$
satisfies the conditions \eqref{wpos}, \eqref{wlim} of \Cref{uniqthm}, then any
$\a_2>\a_1$ with $\a_2-\a_1>0$ arbitrarily small has $z(\a_2)<\infty$.  That is,
$\a_2\in N$.

% \rd{Replaced by $N$} Let
% \be\label{adef}
% A \coloneqq \left\{\;\a>0 \;\middle|\; \exists r>0:\; w(r,\a)=0\;\right\}.
% \ee
%
If $\a\in N$, then $\za<\infty$ and we have $w(\za,\a)=0$ and $w'(\za,\a)<0$.
%\rd{if $N$ is non-empty, then let $\a\in N$?} 
Then, by the implicit function theorem $\za$ is differentiable with respect
to $\a$ on $N$ and we have

$$\was(\za, \a) + w'(\za, \a)\frac{\diff \za}{\diff \a} = 0.$$

The term $\was(\za,\a)<0$ by \cref{qlem}(i) and $w'(\za,\a)<0$ by the
definition of $\za$. Hence 
$$\dfrac{\diff \za}{\diff \a}<0\quad\text{on}~N.$$ 
%
Therefore, $\za$ is monotonically decreasing in $\a>0$. Hence, if $N$ is
non-empty, it is a semi-infinite interval.
    % \end{proof}
   
    % \todo{Proof step 2. $a_1$ is the left endpoint of $A$ if $a_1$ satisifies
    % \eqref{wpos} and \eqref{wlim}.}
In the second proof step, we assume that \eqref{wpos} and \eqref{wlim} hold for
$\a_1>0$, or equivalently, that $\a_1\in G$. Under this assumption, we deduce
that $\a_1$ is the left endpoint of $N$.
% is non-empty and we show that if \eqref{wpos} and \eqref{wlim} hold for
% $a=a_1$, then $A$ is non-empty and $a_1$ is the left endpoint of $A$. 
This will clearly imply \Cref{uniqthm}.
   
Let $\a_1\in G$ and let $\a_2>\a_1$. We show that if $\a_2-\a_1$ is
sufficiently small, then the assumption 
\be\label{wass} w(r, \a_2) > 0\quad\text{on}~(0, \infty)\ee 
leads to a contradiction. Note that assumption \eqref{wass} implies  
$\a_2\not\in N$. % since $w(r, \a_2)$ is assumed positive on $(0, \infty)$.
We put $w_i=w_i(r, \a_i)$ for $i=1, 2$\footnote{Correcting a typo in
the definition of $w_i$ given in \cite{coffm}.}. 

By our assumption on $\a_1$ and ii) of \cref{qlem}, we can choose $r_0>0$
such that

\be\label{ubw1} 3r^{-2}w_1^2<\frac{1}{2},\quad\text{for}~r\geq r_0\ee
and

\be\label{qneg} \was(r_0, \a_1)<0,\quad\text{and}~\was'(r_0, \a_1)<0.\ee

From \cref{qlem} and \eqref{qneg} it follows that if $\a_2>\a_1$ and $\a_2-\a_1$
is sufficiently small, then
\be\label{w2ltw1} 
w_2(r_0,\a_1)<w_1(r_0,\a_1),\quad w_2'(r_0,\a_1)<w_1'(r_0,\a_1).\ee

We put $v = w_1-w_2$, so that $v$ satisfies\footnote{Correcting a 
typo in (4.12) of \cite{coffm}.}
\be\label{vivp} v'' - v + r^{-2}\left(w_1^2+w_1w_2+w_2^2\right)v=0.\ee

We suppose that $w_2(r,\a_2)$ satisfies \eqref{wass} and that for some $r_1>r_0$
\be\label{w2ltw1r0r1} 0<w_2<w_1\quad\text{on}~\left[r_0,r_1\right).\ee

Such an $r_1$ exists by \eqref{w2ltw1}.
% \todo{Untangle the argumentation and make some pictures. Compare to Genoud.}
\begin{itemize}
    \item From \eqref{w2ltw1}, \eqref{vivp} and \eqref{ubw1} it follows that
        $v$ is positive and convex on $\left[r_0, r_1\right)$; 
    \item moreover, from \eqref{w2ltw1}, $v'(r_0)>0$ so that $v$ is
        increasing on $\left[r_0, r_1\right)$. 
    \item Thus \eqref{w2ltw1r0r1} holds at $r=r_1$;
    \item hency by a standard argument we conclude that \eqref{w2ltw1r0r1}
        holds on $\left[r_0, \infty\right)$ and that $v$ is increasing
        there. 
    \item The inequality \eqref{w2ltw1r0r1} on $\left[r_0, \infty\right)$
        implies that
        \be\label{wlimsumsq}
            \underset{r\to\infty}{\lim}\left(w_1^2+w_1w_2+w_2^2\right)=0.
        \ee
    \item Using this fact and the monotone character of $v$, we conclude
        from asymptotic integration of \eqref{vivp} that $v$ grows
        exponentially as $r\to\infty$.
    \item From the definition of $v$, this is clearly a contradiction of
        \eqref{wlim} for $\a=\a_1$ and \eqref{wass}. 
    \item We conclude therefore that \eqref{wass} cannot hold for $\a_2>\a_1$
        and $\a_2-\a_1$ arbitrarily small;
    \item therefore $\a_2\in N$ for all $\a_2>\a_1$ with $\a_2-\a_1$ sufficiently
        small. 
    \item This completes the proof of \Cref{uniqthm}.
\end{itemize}
\end{proof}

% Also, another (now regular) initial value problem in $\delta(r,a)$ is introduced. 
% 
% **Question:** Regularity in which sense? Usually smoothness, here 
% Theorem 4.1 claims uniqueness of the parameter $a>0$ for which $w(r, a)$ is positive on $(0, \infty)$ and vanishes at infinity. This is implied by Lemma 4.2.
% 
By studying the zeroes of $w(r,\a)$, we can show that $N$ has a left endpoint.

\newcommand{\va}{v(r, a)}
\begin{lemma}\label{vlem}
Let $\a>0$ and let $w=w(r,\a)$ either vanish at least once in $(0, \infty)$ or
satisfy \eqref{wlim}; then $\a>\sqrt{2}$, $w(r, \a)=r$ for precisely one value
$r=r_0$ in $(0, \za)$, and $w'(r_0, \a)<0$.
\end{lemma}
\begin{proof}
{\red Proof steps.} The function $\va=r^{-1}w(r, \a)$ satisfies
\be\label{vivp} v'' + 2r^{-1}v' - v + v^3 = 0\ee
and
\be\label{vic} \lim_{r\to0} v(r, \a) = \a,\quad \lim_{r\to 0}v'(r, \a)=0.\ee

We define the function
$$\Phi(r) = (v')^2 + \half v^4 - v^2\quad\left(v=v(r, a)\right)$$
and differentiate with respect to $r$
$$\Phi'(r) = 2v'v'' + 2v^3v' - 2vv'.$$

By rewriting \eqref{vivp} as %, $\Phi(r)$ is a
$$v'' - v + v^3 = -2r^{-1}v',$$
we calculate $\Phi'(r)$ as 
$$\Phi'(r) = 2v'\left(v'' - v + v^3\right)=-4r^{-1}(v')^2.$$

Hence, $\Phi(r)$ is a strictly decreasing function of $r$. 
%for $\a>0$
%We calculate $\Phi'(r)=\frac{\diff\Phi(r)}{\diff r}$ as
%
%We rewrite \eqref{vivp} as
%
%to conclude that 

%is strictly decreasing on $(0, \za)$.
\begin{itemize}
    \item Since $-v^2<\Phi(r)$, it follows from the monotone character of $\Phi$
        that if $\Phi(r_0)\leq 0$ for some $r_0$ in $[0,\infty)$, then $v$ does
        not vanish in $(r_0,\infty)$ and $\liminf_{r\to\infty} v^2(r)>0$.
    \item If $0<\a\leq\sqrt{2}$, then from \eqref{vic} it follows that
        $\Phi(0)\leq 0$ and that $w$ neither vanishes on $(0,\infty)$ nor
        satisfies \eqref{wlim}, since \eqref{wlim} clearly implies
        $\lim_{r\to\infty}v(r)=0$.
    \item Suppose now that $\wa$ were to satisfy the hypothesis of \cref{vlem}
        but that for $r_0\in(0,\za)$, $w(r_0)=r_0$ while $w'(r_0)\geq0$.
    \item Since $w$ is convex as long as $0<w<r$, this assumption would imply
        the existence of an $r_1$ with $r_0<r_1<\za$ such that $w(r_1)=r_1$, so
        that the assertion concerning the sign of the slope where $w$ crosses
        the $45^\circ$ line reduces to the assertion that there is  single such
        crossing in $(0,\za)$.
    \item Suppose that there were two such crossings, $r_0,r_1$. We would then
        have $v(r_0)=v(r_1)=1$, and we can assume that $0<v<1$ on $(r_0,r_1)$.
    \item There is then an $r_3\in(r_0,r_1)$ with $v'(r_3)=0$, but then
        $\Phi(r_3)<0$.
    \item This implies, as indicated above, that $w$ cannot vanish on
        $(r_3,\infty)$ nor satisfy \eqref{wlim}.
    \item Thus the assumption that $w(r)-r$ can vanish twice in $(0,\za)$ has
        led to a contradiction and the proof of the lemma is complete.
\end{itemize}
\end{proof}

For $w=\wa$ as in \Cref{vlem}, it follows from that result that there will exist
positive numbers $a, b, c$ which are, respectively, the least positive values of
$r$ for which $$w'(r)=1,\quad w'(r)=0,\quad w(r)=r.$$

Moreover, by \Cref{vlem}, $0<a<b<c<\za$, $r<w(r)$ on $(0,c)$ and $0<w(r)<r)$ on
$(c,\za)$. Finally, $w$ is concave on $(0,c)$ and convex on $(c, \za)$.

We shall require the following identities which are valid for $w=\wa$,
$\was=\wa$:
\be\label{ids1}
(w'\was-\was'w)'=2r^{-2}w^3\was,
\ee
\be\label{ids2}
(w'\was'-w''\was)'=-2r^{-3}w^3\was,
\ee
\be\label{ids3}
\rd{typo}\left(r(w'\was'-w''\was)'\right.=-2w\was,
\ee
\be\label{ids4}
\left((w'-1)\was'-w''\was\right)'=-r^{-3}\was(w-r)^2(2w+r),
\ee
\be\label{ids5}
\rd{typo}\left[r\left((w'-1)\was'-w''\was)-(w'-1)\was\right)\right]'=r^{-1}\was(w-r)(3w+r).
\ee

Let $y_1$ denote the least positive zero of $\was=\wa$.
\begin{lemma}
    $\alpha < y_1 < \beta.$
\end{lemma}
\begin{proof}
\begin{itemize}
    \item Suppose first that $$y_1\leq\alpha$$ and integrate (4.20) between 0 and
        $y_1$. The expression $q(w-r)(3w+3)$ is positive on $(0,y_1)$, which
        implies $$y_1(w'(y_1)-1)q'(y_1)>0.$$
    \item Because of the definition of $\alpha$ the assumption (4.21) implies
        that $w'(y_1)\geq 1$, while clearly $q'(y_1)<0$, so that the assumption
        (4.21) has led to a contradiction.
    \item Suppose next that $$y_1\geq\beta$$ and integrate (4.18) between 0 and
        $\beta$. This give $$-w''(\beta)q(\beta)<0;$$ but (4.22) implies
        $q(\beta)\geq0$ and clearly $w''(\beta)<0$, so (4.22) has also led to a
        contradiction and the lemma is proved.
\end{itemize}
\end{proof}

\begin{itemize}
    \item Since $w$ is concave on $(0,\gamma)$, we have $w'(r)<1$ on
        $(\alpha,\gamma)$. In particular by \Cref{qlem} $$w'(y_1)<1.$$
\end{itemize}
\begin{proof}[Remainder of proof of \Cref{qneglem}]
\begin{itemize}
    \item Suppose that $q$ has a zero, say $y_2$, in $(y_1,z_1]$ and integrate
        (4.19) between $y_1$ and $y_2$. This gives \be\label{y2wqint}\ldots\diff
        r.\ee
    \item We assume (as we obviously can) that $q<0$ on $(y_1,y_2)$. Then
        $q'(y_1)<0$ and by (4.23), $(w'(y_1)-1)<0$, so that the right side of
        (4.24) is positive. Since $q'(y_2)>0$, (4.24) implies that $w'(y_2)>1$.
        This is clearly a contradiction since $w'<1$ on $(\alpha,\gamma)$ and,
        since $w$ is convex on $(\gamma,z_1]$, $w'<0$ on that interval. Thus
        $q<0$ on $(y_1,z_2]$ and i) of \Cref{qneglem} is proved.

\end{itemize}
\end{proof}

If $w$ satisfies (4.5) and (4.6), then $w'(r)<0$ on $(\beta,\infty)$ and $q<0$
on $(y_1,\infty)$. Integration of (4.19) from $y_1$ to $\gamma$ gives
$q'(\gamma)<0$, and integration of (4.17) from $\gamma$ then gives
$$\lim_{r\to\infty}(w'(r)q'(r)-w''(r)q(r))>0,$$ and this implies that $-q$ grows
exponentially. This comples the proof of \Cref{qneglem}.

%The proof of Theorem 4.1 seems reasonable.
% The proofs of Lemma 4.2, 4.3 and 4.4 seem intricated/confusing.

\subsection{Derivation of equation for radially symmetric solutions}\label{weqv}
Consider radially symmetric solutions to \eqref{upde}. Then $u(x) = u(|x|) =
u(r)$ . This transforms \eqref{upde} to the ODE \eqref{ivp}, restated here for
$N=3$
\be \label{uode} u'' + \frac{2}{r} u' - u + u^3 = 0 \ee 

Furthermore, substituting $u(r) = r^{-1}w(r)$, we calculate the derivatives of
$u$ as
\begin{enumerate}
    \item $u'(r) = -r^{-2}w(r) + r^{-1}w'(r)$
    \item $u''(r) = 2r^{-3}w(r) - 2r^{-2}w'(r) + r^{-1}w''(r)$.
\end{enumerate}
% Then, since $N=3$, \eqref{uode} reads
We substitute in \eqref{uode} to obtain
\begin{multline} 
u''(r) + \frac{2}{r} - u(r) + u^3(r) \\
= 2r^{-3}w(r) - 2r^{-2}w'(r) + r^{-1}w''(r)
+ \frac{2}{r}\left(-r^{-2}w(r) + r^{-1}w'(r)\right) \\
- r^{-1}w(r) + r^{-3}w^3(r) = 0,
\end{multline}
which is simplified to
$$ r^{-1}\left(w'' - w + r^{-2}w^3 \right) = 0. $$
In conclusion, since $r\neq 0$, we obtain
\be \label{wode} w'' - w + r^{-2}w^3 = 0. \ee

% So indeed, positive radially symmetric solutions to \eqref{upde} are of the form 
% $$ u(x) = |x|^{-1} w(|x|) $$
% where $w(|x|)$ satisfies the differential equation \eqref{wode}.
% 
% Therefore, uniqueness of $w(r)$ to
% \be \begin{dcases} \label{wbvp}
%     w'' - w + r^{-2}w^3 = 0,\\
%     0 < \lim_{r\to 0}r^{-1}w(r) < \infty,\\
%     \lim_{r\to\infty}w(r) = 0
% \end{dcases} \ee
% 
% is sufficient for uniqueness of $\phi_1$ to \eqref{upde}.
% 
% We can interpret \eqref{wbvp} as an initial value problem
% \be \begin{dcases} \label{wivp}
%     w'' - w + r^{-2}w^3 = 0,\\
%     \lim_{r\to 0}r^{-1}w(r) = a.
% \end{dcases} \ee


% \section{Introduction}

%In the reference paper \cite{nehari}
