%\begin{abstract} \comm{We consider a  nonlinear Schr\"odinger equation (NLS) posed on a graph or network composed of a generic compact part to which a finite number of half-lines are attached. We call this structure a starlike graph. At the vertices of the graph %structure
%interactions of $\delta$-type can be present and an overall external potential is admitted. Under general assumptions on the potential, we prove that the NLS  is globally well-posed in the energy domain.
%
%We are interested in minimizing the energy of the system on the manifold of constant mass ($L^2$-norm). When existing, the minimizer
%%minimum
%is called ground state and it is the profile of an orbitally stable standing wave for the NLS evolution. We prove that a ground state exists for sufficiently small masses whenever the  quadratic part of the energy admits a simple isolated eigenvalue at the bottom of the spectrum (the linear ground state). This is a wide generalization of a result previously obtained for a star graph with a single vertex.
%The main part of the proof is devoted to prove the concentration compactness principle for starlike structures; this is non trivial
%%obvious 
%due to the lack of translation invariance of the domain. Then we show that a minimizing bounded $H^1$ sequence for the constrained NLS energy with external linear potentials is in fact convergent if its mass is small enough. Examples are provided with discussion of hypotheses on the linear part.}
%\end{abstract}

\begin{abstract}
  In this review, existence and uniqueness of ground state solutions to an NLS-like equation is studied. The nonlinear Schr\"odinger equation (NLS) is the paraxial approximation of the nonlinear Helmholtz equation (NLH). 
\end{abstract}

\maketitle
%\comm{
%\begin{footnotesize}
% \emph{Keywords:} Quantum graphs; non-linear Schr\"odinger equation; concentration-compactness techniques. 
% 
% \emph{MSC 2010:}  35Q55, 81Q35, 35R02.  
% %35Q55  	NLS-like equations (nonlinear Schrödinger) 
% %81Q35  	Quantum mechanics on special spaces: manifolds, fractals, graphs, etc.
%% 35R02  	Partial differential equations on graphs and networks (ramified or polygonal spaces)
% %37K50  	Bifurcation problems
% \end{footnotesize}}
