{\color{gray} Intermezzo: some definitions. Let $\theta(r)\coloneqq-ru'(r)/u(r)$ for $r\in[0,z(\alpha)$ and $\rho\coloneqq\theta^{-1}$. Then PPProperties. Many auxiliary functions will be introduced now. These functions construct the information needed to prove $w$ has a unique zero. Bear with me as the following functions and variables are introduced: $\theta(r),\beta,\rho(\beta),\phi_{\beta}(r),\nu_{\beta}(r)$. In the lemma that follows, even more functions and variables are introduced: $\bar\beta,\sigma(\beta),\xi(r), \Xi(r)\coloneqq\sigma(\beta)^{-1},\beta_0$.}

{\color{teal}Define $\theta(r)\coloneqq-ru'(r)/u(r)$ on $r\in[0,z(\alpha))$. Note $\theta(r)$ has the following properties: \begin{enumerate}[(i)]\item $\theta(0)=0$, \item $\theta'(r)>0$ for all $r\in(0,z(\alpha))$, and \item $\underset{r\to z(\alpha)}{\lim}\theta(r)=\infty$. \end{enumerate} SHOW THETA(R) HAS THESE PROPERTIES.

Define $\rho\coloneqq\theta^{-1}$. Since $\theta(r)$ is continuous and increasing SHOW.., there exists a unique $r=\rho(\beta)>0$ such that $\theta(r)=\beta$. SHOW. Then $\rho(r)$ is continuous and increasing, with $\rho(0)=0$ and $\underset{\beta\to\infty}{\lim}\rho(\beta)=z(\alpha).$ SHOW.

Define $\nu_{\beta}(r)\coloneqq ru'(r)+\beta u(r)=-u(r){\theta(r)-\beta}.$ Let $\beta>0,$ then $\nu_{\beta}(r)>0$ if $r<\rho(\beta)$ and $\nu_{\beta}(r)<0$ if $r>\rho(\beta)$. SHOW.

Define $\phi_{\beta}(r)\coloneqq\left[\beta(p-1)-2\right]V(r)u(r)^p-rV'(r)u(r)^p+2\lambda u(r).$ Now observe $\nu_{\beta}(r)$ satisfies the differential equation $$\nu_{\beta}''(r)+\frac{1}{r}\nu_{\beta}(r)'-\lambda\nu_{\beta}+pV(r)u(r)^{p-1}\nu_{\beta}=\phi_{\beta}(r)$$} SHOW.