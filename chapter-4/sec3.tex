\section{Minimisation of $J$}
This section first states that a solution $u\in L^4$ must belong to $H^1$.
For $u\in L^4$, $u\neq 0$, we define $\sigma(u)$ by 
\be \label{sigmadef} \left(\sigma(u)\right)(x) = c \int g(x-t)u^3(t)\diff t, \ee
where $c>0$ is chosen to normalise $v=\sigma(u)$. That is,
\be\label{vnorm} \int v^4\diff x = 
\int \left(\sigma^4(u)\right)(x) \diff x =
c^4 \int \left(\int g(x-t)u^3(t)\diff t\right)^4 \diff x = 1. \ee

{\red Further details:}
\begin{itemize}
    \item $u\in L^4\implies u^3\in L^{4/3}$
    \item $g\ast (u^3)\in L^4$ and $g\ast (u^3)\neq 0$
    \item Fixed points of $\sigma$ are non-trivial solutions of \eqref{uint}
    \item $\sigma: L^4\to H^1$
    \item $\sigma$ is an operator in $H^1\setminus\{0\}$
    \item $L^4$ solution of \eqref{uint} must belong to $H^1$
\end{itemize}

%($u\in H^1$ and $u\neq 0$)
%Lemma 3.1
\begin{lemma} 
Let $u$ be an admissible solution with
\be\label{uass} \int u^4 \diff x = 1.\ee

Then $\sigma(u)$ is admissible and
\be\label{jsigma}
J(\sigma(u))\leq J(u)
\ee

with equality only if $\sigma(u) = u$. Moreover, $\sigma(u)\in L^\infty$ 
%($\sigma(u)$ bounded) 
and $v=\sigma^2(u)$ has bounded continuous derivatives and satisfies
\be\label{vlim}
\underset{|x|\to\infty}{\lim} v(x)=0;
\ee
finally $\sigma^3(u)\in C^2$.
\end{lemma}
\begin{proof}
{\red Proof steps}
\begin{itemize}
    \item $\sigma(u)$ admissible follows from e) of \Cref{sec2}
    \item By e), \eqref{sigmadef} and \eqref{uass}, $w=\sigma(u)$ satisfies
        $$c=c\int u^4\diff x=\int\left(\nabla w\cdot \nabla u + wu\right)\diff
        x\leq |u|_{1, 2}|w|_{1, 2}.$$
    \item By e), \eqref{vnorm} and \eqref{uass}
        \be\label{wucineq} 
            \int\left(|\nabla w|^2+w^2\right) \diff x = c\int wu^3\diff x\leq
            c|w|_{0, 4}|u|_{0, 4}^3=c.
        \ee
    \item Combining the inequalities yields
        $$ |w|_{1, 2}\leq |u|_{1, 2},$$
    \item which imply \eqref{jsigma} in view of \eqref{sigmadef} and
        \eqref{uass}.
    \item Equality can hold only if $w$ and $u$ are proportional, from the
        application the Schwarz inequality.
    \item The constant of proportionality must be 1, by \eqref{vnorm},
        \eqref{uass} and \eqref{wucineq}.
    \item The boundedness of $\sigma(u)$ follows from applying the Schwarz
        inequality to \eqref{sigmadef} and then using the inequality
        $$\int u^6\diff x\leq 48\left(\int|\nabla u|^2\diff x\right)^3;$$
        from {\red 14, p.12} or {\red 11, Theorem 9.3, p.24}.
    \item The remaining assertions follow from f) and g).
\end{itemize}
\end{proof}

The following two lemmata are corollaries of Lemma 3.1.
\begin{lemma} 
    If $v\in L^4$ is a solution of \eqref{uint} then $v\in C^2$, $v$ has
    bounded first derivatives, and $v$ satisfies \eqref{vlim}.
\end{lemma}
\begin{lemma}\label{jineqlem}
    If $u$ is any (radially symmetric) admissible function, then there is a
    (radially symmetric) admissible function $v\in C^2$ which is positive, has
    bounded first derivatives and satisfies \eqref{vlim} and 
\be\label{jvu}
J(v)\leq J(u).
\ee
    Moreover, unless $u$ itself has the same properties and is a solution of 
    \eqref{uint} (to within a positive factor), then $v$ can be chosen so that
    inequality \eqref{jvu} is strict.
\end{lemma}
\begin{proof}
{\red Insert skeleton.}
\end{proof}

% The implications are that solutions of the integral equation (2.2) are in $C^2$, have bounded first derivatives and vanish at the boundary.
% 
% Furthermore, existence of a (radially symmetric) admissible function implies existence of a (radially symmetric) function in $C^2$ which is positive, has bounded first derivatives, vanishes at the boundary and satisfies (3.5) and satisfies
% 
% Moreover, unless $u$ has the same properties and solves (2.2), the $v$ can be chosen so that (3.7) is a strict inequality.

\begin{theorem}% 3.1 which states that for the ground state
Let
    $$\lambda_1 = \inf\left\{J(u): u~\text{admissible}\right\}.$$
    There exists a $\phi_1\in V$ %(hence radially symmetric)
    with 
    $$J(\phi_1)=\lambda_1.$$ 
    For $u\in H^1$, $J(u)>\lambda_1$ unless $u$ is of the form \eqref{uts}.
\end{theorem}
\begin{proof}
{\red Proof steps}
\begin{itemize}
    \item That $J$ attains an infimum in the class of radially symmetric
        admissible functions was shown in \cite{nehari}. This also follows from
        assertion d) of \Cref{sec2}.
    \item Suppose that $u\in H^1$ but no translate of $u$ is essentially
        radially symmetric. 
    \item For the purpose of showing that $J(u) > \lambda$, we can suppose by
        \Cref{jineqlem} that $u$ is positive and of class $C^2$, and that
        $u(x)\to 0$ as $|x|\to \infty$.
    \item By the Schwarz symmetrization procedure (see {\red 19} or {\red 16,
        Section 8}) we can produce a radially symmetric admissible function $v$
        with $J(v)<J(u)$.
    \item This shows that the infimum over all admissible functions is the same
        as its infimum over the radially symmetric functions and therefore that
        this infimum is attained. 
    \item The final assertion of the theorem follows
        from \Cref{uniqthm} below.
\end{itemize}
\end{proof}

% I have lent the book Polya[19] from the library (mentioned in the proof of 3.1).
% I have downloaded the references Mostow[16] and Moser[15].
The proof of Theorem 3.1 shows the desired results of the paper except for the
last statement. This requires \ref{uniqthm} of the next section.
% The basic idea of Rayleigh quotient minimization is to find the ground state eigenfunction. Another interpretation is that we find the best constant for the imbedding $H^1 \to L^4$.
