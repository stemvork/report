\section{Minimisation of $J$}
This section first states that a solution $u\in L^4$ must belong to $H^1$.
For $u\in L^4$, $u\neq 0$, we define $\sigma(u)$ by 
\be \label{sigmadef} \left(\sigma(u)\right)(x) = c \int g(x-t)u^3(t)~dt \ee

$[\ldots]$

%($u\in H^1$ and $u\neq 0$)
%Lemma 3.1
\begin{lemma} 
If $u$ is an admissible solution, then $\sigma(u)$ is admissible and
\be\label{jsigma}
J(\sigma(u))\leq J(u)
\ee

with equality only if $\sigma(u) = u$. Moreover, $\sigma(u)\in L^\infty$ 
%($\sigma(u)$ bounded) 
and $v=\sigma^2(u)$ has bounded continuous derivatives and satisfies
\be\label{vlim}
\underset{|x|\to\infty}{\lim} v(x)=0;
\ee
finally $\sigma^3(u)\in C^2$.
\end{lemma}
\begin{proof}
\end{proof}

The following two lemmata are corollaries of Lemma 3.1.
\begin{lemma} 
    If $v\in L^4$ is a solution of \eqref{uint} then $v\in C^2$, $v$ has
    bounded first derivatives, and $v$ satisfies \eqref{vlim}.
\end{lemma}
\begin{lemma} 
    If $u$ is any (radially symmetric) admissible function, then there is a
    (raaially symmetric) admissible function $v\in C^2$ which is positive, has
    bounded first derivatives and satisfies \eqref{vlim} and 
\be\label{jvu}
J(v)\leq J(u).
\ee
    Moreover, unless $u$ itself has the same properties and is a solution of 
    \eqref{uint} (to within a positive factor), then $v$ can be chosen so that
    inequality \eqref{jvu} is strict.
\end{lemma}
\begin{proof}
\end{proof}

% The implications are that solutions of the integral equation (2.2) are in $C^2$, have bounded first derivatives and vanish at the boundary.
% 
% Furthermore, existence of a (radially symmetric) admissible function implies existence of a (radially symmetric) function in $C^2$ which is positive, has bounded first derivatives, vanishes at the boundary and satisfies (3.5) and satisfies
% 
% Moreover, unless $u$ has the same properties and solves (2.2), the $v$ can be chosen so that (3.7) is a strict inequality.

\begin{theorem}% 3.1 which states that for the ground state
Let
    $$\lambda_1 = \inf\left\{J(u): u~\text{admissible}\right\}.$$
    There exists a $\phi_1\in V$ %(hence radially symmetric)
    with 
    $$J(\phi_1)=\lambda_1.$$ 
    For $u\in H^1$, $J(u)>\lambda_1$ unless $u$ is of the form \eqref{uts}.
\end{theorem}

% I have lent the book Polya[19] from the library (mentioned in the proof of 3.1).
% I have downloaded the references Mostow[16] and Moser[15].
The proof of Theorem 3.1 shows the desired results of the paper except for the
last statement. This requires \ref{uniqthm} of the next section.
% The basic idea of Rayleigh quotient minimization is to find the ground state eigenfunction. Another interpretation is that we find the best constant for the imbedding $H^1 \to L^4$.
