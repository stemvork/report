\section{Introduction}
% In this chapter, we study the paper "Uniqueness of the Ground State Solution for
% $\Delta u - u + u^3 = 0$ and a Variational Characterization of Other Solutions"
% written in 1972 by Charles V. Coffman. 

In this chapter, we study a paper from 1972 by Charles V. Coffman \cite{coffm}. 
The paper proves uniqueness of the positive radially symmetric (ground
state) solution $u=\phi_1\in C^2\cap L^4$ for the equation
\be\label{upde}
\Delta u - u + u^3 = 0\quad\text{in }\R^3.
\ee

Note that all function spaces consist of real valued functions on $\R^3$.
Furthermore, radial symmetry is with respect to the origin only. 

The existence of such a function $\phi_1$ was shown in \cite{nehari},
where $\phi_1=v_1(|x|)$ solves \eqref{upde}. In fact, there exist functions
$v_n(|x|)\in C^2([0, \infty))$, $n=1, 2, \ldots$, such that for each $n$, $v_n$
has exacly $n-1$ isolated zeroes in $[0, \infty)$, decays exponentially as
$r\to\infty$. This was shown in \cite{ryder, berger}

Moreover, Theorem 3.1 of \cite{coffm} improves the result of \cite{robinson}, 
which also studied \eqref{upde} (in the context of variational calculus). 
In \cite{robinson}, they show that the Lagrangian associated with \eqref{upde}
is zero in its first variation and the second variation is positive if
$\lambda_1 > 1$. The latter is shown only through approximations. Theorem 3.1 of
\cite{coffm} shows that the Rayleigh quotient $J$ associated with \eqref{upde}
\be\label{rayleigh} 
J(u) = \dfrac{\left(\int\left|\nabla u\right|^2+u^2~dx\right)^2}{\int u^4~dx}
\ee

is indeed minimal for $u=\phi_1$ and for 
\be\label{uts}
u(x)=k\phi_1(x+x_0)
\ee
for any $k\neq 0$ and
$x_0\in \R^3$.
