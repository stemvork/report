\newcommand{\intrrr}{\int_{\R^3}}
\chapter{Uniqueness of ground state}

%sectionm
\section{Introduction}
% In this chapter, we study the paper "Uniqueness of the Ground State Solution for
% $\Delta u - u + u^3 = 0$ and a Variational Characterization of Other Solutions"
% written in 1972 by Charles V. Coffman. 

% \todo{Make connection to chapter 2 and discuss similarities and differences.}
In \Cref{existence} we studied the existence result \cite{ber81} for
\be\label{updeber}
\Delta u - u = f(u)\quad\text{in }\R^3,
\ee
where $f(u)$ satisfies certain general conditions. Motivated by \Cref{physics},
we are interested in the specific case $f(u) = u - u^3$.  In the present
chapter, we study the uniqueness result \cite{coffm} for the problem 
\be\label{upde}
\Delta u - u + u^3 = 0\quad\text{in }\R^3.
\ee

In particular, \cite{coffm}  proves uniqueness of the positive radially
symmetric (ground state) solution $u=\phi_1\in C^2\cap L^4$ for \eqref{upde}.
All function spaces considered in this chapter consist of real valued functions
on $\R^3$. Furthermore, radial symmetry is with respect to the origin only. 

The existence of such a function $\phi_1$ was shown in \cite{nehari},
where $\phi_1=v_1(|x|)$ solves \eqref{upde} as a specific case of $f(u) = u -
u^k$ with $1<k\leq 4$. 

In addition, \cite{coffm} refers to the existence of functions $v_n(|x|)\in
C^2([0, \infty))$, $n=1, 2, \ldots$, such that for each $n$, $v_n$ has exacly
$n-1$ isolated zeroes in $[0, \infty)$ and decays exponentially as $r\to\infty$.
This was shown in \cite{ryder}, which considered $f(u) = u - ug(u^2)$,
as well as \cite{berger}, which considered
$$i\frac{\partial u}{\partial t} = \Delta u + f(|x|, |u|^2)u$$ 
under the assumption that $f(|x|, |u|^2)=k(|x|)|u|^\sigma$ for $0<\sigma<4$ and
$k(|x|)$ a Lipschitz continuous positive bounded function.

Furthermore, Theorem 3.1 of \cite{coffm} improves the result of \cite{robinson},
which also studied \eqref{upde} (in the context of variational calculus).  In
\cite{robinson}, they show that the Lagrangian associated with \eqref{upde} is
zero in its first variation and the second variation is positive if $\lambda_1 >
1$. The latter is shown only through approximations. Theorem 3.1 of \cite{coffm}
shows that the Rayleigh quotient $J$ associated with \eqref{upde}
\be\label{rayleigh} 
J(u) = \dfrac{\left(\int\left|\nabla u\right|^2+u^2~dx\right)^2}{\int u^4~dx}
\ee

is indeed minimal for %$u=\phi_1$ and for 
\be\label{uts}
u(x)=k\phi_1(x+x_0),\quad\text{where}~k\neq0~\text{and}~x_0\in\R^3
\ee
% for any $k\neq 0$ and $x_0\in \R^3$. 
The right hand side of \eqref{rayleigh} is
meaningful for admissible functions.

\begin{definition} 
    A function $u$ is \emph{admissible} if $u\in H^1$ and $u\neq 0$.
\end{definition}

\section{Preliminary results for the integral equation}
The problem \eqref{upde} subject to $u\in L^4$ is equivalent to the integral
equation in $L^4$
\be \begin{cases}\label{uint}
    u(x) = \int g(x-y) u^3(y)\diff y,\quad\text{where}\\
    g(x) = (4\pi)^{-1} |x|^{-1} e^{-|x|}. %\# research
\end{cases} \ee 

Here $g(x)$ is the Yukawa (screened Coulomb) potential. This potential 
is associated with the equation
\be\label{yukawa}\Delta u - u = 0.\ee 
We consider radially symmetric solutions to \eqref{yukawa}, $r=|x|$, which solve
\be\label{yukrs}\frac{\diff}{\diff r^2}\left(ru\right)=ru.\ee 
Hence, the Yukawa potential $u(r)$ is of the form $ru = e^{-r}
\iff u(r) = r^{-1}e^{-r}$.

% This is the Yucawa potential. I have downloaded the reference
% Ladyzhenskaya. I have not been able to download the references Nirenberg,
% Friedman, Coffman.

The following two subsections discuss (mostly) standard results regarding the
Sobolev space $H^1$ and the convolution operator $\tau: u \to g\ast u$. 

% Further details can obtained from \#3 and \#14.
% This section summarises results concerning the space $H^1$ and the convolution
% operator $\tau: u\to g\ast u$.

\subsection{Some results regarding $H^1$} 
First, concerning the space $H^1$, we have the following results: 
\begin{enumerate}[a)] 
    \item $C^\infty_0$ is dense in $H^1$.  
    \item If $u\in H^1$, then $v=|u|\in H^1$ and $$|u|_{1,2}=|v|_{1,2}.$$ 
    \item If $u\in H^1$, then $u\in L^4$ and \be|u|_{0,4} \leq 2^{-1/4}|u|_{1,2}.\ee 
    \item Let $V$ denote the subspace of $H^1$ consisting of radially symmetric functions. The embedding $V \to L^4$ is compact.  
\end{enumerate}

\subsection{Some results regarding the convolution operator} 
\begin{enumerate}[a)] \setcounter{enumi}{4} 
    \item If $u\in L^{4/3}$, then $v = g\ast u \in H^1\subseteq L^4$, $\int u~v~\mathrm{d} x>0$ unless $u=0$, and $v$ is a weak solution of \be \label{vpde} -\Delta v + v = u. \ee 
    \item If $u \in L^1\cap L^\infty$, then $v = g\ast u$ has bounded continuous
        first derivatives and $$\underset{|x|\to\infty}{\lim} v(x) = 0$$.  
    \item If $u\in L^1\cap L^\infty\cap C^1$, then $v = g\ast u\in
        C^2$ and $v$ satisfies \eqref{vpde}.  
    \item Let $X$ and $Y$ denote the subspaces of $L^{4/3}$ and $L^4$ respectively, consisting of radially symmetric
functions. Then $Y = X^\ast$ and $\tau : X \to Y$ is compact.  
\end{enumerate}


%sectionm
\section{Minimisation of $J$}
This section first states that a solution $u\in L^4$ must belong to $H^1$.
For $u\in L^4$, $u\neq 0$, we define $\sigma(u)$ by 
\be \label{sigmadef} \left(\sigma(u)\right)(x) = c \int g(x-t)u^3(t)~dt \ee

$[\ldots]$

%($u\in H^1$ and $u\neq 0$)
%Lemma 3.1
\begin{lemma} 
If $u$ is an admissible solution, then $\sigma(u)$ is admissible and
\be\label{jsigma}
J(\sigma(u))\leq J(u)
\ee

with equality only if $\sigma(u) = u$. Moreover, $\sigma(u)\in L^\infty$ 
%($\sigma(u)$ bounded) 
and $v=\sigma^2(u)$ has bounded continuous derivatives and satisfies
\be\label{vlim}
\underset{|x|\to\infty}{\lim} v(x)=0;
\ee
finally $\sigma^3(u)\in C^2$.
\end{lemma}
\begin{proof}
\end{proof}

The following two lemmata are corollaries of Lemma 3.1.
\begin{lemma} 
    If $v\in L^4$ is a solution of \eqref{uint} then $v\in C^2$, $v$ has
    bounded first derivatives, and $v$ satisfies \eqref{vlim}.
\end{lemma}
\begin{lemma} 
    If $u$ is any (radially symmetric) admissible function, then there is a
    (raaially symmetric) admissible function $v\in C^2$ which is positive, has
    bounded first derivatives and satisfies \eqref{vlim} and 
\be\label{jvu}
J(v)\leq J(u).
\ee
    Moreover, unless $u$ itself has the same properties and is a solution of 
    \eqref{uint} (to within a positive factor), then $v$ can be chosen so that
    inequality \eqref{jvu} is strict.
\end{lemma}
\begin{proof}
\end{proof}

% The implications are that solutions of the integral equation (2.2) are in $C^2$, have bounded first derivatives and vanish at the boundary.
% 
% Furthermore, existence of a (radially symmetric) admissible function implies existence of a (radially symmetric) function in $C^2$ which is positive, has bounded first derivatives, vanishes at the boundary and satisfies (3.5) and satisfies
% 
% Moreover, unless $u$ has the same properties and solves (2.2), the $v$ can be chosen so that (3.7) is a strict inequality.

\begin{theorem}% 3.1 which states that for the ground state
Let
    $$\lambda_1 = \inf\left\{J(u): u~\text{admissible}\right\}.$$
    There exists a $\phi_1\in V$ %(hence radially symmetric)
    with 
    $$J(\phi_1)=\lambda_1.$$ 
    For $u\in H^1$, $J(u)>\lambda_1$ unless $u$ is of the form \eqref{uts}.
\end{theorem}

% I have lent the book Polya[19] from the library (mentioned in the proof of 3.1).
% I have downloaded the references Mostow[16] and Moser[15].
The proof of Theorem 3.1 shows the desired results of the paper except for the
last statement. This requires \ref{uniqthm} of the next section.
% The basic idea of Rayleigh quotient minimization is to find the ground state eigenfunction. Another interpretation is that we find the best constant for the imbedding $H^1 \to L^4$.

%sectionm
\section{Uniqueness theorem}
The radially symmetric solutions of \eqref{upde} are of the form
$$u(x)=|x|^{-1}w(|x|),$$ 
where $w(r)$ ($r=|x|$) solves 
\be\label{wvp}
w''-w+r^{-2}w^3=0.
\ee
We refer to \ref{weqv} for the details.

To prove the uniqueness of ground state solution $\phi_1$ for \eqref{upde}, it
suffices to prove that \eqref{wvp} has at most one positive solution satisfying
the following boundary conditions
\be\label{wbc}
0 < \underset{r\to0}{\lim}~r^{-1}w(r) < \infty,\quad
\underset{r\to\infty}{\lim} w(r)=0.
\ee

The problem \eqref{wvp} is transformed to an initial value problem where 
\be\label{wic}
\underset{r\to0}{\lim}~r^{-1}w(r)=a>0.
\ee

The basic facts regarding the problem \eqref{wvp}, \eqref{wic} are summarised in
Lemma 4.1. The proofs are omitted in \cite{coffm}. 
\begin{lemma}\label{dlem}
    For each $a>0$ the equation \eqref{wvp} has a unique solution $w=w(r,a)$ 
    which is of class $C^2$ on $\left(0, \infty\right)$ and satisfies
    \eqref{wic}.  The partial derivatives $\partial w(r,a)/\partial a$ and
    $\partial w'(r, a)/\partial a$ exist for all positive $r$ and $a$.
    Furthermore, $\partial w(r, a)/\partial a$ coincides on $(0, \infty)$ with
    the solution $\delta=\delta(r, a)$ of the regular initial value problem 
    \be\label{dvp} \begin{cases}
    \delta'' - \delta + 3r^{-2}w^2\delta=0,\\ 
    \delta(0)=0,\quad\delta'(0)=1,
    \end{cases}\ee
    with $w=w(r, a)$; $\partial w'(r, a)/\partial a=\delta'(r, a)$.  
\end{lemma}

It is clear that a solution of \eqref{wvp} which satisfies
\eqref{wbc} belongs to the one-parameter family $w=w(r, a)$, $a>0$; we
therefore formulate our uniqueness result as follows.

\begin{theorem}\label{uniqthm}
    There is at most one positive value of $a$ for which
    \be\label{wpos} w(r,a)>0,\quad 0<r<\infty\ee
    and
    \be\label{wlim}\underset{r\to\infty}{\lim} w(r, a)=0.\ee
\end{theorem}

\Cref{uniqthm} is implied by the following lemma.

\begin{lemma}
    \begin{enumerate}[(i)]
        \item If $a>0$ and $w(r, a)>0$ on $(0, z_1)$ with $w(z_1, a)=0$, then
            $\delta(z_1, a)<0$.
        \item If $a>0$ and $w(r, a)$ satisfies (4.5) and (4.6) then
            \be\label{dlim}
                \underset{r\to\infty}{\lim} e^{-r}\delta(r, a)<0.
            \ee
    \end{enumerate}
\end{lemma}

\begin{proof}[Proof of Theorem \ref{uniqthm}]
    Let
    \be\label{adef}
    A = \left\{a>0 : \exists r_0>0 : w(r_0,a)=0\right\}.
    \ee
    
    If $a\in A$ and $z_1=z_1(a)$ is the least positive zero of $w(r,a)$, then,
    by \cref{dlem} and the implicit function theorem, $z_1$ is a differentiable
    function of $a$ on $A$ and

    $$w'(z_1, a)\frac{\diff z_1}{\diff a} + \delta(z_1, a) = 0.$$

    Since $w'(z_1, a)<0$ it follows from i) of \cref{dlem} that $\frac{\diff
    z_1}{\diff a}<0$ on $A$; therefore $z_1$ moves monotonically to the left as
    $a$ increases. Thus if $A$ is non-empty it is a semi-infinite interval. We
    next show that if \eqref{wpos} and \eqref{wlim} hold for $a=a_1$, then $A$
    is non-empty and $a_1$ is the left endpoint of $A$. This will clearly imply
    \Cref{uniqthm}. Let $a_1$ be as above and let $a_2>a_1$; we shall show that
    if $a_2-a_1$ is sufficiently small, then the assumption
    \be\label{wass} w(r, a_2) > 0\quad\text{on}~(0, \infty)\ee
    leads to a contradiction. We put $w_i=w_i(r, a)$ for $i=1, 2$. By our
    assumption on $a_1$ and ii) of \cref{dlem}, we can choose $r_0$

\end{proof}

% Also, another (now regular) initial value problem in $\delta(r,a)$ is introduced. 
% 
% **Question:** Regularity in which sense? Usually smoothness, here 
% Theorem 4.1 claims uniqueness of the parameter $a>0$ for which $w(r, a)$ is positive on $(0, \infty)$ and vanishes at infinity. This is implied by Lemma 4.2.
% 
By studying the zeroes of $w(r,a)$, we can show that $A$ (the set of $a>0$
such that $w(r, a)$ has at least one zero in $(0, \infty)$) has a left
endpoint.

\begin{lemma}
Let $a>0$ and let $w=w(r,a)$ either vanish at least once in $(0, \infty)$
    or satisfy \eqref{wlim}; then $a>\sqrt{2}$, $w(r, a)=r$ for precisely one value
$r=r_0$ in $(0, z_1)$ and $w'(r_0, a)<0$.
\end{lemma}
\begin{proof}
\end{proof}

\begin{lemma}
    $\alpha < y_1 < \beta.$
\end{lemma}
\begin{proof}
\end{proof}

%The proof of Theorem 4.1 seems reasonable.
% The proofs of Lemma 4.2, 4.3 and 4.4 seem intricated/confusing.

\subsection{Derivation of equation for radially symmetric solutions}\label{weqv}
Consider radially symmetric solutions to \eqref{upde}. Then $u(x) = u(|x|) =
u(r)$ . This transforms \eqref{upde} to the ODE \eqref{ivp}, restated here for
$N=3$
\be \label{uode} u'' + \frac{2}{r} u' - u + u^3 = 0 \ee 

Furthermore, substituting $u(r) = r^{-1}w(r)$, we calculate the derivatives of
$u$ as
\begin{enumerate}
    \item $u'(r) = -r^{-2}w(r) + r^{-1}w'(r)$
    \item $u''(r) = 2r^{-3}w(r) - 2r^{-2}w'(r) + r^{-1}w''(r)$.
\end{enumerate}
% Then, since $N=3$, \eqref{uode} reads
We substitute in \eqref{uode} to obtain
\begin{multline} 
u''(r) + \frac{2}{r} - u(r) + u^3(r) \\
= 2r^{-3}w(r) - 2r^{-2}w'(r) + r^{-1}w''(r)
+ \frac{2}{r}\left(-r^{-2}w(r) + r^{-1}w'(r)\right) \\
- r^{-1}w(r) + r^{-3}w^3(r) = 0,
\end{multline}
which is simplified to
$$ r^{-1}\left(w'' - w + r^{-2}w^3 \right) = 0. $$
In conclusion, since $r\neq 0$, we obtain
\be \label{wode} w'' - w + r^{-2}w^3 = 0. \ee

% So indeed, positive radially symmetric solutions to \eqref{upde} are of the form 
% $$ u(x) = |x|^{-1} w(|x|) $$
% where $w(|x|)$ satisfies the differential equation \eqref{wode}.
% 
% Therefore, uniqueness of $w(r)$ to
% \be \begin{dcases} \label{wbvp}
%     w'' - w + r^{-2}w^3 = 0,\\
%     0 < \lim_{r\to 0}r^{-1}w(r) < \infty,\\
%     \lim_{r\to\infty}w(r) = 0
% \end{dcases} \ee
% 
% is sufficient for uniqueness of $\phi_1$ to \eqref{upde}.
% 
% We can interpret \eqref{wbvp} as an initial value problem
% \be \begin{dcases} \label{wivp}
%     w'' - w + r^{-2}w^3 = 0,\\
%     \lim_{r\to 0}r^{-1}w(r) = a.
% \end{dcases} \ee


% \section{Introduction}

%In the reference paper \cite{nehari}

\references{dissertation}
