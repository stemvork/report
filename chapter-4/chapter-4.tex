\newcommand{\intrrr}{\int_{\R^3}}
\chapter{Uniqueness}
\section{Introduction}
% In this chapter, we study the paper "Uniqueness of the Ground State Solution for
% $\Delta u - u + u^3 = 0$ and a Variational Characterization of Other Solutions"
% written in 1972 by Charles V. Coffman. 

In chapter 2 \# reference.

In this chapter, we study a paper from 1972 by Charles V. Coffman \cite{coffm}. 
{\green This paper proves uniqueness of the positive radially symmetric ground
state solution in $\R^3$.} {\red That is, there is precisely one positive radially
symmetric solution that is twice differentiable and belongs to $L^4$.  Moreover,
the Rayleigh quotient $J(u)$ attains its infimum.}

{\red The Rayleigh quotient is generally obtained by multiplying with the solution and
integrating with respect to \# the argument. In this way, we obtain the
expression $$J(u) = \dfrac{\int\left|\nabla u\right|^2+u^2~dx}{\int u^4~dx}$$

which is not a good minimiser. The functional has to be invariant in the same
sense as the desired solution. Hence, square the numerator.}

% The introduction refers to \# Finklestein, Nehari, Ryder and Berger.

\hfill\hrule


From \cite{nehari}, we know that
\be\label{upde}
\Delta u - u + u^3 = 0
\ee

in $\R^3$ has a positive radially symmetric solution $u$. {\red in which space}
This positive radially symmetric solution is unique. 

Moreover, $u \in H^1$ and $u\neq 0$ implies $J(\phi_1) < J(u)$ unless
$u(x) = \lambda \phi_1 \left( x + x_0 \right)$ for some non-zero, real $\lambda$
and $x_0 \in \R^3$. Here $J$ is the Rayleigh quotient associated with
\eqref{upde},

\be\label{rayl}
% J(u) = \frac{\left( \intrrr |\mathrm{grad} u|^2 + u^2 dx
J(u) = \frac{\left( \intrrr |\nabla u|^2 + u^2 \diff x \right)^2}{\int_{R^3} u^4 \diff x}
\ee

The right hand side of expression \eqref{rayl} is meaningful for $u\in H^1$ and
$u\neq 0$. Such functions will be referred to as \emph{admissible} functions.

In \cite{finkelstein} the equation \eqref{upde} was considered.

\begin{lemma}\label{coff1}
There exist $v_n(r) \in C^2\left( \left[0, \infty \right) \right)$ where $n=1, 2,
\ldots$, such that for each $n$, $v_n$ has exactly $n-1$ isolated zeroes in
$\left[0, \infty \right)$, decays exponentially as $r\to\infty$ and $\phi_n(x) =
v_n(|x|)$ is a solution of \eqref{upde}.
\end{lemma}

The proofs were given in % TODO. 
\cite{nehari} and \cite{ryder}. The same
results were obtained in \cite{berger} using Lyusternik-Schnirelman theory.

This paper proves the main result %TODO
below to answer the questions raised in \cite{robinson}.

We seek solutions to \eqref{upde} subject to the boundary condition at infinity
given by
\be\label{boundcond} u \in L^4. \ee

The problem \eqref{upde}, \eqref{boundcond} is equivalent to the integral
equation
\be\label{inteq} u(x) = \intrrr g(x - t) u^3(t) \diff t, \ee

in $L^4$, where
\be\label{testfunc} g(x) = (4\pi)^{-1} \left| x \right|^{-1} e^{-|x|}. \ee

The details to the results below (as well as a more complete bibliography
concering \eqref{upde}) can be found in \cite{coffmcarn}.

\begin{itemize}
\item $C_0^\infty$ is dense in $H^1$.
\item If $u \in H^1$ then $v = |u| \in H^1$ and
\be |u|_{1, 2} = |v|_{1, 2}. \ee
\item If $u \in H^1$ then $u \in L^4$ and
\be\label{ineq} |u|_{0, 4} \leq 2^{-\half} |u|_{1, 2}. \ee
\item Let $V$ denote the subspace of $H^1$ consisting of radially symmetric
functions. The embedding $V \to L^4$ is compact.
\end{itemize}
% TODO: notation of $|u|_{0, 4}$ related to Sobolev spaces?

Except for the constant, \eqref{ineq} follows from \cite{Ladyzhenskaya} or from
the more general inequality in \cite{nirenberg} or \cite{friedman}. The constant
can be obtained from the representation $u = g\ast w$ where $w = - \Delta u +
u$. It suffices to prove \eqref{ineq} for $u \in C_0^\infty$. The assertion d)
follows in a straightforward way from the Sobolev imbedding theorem and the
inequality
\be 4\pi \int_{|x|\geq \rho} |v(x)|^4 \diff x \leq 2\rho^{-1} |v|_{1, 2}^4 \ee

for $v \in V$, $\rho > 0$.

    Concerning the convolution operator $\tau : u \to g \ast u$, where 
    \be (g \ast u)(x) = \intrrr g(x-t) u(t) \diff t \ee
    and $g$ is given by \eqref{testfunc}, we have the following results:

\begin{itemize}
\item $u \in L^{4/3}$ then $v = g \ast u \in H^1 \subseteq L^4$, $\intrrr u v
\diff x > 0$ unless $u = 0$, and $v$ is a weak solution of
\be\label{vpde} -\Delta v + v = u. \ee
\item If $u \in L^1 \cap L^\infty$ then $v = g \ast u$ has bounded continuous
first derivatives and
\be \underset{|x|\to\infty}{\lim} v(x) = 0. \ee
\item If $u \in L^1 \cap L^\infty \cap C^1$ then $v = g \ast u \in C^2$ and $v$
satisfies \eqref{vpde}.
\item Let $X$ and $Y$ denote the subspaces of $L^{4/3}$ and $L^4$ respectively,
consisting of radially symmetric functions. Then $Y = X^*$ and $\tau : X \to Y$
is compact.
\end{itemize}

The first assertion of h) is obvious, the second follows immediately from d) and
e).

\textbf{Remark} For consideration of the equation
\be\label{upde2} \Delta u - u + |u|^{p-2} u = 0,\ee

one replaces $L^4$ by $L^p$ and $L^{4/3}$ by $L^q$ where $p^{-1} + q^{-1} = 1$.
If $2 < p < 6$, then c), d), e) and h) remain valid in this more general case
(except for a change in the constant in \eqref{ineq}; e) and h) of course fail
for $p=2$.

\section{Minimization of $J$}
For $u \in L^4$, $u\neq 0$, we define $\sigma(u)$ by
\be\label{sigmadef} \left( \sigma(u) \right)(x) = c \intrrr g(x-t) u^3(t) \diff
t \ee

where $c > 0$ is chosen so that $v = \sigma(u)$ satisfies
\be\label{sigmaint} \intrrr v^4 \diff x = 1.\ee

This is possible since $u \in L^4$ implies $u^3 \in L^{4/3}$; thus by e), $g
\ast (u^3) \in L^4$ and is non-zero. It is clear that up to positive factors the
fixed points of $\sigma$ are precisely the non-trivial solutions of
\eqref{rayl}. From e) above it follows that $\sigma$ actually maps $L^4$ into
$H^1$ and thus by c), $\sigma$ can also be regarded as an operator in
$H^1\setminus\{0\}$. In particular it follows that an $L^4$ solution of \eqref{rayl}
must belong to $H^1$.

\begin{lemma} Let $u$ be an admissible function with
    \be\label{uint} \intrrr u^4 \diff x = 1. \ee
    Then $\sigma(u)$ is admissible and
    \be\label{jineq} J\left(\sigma(u)\right)\leq J(u) \ee
    with equality only if $\sigma(u) = u$. Moreover, $\sigma(u) \in L^\infty$,
    and $v=\sigma^2(u)$ has bounded continuous derivatives and satisfies
    \be\label{vlim} \underset{|x|\to\infty}{\lim} v(x) = 0;\ee
    finally $\sigma^3(u) \in C^2$.
\end{lemma}
\begin{proof}
The admissibility of $\sigma(u)$ follows from e). By e), \eqref{sigmadef} and
\eqref{uint}, $w=\sigma(u)$ satisfies
\be \label{cineq} c = c \intrrr u^4 \diff x = \intrrr \left( \grad w \cdot \grad
u + wu \right) \diff x \leq |u|_{1, 2}|w|_{1, 2}. \ee

\end{proof}


% \section{Introduction}

%In the reference paper \cite{nehari}

\references{dissertation}
