\section{Uniqueness theorem}
The radially symmetric solutions of \eqref{upde} are of the form
$$u(x)=|x|^{-1}w(|x|),$$ 
where $w(|x|)=w(r)$ solves 
\be\label{wvp}
w''-w+r^{-2}w^3=0.
\ee
We refer to \ref{weqv} for more details on \eqref{wvp}.
%
% \todo{Rephrase this paragraph}
To prove the uniqueness of the ground state solution $\phi_1$ for \eqref{upde},
it suffices to prove that \eqref{wvp} has at most one positive solution
satisfying the following boundary conditions
\be\label{wbc}
% \underset{r\to0}{\lim}~r^{-1}w(r) in some sense is u(0) \in (0, \infty)
0 < \underset{r\to0}{\lim}~r^{-1}w(r) < \infty,\quad
\underset{r\to\infty}{\lim} w(r)=0
\ee

% The problem \eqref{wvp},\eqref{wbc} is transformed to an initial value problem where 
Similar to the shooting method in \Cref{existence}, we consider the initial
condition 
\be\label{wic}
\underset{r\to0}{\lim}~r^{-1}w(r)=a>0.
\ee
rather than the boundary conditions \eqref{wbc}. The basic facts regarding the
problem \eqref{wvp}, \eqref{wic} are summarised in Lemma 4.1. {\red The proofs
are omitted in the original paper \cite{coffm}}. 

\begin{lemma}\label{qlem}
    For each $a>0$ the equation \eqref{wvp} has a unique solution $w=w(r,a)$ 
    which is of class $C^2$ on $\left(0, \infty\right)$ and satisfies
    \eqref{wic}.  The partial derivatives $w_a(r, a) = \partial_a w(r, a) =
    \frac{\partial w(r,a)}{\partial a}$ and $w_a'(r, a) = \partial_a w'(r, a) =
    \frac{\partial w'(r, a)}{\partial a}$ exist for all $r>0$ and $a>0$.
    %Let $q(r, a)$ denote $w_a(r, a)$. 
    Furthermore, $w_a(r, a)$ coincides on $(0, \infty)$ with the solution
    $q=q(r, a)$ of the regular initial value problem 
    \be\label{qvp} \begin{cases}
    q'' - q + 3r^{-2}w^2q=0,\\ 
    q(0)=0,\quad q'(0)=1,
    \end{cases}\ee
    with $w=w(r, a)$; $q'(r, a)=w_a'(r, a)$; $q''(r, a)=w_a''(r,
    a)=\frac{\partial w''(r, a)}{\partial a}$.  
\end{lemma}

% \todo{Rephrase result.}
The solution $w(r)$ to problem \eqref{wvp}, \eqref{wic} depends on the parameter
$a>0$ (initial condition). This motivates us to write $w=w(r, a)$ and formulate
the uniqueness result in terms of $a>0$.

\begin{theorem}\label{uniqthm}
There is at most one positive value of $a$ for which
\be\label{wpos}w(r,a)>0,\quad 0<r<\infty\ee
and
\be\label{wlim}\underset{r\to\infty}{\lim} w(r, a)=0.\ee
\end{theorem}

% \todo{Expand explanation}
Similar to the approach in \Cref{existence}, the solutions $w(r,a)$ are of one
of three types: (i) solutions that have a finite zero, (ii) solutions that are
positive everywhere and (iii) solutions that are positive everywhere and vanish
at infinity. We refer to definitions \eqref{nset}, \eqref{pset} and \eqref{gset}
respectively. {\red Revise notation of this chapter to match/contract the
notation in \Cref{existence} where necessary.} Let 
\be\label{z1def}
z_1 \coloneqq \inf\left\{\; r>0 \;\middle|\; w(r, a)=0\;\right\}.
\ee
We note that $z_1<\infty$ for $a\in N$ and we write $z_1=\infty$ for $a\in G\cup
N$. \Cref{uniqthm} is implied by the following lemma.

\begin{lemma}\label{qneglem}
    \begin{enumerate}[(i)]
        \item If $a>0$ and $w(r, a)>0$ on $(0, z_1)$ with $w(z_1, a)=0$, then
            $q(z_1, a)<0$.
        \item If $a>0$ and $w(r, a)$ satisfies \eqref{wpos} and \eqref{wlim} then
            \be\label{qexplim}
                \underset{r\to\infty}{\lim} e^{-r}q(r, a)<0.
            \ee
    \end{enumerate}
\end{lemma}

\begin{proof}[Proof of Theorem \ref{uniqthm}]
    We assume \Cref{qneglem}.  We first show that $z_1$ is a monotonically
    decreasing function of $a>0$. 
    Let
    \be\label{adef}
    A \coloneqq \left\{\;a>0 \;\middle|\; \exists r>0 \;:\; w(r,a)=0\;\right\}.
    \ee

    % and let $z_1=z_1(a)$ be the least positive zero of $w(r, a)$ for any $a\in A$.
    % If $a\in A$ and $z_1=z_1(a)$ is the least positive zero of $w(r,a)$, then,
    By \cref{qlem} and the implicit function theorem ({\red expand}), $z_1$ is
    differentiable with respect to $a$ and

    $$w'(z_1, a)\frac{\diff z_1}{\diff a} + \delta(z_1, a) = 0.$$

    Since $w(r, a)>0$ on $(0, z_1)$ and $w(z_1, a)=0$, we have $w'(z_1, a)<0$. 
    By \cref{qlem}(i), this implies $q(z_1, a)<0$.  Hence 
    $$\dfrac{\diff z_1}{\diff a}<0\quad\text{on}~A.$$ 

    Therefore, $z_1$ moves monotonically to the left as $a$ increases. 
    Thus if $A$ is non-empty it is a semi-infinite interval. 
   
    % \todo{Proof step 2. $a_1$ is the left endpoint of $A$ if $a_1$ satisifies
    % \eqref{wpos} and \eqref{wlim}.}
    Next, we show that if \eqref{wpos} and \eqref{wlim} hold for $a=a_1$, then
    $A$ is non-empty and $a_1$ is the left endpoint of $A$. This will clearly
    imply \Cref{uniqthm}.
   
    Let $a_1$ be as above and let $a_2>a_1$; we shall show that if $a_2-a_1$ is
    sufficiently small, then the assumption 
    \be\label{wass} w(r, a_2) > 0\quad\text{on}~(0, \infty)\ee 
    leads to a contradiction. Note that the assumption \eqref{wass} implies  
    $a_2\not\in A$, since $w(r, a_2)$ is assumed positive on $(0, \infty)$.

    We put $w_i=w_i(r, a_i)$ for $i=1, 2$\footnote{Correcting a typo in
    the definition of $w_i$ given in \cite{coffm}.}. 

    By our assumption on $a_1$ and ii) of \cref{qlem}, we can choose $r_0>0$
    such that

    \be\label{ubw1} 3r^{-2}w_1^2<\frac{1}{2},\quad r\geq r_0\ee
    and

    \be\label{qneg} q(r_0, a_1)<0,\quad\text{and}~q'(r_0, a_1)<0.\ee

    From \cref{qlem} and \eqref{qneg} it follows that if $a_2>a_1$ and $a_2-a_1$
    is sufficiently small, then
    \be\label{w2ltw1} w_2(r_0)<w_1(r_0),\quad w_2'(r_0)<w_1'(r_0).\ee

    We put $v = w_1-w_2$, so that $v$ satisfies\footnote{Correcting a 
    typo in (4.12) of \cite{coffm}.}
    \be\label{vivp} v'' - v + r^{-2}\left(w_1^2+w_1w_2+w_2^2\right)v=0.\ee

    We suppose that $w_2(r,a_2)$ satisfies \eqref{wass} and that
    \be\label{w2ltw1r0r1} 0<w_2<w_1\quad\text{on}~\left[r_0,r_1\right)\ee

    for some $r_1>r_0$. Such an $r_1$ exists by \eqref{w2ltw1}.
    % \todo{Untangle the argumentation and make some pictures. Compare to Genoud.}
    \begin{itemize}
        \item From \eqref{w2ltw1}, \eqref{vivp} and \eqref{ubw1} it follows that
            $v$ is positive and convex on $\left[r_0, r_1\right)$; 
        \item moreover, from \eqref{w2ltw1}, $v'(r_0)>0$ so that $v$ is
            increasing on $\left[r_0, r_1\right)$. 
        \item Thus \eqref{w2ltw1r0r1} holds at $r=r_1$;
        \item hency by a standard argument we conclude that \eqref{w2ltw1r0r1}
            holds on $\left[r_0, \infty\right)$ and that $v$ is increasing
            there. 
        \item The inequality \eqref{w2ltw1r0r1} on $\left[r_0, \infty\right)$
            implies that
            \be\label{wlimsumsq}
                \underset{r\to\infty}{\lim}\left(w_1^2+w_1w_2+w_2^2\right)=0.
            \ee
        \item Using this fact and the monotone character of $v$, we conclude
            from asymptotic integration of \eqref{vivp} that $v$ grows
            exponentially as $r\to\infty$.
        \item From the definition of $v$, this is clearly a contradiction of
            \eqref{wlim} for $a=a_1$ and \eqref{wass}. 
        \item We conclude therefore that \eqref{wass} cannot hold for $a_2>a_1$
            and $a_2-a_1$ arbitrarily small;
        \item therefore $a_2\in A$ for all $a_2>a_1$ with $a_2-a_1$ sufficiently
            small. 
        \item This completes the proof of \Cref{uniqthm}.
    \end{itemize}
\end{proof}

% Also, another (now regular) initial value problem in $\delta(r,a)$ is introduced. 
% 
% **Question:** Regularity in which sense? Usually smoothness, here 
% Theorem 4.1 claims uniqueness of the parameter $a>0$ for which $w(r, a)$ is positive on $(0, \infty)$ and vanishes at infinity. This is implied by Lemma 4.2.
% 
By studying the zeroes of $w(r,a)$, we can show that $A$ (the set of $a>0$
such that $w(r, a)$ has at least one zero in $(0, \infty)$) has a left
endpoint.

\begin{lemma}
Let $a>0$ and let $w=w(r,a)$ either vanish at least once in $(0, \infty)$
    or satisfy \eqref{wlim}; then $a>\sqrt{2}$, $w(r, a)=r$ for precisely one value
$r=r_0$ in $(0, z_1)$ and $w'(r_0, a)<0$.
\end{lemma}
\begin{proof}
{\red Proof steps.} The function $v(r, a)=r^{-1}w(r, a)$ satisfies
\be\label{vivp} v'' + 2r^{-1}v' - v + v^3 = 0\ee
and
\be\label{vic} \lim_{r\to0} v(r, a) = a,\quad \lim_{r\to 0}v'(r, a)=0.\ee

Upon differentiating the function
$$\Phi(r) = (v')^2 + \half v^4 - v^2\quad\left(v=v(r, a)\right)$$
and using \eqref{vivp}, \cite{coffm} concludes that, for $a>0$, $\Phi(r)$ is a
strictly decreasing function of $r$. 

We calculate $\Phi'(r)=\frac{\diff\Phi(r)}{\diff r}$ as
$$\Phi'(r) = 2v'v'' + 2v^3v' - 2vv'.$$

We rewrite \eqref{vivp} as
$$v'' - v + v^3 = -2r^{-1}v'$$

to conclude that 
$$\Phi'(r) = 2v'\left(v'' - v + v^3\right)=-4r^{-1}(v')^2$$
\end{proof}

is strictly decreasing {\red on $(0, z_1)$}.

\begin{lemma}
    $\alpha < y_1 < \beta.$
\end{lemma}
\begin{proof}
{\red Requires skeleton.}
\end{proof}

%The proof of Theorem 4.1 seems reasonable.
% The proofs of Lemma 4.2, 4.3 and 4.4 seem intricated/confusing.

\subsection{Derivation of equation for radially symmetric solutions}\label{weqv}
Consider radially symmetric solutions to \eqref{upde}. Then $u(x) = u(|x|) =
u(r)$ . This transforms \eqref{upde} to the ODE \eqref{ivp}, restated here for
$N=3$
\be \label{uode} u'' + \frac{2}{r} u' - u + u^3 = 0 \ee 

Furthermore, substituting $u(r) = r^{-1}w(r)$, we calculate the derivatives of
$u$ as
\begin{enumerate}
    \item $u'(r) = -r^{-2}w(r) + r^{-1}w'(r)$
    \item $u''(r) = 2r^{-3}w(r) - 2r^{-2}w'(r) + r^{-1}w''(r)$.
\end{enumerate}
% Then, since $N=3$, \eqref{uode} reads
We substitute in \eqref{uode} to obtain
\begin{multline} 
u''(r) + \frac{2}{r} - u(r) + u^3(r) \\
= 2r^{-3}w(r) - 2r^{-2}w'(r) + r^{-1}w''(r)
+ \frac{2}{r}\left(-r^{-2}w(r) + r^{-1}w'(r)\right) \\
- r^{-1}w(r) + r^{-3}w^3(r) = 0,
\end{multline}
which is simplified to
$$ r^{-1}\left(w'' - w + r^{-2}w^3 \right) = 0. $$
In conclusion, since $r\neq 0$, we obtain
\be \label{wode} w'' - w + r^{-2}w^3 = 0. \ee

% So indeed, positive radially symmetric solutions to \eqref{upde} are of the form 
% $$ u(x) = |x|^{-1} w(|x|) $$
% where $w(|x|)$ satisfies the differential equation \eqref{wode}.
% 
% Therefore, uniqueness of $w(r)$ to
% \be \begin{dcases} \label{wbvp}
%     w'' - w + r^{-2}w^3 = 0,\\
%     0 < \lim_{r\to 0}r^{-1}w(r) < \infty,\\
%     \lim_{r\to\infty}w(r) = 0
% \end{dcases} \ee
% 
% is sufficient for uniqueness of $\phi_1$ to \eqref{upde}.
% 
% We can interpret \eqref{wbvp} as an initial value problem
% \be \begin{dcases} \label{wivp}
%     w'' - w + r^{-2}w^3 = 0,\\
%     \lim_{r\to 0}r^{-1}w(r) = a.
% \end{dcases} \ee


% \section{Introduction}

%In the reference paper \cite{nehari}
