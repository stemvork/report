\documentclass[11pt,a4paper]{amsart}
\usepackage{amsfonts}
\usepackage{amsthm}
\usepackage{amsmath}
\usepackage{amssymb}
\usepackage{amsxtra}

% \usepackage{geometry}
% \usepackage[moderate]{savetrees}

\usepackage{hyperref}
\usepackage[english]{babel}
\usepackage{url}
\usepackage{lineno}
\usepackage{mathrsfs}
\usepackage{fancyhdr}
\usepackage{enumerate}
\usepackage{graphicx}
% \usepackage{pdfsync}
\usepackage{psfrag}
\usepackage{tikz}
\usepackage{soul}
\usepackage{listings}
\usepackage{cleveref}
\usepackage{empheq}
\usepackage{marvosym}
\usepackage{mathtools}

\usepackage{markdown}
\usepackage{outlines}

\tikzstyle{nodino}=[circle,draw,fill,inner sep=0pt,minimum size=0.5mm]
\tikzstyle{infinito}=[circle,inner sep=0pt,minimum size=0mm]
\tikzstyle{nodo}=[circle,draw,fill,inner sep=0pt,minimum size=0.5*\widthof{k}]

\newcommand{\blu}[1]{{\color{blue}#1}}
\newcommand{\ros}[1]{{\color{red}#1}}
\newcommand{\nero}[1]{{\color{black}#1}}
\newcommand{\verde}[1]{{\color{green}#1}}
\newcommand{\magenta}[1]{{\color{magenta}#1}}
\newcommand{\ciano}[1]{{\color{cyan}#1}}

%%%%%%%%%%%%%%%%%%%%%%%%%%%
% DOCUMENT LAYOUT
%%%%%%%%%%%%%%%%%%%%%%%%%%%
\setlength{\textwidth}{16cm}
\setlength{\textheight}{24cm}
\setlength{\oddsidemargin}{0cm}
\setlength{\evensidemargin}{0cm}
\setlength{\marginparwidth}{2cm}
\hoffset=0truecm
\voffset=-1.5truecm
\footskip = 30pt
\marginparsep=-0.2cm

%%%%%%%%%%%%%%%
%theorem's style
%%%%%%%%%%%%%%%
\newtheorem{theorem}{Theorem}
\newtheorem{prob}{Problem}
\newtheorem{lemma}{Lemma}[section]
\newtheorem{proposition}[lemma]{Proposition}
\newtheorem{corollary}{Corollary}
\newtheorem{definition}[lemma]{Definition}
\newtheorem{assumption}{Assumption}

\theoremstyle{definition}
\newtheorem{remark}[lemma]{Remark}

%%%%%%%%%%%%%%%
%numbering equations
%%%%%%%%%%%%%%%
\numberwithin{equation}{section}

%%%%%%%%%%%%%%%%%%
%TEMPORARY NEWCOMMANDS
%AND PACKAGES
%%%%%%%%%%%%%%%%%%
%\newcommand{\comment}[1]{}
%\def\sidenote#1{\begin{small}\marginpar{\bf #1}\end{small}}
%\usepackage[notref,notcite]{showkeys}
%\usepackage{refcheck}
%\newcommand{\claudio}[1]{{\bf claudio: }{#1}\\ \noindent}
%\newcommand{\diego}[1]{{\bf diego: }{#1}\\ \noindent}


%%%%%%%%%%%%%%%%%%%%%%%%%%%%%%%
%NEWCOMMANDS
%%%%%%%%%%%%%%%%%%%%%%%%%%%%%%%
\newcommand{\pd}[2]{\frac{\partial {#1}}{\partial {#2}}}
\newcommand{\beq}{\begin{equation}}
\newcommand{\eeq}{\end{equation}}
\newcommand{\be}{\begin{equation*}}
\newcommand{\ee}{\end{equation*}}
\newcommand{\n}{\noindent}
\newcommand{\Wbb}{\mathbb{H}^{in}}
%\newcommand{\Tau}{\mathcal{T}}

\newcommand{\vertiii}[1]{{\left\vert\kern-0.25ex\left\vert\kern-0.25ex\left\vert #1
    \right\vert\kern-0.25ex\right\vert\kern-0.25ex\right\vert}}

\newcommand{\RE}{\mathbb R}
\newcommand{\erre}{\mathbb R}
\newcommand{\CO}{\mathbb C}
\newcommand{\NA}{\mathbb N}
\newcommand{\II}{\mathbb I}
\newcommand{\Obb}{\mathbb O}
\newcommand{\Gbb}{\mathbb G}
\newcommand{\Hbb}{\mathbb{H}}
%\newcommand{\PP}{\mathbb P}
\newcommand{\BB}{\mathscr B}
\newcommand{\DD}{\mathcal D}
\newcommand{\OO}{\mathcal{O}}
%\newcommand{\QQ}{\mathcal{Q}}
\newcommand{\QQ}{Q}
\newcommand{\CC}{\mathcal{C}}
\newcommand{\HH}{\mathbb{H}}
\newcommand{\GG}{\mathcal{G}}
\newcommand{\UU}{\mathcal{U}}
\newcommand{\ulim}{\operatorname{u}-\lim}
\newcommand{\sgn}{\operatorname{sgn}\,}
\newcommand{\supp}{\operatorname{supp}\,}
\newcommand{\Ran}{\operatorname{Ran}\,}
\newcommand{\Ker}{\operatorname{Ker}\,}
\newcommand{\lf}{\left}
\newcommand{\ri}{\right}
\newcommand{\ve}{\varepsilon}
\newcommand{\al}{\alpha}
\newcommand{\ual}{\undeline{\al}}
\newcommand{\bt}{\beta}
\newcommand{\Si}{\Sigma}
\newcommand{\si}{\sigma}
\newcommand{\Ga}{\Gamma}
\newcommand{\ga}{\gamma}
\newcommand{\La}{\Lambda}
\newcommand{\la}{\lambda}
\newcommand{\de}{\delta}
\newcommand{\De}{\Delta}
\newcommand{\ci}{\mathbb{C}}
\newcommand{\wconv}{\xrightarrow{w}}
\newcommand{\ome}{\omega}
\DeclareMathOperator{\sech}{sech}
\DeclareMathOperator{\arctanh}{arctanh}
\providecommand{\ove}[1]{\overline{#1}}
\newcommand{\lan}{\langle}
\newcommand{\ran}{\rangle}
%RENEWCOMMAND
\renewcommand{\Im}{\operatorname{Im}\,}
\renewcommand{\Re}{\operatorname{Re}\,}
\renewcommand{\neg}{\operatorname{Neg}\,}

\renewcommand{\leq}{\leqslant}
\renewcommand{\geq}{\geqslant}

%MORENEWCOMMANDS
\newcommand{\x}{\underline{x}}
\newcommand{\y}{\underline{y}}
\renewcommand{\v}{\underline{v}}
\newcommand{\vd}{\underline{v}_2}
\newcommand{\vu}{\underline{v}_1}
\newcommand{\f}{\frac}
\newcommand{\EE}{\mathcal E}
\newcommand{\C}{\mathbb{C}}
\newcommand{\VV}{R}
\newcommand{\WW}{S}
\newcommand{\ZZ}{Z}
\newcommand{\omestar}{\tilde{\ome}}

\newcommand{\vol}{\operatorname{Vol}}
\newcommand{\one}{{\bf 1}}
% \renewcommand{\a}{\underline{a}}
\renewcommand{\b}{\underline{b}}

%%%%%%%%%%%%%%%%%%%%%%%%%%%%%%%%%
%ADDITIONS BY JASPER EENHOORN
%%%%%%%%%%%%%%%%%%%%%%%%%%%%%%%%%
\renewcommand{\a}{\alpha}
\newcommand{\uowd}{\left(\frac{u}{w}\right)'}
\newcommand{\nub}{\nu_{\beta}}
\newcommand{\phib}{\phi_{\beta}}

\newcommand{\comm}[1]{}
\setlength\parindent{24pt}
\newcommand{\limrtoinf}{\underset{r\to\infty}{\lim}}
\newcommand{\za}{z(\alpha)}
\def\at{
  \left.
  \vphantom{\int}
  \right|
}
\newcommand{\rplus}{\mathbb{R}_+}
\def\seperate{\vspace{1em}\hrule\vspace{1em}}
%%%%%%%%%%%%%%%%%%%%%%%%%%%%%%%%%
%AUTHOR AND TITLE
%%%%%%%%%%%%%%%%%%%%%%%%%%%%%%%%%
\title[]{On existence and uniqueness of ground state solutions to NONLINEAR SCHR\"ODINGER EQUATION}

\author[]{Jasper Eenhoorn}
\address{Department of Applied Physics, TU Delft, Lorentzweg 1, 2628CJ, Delft, Netherlands, EU}
\email{j.s.eenhoorn@student.tudelft.nl}%

\date{}

%\begin{document}

%\thanks{}
\begin{document}
\begin{outline}
  \1 Define Lyapunov function $E(r)$.
  \1 Rewrite IVP to seperate $-\frac{1}{r}u'(r)$.
  \1 Calculate $E'(r)$.
  \1 Use IVP to simply expression.
  \1 Conclude from sign of $E'(r)$ that $E(r)$ is nonincreasing.
    \2 Note that $V'(r)\leq0$ and $u'(r)^2\geq0$, $r\geq0$ and $u(r)^{p+1}\geq0$.
  \1 Evaluate $E(0)$. Analyse the sign of $E(0)$ as a function of $\alpha$.
    \2 Solve $E(0)=0$ for $\bar\alpha$ and note $\alpha<\bar\alpha\implies E(0)<0$ and vice versa.
  \1 Now analyse the behaviour of solutions with $\alpha<\bar\alpha$ using proof by contradiction.
    \2 $\alpha<\bar\alpha\in N$ would imply $u(z(\alpha))=0$ and $E(z(\alpha))\geq0$ contradicting $E<0$.
    \2 $\ldots\in G$ would imply $\underset{r\to\infty}{\lim}E(r)=0$ contradicting $E<0$.
  \1 Conclusion: $\alpha\in P$.
\end{outline}
\seperate

\begin{outline}
  % This part is already in Lemma 5.1?
  % TODO: What is the actual interval of definition of $E(r)$?
  \1 As $\alpha\in G\cup N$, $E(r)\geq0$ for $r\in(0,z(\alpha))$.
  \1 Even in $z(\alpha)$, $E(z(\alpha))\geq0$ (SHOW).

  \1 By IVP and Lemma 5.1: $u''(0)<0$.
  % TODO: Compare with $\lim_{r\to0}ru'(r)=0$.
  \1 Using an argument from Chapter 4: $u''(0)$ and $u'(0)=0$ would imply $u\equiv\alpha$.

  % TODO: Build the actual argument that contradicts E non-increasing.
  \1 On the other hand $u''(0)>0$ and $u'(0)=0$ would imply $u(r)>u(0)=\alpha$ for $r>0$ and small.
  \1 This contradicts $E'(r)\leq0$? (SHOW)
  \1 Conclusion: $u''(0)<0$ and $u'(r)<0$ for $r>0$ and small.

  % TODO: Draw figure F1.
  \1 To show $u'(r)<0$ for $r\in(0,z(\alpha))$:
    \2 Suppose by contradiction that there exists $r_0$ such that $u'(r_0)=0$.
    \2 Let $r_0\coloneqq\{r>0:u'(r;\a)=0\}$.
    \2 By the IVP: $u''(r_0)=\lambda u(r_0)-V(r_0)u(r_0)^p$.

    \2 From the previous concavity argument, $u''(r_0)\geq0$ and $u''(r_0)=0$ would imply $u\equiv u(r_0)$.
    \2 This concavity argument is Figure F1.

    \2 Then $u''(r_0)>0$ which can be used in analysis of $E(r_0)$.
    \2 Evaluate $E(r_0)=\frac{1}{2}u'(r_0)^2-\frac{\lambda}{2}u(r_0)^2 + \frac{1}{p+1}V(r_0)u(r_0)^{p+1}\geq 0$.
    \2 An interesting property of $u(r_0)$ by IVP, $u''(r_0)>0$ and $u'(r_0)=0$ follows
      \3 $u(r_0)\leq\left[\frac{\lambda}{V(0)}\right]^{\frac{1}{p-1}}<
      \left[\frac{\lambda}{V(0)}\frac{p+1}{2}\right]^{\frac{1}{p-1}} \iff -\frac{\lambda}{2}u(r_0)^2+\frac{1}{p+1}V(r_0)u(r_0)^{p+1}>0 \iff E(r_0)>0$
      \3 Which contradicts $E(r)\geq0$ for $r\in(0,z(\alpha))$.
    \2 Hence $u'(r)<0$ for $r\in(0,z(\alpha))$.
  \1 If $\alpha\in N$ then $u'(z(\alpha))=0$ then $u\equiv0$.
    \2 Concavity prevents $u'(z(\alpha))>0$. (SHOW)
    \2 What does this yield if $\alpha\in G$? (SHOW)
  \1 So for $\alpha\in N$: $u'(r)<0$ for $r\in(0,z(\alpha)]$.
\end{outline}
%
% \emph{\color{teal}So $u(r)$ has initial condition $u(0)=\alpha>0$. And then? Depens on the solution set. For solutions in $P$, the solution will remain positive. Without any further information on global behaviour. For solutions in $G\cup N$, there will be a first zero. This might be at infinity. Clearly, the solution is decreasing. In fact, the solution will prove to be strictly decreasing.}

% \emph{\color{red}Now, the Lyapunov function $E(r)$ is well-defined for $r\in(0,z(\alpha))$ and non-negative! For deriving $E(r)\geq0$ in $z(\alpha)$ use L'Hopital rule for limit? From the lemma before and the IVP it follows ``EASILY'' that $u''(0)<0$ and so $u'<0$ immediately to the right of $r=0$....... Even for $V(0)=\infty$ this property follows from the equation.\\ \\ Suppose by contradiction that $\exists r_0\in(0,z(\alpha))$ where $u'(r_0)=0$. Then by the IVP: $\lambda u(r_0) - V(r_0)u(r_0)^p = u''(r_0)\geq 0$. And so $u(r_0)$ by lemma 2.1 has $E(r_0)<0$: a contradiction. Thus $u'(r)<0$ for $r\in(0,z(\alpha))$. Also $u'(z(\alpha))<0$ to prevent $u\equiv0$.}
%
\seperate

\begin{outline}
  \1 To conclude that $w$ has one zero in $(0,\za)$, use Lagrange identity.
  \1 The proofs for $\a\in G$ and $\a\in N$ will be done seperately.
  \1 First, suppose $\a\in N$.
  \1 Rewrite IVP and \emph{w-d.e.} to the following:
    \2 $(ru')'+r\left[-\lambda u+Vu^p\right]=0$
    \2 $(rw')'+r\left[-\lambda w+pVu^{p-1}w\right]=0$
  \1 Multiply by $w$ and $u$ respectively, then integrate from 0 to $\za$:
    \2 $\int_0^{\za}w(ru')'-u(rw')'dr = \int_0^{\za}r\left[pVu^pw-Vu^pw\right]dr$
  \1 By partial integration for left hand side, one obtains:
    \2 $rwu'\at_0^{\za}-ruw'\at_0^{\za}-\int_0^{\za}\left[ru'w'-ru'w'\right]dr=(p-1)\int_0^{\za}rVu^pwdr$
  \1 Use $u(\za)=0$ to obtain:
    \2 $\za w(\za)u'(\za)-\za u(\za)w'(\za) = (p-1)\int_0^{\za}rVu^pwdr$
  \1 Suppose $w>0$ on $(0,\za)$ then left hand side $\leq0$ as $\za>0$, $w(\za)\geq0$ (?) and $u'(\za)<0$.
  \1 {\color{red} To resolve this, suppose $w>0$ on $(0,\za)$ then $\za>0$, $w(\za)>0$ and $u'(\za)<0\implies\text{ l.h.s. }<0$. That \emph{is} sufficient to show contradiction with $\text{r.h.s.}>0$ as ...}
  \1 {\color{blue} To \emph{actually} resolve this, the initial supposition was correct: $w>0$ on $(0,\za)$ implies $\text{l.h.s.}\leq0$ as $\za>0$, \emph{importantly} $w(\za)>0$ and $u'(\za)<0$. Why can't $w(\za)=0$? By Sturm comparison! Since $pVwu^{p-1}\neq Vu^p$, the zeroes of $w$ and $u$ will not coincide!}
    \2 Note work remains to be done to clarify this part of the argument. From \'Suppose $w>0$...\' to the contradiction.
  \1 By this contradiction then, $w(r)$ has at least one zero on $(0,\za)$.
  \1 To conclude the same for $\a\in G$, again, assume $w>0$ on $(0,\za)$ then still r.h.s. of \ref identity $>0$.
  \1 As for the l.h.s. regard the expression $\frac{u}{w}$...
    \2 Write $\left(\frac{u}{w}\right)'=\frac{wu'-uw'}{w^2}$.
    \2 Note that $u(0)>0$ and $w(0)>0$ implies $\frac{u}{w}(0)>0$.
    \2 Rewrite the \ref identity to read:
      \3 $rwu'-ruw'=(p-1)\int_0^{\za}rVu^pwdr$
      \3 $\frac{wu'}{w^2}-\frac{uw'}{w^2}=\frac{p-1}{rw^2}\int_0^{\za}rVu^pwdr>0$
      \3 $\frac{wu'-uw'}{w^2}>0\implies \left(\frac{u}{w}\right)'>0$
    \2 So $\uowd$ is increasing.
  \1 Now, there is a hole.. Apparantly, this also implies $w>0$ yields contradiction.
    \2 Intuitively, $\uowd$ increasing means that $w$ decays faster than $u$ everywhere.
    \2 Then, as for $\a\in G$ the solution decays to 0, so must $w$.
    \2 But, since $w$ decays faster than $u$, the zero of $w$ must be to the left of $\za=\infty$.
    \2 Hence, $w$ has a zero in $(0,\za)$.
      \3 Hooray!
  \1 Might need to formalise this a bit further.
\end{outline}
\seperate

% LEMMATA 4 through 8 still coming..

\end{document}
