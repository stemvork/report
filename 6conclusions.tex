\comm{\section{Main theorem\label{s:mainth}}
\begin{proof}[Proof of Theorem \ref{t:prob1}]
We prove first that $mE_0<\nu<\infty$. The lower bound $\nu>mE_0$ is a direct consequence of the fact that $E[\Psi] < E^{lin}[\Psi]$, for all $\Psi\in \EE$, and that, by the definition of $E_0$  and $E^{lin}$, one has
\[\inf \{ E^{lin}[\Psi] \text{ s.t. } \Psi\in \EE , \, M[\Psi]=m  \} = -mE_0.\]\\
To prove that $\nu<+\infty$ we first note that, by using  H\"older and   Gagliardo-Nirenberg inequalities, one can prove the bounds:  
\begin{equation}\label{4.1a}
 \|\Psi\|_{{2\mu+2}}^{2\mu+2} \leq c \|\Psi\|_{H^1}^{\mu} \|\Psi\|^{2+\mu}; \end{equation}
\begin{equation}\label{W-1}
(\Psi,W_- \Psi)  \leq \|W_-\|_r \|\Psi\|^2_{2r/(r-1)} \leq c \|W_-\|_r  \|\Psi\|_{H^1}^{2\alpha}\|\Psi\|_q^{2(1-\alpha)} \end{equation}
for all $q\in[2,2r/(r-1)]$ and with $ \alpha= \frac{2}{2+q}\left(1-\frac{q(r-1)}{2r}\right)$; and 
\begin{equation}\label{4.1b}
 |\Psi(\v)|^2\leq \|\Psi\|_{\infty}^2\leq c  \|\Psi\|_{H^1}\|\Psi\| \quad  \forall \v \in V \,.
\end{equation}
We remark that the inequalities \eqref{4.1a} - \eqref{4.1b} hold true for any connected finite graph. 
 If $M[\Psi] = m$, by \eqref{4.1a} - \eqref{4.1b} we have 
\[
 E[\Psi] + m
 \geq \| \Psi \|_{H^1}^2 -C \frac{m^{\frac{2+\mu}{2}}}{\mu+1}\|\Psi\|_{H^1}^{\mu}
  - C \sqrt m \sum_{\underline v\in V_v}|\alpha(\underline v)|   \|\Psi\|_{H^1} -  Cm^{1-1/(2r)} \|W_-\|_r  \|\Psi\|_{H^1}^{1/r}.
\]
We notice that for any $a,b,c,d>0$, $r\geq 1$, and   $0< \mu < 2$ there exist $\de,\beta >0$ such that $a x^2 - bx^\mu  -cx -d x^{1/r} > \de x^2 - \beta $, for any $x\geq0$, then  
\beq
\label{e:dec}
 E[\Psi] + m
 \geq \de \| \Psi\|_{H^1}^2 -\beta \,,
\eeq
which implies $\nu \leq \beta + m$.\\

In the  remaining part of the proof we shall prove that we can choose $m^*$ such that for $m<m^*$ minimizing sequences have a  convergent subsequence.

Let $\{\Psi_n\}_{n\in\NA}$ be a minimizing sequence, i.e., $\Psi_n \in\EE$, $M[\Psi_n]=m$, and $\lim_{n\to\infty} E[\Psi_n] = -\nu$. Concerning the mass constraint, we remark that it is enough to assume $M[\Psi_n]\to m$ as $n\to \infty$, in such a case one can define $\widetilde \Psi_n = \sqrt{m} \Psi_n /\|\Psi_n\|$ and note that $\lim_{n\to\infty} E[\widetilde\Psi_n] = \lim_{n\to\infty} E[\Psi_n] $. 

We shall prove that there exists $\hat \Psi \in H^1 (\GG)$ such that $M[\hat \Psi] = m $, $E[\hat \Psi] =-\nu$ and $\Psi_n \to \hat \Psi$ in $ H^1 (\GG)$.

We can assume that 
$E[\Psi_n] \leq -\nu/2$ then by inequality \eqref{e:dec}, up to taking a subsequence,  we can assume that 
\[
\sup_{n\in\NA}\|\Psi_n\|_{H^1}\leq \infty,\]
moreover the following lower bound holds true
\beq
\label{e:floor}
 \frac{1}{\mu+1} \| \Psi_n\|_{2\mu+2}^{2\mu+2} +(\Psi,W_-\Psi) +  \sum_{\v \in V_- } |\al(v)| |\Psi_{n}(\v)|^2 \geq
\frac{\nu}2 \,.
\eeq

Next we use Lem. \ref{l:cc} and  prove that  vanishing and dichotomy
cannot occur for $\{\Psi_n\}_{n\in\NA}$. Set $\tau = \lim_{t\to\infty}\liminf_{n\to\infty}
\rho(\Psi_n,t)$. First we prove that  vanishing cannot occur. If
$\tau=0$,  then by Lem. \ref{l:cc} there would exist a subsequence $\Psi_{n_k}$
such that $\| \Psi_{n_k}\|_{p} \to 0 $ for all $2<p\leq \infty$ but
this, together with Eqs. \eqref{W-1} and \eqref{4.1b},   would contradict 
\eqref{e:floor}. \\
To prove that dichotomy cannot occur, suppose $0<\tau<m$, then there
would exist $\VV_k$ and $\WW_k$ satisfying \eqref{dic1}-\eqref{dic7}.
In particular we know that
\[
 \liminf_{k\to \infty} \left(\|\Psi_{n_k}' \|^2 - \| \VV_k' \|^2
- \| \WW_k' \|^2 \right)  \geq 0
\]
\[
\lim_{k\to \infty} \lf( \| \Psi_{n_k}\|_p^p - \|\VV_k \|_p^p -
\| \WW_k \|_p^p \ri)=0 \qquad 2\leq p < \infty 
\]
and 
\be
\lim_{k\to \infty}\left||\Psi_{n_k}(\v)|^2-  |\VV_{k}(\v)|^2 -| \WW_{k}(\v)|^2\right| = 0\,.
\ee
Moreover we claim that 
\begin{equation}\label{claim1}
\lim_{k\to\infty}  (\Psi_{n_k}, W\Psi_{n_k}) - (\VV_k,W \VV_k) - (\WW_k, W\WW_k) \geq  0  ,
\end{equation}
we postpone the proof of this claim to the end of the discussion.  Summing up, we arrive at
\be
\liminf_{k\to\infty} \lf(
E[\Psi_{n_k} ] - E[ \VV_k] - E[\WW_k]
\ri) \geq 0 \,,
\ee
which implies
\beq
\label{e:black-1}
\limsup_{k\to\infty} \lf(
 E[ \VV_k] + E[\WW_k]
\ri) \leq -\nu \,.
\eeq
Notice that, given $\Psi\in\EE$ and $\de >0$, then
\[
E[\Psi] = \frac{1}{\de^2} E[\de \Psi] + \frac{\de^{2\mu} -1}{\mu+1} \| \Psi
\|_{2\mu+2}^{2\mu+2}.
\]
We remark that $\VV_k, \WW_k \in \EE$, since $\Psi_{n_k}$ satisfies   the continuity condition at
the vertices  and the multiplication with the cut-off functions
preserves that. Let $\de_k= \sqrt {m / M[\VV_k]}$ and $\ga_k = \sqrt {m / M[\WW_k]}$  such that $M[\de_k \VV_k] ,\,
M[\ga_k \WW_k] =m$. Then, using the above equality and the fact that
$E[\de_k \VV_k], E[\ga_k \WW_k] \geq -\nu$,  one has
\[
E[\VV_k] \geq - \frac{\nu}{\de^2_k} + \frac{\de^{2\mu}_k -1}{\mu+1} \| \VV_k
\|_{2\mu+2}^{2\mu+2}
\]
\[
E[\WW_k] \geq - \frac{\nu}{\ga^2_k} + \frac{\ga^{2\mu}_k -1}{\mu+1} \| \WW_k
\|_{2\mu+2}^{2\mu+2}
\]
from which 
\[
E[\VV_k]+E[\WW_k] \geq -\nu \lf( \frac{1}{\de^2_k} + \frac{1}{\ga^2_k} \ri) +
\frac{\de^{2\mu}_k -1}{\mu+1} \| \VV_k \|_{2\mu+2}^{2\mu+2} +
\frac{\ga^{2\mu}_k -1}{\mu+1} \| \WW_k \|_{2\mu+2}^{2\mu+2}\,.
\]
Notice that  by \eqref{dic4}
\[
\frac{1}{\de^2_k} \to \frac{\tau}{m} \qquad \qquad \frac{1}{\ga^2_k} \to 1-\frac{\tau}{m}\,.
\]
Let $\theta = \min \{ (\tau/m)^{-\mu} , (1-\tau/m)^{-\mu} \}$ and notice that $\theta
>1$ since $0<\tau/m <1$. Therefore
\begin{align}
\label{e:black-2}
\liminf_{k\to\infty} \lf(
 E[ \VV_k] + E[\WW_k]
\ri) 
&\geq -\nu + \frac{\theta -1}{\mu+1} \liminf_{k\to\infty} \| \Psi_{n_k}
\|_{2\mu+2}^{2\mu+2} > -\nu,
\end{align}
where we used the fact that $\liminf_{k\to\infty} \| \Psi_{n_k}
\|_{2\mu+2}^{2\mu+2} \neq 0$. The latter claim is proved by noticing
that $\liminf_{k\to\infty} \| \Psi_{n_k} 
\|_{2\mu+2}^{2\mu+2} = 0$,  together with $\| \Psi_{n_k}
\|_{H^1}$ bounded and Eqs. \eqref{W-1} and \eqref{4.1b},  would imply $\liminf_{k\to\infty} (\Psi_{n_k},W_-\Psi_{n_k}) =0$
 and $\liminf_{k\to\infty} \|\Psi_{n_k}\|_\infty =0$. Hence, there would be a contradiction with   inequality \eqref{e:floor}. We conclude that if $0<\tau<m$  we get a contradiction, cfr. inequalities \eqref{e:black-1} and \eqref{e:black-2}. To end the analysis of the case $0<\tau<m$ we are left to prove the claim \eqref{claim1}.  We rewrite $W = W_+- W_-$ and consider first the term with $W_+$. We have that 
\[\begin{aligned}
 &(\Psi_{n_k}, W_+\Psi_{n_k}) - (\VV_k,W_+ \VV_k) - (\WW_k, W_+\WW_k)     \\ 
=&   \sum_e \int_{I_e}(W_+)_e \left[1 - (\Theta_k)_e^2 - (\Phi_k)_e^2\right] |(\Psi_{n_k})_e|^2 dx \geq 0. 
\end{aligned}\]
Since $\VV_k$ and $\WW_k$ have disjoint supports, we have that 
\begin{equation*}\begin{aligned}
&\left| (\Psi_{n_k}, W_-\Psi_{n_k}) - (\VV_k,W_- \VV_k) - (\WW_k, W_-\WW_k) \right|  \\ 
\leq &  |(\ZZ_k, W_-\ZZ_k)| +2|(\VV_k,W_- \ZZ_k)| +2 |(\WW_k, W_-\ZZ_k) | \\  
\leq & | (\ZZ_k, W_-\ZZ_k)| +2(\VV_k,W_-\VV_k)^{1/2 }(\ZZ_k,W_- \ZZ_k)^{1/2} +2(\WW_kW_-\WW_k)^{1/2 }(\ZZ_k,W_- \ZZ_k)^{1/2} .
\end{aligned}\end{equation*}
The terms containing $\VV_k$ and $\WW_k$ are bounded by Lemma \ref{l:cc} and inequality \eqref{W-1}. The terms containing $\ZZ_k$, go to zero by  inequality \eqref{W-1}  and because $\|\ZZ_k\|\to 0$ by Eq. \eqref{ZZk}. From which the claim \eqref{claim1} follows. 

Since  $0\leq\tau<m$ leads us to a contradiction, it must be $\tau=m$. 

Now we prove that for $m<m^\ast$ the minimizing sequence is not {\em runaway}. Here the limitation on the mass plays a role for the first time.
By absurd suppose that $\{\Psi_n\}_{n\in\NA} $ is {\em runaway}, then we have that 
\begin{equation}\label{limit}
\lim_{n\to\infty} \Psi_{n} (\v) =0\quad \forall \underline{v}\in V\qquad \text{and}\qquad \lim_{n\to\infty}(\Psi_n, W_- \Psi_n)= 0.
\end{equation} The first limit  is a direct consequence of Lem. \ref{l:cc}, Eq. \eqref{e:run-1}. To prove the second one, assume that $\Psi_n$ escapes at infinity on the external  edge $e^*$ (this can always be done up to taking a subsequence). We note that
\begin{equation*}
\lim_{n\to\infty}\int_{I_e} (W_-)_e |(\Psi_n)_e|^2 dx = 0 \qquad \forall e \neq e^* ,
\end{equation*}
this is a  direct consequence of Lemma \ref{l:cc} and inequality  \eqref{W-1} applied to the edge $I_e$. We are left to prove that 
\begin{equation}\label{holiday}
\lim_{n\to \infty}\int_{0}^{+\infty} (W_-)_{e^*} |(\Psi_n)_{e^*}|^2 dx = 0 . 
\end{equation}
 We start by noticing that  $\|\Psi_n\|_{H^1}$ is uniformly bounded, hence, so is $\|\Psi_n\|_p$ for all $p\in [2,+\infty]$, by \eqref{gajardo3} (with $q=2$). As a consequence, we have that  for any $\ve>0$ there exists $R>0$ (independent of $n$) such that
 \begin{equation*}
\int_{R}^{+\infty} (W_-)_{e^*} |(\Psi_n)_{e^*}|^2 dx \leq \|(W_-)_{e^*}\|_{L^r(R,\infty)} \|\Psi_n\|_{2r'}^2 \leq \ve,
\end{equation*}
with $r'$ such that $r^{-1}+{r'}^{-1} =1 $.  For such $R$, there exists $n_0$ such that for all $n>n_0$ one has 
\begin{equation*}
\int_{0}^{R} (W_-)_{e^*} |(\Psi_n)_{e^*}|^2 dx \leq  \|W_-\|_r \|(\Psi_n)_{e^*}\|_{L^{2r'}(0,R)}^2 \leq \ve  
\end{equation*}
by \eqref{e:run-1} (see also Rem. \ref{r:2.8}), from which the second limit in \eqref{limit}. 

Recalling that, by Lem. \ref{l:cc} - Eq. \eqref{e:run-1}, one has $\lim_{n\to\infty}\|(\Psi_n)_e\|_{L^{2\mu+2}(I_e)} =0$ for all $e\neq e^*$, and by Eq. \eqref{limit}, we infer 
 \begin{equation}
 \label{little}
 \lim_{n\to \infty} E[\Psi_n]  \geq \lim_{n\to\infty }  \int_0^\infty |(\Psi_n)_{e^*}'|^2 dx -\frac{1}{\mu+1} \int_0^\infty  |(\Psi_n)_{e^*}|^{2\mu+2} dx.
 \end{equation}
 Let  $\chi:\RE_+ \to [0,1]$ be a  function such that $\chi \in C^\infty(\RE_+)$, $\chi(0) = 0$ and $\chi(x)=1$ for all $x\geq 1$.  Define 
\[\psi_n^*(x) :=  \chi(x)(\Psi_n)_{e^*}(x) , \]
so that $\psi_n^*(0)= 0$, and ${\|\psi_n^*}'\|_{L^2(\RE_+)}^2 \leq c$. By Lem. \ref{l:cc} - Eq. \eqref{e:run-1}, for all $p\geq 2$,
\begin{equation}\label{psinstar1}
\lim_{n\to\infty} \|\Psi_n\|_p^p = \lim_{n\to\infty} \|(\Psi_n)_{e^*}\|_{L^p((0,\infty))}^p = \lim_{n\to\infty} \|\psi_n^*\|_{L^p((0,\infty))}^p,
\end{equation}
where we used the fact that $\lim_{n\to\infty} \|(\Psi_n)_{e^*}\|_{L^p((0,1))} = 0  $, and the trivial bound $ \|\psi_n^*\|_{L^p((0,1))} \leq  \|\chi\|_{L^\infty((0,1))} \|(\Psi_n)_{e^*}\|_{L^p((0,1))}$.  In particular, $ \lim_{n\to\infty} \|(\Psi_n)_{e^*}\|_{L^2((0,\infty))}^2 = \lim_{n\to\infty} \|\psi_n^*\|_{L^2((0,\infty))}^2 = m$. 
Moreover we have that 
 \begin{equation}\label{psinstar2}
  \lim_{n\to\infty } \frac12 \int_0^\infty |(\Psi_n)_{e^*}'|^2 dx \geq   \lim_{n\to\infty } \frac12 \int_0^\infty |{\psi_n^*}'|^2 dx. 
\end{equation}
 To prove the latter inequality, we note that 
\[\begin{aligned}
 \lim_{n\to\infty } \int_0^\infty |(\Psi_n)_{e^*}'|^2 - |{\psi_n^*}'|^2 dx 
 =&   
 \lim_{n\to\infty } \int_0^\infty |(\Psi_n)_{e^*}'|^2\left(1-\chi^{2}\right)  dx  \\ 
 & +
 \lim_{n\to\infty } \int_0^1 |(\Psi_n)_{e^*}|^2 {\chi'}^{2} + 2\chi {\chi'} \Re \overline {(\Psi_n)_{e^*}'}(\Psi_n)_{e^*} dx \\
 = &   \lim_{n\to\infty } \int_0^1 |(\Psi_n)_{e^*}'|^2\left(1- \chi^{2}\right)  dx  \geq 0 ,
\end{aligned}\]
where we used again Lem. \ref{l:cc} - Eq. \eqref{e:run-1} and the bounds $\|\chi\|_\infty, \|\chi'\|_\infty\leq c $. 

We have the following chain of inequalities/identities  
\begin{align}
&\lim_{n\to \infty} E[\Psi_n] \nonumber \\
& \geq 
\lim_{n\to\infty } \int_0^\infty |{\psi_n^*}'(x)|^2 dx -\frac{1}{\mu+1} \int_0^\infty  |\psi_n^*(x)|^{2\mu+2} dx \nonumber\\
&\text{(we used Eqs. \eqref{little}, \eqref{psinstar1} and \eqref{psinstar2})} \nonumber \\
&   \geq \inf \Big\{ \int_0^\infty |\psi ' (x)|^{2} dx -\frac{1}{\mu+1} \int_0^\infty |\psi (x)|^{2\mu+2} \, dx  \text{ s.t. } \psi\in H^1(\RE^+), \, \psi(0)=0\, ,\| \psi\|_{L^2(\RE^+)}^2 =m \Big\} \nonumber \\ 
&\text{(we used the fact that $\psi_n^*\in H^1(\RE_+)$,  $\psi_n^*(0)= 0$, and $\|\psi_n^*\|_{L^2(\RE_+)}^2 \to  m$ as $n\to\infty$)} \nonumber
\\ 
&  =  \inf \Big\{  \int_\RE |{\psi}' (x)|^{2} dx-\frac{1}{\mu+1} \int_\RE|{\psi} (x)|^{2\mu+2} \, dx   \text{ s.t. } \psi\in H^1(\RE), \, \psi(x)=0 \; \forall x\leq 0 ,\| \psi\|_{L^2(\RE)}^2 =m  \Big\}\nonumber \\
&\text{(where we used the fact that $\psi\in H^1(\RE_+)$ and $\psi(0)= 0$ if and only if its zero extension} \nonumber\\
&\text{belongs to $H^1(\RE)$, see, e.g., \cite[Th. 5.29]{AF03}})\nonumber
\\ 
& \geq \inf \left\{  \int_\RE |\psi  ' (x)|^{2} dx-\frac{1}{\mu+1} \int_\RE|\psi (x)|^{2\mu+2} \, dx  \text{ s.t. } \psi\in H^1(\RE), \, \| \psi\|_{L^2(\RE)}^2 =m  \right\} \label{infsol}\\
&\text{(we enlarged the set on which the $\inf$ is taken).} \nonumber
\end{align}

It is well known that the infimum in the latter minimization problem is indeed  attained and that the minimizing function (up to translations and phase multiplications) is given by the soliton profile 
\begin{equation*}%\label{soliton}
\phi(x) = [ (\mu + 1) \ome_{\RE}]^{\frac{1}{2\mu}} \sech^{\frac{1}{\mu}} (\mu \sqrt{\ome_{\RE}} x).
\end{equation*}
The frequency   $\omega_\RE$ is fixed by the mass constraint through the relation  
\begin{equation*}%\label{solmass}
m =\|\phi\|^2_{L^2(\RE)} =   2\f{(\mu+1)^{\f 1 \mu }  }{\mu} \ome_\RE^{ \f 1 \mu - \f 1 2} \int_0^1 (1-t^2)^{\f 1 \mu -1} dt,
\end{equation*}
which gives 
\begin{equation*}%\label{solmass2}
\omega_\RE = \left( 2\f{(\mu+1)^{\f 1 \mu }  }{\mu}  \int_0^1 (1-t^2)^{\f 1 \mu -1} dt\right)^{-\frac{2\mu}{2-\mu}} m^{\frac{2\mu}{2-\mu}}.
\end{equation*}
The infimum in the minimization problem \eqref{infsol} is given by  the nonlinear energy  of the soliton 
\begin{equation*}%\label{solen}
  	  \int_\RE |\phi'(x) |^{2} dx-\frac{1}{\mu+1} \int_\RE|\phi(x)|^{2\mu+2} \, dx  =
- \f{2-\mu}{2+\mu}  \; \ome_\erre \, m = -\gamma_\mu m^{1+\frac{2\mu}{2-\mu}}, 
\end{equation*}
with $\gamma_\mu= \f{2-\mu}{2+\mu}\left( 2\f{(\mu+1)^{\f 1 \mu }  }{\mu}  \int_0^1 (1-t^2)^{\f 1 \mu -1} dt\right)^{-\frac{2\mu}{2-\mu}}$. So that by the inequality \eqref{infsol}, we conclude that  if $\Psi_n$ is a runaway sequence  it must be 
\begin{equation}\label{lower}
\lim_{n\to \infty} E[\Psi_n] \geq  -\gamma_\mu m^{1+\frac{2\mu}{2-\mu}} .
\end{equation}


To show that for $m$ small enough a minimizing sequence cannot be runaway we  compute the energy   on a trial function.  As trial function we choose the  function $\Phi(\ome)$, with $\omega = \omega(m)$,  given in Th. \ref{t:bif}. By the same theorem we have that the energy  $ E[\Phi(\ome)] = -E_0 m + o(m)$, and by a simple continuity argument we infer that there exists $m^*$  such that $ E[\Phi(\ome)]  < -\gamma_\mu m^{1+\frac{2\mu}{2-\mu}}$ for all  $0<m<m^*$. This, together with the lower bound \eqref{lower},   imply that a minimizing sequence cannot be runaway. 

By Lem. \ref{l:cc} we conclude that for all $0<m<m^*$ there exists a state $\hat\Psi \in\EE$ such that  minimizing sequences converge, up to taking  subsequences, to $\hat\Psi $ in $L^p$ for $p  \geq 2$. In particular, $M[\hat \Psi] = m$, and the potential, vertices, and nonlinear terms in $E[\Psi_n]$ converge to the corresponding ones in $E[\hat\Psi]$. Taking into account also the weak lower continuity of the $H^1$ norm we have
\[
E[ \hat \Psi] \leq \lim_{n\to \infty} E[ \Psi_n] = -\nu
\]
which implies that $E[ \hat \Psi] = -\nu$. Since $E[ \hat \Psi] = \lim_{n\to \infty}  E[ \Psi_n]$ then $\| \hat \Psi ' \| = \lim_{n\to \infty} \|  \Psi_n ' \|$ and we
have proved that $\Phi_n \to \hat \Psi$ in $H^1$.
\end{proof}}