% !TEX root = main.tex
\newpage
\begin{lemma}
  Let $\alpha\in G\cup N$, and $u=u(\alpha,r)$.
  Then $u'(r)<0$ for all $r\in(0,z(\alpha))$ and $u'(z(\alpha))<0$ if $\alpha\in N$.

\begin{proof}
% TODO: improve introduction
The lemma shows $u(r)$ strictly decreasing on $(0,z(\a))$ for $\a\in G\cup N$. The argument uses this to conclude that $w(r)$ has a unique zero on $(0,z(\a))$. Finally, analysis of solution sets $N$ and $G$ leads to the uniqueness result.
Write $z(\alpha)=\infty$ when $\alpha\in G$, since $u(\alpha,r)\to0$ as $r\to\infty$.
%Thus $z(\alpha)=\infty$.
Let $\alpha\in G\cup N$.
By lemma \ref{lya}, $E(r)\geq0$ on $[0,z(\a)]$ and non-increasing.
% By lemma \ref{lya}, $E(r)$ is well-defined for $r\in[0,z(\alpha))$ at least and $\lim_{r\to z(\alpha)}E(r)\geq0.$
% Evaluate:
% \begin{empheq}{align*}
%   \underset{r\to z(\alpha)}{\lim}E(r)&=
%   \underset{r\to z(\alpha)}{\lim}\bigg[
%     \frac{1}{2}u'(r)^2-\frac{\lambda}{2}u(r)^2+\frac{1}{p+1}V(r)u(r)^{p+1}
%   \bigg] \\
%   &=u'(z(\alpha))^2\geq0.
% \end{empheq}
% Since $E(r)$ is non-increasing, $E(r)\geq0$ for all $r\in[0,z(\alpha)]$.

% \seperate

{\color{red} Needs argument why $u''(0)<0$. And why $u'(r)<0$ for $r$ sufficiently small.}

% The following argument shows that $u''(0)<0$. Evaluate:
% \begin{align*}
%   E(0)&=\frac{1}{2}u'(0)^2+\frac{\lambda}{2}u(0)^2+\frac{1}{p+1}V(0)u(0)^{p+1}\geq0\\
% %u''(0)&=\underbrace{\lambda\a-V(0)\a^p}_{\text{finite}}-\underset{r\downarrow0}{\lim}\frac{u'(r)}{r}
%   u''(0)&=\lambda\a-V(0)\a^p-\underset{r\downarrow0}{\lim}\frac{u'(r)}{r}=\lambda\a-V(0)\a^p-\underset{r\downarrow0}{\lim}\frac{ru'(r)}{r^2}\\
%   % u''(0)
% \end{align*}


% To further study the behaviour of $u(r)$ for $\a\in G\cup N$, consider the value of $u''(0)$.
% % TODO: Compare with $\lim_{r\to0}ru'(r)=0$.
% % Suppose $u''(0)=0$ and note $u'(0)=0$, so $u\equiv\alpha$ \Lightning.
% Suppose $u''(0)=0$ then \ref{ivp} implies:
% \begin{align*}
%     u''(0)+\underset{\lim}{r\downarrow0}\frac{}{}
% \end{align*}
% % TODO: Build argument that contradicts $E(r)$ non-increasing.
% Alternatively suppose $u''(0)>0$ then $u'(r)>0$ and $u(r)>u(0)=\alpha$ for $r>0$ and small.
% Now $\alpha\in(G\cup N)\iff -\frac{\lambda}{2}\alpha^2+\frac{1}{p+1}V(0)\alpha^{p+1}\geq0$, hence
% $$E(r)-E(0)=\frac{1}{2}\left[u'(r)^2-u'(0)^2\right]-\frac{\lambda}{2}\left[u(r)^2-u(0)^2\right]+\frac{1}{p+1}\left[V(r)u(r)^{p+1}-V(0)u(0)^{p+1}\right]\geq0.$$
% This contradicts $E(r)$ nonincreasing.
% Hence $u''(0)<0$ and $u'(r)<0$ for $r>0$ and small.

\seperate
These results can be extended to show $u'<0$ on $(0,z(\alpha))$.
Suppose by contradiction $0<r_0=\inf(0<r<z(\a),u'(r)=0)$ exists.
Note how $u''(r_0)<0\implies u'(r)>0$ somewhere on $(0,r_0)$.
{\color{red}See also figure F1}
% Because $r_0$ is the infimum over zeroes of $u'(r)$ on $(0,z(\a))$,
% this contradicts $u'(r)<0$ on $(0,r_0)$.
This contradicts $u'(r)<0$ on $(0,r_0)$.
% TODO: make the argument that u'' and u' = 0 implies constant function...
Again, the combination of $u''(0)=0$ and $u'(0)=0$ would imply $u\equiv u(r_0)$.
% TODO: Illustrate above with F1.
Hence $u''(r_0)>0$. Invoke \ref{ivp}:
%Note $u'(r_0)=0$ and
\begin{align*}
  &u''(r_0)=\lambda u(r_0)-V(r_0)u(r_0)^p>0\\
  \implies &u(r_0)<\left[\frac{\lambda}{V(r_0)}\right]^{1/(p-1)}
    <\left[\left(\frac{p+1}{2}\right)\frac{\lambda}{V(r_0)}\right]^{1/(p-1)}\\
  \iff &u(r_0)^{p-1}<\left(\frac{p+1}{2}\right)\frac{\lambda}{V(r_0)}\\
  \iff &\frac{1}{p+1}V(r_0)u(r_0)^{p+1}<\frac{\lambda}{2}u(r_0)^2\\
  \iff &-\frac{\lambda}{2}u(r_0)^2+\frac{1}{p+1}V(r_0)u(r_0)^{p+1}<0
  % \implies E(r_0)=-\frac{\lambda}{2}u(r_0)^2+\frac{1}{p+1}V(r_0)u(r_0)^{p+1}<0,
\end{align*}
Then using $u'(r_0)=0$, this yields $E(r_0)<0$:
$$ E(r_0)=-\frac{\lambda}{2}u(r_0)^2+\frac{1}{p+1}V(r_0)u(r_0)^{p+1}<0 $$
%\begin{empheq}{align*} \eqref{ivp}:\quad&u''(r)+\frac{N-1}{r}u'(r)+g(u(r))=0 \\ r=r_0, u'(r_0)=0:\quad&u''(r_0)+g(u(r))=0 \\ &-g(u(r_0))=u''(r_0) \\ &\lambda u(r_0)-V(r_0)u(r_0)^p=u''(r_0)\geq 0 \\ &\lambda u(r_0)-V(r_0)u(r_0)^p\geq0\quad[[]]\\ \implies &u(r_0)\leq\Bigg[\frac{\lambda}{V(r_0)}\Big]^{1/(p-1)}<\Bigg[\Bigg(\frac{p+1}{2}\Bigg)\frac{\lambda}{V(r_0)}\Big]^{1/(p-1)}\Bigg]\end{empheq}
But $E(r_0)<0$ contradicts $E(r)\geq0$, so $u'<0$ on $(0,z(\alpha))$.

\seperate

It remains to show $u'(z(\alpha))<0$ whenever $\alpha\in N$.
Suppose $u'(z(\alpha))=0$ and remember $u(z(\a))=0$.
Then $u\equiv0$, because $u''(z(\a))=\lambda u(z(\a))-V(z(\a))u(z(\a))^p=0$.
\underline{Conclusion:} $u'(z(\alpha))<0$.
%The following claim contradicts $\alpha\in(G\cup N)$. \underline{Claim:} $E(r_0)<0$. To see this, evaluate $E(r)$ in $r_0$ as follows: \begin{empheq}{align*} E(r_0)=~&\frac{1}{2}u'(r_0)^2-\frac{\lambda}{2}u(r_0)^2+\frac{1}{p+1}V(r_0)u(r_0)^{p+1} \\ =&-\frac{\lambda}{2}u(r_0)^2+\frac{1}{p+1}V(r_0)u(r_0)^{p+1} \end{empheq}

%\underline{$u'(z(\alpha))<0$} Now if $\alpha\in N$, note that by the above $u'(r)<0$ on $(0,z(\alpha))$. If $u'(z(\alpha))=0$ then since $u(z(\alpha))=0$, the solution would be trivial, $u\equiv0$. Hence $u'(z(\alpha))<0$ and the proof is complete.
\end{proof}
\end{lemma}
