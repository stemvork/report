\newpage
\begin{lemma}Let $\alpha\in G\cup N$, and $u=u(\alpha,r)$. Then $u'(r)<0$ for all $r\in(0,z(\alpha))$ and $u'(z(\alpha))<0$ if $\alpha\in N$.
\begin{proof} 
Write $z(\alpha)=\infty$ when $\alpha\in G$, since: $u(\alpha,r)\to0$ as $r\to\infty$. %Thus $z(\alpha)=\infty$. 
Let $\alpha\in G\cup N$. By lemma \ref{lya} then, $E(r)$ is well-defined for $r\in[0,z(\alpha))$ at least and $\lim_{r\to z(\alpha)}E(r)\geq0.$ Evaluate: \begin{empheq}{align*}\underset{r\to z(\alpha)}{\lim}E(r)&=\underset{r\to z(\alpha)}{\lim}\bigg[\frac{1}{2}u'(r)^2-\frac{\lambda}{2}u(r)^2+\frac{1}{p+1}V(r)u(r)^{p+1}\bigg] \\ &=u'(z(\alpha))^2\geq0.\end{empheq} 
Since $E(r)$ is non-increasing, $E(r)\geq0$ for all $r\in[0,z(\alpha)]$.

Suppose $u''(0)=0$ and note $u'(0)=0$, so $u\equiv\alpha$ \Lightning. Suppose $u''(0)>0$ then $u'(r)>0$ and $u(r)>u(0)=\alpha$ for $r>0$ and small. Now $\alpha\in(G\cup N)\iff -\frac{\lambda}{2}\alpha^2+\frac{1}{p+1}V(0)\alpha^{p+1}\geq0$, hence $$E(r)-E(0)=\frac{1}{2}\left[u'(r)^2-u'(0)^2\right]-\frac{\lambda}{2}\left[u(r)^2-u(0)^2\right]+\frac{1}{p+1}\left[V(r)u(r)^{p+1}-V(0)u(0)^{p+1}\right]\geq0.$$ This contradicts $E(r)$ nonincreasing. Hence $u''(0)<0$ and $u'(r)<0$ for $r>0$ and small.

Claim: $u'<0$ on $(0,z(\alpha))$. Suppose by contradiction $r_0=\inf(r>0,u'(r)=0)<z(\alpha)$. Note how $u''(r_0)<0\implies u'(r)>0$ on $(0,r_0)$ \Lightning. Hence $u''(r_0)\geq0$. Invoke \ref{ivp}:
%Note $u'(r_0)=0$ and 
\begin{gather*}u''(r_0)=\lambda u(r_0)-V(r_0)u(r_0)^p\geq0\\u(r_0)\leq\left[\frac{\lambda}{V(r_0)}\right]^{1/(p-1)}<\left[\left(\frac{p+1}{2}\right)\frac{\lambda}{V(r_0)}\right]^{1/(p-1)}\\
\implies-\frac{\lambda}{2}u(r_0)^2+\frac{1}{p+1}V(r_0)u(r_0)^{p+1}<0,\end{gather*}Then using $u'(r_0)=0$, evaluate $E(r_0)$: $$E(r_0)=-\frac{\lambda}{2}u(r_0)^2+\frac{1}{p+1}V(r_0)u(r_0)^{p+1}<0.$$ %\begin{empheq}{align*} \eqref{ivp}:\quad&u''(r)+\frac{N-1}{r}u'(r)+g(u(r))=0 \\ r=r_0, u'(r_0)=0:\quad&u''(r_0)+g(u(r))=0 \\ &-g(u(r_0))=u''(r_0) \\ &\lambda u(r_0)-V(r_0)u(r_0)^p=u''(r_0)\geq 0 \\ &\lambda u(r_0)-V(r_0)u(r_0)^p\geq0\quad[[]]\\ \implies &u(r_0)\leq\Bigg[\frac{\lambda}{V(r_0)}\Big]^{1/(p-1)}<\Bigg[\Bigg(\frac{p+1}{2}\Bigg)\frac{\lambda}{V(r_0)}\Big]^{1/(p-1)}\Bigg]\end{empheq}
But $E(r_0)<0$ contradicts $E(r)\geq0$, so $u'<0$ on $(0,z(\alpha))$.

It remains to show $u'(z(\alpha))<0$ whenever $\alpha\in N$. Suppose $u'(z(\alpha))=0$. Then $u\equiv0$, since $u(z(\alpha))=0$. Hence $u'(z(\alpha))<0$.
%The following claim contradicts $\alpha\in(G\cup N)$. \underline{Claim:} $E(r_0)<0$. To see this, evaluate $E(r)$ in $r_0$ as follows: \begin{empheq}{align*} E(r_0)=~&\frac{1}{2}u'(r_0)^2-\frac{\lambda}{2}u(r_0)^2+\frac{1}{p+1}V(r_0)u(r_0)^{p+1} \\ =&-\frac{\lambda}{2}u(r_0)^2+\frac{1}{p+1}V(r_0)u(r_0)^{p+1} \end{empheq}

%\underline{$u'(z(\alpha))<0$} Now if $\alpha\in N$, note that by the above $u'(r)<0$ on $(0,z(\alpha))$. If $u'(z(\alpha))=0$ then since $u(z(\alpha))=0$, the solution would be trivial, $u\equiv0$. Hence $u'(z(\alpha))<0$ and the proof is complete.
\end{proof}
\end{lemma}