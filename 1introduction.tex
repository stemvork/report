\section{Introduction}
\begin{markdown}
- PDEs and engineering (refer to Haberman?)
    - partial differential equations and ordinary differential equations
    - diverse examples of equations solved in engineering context
    - review of types of pde and ode
        - homogenous or inhomogenous
        - linear or nonlinear
        - initial values and boundary values
        - conversion between bv problems and iv problems
        - TODO: is our problem an iv formulation of bv problem?
    - discussion of nls (refer to Fibich)

- Analytical and numerical
    - to determine solutions, often numerical methods
    - some problems have analytical solutions
    - existence and uniqueness are analytical problems
    - analysis of problem as part of solving problem

- Optics and physics
    - principles of optics
        - maxwell equations for em waves
        - light regarded as em wave
        - geometric optics and diffraction optics
        - components of em field and coupling
        - helmholtz: decoupled components of em field
        - helmholtz solving and history
        - principles of bridge to nls
    - derivation of nls from first principles
        - TODO: sketch steps of derivation
    - relation to self-focusing research (history)
    - physical interpretation of our problem

- Solutions and mathematics
    - TODO: what about solutions and existence and mathematics

- Existence and uniqueness
    - definitions of existence and uniqueness
        - existence: there is at least one solution to the problem, possibly infinite
        - uniqueness: there is at most one solution to the problem, if it exists
    - methods of proving existence and uniqueness
    - e. and u. and well-posedness

- Ground states
    - TODO: look into eigenvalues and eigenfunctions, ground state is first eigenfunction, or solution corresponding to first eigenvalue of the operator applied to the problem (iv bv problem)
\end{markdown}

% ...  This study comprised three parts regarding the existence and uniqueness of ground state solutions to the Nonlinear Schr\"odinger equation (NLS). The report should clarify these results on the level of an interested student with a bachelor in physics or mathematics. {\color{lightgray}In the remainder of this introduction, these results are named and an overview of the report is given. They are named in chronological order of when they were published, which coincides with the order in which they were reviewed for this report.}

%Firstly, the publication of Berestycki, Lions and Peletier [ref] on the existence of ground state solutions to the problem (IVP). In this paper an existence theorem is stated and proved by a shooting argument. It is split into two main lemmata.

%Secondly, the publication of Genoud [ref] (and Kwong [ref]) on the uniqueness of ground state solutions to the same problem. To prove uniqueness, the set of ground state solutions needs to be a singleton (set with one element). This proof is by eight lemmata.

% Lastly, the book on NLS by Fibich discusses both the mathematics and the physics of the derivation and setting of the problem as it arises in nonlinear optics.

% This report discusses the above in a different order. First, an introduction (Chapter \ref{phy}) to the physics of laser beams and the NLS. Then, in Chapter \ref{not} the differential equation is stated and the notation of initial conditions, solutions and solution sets is clarified. These were different among the two papers and the book. For this report, the same notation will be used in all chapters. In Chapter \ref{exi} the existence theorem is stated and a proof is given. The uniqueness theorem is stated in Chapter \ref{uni} and proven through various lemmata. Lastly, Chapter \ref{con} ...
