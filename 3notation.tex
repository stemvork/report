\section{Solution sets and notation}\label{not}
The previous chapter leads to the following initial value problem (IVP). \underline{What is $u$?} Another important identity is achieved by integrating the IVP over $r$. \underline{The solution sets are disjoint.}
\begin{gather*}-u''(r)-\frac{N-1}{r}u'(r)=g(u(r))\tag{IVP}\label{ivp}\\ u(0)=\alpha,~u'(0)=0\end{gather*}
\begin{equation}\label{ivpint}
\end{equation}

The possible initial values $u(0)=\alpha>0$ categorise the solutions in three solution sets: solutions that become negative ($N$), solutions that are positive ($P$) for all $r$ and solutions that vanish ($G$, for ground state). In terms of zeroes, solutions with initial condition in $N$ have at least one zero, solutions with initial condition in $P$ have no zeroes and solutions with initial condition in $G$ tend to zero for $r$ to infinity. As these solution sets describe all possible behaviour, the positive $r$ axis is the disjoint union of $P$, $G$ and $N$. Lastly, let $z(\alpha)$ describe the smallest zero of the solution. That is, the supremum of $s>0$ such that the solution is positive for $r\in[0,s)$. Formally:
\setlength{\jot}{1em}
\\\begin{gather*}
  N \coloneqq \text{\{$\alpha>0$: there exists $r>0$ such that $u(r;\alpha)=0$\}} \\
  G \coloneqq \text{\{$\alpha>0$: $u(r;\alpha)>0$ for all $r>0$ and $\underset{r\to\infty}{\lim}u(r;\alpha)$\}} \\
  P \coloneqq \text{\{$\alpha>0$: $u(r;\alpha)>0$ for all $r>0$ and $\alpha\notin G$\}}\\
  (0,\infty) = P\cup G\cup N\\
  z(\alpha) \coloneqq \text{$\sup\{s>0$: $u(r;\alpha)>0$ for all $r\in[0,s)$\}}
\end{gather*}
