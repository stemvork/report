\newcommand{\ns}{\enspace}
\newcommand*\diff{\mathop{}\!\mathrm{d}}
\newcommand*\R{\mathbb{R}}
\newcommand{\cupdot}{\mathbin{\dot{\cup}}}
%\newcommand*\limrtoinf{\underset{r\to\infty}{\lim}}

\chapter{Existence of ground state} 
\label{existence} 

\section{Initial value problem and nonlinearity} 
In this chapter, we will study an existence proof for the
initial value problem
\be \label{ivp} 
-u''(r) - \frac{n-1}{r} u'(r) = f(u(r)),\quad\text{on}\; 0<r<\infty,
\ee
satisfying initial conditions and an integrability condition
\be \label{ivpcond} \begin{dcases}
~u(0) = \a,\\
~u'(0) = 0\\
~\limrtoinf u(r) = 0.
\end{dcases} \ee
%
The existence proof will be based on \cite{ber81}, which generalises earlier
results. One of these is the uniqueness result \cite{coffm}, which was
later generalised in \cite{kwong}, which forms the basis for the next chapter. 

The proof will be by a shooting method, where we categorise the solutions based
on their asymptotic behaviour. Furthermore, solutions to the initial value
problem \cref{ivp} are also positive radial solutions to the more general
problem
\be \label{pde} 
-\Delta u = f(u)\quad\text{in}\;\mathbb{R}^n,
\ee
where $f(u)$ is a given nonlinear function. This partial differential
equation is relevant to many areas of mathematical physics.

The solutions $R(r)$ to \cref{sol} are solutions $u(r)$ to \eqref{ivp} with
$n=2$ and 
\[ f(u) = -u + u^3. \]
%
\section{Definitions of solution sets} 
A \textbf{ground state solution} is strictly decreasing everywhere and has no
finite zeroes. Yet, the solution should vanish in the limit as $r\to\infty$.

We define the set $G$ of ground state initial conditions as 
\be \label{gset} 
G\coloneqq\left\{\;\a>0 \;\middle|\; u(r,\a)>0 \;\text{and}\;
u'(r,\a)<0\; \text{for all}\; r>0\; \text{and}\;
\limrtoinf u(r,\a)=0\;\right\}.  
\ee
We consider two alternatives: either (i) the derivative vanishes, or (ii) the
solution vanishes. We define the set $P$ of initial conditions with a vanishing
derivative as
\be \label{pset}
P\coloneqq\Big\{\;\a>0 \;\Big|\; \exists r_0: 
% u'(r_0,\a)=0 \;\text{and}\; u(r,\a)>0 \;\text{for}\; r\in(0,r_0] \;\Big\}.
u'(r_0,\a)=0 \;\text{and}\; u(r,\a)>0 \;\text{for all}\; r\leq r_0 \;\Big\}.
\ee

We define the set $N$ of initial conditions with a vanishing solution as
\be \label{nset}
N\coloneqq\Big\{\;\a>0 \;\Big|\; \exists r_0: 
% u(r_0,\a)=0 \;\text{and}\; u'(r,\a)<0 \;\text{for}\; r\in(0,r_0] \;\Big\}.
u(r_0,\a)=0 \;\text{and}\; u'(r,\a)<0 \;\text{for all}\; r\leq r_0 \;\Big\}.
\ee

\revgroup These solution sets are disjoint by definition and we write the union of initial
conditions as $I=P\cupdot G\cupdot N$.\endgroup

% \# As noted in \cite{ber81}, for $u(r,\a)\leq 0$ the nonlinearity $f(t)$ is
% extended as
% $$f(t)=0,\quad\text{for}\; t\leq0.$$ 

\section{Assumptions on $f$}
We assume that $f$ is locally Lipschitz continuous from $\R_+\to\R$ and
satisfies $f(0)=0$. Local Lipschitz continuity is an important condition for the
Picard-Lindel\"of local existence and uniqueness theorem. Additionally, we
assume that hypotheses (H1)--(H5) are satisfied. Firstly,
\be \label{h1} f(\kappa) = 0,\text{ for some }\kappa>0.\tag{H1}\ee
Secondly, defining $F(t)$ as the integral of $f(t)$
\be \label{bigg} F(t) \coloneqq \int_0^t f(s)\diff s, \ee
there exists an initial condition $\a>0$ such that $F(\a)>0$. We define
\be \label{h2} 
\a_0 \coloneqq \inf\left\{\a > 0\;\middle|\; F(\a)>0\;\right\}.
\tag{H2} \ee
Thirdly, the right-derivative of $f(s)$ at $\kappa$ is positive
\be \label{h3} f'(\kappa^+) = 
\underset{s\downarrow\kappa}{\lim}\;\frac{f(s)-f(\kappa)}{s-\kappa} > 0,\tag{H3}
\ee
and fourthly, we have 
\be \label{h4} f(s) > 0\quad\text{for}~s\in\left(\kappa, \a_0\right].\tag{H4} \ee
We define
\be \lambda \coloneqq \inf\left\{\;\a>\a_0 \;\middle|\; f(\a) = 0 \;\right\}, \ee
and note that $\a_0<\lambda\leq \infty$. In the situation where $\lambda =
\infty$, we assume
%\be f(t)>0\quad\text{for}~\a>\kappa, \ee
%then we make the final assumption
\be \underset{s\to\infty}{\lim} \frac{f(s)}{s^l} = 0,\quad\text{with}~
l<\frac{n+2}{n-2}. \tag{H5} \ee

\section{Main theorem}
\begin{theorem} Let $f\text{ be a locally Lipschitz continuous function on
}\mathbb{R}_+=[0,\infty)$ such that $f(0)=0$ and $f$ satisfies
hypothese $(H1)-(H5)$. Then there exists a number $\a\in(\a_0,\lambda)$ such
that the solution $u(r,\a)\in C^2(\mathbb{R}_+)$ of the initial value problem
\eqref{ivp} has 
% the properties positive and decreasing 
% for $r\in(0, \infty)$ and limit
\be \label{lemlimcas}
\begin{dcases}
u(r,\a)>0\quad\text{for all}\;r\geq0,\\%\quad\text{and}\\
u'(r,\a)<0\quad\text{for all}\;r>0,
\end{dcases}
\ee
and \[ \underset{r\to\infty}{\lim} u(r) = 0. \]
Then $\a\in G$. Hence $G$ is non-empty.

% r to infty vanish \todogroup math \endgroup 
If in addition, we assume that $f$ satisfies ?? then there exists constants
such that etc... 
\end{theorem}


\section{Interval of definition}\label{semidef}
Existence of local unique solutions is guaranteed by the Picard-Lindel\"of
theorem, see for example \cite[Theorem.~2.2]{teschl}. 
% \todogroup about local existence, local unique existence, lipschitz condition,
% maximal intervals and global solutions \endgroup

In these circumstances, boundedness of the solution $u(r,\a)$ is a sufficient
condition for the solution to be defined on the maximal interval $[0, \infty)$.
This is also called the \emph{blow-up alternative}. Either (i) for some
$r_0 > 0$ we have 
\[ \left|u(r_0,\alpha)\right|>M,\quad\text{ for all }M>0, \] 
and the solution is defined on $[0,r_0)$. Or (ii) for some $M>0$ we have
\[ \left|u(r,\a)\right|\leq M,\quad\text{for all}~r\geq 0, \]
and the solution is defined for all $r\geq0$.


In this section, we will derive an upper and a lower bound for $u(r,\a)$. Since
the solution is initially decreasing, possibly the initial condition $\a$ is an
upper bound.
\begin{lemma}
$u(r,\a)\leq u(0,\a)=\a$\quad\text{for }$r\geq0$.
\end{lemma}
\begin{proof}
    \todogroup Seperate into lemma about the quantity \eqref{gint}.
In this proof, we write $u(r)=u(r,\a)$ for brevity. We start with \eqref{ivp}
and multiply by $u'(r)$. Then we integrate from $0$ to $r$ to obtain
\be \label{gintstart}
-\int_0^r\left[u'(s)u''(s)\right]ds
-\int_0^r\left[\frac{n-1}{s}[u'(s)]^2\right]\diff s=
\int_0^r\left[u'(s)f(u(s))\right]\diff s.
\ee

We use the chain rule simplify the first term in \eqref{gintstart} and obtain
\[ \od{}{r}[u'(r)^2]=2u'(r)u''(r)\stackrel{\eqref{ivpcond}}{\iff}
   \half[u'(r)]^2 = \int_0^r\left[u'(s)u''(s)\right]\diff s. \]

Then, we rewrite the right-hand side of \eqref{gintstart} using the fundamental
theorem of calculus
\[ \int_0^r\left[u'(s)f(u(s))\right]\diff s=
   \int_0^r\left[\od{u}{s}f(u(s))\right]\diff s= 
   \int_{u(0)}^{u(r)} f(u)\diff u=
   F(u(r))-F(u(0)). \]

Finally, using $u(0)=\a$, we have rewritten \eqref{gintstart} as
\be \label{gint}
-\half\left[u'(r)\right]^2
-(n-1)\int_0^r\left[u'(s)\right]^2\frac{\diff s}{s}=
F(u(r))-F(\alpha).
\ee
\endgroup

%This expression will be useful in the following proof by contradiction. 
We suppose by contradiction that 
\be \label{ubas} u(r_0)>\a,\quad\text{for some}\;r_0>0.\ee

\todogroup Be more specific about: in which quantity, with which assumption, do we have which result.
By the assumptions on $f(u)$, we have $f(u)>0$ on $(\a_0,\infty)$. As a result,
$F(u)$ is increasing on $(\a_0,\infty)$. Using assumption \eqref{ubas} and
$\a>\kappa$, we deduce that
\[ F(u(r_0))>F(\a) 
\iff F(u(r_0))-F(\a)>0. \]
%\[ F(u(r_0,\alpha))-F(u(0,\alpha))=F(u(r_0,\alpha))-F(\alpha)>0 \Lightning\]
This contradicts \eqref{gint}, as the left-hand side is clearly non-positive. 
\end{proof}\endgroup

% The upper bound on $u(r,\a)$ is
% \[ u(r,\alpha)\leq\alpha\;\text{for}\;r\geq0. \]
% 
% \subsection{Lower bound} 
We will show that $u(r,\a)$ has a lower bound for $r<\infty$. Let $r_0$ be the first
zero of $u(r,\a)$
\be \label{r0inf}
r_0\coloneqq\inf\left\{\;r>0 \;\middle|\; u(r,\a)=0\;\right\}.
\ee
If $r_0=\infty$, then we have $u(r,\a)>0$ for all $r>0$. Suppose to the contrary 
that $r_0<\infty$, then we have the following bound on the derivative $u'(r,\a)$.

\begin{lemma} \label{thm:dechyp}
Suppose that $r_0<\infty$. Then for $r\geq r_0$, we have
\be \label{dechyp}
u'(r,\a)=\left(\frac{r_0}{r}\right)^{n-1}u'(r_0,\a)\geq u'(r_0,\a).
\ee
\end{lemma}
\begin{proof}
We consider the sign of $u'(r_0, \a)$. Firstly, if $u'(r_0,\a)=0$ then $u$
and $u'$ vanish simultaneously in $r_0.$ Then from \eqref{ivp} we have
\[ u'' = 0,\quad\text{with }u(r_0)=u'(r_0)=0, \]
which is solved by
\[ u(r) = c_1 r + c_2, \] %\quad\text{and }u'(r) = c_1,   \]
where we must have $c_1 = c_2 = 0$ to satisfy the conditions at $r_0$, so that
$u\equiv 0$. But this contradicts $u(0, \a)=\a>0$. Hence, $u$ and $u'$ cannot
vanish simultaneously for $\a>0$. 

Secondly, if $u'(r_0,\a)>0$ we also reach a contradiction. By \eqref{ivp} with
$u(r_0,\a)=0$,
    \[ u''(r_0, \a) + \frac{n-1}{r}u'(r_0, \a) = -f(0) = 0 \]

we see that $u''$ and $u'$ have opposite signs in $r_0$. Then either:
\be \begin{dcases}
\text{(i)}~u''>0\quad\text{and}\quad u'<0\quad\text{in}~r_0,\quad\text{or}\\
\text{(ii)}~u''<0\quad\text{and}\quad u'>0\quad\text{in}~r_0. 
\end{dcases} \ee
The latter case implies that $u(r,\a)<0$ in a left neighborhood of $r_0$, which
contradicts $u(r,\a)>0$ on $[0, r_0)$. Thus, we have $u'(r_0,\a)<0$.
% The latter case implies that $u(r,\a)<0$ for $r\in (r_0-\epsilon, r_0)$
% with $\epsilon>0$ sufficiently small. 
% Thus $u'(r_0,\a)<0$. % For the allowed initial conditions are $I=(0,\infty)$ or a
%restriction thereof.
% \begin{comment}
% Together with \eqref{ivp}. 
% Since the solution is decreasing, suppose for some $r_0>0$ to
% have $u(r_0,\a)=0$. Then by condition A1, for $r>r_0$ near $r_0$ the solution is
% negative and the nonlinearity is extended with $f(u(r,\a))=0$. 
% \end{comment}
% \todogroup MOVE When $\a>\kappa$, the solution $u(r,\a)$ will initially be
% decreasing. \endgroup
% 
% To see this, note that the IVP \eqref{ivp} reads (Initially, the solution will be decreasing. More precisely, for $r>0$ small
% enough, we have $u'(r,\a)<0$. \#) 
%whenever $u(r,\a)\leq0$, say on $[r_0,r_1)$ for some $r_1>0$. 

In the following, we extend $f(u)=0$ for $u\leq0$. Then for $u(r,\a)\leq0$ the
IVP \eqref{ivp} reads 
\be \label{ivpuneg} 
-u''(r,\a)-\frac{n-1}{r}u'(r,\a)=0,
\ee
% zero on the right hand side follows from extending $f(u)$ with $f(u)=0$ for $u\leq 0$. 

We solve \eqref{ivpuneg} for $u'=u'(r,\a)$ and seperate the variables, resulting
in
\[ \frac{\diff u'}{u'}=-\frac{n-1}{r}\diff r. \]

We integrate the expression from $r_0$ to $r$ and evaluate the limits to get
\[ \left.\ln{u'}\right\rvert_{r_0}^r=\left[(n-1)\ln{r}\right]_r^{r_0}\iff
  \ln{u'(r)}-\ln{u'(r_0)}=(n-1)\left[\ln{r_0}-\ln{r}\right]. \]

Then, we rewrite the expression to arrive at the desired result
%\be \label{updec} 
\[
\frac{u'(r)}{u'(r_0)}=\left(\frac{r_0}{r}\right)^{n-1}\iff
  u'(r,\a)=\left(\frac{r_0}{r}\right)^{n-1}u'(r_0,\a)\geq u'(r_0,\a). 
  \qedhere
\]
\end{proof}
%\ee

In conclusion, the solution $u(r,\a)$ is bounded for bounded $r$. More
specifically, in the case of everywhere positive solutions, we have
$$ 0<u(r,\a)\leq\a\quad\text{for all}~r>0.$$

\todogroup Explicit expression for $u(r,\a)$.\endgroup
Alternatively, for solutions with $u(r_0,\a)=0$ for some $r_0>0$, by
Lemma \ref{thm:dechyp}
we have 
%$u'(r,\a)$ bounded for $r$ bounded by \eqref{dechyp}. Thus, we have
\be \label{dechypsol} u(r,\a) \geq 
%\;\stackrel{u(r_0,\a)=0}{\geq}~
\int_{r_0}^r 
% u'(s,\a)ds\geq
\left(\frac{r_0}{s}\right)^{n-1} u'(r_0,\a)\diff s >
-\infty\quad\text{for}~r_0<r<\infty.
\ee

\section{Asymptotics of positive decreasing solutions}
We will show that everywhere positive decreasing solutions $u(r,\a)$ vanish in
the limit as $r\to\infty$. The proof will be in \todogroup three steps: (i) we show that
the nonlinearity $f(u)\to 0$ in the limit as $r\to\infty$, (ii) we show
via a translation $\nu(r)=u(r)-\kappa$ that $l=\kappa$ does not satisfy the IVP,
such that $l=0$.\endgroup

\begin{lemma}\label{lemlim}
Let $f:\R^+\to\R$ be a locally Lipschitz continuous function such that $f(0)=0$.
Let $u(r,\a_1)$ be a solution to initial value problem \eqref{ivp} 
with $\a_1\in(0,\infty)$ such that 
\be \label{lemlimcas}
\begin{dcases}
u(r,\a_1)>0\quad\text{for all}\;r\geq0,\quad\text{and}\\
u'(r,\a_1)<0\quad\text{for all}\;r>0.
\end{dcases}
\ee
Then the number $l \coloneqq \underset{r\to\infty}{\lim}u(r,\a_1)$ satisfies
$f(l)=0$.

If additionally $f(u)$ satisfies \eqref{h3}, then $l=0$.

\end{lemma}
\begin{proof}[Proof step 1] 
By assumption \eqref{lemlimcas} on $u(r,\a_1)$ and the monotone convergence
theorem, we have $0\leq l<\a_1$. Then $f(l)<f(\a_1)$. We consider the
limit as $r\to\infty$ of the IVP \eqref{ivp}
\be \label{ivplim}
\limrtoinf\left[-u''(r,\a_1)-\frac{n-1}{r}u'(r,\a_1)\right]=f(l)<\infty.
\ee


We restate \cref{gint}
\[
\half\left[u'(r,\a_1)\right]^2
+(n-1)\int_0^r\left[u'(s,\a_1)\right]^2\frac{\diff s}{s}=
F(\alpha_1)-F(u(r,\a_1)),
\]
and consider the limit as $r\to\infty$
\[ 
\underset{r\to\infty}{\lim} \left[\half\left[u'(r,\a_1)\right]^2
+(n-1)\int_0^r\left[u'(s,\a_1)\right]^2\frac{\diff s}{s}\right]
= F(\a_1) - \underset{r\to\infty}{\lim} F(u(r,\a_1)), 
\]

where we use $\underset{r\to\infty}{\lim} F(u(r,\a_1)) = F(l)<\infty$ to write
\be \label{gintlim} 
\underset{r\to\infty}{\lim}~\half[u'(r,\a_1)]^2
%(n-1)\int_0^\infty[u'(\a_1,s)]^2\frac{ds}{s}
+(n-1)\int_0^\infty u'(s,\a_1)^2 \frac{\diff s}{s}
=F(\a_1)-F(l).
\ee

Note that 
\[ 
F(\a_1)-F(l) < \infty \implies 
\int_0^\infty u'(s,\a_1)^2 \frac{\diff s}{s}<\infty,
\]
such that $u'(r,\a_1)^2/r$ converges as $r\to\infty$ by the Levi monotone
convergence theorem. Then $u'(r,\a_1)^2$ converges, because... and since
$u'(r,\a_1)<0$ everywhere, we deduce that $u'(r,\a_1)$ converges. However, since
$0\leq u(r,\a_1)\leq \a_1$, we have
\be \label{udevlim}
\underset{r\to\infty}{\lim}~u'(r,\a_1) = 0.
\ee


% We note that the right-hand side of \eqref{gintlim} is finite.  Then both terms
% of the left-hand side should be finite too.  This can only be true when
% $u'(r,\a_1)$ converges. Remember that $u(r,\a_1)$ is bounded, so if the
% derivative converges, it must converge to zero
% $$\underset{r\to\infty}{\lim}u'(r,\a_1)=0.$$

Now, we return to \cref{ivplim} and use $\limrtoinf u'(r,\a_1)=0$ to obtain
%\begin{align*}
%\limrtoinf\left[u''(r,\a_1)+\frac{N-1}{r}u'(r,\a_1)\right]&=-f(l)\\
%
% \[ -\limrtoinf\left[u''(r,\a_1)\right]
%  -\limrtoinf\left[\frac{N-1}{r}u'(r,\a_1)\right]=f(l), \]
% 
% where we can use the acquired information, 
% $\limrtoinf u'(r,\a_1)=0$ to obtain
\[ -\limrtoinf\left[u''(r,\a_1)\right]=f(l).\]
% -\limrtoinf\left[u''(r,\a_1)\right]
%  -\limrtoinf\left[\frac{N-1}{r}u'(r,\a_1)\right]&=\limrtoinf f(u(r,\a_1))\\
%
% -\limrtoinf\left[u''(r,\a_1)\right]&=f(l) 
% \end{align*} 

% However, we know that $u'$ is bounded and even has to converge to 0. 
%However, we found that $u'(r,\a_1)$ needs to be bounded and even converge to 0. 
We have \eqref{udevlim} and hence, we have 
% Suppose that $u''$ is not bounded, then neither is $u'$. 
% Hence, $u''(r,\a_1)$ needs to bounded and converge to 0. 
$$\limrtoinf u''(r,\a_1)=0.$$ 
%$u''$ must converge as $r$ tends to infinity as well, 
%since $u$ is bounded. Suppose $u''$ is not bounded, 
%then neither is $u'$ and neither is $u$. 
%Since $u'$ converges to $0$, so must $u''$. 

The desired result follows: $f(l)=0$.
\end{proof}

\begin{proof}[Proof step 2.]
The nonlinearity $f(u)$ has more than one zero.  Both $f(0)=0$ and
$f(\kappa)=0$. We will show that only $l=0$ satisfies the assumptions.

Suppose to the contrary that $l=\kappa$. We will use the substitution
\[ \nu(r)=r^{(1/2)(n-1)}\left[u(r,\a_1)-\kappa\right] \] 
in \cref{ivp} to obtain a differential equation in $\nu(r)$. In the remainder of
the proof, we will abbreviate $u(r,\a_1) = u(r)$. We note that
$\nu(r)>0$ by definition, as the assumption is that 
$u(r)>\kappa$ for $r>0$ and $u(r)\downarrow\kappa$.

We proceed to calculate the first derivative
%\begin{align*}\nu(r)&=r^{(N-1)/2}\left[u(r)-\kappa\right] \\
\[ \nu'(r)=\frac{1}{2}(n-1)r^{(n-3)/2}\left[u(r)-\kappa\right]+
r^{(n-1)/2}u'(r),\]

and the second derivative, where we gather the terms by $u(r), u'(r)$ and
$u''(r)$ as
\be \label{nupp}
\nu''(r) = \frac{1}{4}(n-1)(n-3)r^{(n-5)/2}\left[u(r)-\kappa\right]
+ (n-1)r^{(n-3)/2}u'(r)+r^{(n-1)/2}u''(r).
\ee

We multiply the IVP \eqref{ivp} by $r^{(n-1)/2}$ to obtain
\be \label{ivprn} 
-r^{(n-1)/2}u''(r)-(n-1)r^{(n-1)/2}r^{-1}u'(r) = f(u(r))r^{(n-1)/2}.
\ee

We can use this to simplify \eqref{nupp}
\[
\nu''(r) = \frac{1}{4}(n-1)(n-3)r^{(n-1)/2}r^{-2}\left[u(r)-\kappa\right]
-f(u(r))r^{(n-1)/2}.
\]

Now we factor out $\nu(r)=r^{(n-1)/2}\left[u(r)-\kappa\right]$ to obtain
\[
\nu''(r) = r^{(n-1)/2}\left[u(r)-\kappa\right]
\left\{ \frac{1}{4}(n-1)(n-3)r^{-2}-\frac{f(u)}{u(r)-\kappa} \right\}.
\]

Lastly, we multiply by $-1$ to obtain the exact expression from \cite{ber81} as
%\[ \nu''(r) = \nu\left\{\frac{(N-1)(N-3)}{4r^2}
%-\frac{f(u)}{u(r)-\kappa}\right\}. \]
\be \label{nuivp}
-\nu''(r) = \left\{ \frac{f(u)}{u(r)-\kappa}
-\frac{(n-1)(n-3)}{4r^2} \right\}\nu.
\ee

We can show that there exist $\omega>0$ and $R_1>0$, such that
\be \label{omrlemeq} 
\frac{f(u)}{u(r)-\kappa}
-\frac{(n-1)(n-3)}{4r^2}\geq\omega
\quad\text{for all}~r\geq R_1.
\ee
This will be done in proof step 3. We will first show how this leads to
 $l=0$ to conclude proof step 2.

We have $\nu''(r)<0$ for $r\geq R_1$, which implies that
\[ \nu'(r)\downarrow L\geq-\infty,\quad\text{as}~r\to\infty. \]

Suppose that $L<0$, then $\nu(r)\to\-\infty$ as $r\to\infty$, which contradicts
$\nu(r)>0$. On the other hand, suppose that $L\geq 0$, then $\nu(r)\geq
\nu(R_1)>0$ for $r\geq R_1$. Substituting in \eqref{omrlemeq}, we have
\[ -\nu''(r)\geq \omega\nu(R_1)>0, \]

such that $\nu'(r)\to-\infty$ as $r\to\infty$. This contradicts $L\geq0$.
Since $l=\kappa$ is contradictory in any case, we have $l=0$.
\end{proof}

\begin{proof}[Proof step 3.]
The first term \eqref{omrlemeq} is non-negative and decreasing by \eqref{h3}. We will write
\be \label{mest}
M(r)\coloneqq \frac{f(u)}{u(r)-\kappa} > 0,
\ee

% We define
% \[ M = \underset{r\geq0}{\max}~\frac{f(u(r)}{u(r)-\kappa}, \]
% 
and rewrite \eqref{omrlemeq} to obtain
\be \label{omrm}
M(r) \geq \frac{(n-1)(n-3)}{4r^2} + \omega.
\ee

We choose $2\omega = \underset{r>0}{\max}\;M(r)$ and choose $R_1>0$ such that
\[ \frac{(n-1)(n-3)}{4r^2} \leq \half M(r)\quad\text{for}~r\geq R_1. \qedhere \]
\end{proof}

\section{$P$ is non-empty and open}
In this section we will show that $P$ is non-empty and open. \todogroup Refer
to main theorem.\endgroup The existence of solutions in $F$ also requires that
$N$ is non-empty and open. The proof that $N$ is open is similar to the proof
given for $P$. For the proof that $N$ is non-empty, we refer to "$I_-$ is non-empty" in \cite[p. 147]{ber81}.
% \subsection{$P$ is open}
\begin{lemma}
Solution set $P$ as defined in \eqref{pset}
\[ P\coloneqq\Big\{\;\a>0 \;\Big|\; \exists r_0: 
u'(r_0,\a)=0 \;\text{and}\; u(r,\a)>0 \;\text{for all}\; r\leq r_0 \;\Big\} \]

is non-empty and open.
\end{lemma}
\begin{proof}[Proof step 1.]
We will show that solution set $P$ is non-empty. Let $\a\in(\kappa,\a_0]$. 
\todogroup Refer to definition of $\a_0$. \endgroup % We know that $F(\a)$ is increasing for $\a>\kappa$ and $F(\a_0)=0$ 
% by assumptions on \eqref{glip} on $g$. Since $\a<\a_0$, we have $F(\a)<0$.
Considering all initial conditions $(0,\infty)$ and the disjoint subsets $P$
and $N$, if $\a\notin N$ and $\a\notin\left(P\cupdot N\right)$, then $\a\in P$.

First, we suppose by contradiction that $\a\in N$.  By the definition of $N$ in
\eqref{nset} there exists a number $r_0>0$ such that
\be \label{anprop} \begin{dcases}
u(r_0,\a)=0,\quad\\
u'(r,\a)<0\quad\text{for }~r\leq r_0.
\end{dcases} \ee

\todogroup Refer to gint lemma. \endgroup We restate \cref{gint} for $r=r_0$
\be \label{gintr0}
\half\left[u'(r_0,\a)\right]^2
+(n-1)\int_0^{r_0}u'(s,\a)^2\frac{\diff s}{s}=
F(\alpha)-F(u(r_0,\a)).
\ee

The left hand side of \eqref{gintr0} is positive. But $F(u(r_0,\a))=F(0)=0$, by ...
Furthermore, for $\a\in(\kappa,\a_0]$, we have $F(\a)<0$. Hence $\a\notin N$.

Next, we suppose that $\a\notin (P\cupdot N)$.  
\todogroup Precise refs \endgroup Thus, by the definitions of $P$ and $N$, we have...
\be \label{gsol} 
    \begin{dcases}
u(r,\a)>0\quad\text{for}~r\geq0,~\text{and}\\
u'(r,\a)<0\quad\text{for}~r>0. 
    \end{dcases}
\ee
This implies \todogroup Be specific about the lemma statement and proof steps I refer to. \endgroup
\be \label{gsolinf} 
u(r,\a)\downarrow l\geq0\quad\text{as}~r\uparrow\infty, 
\ee

and by Lemma \eqref{lemlim}, we know that
\[ \underset{r\to\infty}{\lim} u'(r,\a)=0,\quad\text{and}~l=0. \] 

Thus, 
\[ (n-1)\int_0^\infty u'(s,\a)^2 \frac{\diff s}{s} = F(\a)<0. \]

By this contradiction we have $\a\notin \left(P\cupdot N\right)$. Hence $(\kappa, \a_0]\subset P$.
% \[ (\kappa,\a_0]\subset P. \qedhere\]
\end{proof}

\begin{proof}[Proof step 2.] 
We will show that $P$ is open. 
\todogroup Read Teschl or CodLev to be more precise about the circumstances
that I assume, and the type of continuity this implies. \endgroup We know that the
solution $u(r,\a)$ and its
derivative $u'(r,\a)$ depend continuously on $\a$. Let $\a\in P$. There exists
\[ r_0 \coloneqq \inf\left\{r>0 \;\middle|\;
u'(r,\a)=0~\text{and}~u(r,\a)>0\right\} \]

such that
\be \label{posdec}
\begin{dcases}
u(r,\a)>0\quad\text{for all }r\in[0,r_0]\\
u'(r,\a)<0\quad\text{for all }r\in(0,r_0).
\end{dcases}
\ee

Evaluating the IVP \eqref{ivp} in $r_0$ yields
$$ u''(r_0,\a) = -f(u(r_0,\a)). $$

Suppose that $u''(r_0,\a)=0$. Then $-f(u(r_0,\a))=0$. Therefore $u(r_0,\a)=\kappa$ or
$u(r_0,\a)=0$. These would imply $u(r,\a)\equiv\kappa$ or $u(r,\a)\equiv 0$,
which are both impossible in light of \eqref{posdec}. 
\todogroup Mention uniqueness of the IVP, be precise about implication \endgroup

Suppose that $u''(r_0,\a)\neq 0$. Since 
\be
\begin{dcases}
u'(r_0,\a)=0\quad\text{and}\\
u'(r,\a)<0\quad\text{for}~r<r_0,
\end{dcases}
\ee  
 we have $u''(r_0, \a)>0$. Then for $r_1>r_0$ near $r_0$, we have 
\[ u(r,\a) > u(r_0,\a)\quad\text{for all }r\in(r_0,r_1]. \]

By the continuous dependence on $\a$, we have
\[ \forall\epsilon>0\,
\exists\;\delta>0:~\left|u(r,\a)-u(r,\beta)\right|<\epsilon
\impliedby~\left|\a-\beta\right|<\delta.
\]

% In particular, for any $\epsilon$-neighborhood around $u(r_0,\a)$ and
% $u(r_1,\beta)$, there exists a $\delta$-neighborhood around $\a$ that maps to
% the $\epsilon$-neighborhood.
% 
We define 
\[ \epsilon\coloneqq \half\left(u(r_1,\a)-u(r_0,\a)\right). \]

For $\delta_{r_0}>0$ sufficiently small, we have
\[ \left|u(r_0,\a)-u(r_0,\beta)\right|<\epsilon, \]

and for $\delta_{r_1}>0$ sufficiently small, we have
\[ \left|u(r_1,\a)-u(r_1,\beta)\right|<\epsilon. \]

That is, for $\beta$ near $\a$ we have
\be \label{asposdec}
\begin{dcases}
u(r_1,\beta) > u(r_0,\beta)\\
\beta > u(r,\beta) > 0\quad\text{for all }r\in(0, r_1].
\end{dcases}
\ee

Thus $\beta\in P$ and $P$ is open.
\end{proof}
\references{dissertation}
