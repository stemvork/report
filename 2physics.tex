% !TEX root = main.tex
\section{Physics of NLS}
% \reqnomode
In this section, first, the nonlinear Schr\"odinger equation (NLS) will be derived from Maxwell's laws.
A brief discussion of the assumptions involved in this derivation
and the interpretation of (intermediate) results is given.
The discussion is due to Gadi Fibich, for more details, see his book
``The Nonlinear Schr\"odinger Equation''.
In the end, the NLS is given as
\begin{equation}
  2ik_0\psi_z(x,y,z) + \underbrace{\Delta_\bot\psi}_{\text{diffraction}}
+ \underbrace{k_0^2\frac{4n_2}{n_0}|\psi|^2\psi}_{\text{Kerr nonlinearity}} = 0.
\end{equation}

% \seperate
\subsection{Vector electromagnetic fields: Maxwell's laws}~\\

The propagation of electromagnetic waves in a medium is governed by Maxwell's laws.
(In absence of external charges or currents.)
Remember that Maxwell's laws for the electric field $\mathcal{E}$, magnetic field $\mathcal{H}$,
induction electric field $\mathcal{D}$ and induction magnetic field $\mathcal{B}$ are given by:
\begin{equation}
\begin{gathered}
    % \nabla\times\bar\mathscr{E}=-\pd{\bar\mathscr{B}}{t}\\
    \nabla\times\vec{\mathcal{E}} = -\frac{\partial \vec{\mathcal{B}}}{\partial t},\quad
    \nabla\times\vec{\mathcal{H}} = -\frac{\partial \vec{\mathcal{D}}}{\partial t},\\
    \nabla\cdotp\vec{\mathcal{D}}=0,\quad
    \nabla\cdotp\vec{\mathcal{B}}=0.
\end{gathered}
\end{equation}
These are vector fields: $\vec{\mathcal{E}}=(\mathcal{E}_1,\mathcal{E}_2,\mathcal{E}_3)$ in $x,y,z$ coordinates.
In vacuum, the relations between the electric or magnetic fields and induction fields are given as:
\begin{equation}
    \vec{\mathcal{B}}=\mu_0\vec{\mathcal{H}},\quad \vec{\mathcal{D}}=\epsilon_0\vec{\mathcal{E}}
\end{equation}

% \seperate
\subsection{Wave equation from Maxwell's laws}~\\

From these relations and the vector identity for the curl of the curl, the wave equation can be derived.
In particular,
\begin{align}
  \nabla\times\nabla\times\vec{\mathcal{E}}=
  \nabla\times(-\frac{\partial \vec{\mathcal{B}}}{\partial t})=
  -\frac{\partial}{\partial t}(\nabla\times\vec{\mathcal{B}}),
  \quad&\text{ by Maxwell's laws, and}\\
  \nabla\times\nabla\times\vec{\mathcal{E}}=
  \nabla(\nabla\cdotp\vec{\mathcal{E}})-\nabla^2\vec{\mathcal{E}}=
  \nabla(\nabla\cdotp\vec{\mathcal{E}})-\Delta\vec{\mathcal{E}},
  \quad&\text{ by vector calculus.}
\end{align}
Also, calculate the curl of the magnetic field:
$\nabla\times\vec{\mathcal{B}}=\mu_0\frac{\partial\vec{\mathcal{D}}}{\partial t}$.
Then combining these results:
\begin{align}
  \Delta \vec{\mathcal{E}}-\nabla(\nabla\cdotp\vec{\mathcal{E}})=
  \mu_0\frac{\partial^2\vec{\mathcal{D}}}{\partial t^2}\\
  % \Delta \vec{\mathcal{E}}-\nabla(\epsilon_0\nabla\cdotp\vec{\mathcal{D}})=
  % \mu_0\frac{\partial^2\vec{\mathcal{D}}}{\partial t^2}\\
  \Delta \vec{\mathcal{E}}-\nabla(\frac{1}{\epsilon_0}\nabla\cdotp\vec{\mathcal{D}})=
  \mu_0\epsilon_0\frac{\partial^2\vec{\mathcal{E}}}{\partial t^2}.
  % \Delta\vec{\mathcal{E}}=\frac{1}{c^2}\frac{\partial^2\vec{\mathcal{E}}}{\partial t^2}.
\end{align}
And using $\nabla\cdotp\vec{\mathcal{D}}=\nabla\cdotp\epsilon_0\vec{\mathcal{E}}=0$,
this yields the wave equation, where $\mu_0\epsilon_0=1/c^2$:
\begin{equation}
  \Delta\vec{\mathcal{E}}=\frac{1}{c^2}\frac{\partial^2\vec{\mathcal{E}}}{\partial t^2}.
\end{equation}

% \seperate
\subsection{Scalar wave equation and solutions}~\\

The components of the vector electric field are decoupled.
Thus components of solutions to the vector wave equation
satisfy a scalar wave equation:
\begin{equation}
  \Delta \mathcal{E}_i =
  \sum_{i=1}^3\left[\frac{\partial^2\mathcal{E}_j}{\partial x_i^2}\right] =
  \frac{1}{c^2} \frac{\partial^2 \mathcal{E}_j}{\partial t^2}.
\end{equation}
Decoupled means that the second derivative of any component with respect to time is related to
the Laplacian of that component only. There is no term relating $\mathcal{E}_x$
to $\mathcal{E}_y$ or $\mathcal{E}_z$ and vice versa.
This can be seen from the following explication of the vector wave equation:
\begin{equation}
  \Delta \vec{\mathcal{E}} = \Delta \begin{bmatrix}
    \mathcal{E}_x\\
    \mathcal{E}_y\\
    \mathcal{E}_z
  \end{bmatrix} = \begin{bmatrix}
    \frac{\partial^2\mathcal{E}_x}{\partial x^2}+
    \frac{\partial^2\mathcal{E}_x}{\partial y^2}+
    \frac{\partial^2\mathcal{E}_x}{\partial z^2}\\
    \frac{\partial^2\mathcal{E}_y}{\partial x^2}+
    \frac{\partial^2\mathcal{E}_y}{\partial y^2}+
    \frac{\partial^2\mathcal{E}_y}{\partial z^2}\\
    \frac{\partial^2\mathcal{E}_z}{\partial x^2}+
    \frac{\partial^2\mathcal{E}_z}{\partial y^2}+
    \frac{\partial^2\mathcal{E}_z}{\partial z^2}
  \end{bmatrix} = \frac{1}{c^2} \begin{bmatrix}
    \frac{\partial^2\mathcal{E}_x}{\partial t^2}\\
    \frac{\partial^2\mathcal{E}_y}{\partial t^2}\\
    \frac{\partial^2\mathcal{E}_z}{\partial t^2}
  \end{bmatrix}
\end{equation}
The scalar solutions are of the form:
\begin{equation}
  \label{pwpz}\mathcal{E}_j=E_c\exp^{i(k_0z-\omega_0t)}+\text{ c.c.,}
\end{equation}
where $k_0^2=\omega_0^2/c^2$ is the dispersion relation for a plane wave travelling in the positive $z$-direction.
For plane waves propagating in a general direction:
\begin{equation}
  \label{pwgd}\mathcal{E}_j=E_c\exp^{i(\vec{k_0}\cdotp\vec{r}-\omega_0t)}+\text{ c.c.,}
\end{equation}
where $\vec{k_0}=k_x^2\hat{i}+k_y^2\hat{j}+k_z^2\hat{k}$, $\vec{r}=x\hat{i}+y\hat{j}+z\hat{k}$
and $|\vec{k_0}|^2=\omega_0^2/c^2$.
In these expressions ``c.c.'' stands for complex conjugate, so that \ref{pwpz} actually reads (similar for \ref{pwgd}):
\begin{equation}
  \mathcal{E}_j=E_c\exp^{i(k_0z-\omega_0t)}+E_c\exp^{-i(k_0z-\omega_0t)}
\end{equation}
as $E_c$ is a constant real number.

% \seperate
\subsection{Linear polarisation and interpretation}~\\

For a plane wave propagating in the positive $z$ direction, in what directions can the electric field point?
Suppose $\vec{\mathcal{E}}=(\mathcal{E}_1,0,0)$ then $\vec{\mathcal{E}}$ solves the wave equation if
\begin{equation}
  \mathcal{E}_1=E_c\exp^{i(k_0z-\omega_0t)}+\text{ c.c.}
\end{equation}
which is actually inconsistent with Maxwell's laws.
To see this, calculate
\begin{equation}
  0 = \nabla\cdotp\vec{\mathcal{D}}=\epsilon_0\nabla\cdotp\vec{\mathcal{E}}=\epsilon_0(\mathcal{E}_1)_x
\end{equation}
which implies the electric field is not localised in $x$.
% But physical laser beams are localised in the $x,y$-plane.
Then how can one speak of linearly polarised laser beams?
The resolution is that in reality, $\mathcal{E}_2$ and $\mathcal{E}_3$ are not actually 0,
such that $\vec{\mathcal{E}}$ is linearly polarised in the sense that $\mathcal{E}_1>>\mathcal{E}_2,\mathcal{E}_3$.

\seperate
\subsection{Paraxial beams and Helmholtz equation solutions}~\\

To continue the discussion of solutions to the wave equation, consider a sum of plane waves.
In particular, consider a laser beam representated by a sum of plane waves.
For this more general situation, consider time-harmonic (monochromatic) solutions given by:
\begin{equation}
  \mathcal{E}_j(x,y,z,t) = \exp^{-i\omega_0t}E(x,y,z) +\text{ c.c.},
\end{equation}

solving a specific case of the (scalar) wave equation known as the \emph{Helmholtz} equation.
\begin{equation}
  \Delta E+k_0^2E=0\tag{Helmholtz}
\end{equation}

For a laser beam propagating in the $z$-direction, write the incoming field $E_0^{inc}(x,y)$ as a sum of plane waves:

\begin{gather}
  E_0^{inc}(x,y) = \frac{1}{2\pi}\int E_c(k_x,k_y)\exp^{i(k_x+k_y)}\mathrm{d}k_x\mathrm{d}k_y\text{ such that }\\
  E(x,y,z) =  \frac{1}{2\pi}\int E_c(k_x,k_y)\exp^{i(k_x+k_y+\sqrt{k_0^2-k_x^2-k_y^2}z)}\mathrm{d}k_x\mathrm{d}k_y
\end{gather}

However, since a laser beam is concentrated around the $z$-axis, $k_z\approx k_0$.
Then $k_\perp=k_x^2+k_y^2<<k_z^2$.
In other words, as $k_0^2\coloneqq k_x^2+k_y^2+k_z^2$, $k_z = \sqrt{k_0^2-k_x^2-k_y^2}$.
Then $E(x,y,z)=E_c\exp^{i(k_x x+k_y y+k_z z)}=\exp^{ik_0z}\psi(x,y,z)\text{ with }\psi=E_c\exp^{i(k_x x+k_y y+(k_z-k_0)z)}$.

This $\psi$ function is an envelope, of which the amplitude varies slowly in $z$.
The envelope satisfies its own Helmholtz equation:

\begin{equation}
  \psi_{zz}+2ik_0\psi_z+\Delta_\perp\psi=0,
\end{equation}

where $\Delta_\perp=\psi_{xx}+\psi_{yy}$.
In this equation, the $\psi_{zz}$ term can be neglected as $\psi_{zz}<<k_0\psi_z,~\psi_{zz}<<\Delta_\perp\psi$.
The resulting equation is called the linear Schr\"odinger equation:
\begin{equation}
  2ik_0\psi_z+\Delta_\perp\psi=0.\tag{Linear Schr\"odinger}
\end{equation}

\seperate

Split into $E=\exp^{ik_0z}\psi(x,y,z)$, with $\psi$ an envelope function varying slowly in z.
This $\psi$ solves a Helmholtz equation.
Neglect $\psi_{zz}$ (paraxial) to obtain the linear Schr\"odinger equation for $\psi$.

\seperate

{\color{red} Now the step needs to be made to the nonlinear Helmholtz equation. The different polarisations need to be described, weakly nonlinear and Kerr nonlinear. Then obtain nonlin HH to apply parax. approx. to obtain NLS. Then make steps to dimensionless NLS, consider solitary waves and specify the $\omega$...}

Polarisation, linear polarisation, weakly nonlinear polarisation, Kerr nonlinearity.
This all leads to nonlinear Helmholtz, apply paraxial approximation to obtain NLS.

\seperate

Step over to dimensionless NLS and consider solitary waves.

\seperate

{\color{red} Fill in details.} Then, by considering radially symmetric solitary wave solutions, one obtains:
$$ R'' + \frac{1}{r}R' - R + R^3 = 0, $$
with initial condition $R'(0)=0$ and finite {\color{red}power}: $\limrtoinf R(r) = 0$.
This is the equation for which existence and uniqueness of solutions will be discussed.

\seperate

% \leqnomode

% Physics - Chapter 1
% NLS as leading-order model for propagation of intense laser beams in isotropic bulk medium. A laser pulse is an electromagnetic wave and is governed by Maxwell’s equations.
% Deriving the wave equation from Maxwell is a classic result: Lemma 1.1.
% Each component of the electric field satisfies a scalar wave equation and has plane waves as solution with certain dispersion relation. More generally, these plane waves are expressed … with c.c. the complex conjugate.
% Linear polarisation of a plane wave. Actually, laser pulse is not linearly polarised, but the leading-order approximation of the laser pulse is linearly polarised.
% Now laser beams are (time harmonic) continuous wave solutions to the scalar wave equation. And these yield the scalar linear Helmholtz equation. Plane waves are solutions to a specific scalar linear Helmholtz equation.
% Laser beam propagating in $z$-direction, $E_0^inc$ known at $z=0$. Unlike the plane wave, the laser beam electric field decays to zero as the distance from the beam axis goes to infinity. Nevertheless, it can be expressed as a linear superposition of plane waves: and has a corresponding right-propagating solution that is a solution to the Helmholtz equation (each mode is a plane wave, equal to $E_0^inc$ at $z=0$ and propagating in positive $z$-direction. Most of the modes travel parallel to the $z$-axis and are called paraxial plane waves that satisfy .... The favoured solutions to the Helmholtz equation are travelling in the $z$-direction and have electric field envelope that has its own Helmholtz equation with a diffraction term. For example, plane waves can be expressed ... and have an envelope slowly varying as a function of $z$, compared to the carrier oscillations. Lemma 1.2
%
% If the paraxial laser beam mostly consists of paraxial plane waves, suggesting to neglect the ... , and this paraxial approximation yields a linear Schrödinger equation for the envelope. Worth noting: this is a SE arising from classical physics. Also: neglecting the highest-order derivative changes the problem from a boundary problem (Helmholtz) to an initial value problem (Schrödinger). Nevertheless, the results of the approximation agree with experiments. When the electric field is applied to a dielectric medium it induces an additional electric field called polarisation field. There are three mechanisms contributing. To simplify, the electric field is assumed to be linearly polarised and continuous wave, and the dielectric medium is isotropic and homogenous.
%
% LINEAR, WEAKLY NONLINEAR, KERR NONLINEAR, VECTORIAL KERR NONLINEAR, WEAKLY NONLINEAR HELMHOLTZ, NONLINEAR SCHRODINGER, RELATIVE MAGNITUDE OF KERR, VALIDITY OF NLS.
%
%
%
% Chapter 2
% Laser beams in a Kerr medium can become narrower with propagation (self-focusing) and is studies through geometric approximation. Start with the dimensional linear Helmholtz equation. Geometric approximation is valid when changes occur over distances much greater than the wavelength, so locally the medium is homogenous. Then create the dimensionless linear Helmholtz. There is the Eikonal equation. This is a nonlinear first-order PDE. The rays (...) are determined by the system of six linear ODEs. Lemma 2.1 states that under the geometric approximation, the rays of the Eikonal equation are perpendicular to the wavefronts of the Helmholtz equation. Fermats principle of least time was refined to the principle of stationary time. Rays connecting two points are extremals of the Traveltime functional. (Geometric approximation -- ray description of light propagation.) Transport equation implies that the power of a ray bundle at a cross-section is constant. This implies that a focused beam has all beam power in a focal point.
%
% \begin{enumerate}
% \item PARAGRAPH 2 IS MANY APPLICATIONS, ALSO WAVEGUIDES.
% \item PARAGRAPH 3 IS COLLIMATED BEAMS.
% \item PARAGRAPH 4 IS FUNDAMENTAL SOLUTION TO HELMHOLTZ.
% \item PARAGRAPH 5 IS HELMHOLTZ FOCUSED BEAMS.
% \item PARAGRAPH 6 IS SINGULARITY IN HELMHOLTZ.
% \item PARAGRAPH 7 IS GLOBAL EXISTENCE IN THE LINEAR HELMHOLTZ.
% \item PARAGRAPH 8 IS REPRESENTATION OF INPUT BEAMS.
% \item PARAGRAPH 9 IS GEOMETRICAL ANALYSIS OF PARAXIAL PROPAGATION.
% \item PARAGRAPH 10 IS ARREST OF LINEAR COLLAPSE BY DIFFRACTION (GAUSSIAN).
% \item PARAGRAPH 11 IS DIFFRACTION LENGTH.
% \item PARAGRAPH 12 IS DIMENSIONLESS LINEAR SCHRODINGER.
% \item PARAGRAPH 13 IS FOURIER TRANSFORM.
% \item PARAGRAPH 14 IS NORMS.
% \item PARAGRAPH 15 IS LINEAR SCHRODINGER ANALYSIS.
% \item PARAGRAPH 16 IS NON-DIRECTIONALITY.
% \item PARAGRAPH 17 IS LINEAR AND NONLINEAR SINGULARITIES.
% \end{enumerate}
%
% Chapter 3
% Early Self-Focusing Research (Include 3.1, 3.2, 3.3, 3.4, Omit 3.5 and 3.6 and Include 3.7)
