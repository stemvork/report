% !TEX root = main.tex
\newpage
\begin{lemma}\label{lya}
  Suppose that $V(0):=\underset{r\to0}{\lim}~V(r)$ exists and is finite.
  Then $$0<\alpha<\left[\left(\frac{p+1}{2}\right)\frac{\lambda}{V(0)} \right]^{1/(p-1)}\implies \alpha\in P.$$

% \emph{\color{teal}Some remarks: as mentioned, the function $V(r)$ possibly has a singularity in $r=0$. In this lemma, we suppose the limit of $V(r)$ for $r\to0$ exists and is finite. Then a set of initial conditions is implied for which the solution is positive everywhere. This set is bound below by 0 and the upper bound is given by analysis of the Lyapunov function for this problem. Then, as the Lyapunov function is nonincreasing, for negative Lyapunov initial value ($E(0)<0$) the Lyapunov function remains negative. However, evaluating the Lyapunov function in $z(\alpha)$ or for $r\to\infty$ respectively, it follows that certain initial conditions yield nonnegative Lyapunov function values! From these contradictions, the set of initial conditions for which the solution belongs to $P$ is concluded.}

% \emph{\color{red}I would like to derive the Lyapunov function and explain/show how it is analogous to the classical dynamical potential. PHYSICS :). The argument that these initial conditions lead to negative $E(r)$ is given, but needs restructuring. Now, are these all initial conditions yielding solutions of $P$? Is it possible for $V(0)$ to be infinite and still yield $u\in P$?}

\begin{proof}
% TODO: Actually, don't need this r_0 business.. $\a\in P\implies\text{ PD for all }r>0$.
% Remember solutions $u(r;\a)$ to \ref{ivp} with $u(0)=\alpha\in P$ are positive definite on $(0,r_0)$ for some $r_0$ that satisfies $u'(r_0)=0$. Do such initial conditions exist? What interval of initial conditions belongs to $P$?
Remember solutions $u(r;\a)$ to \ref{ivp} with $u(0)=\alpha\in P$ are positive everywhere. Do such initial conditions exist? (See Chapter 4: $P$ is nonempty.) What interval of initial conditions belongs to $P$?
% TODO: define _the_ Lyapunov function $E(r)$ as
% Is it the Lyapunov function or some Lyapunov function?
% When is a function a Lyapunov function? Does this function satisfy those conditions?
To determine such an interval (of initial conditions), define (the Lyapunov or the energy) function $E(r)$ on $(0,z(\a))$ as:
$$E(r)\coloneqq\frac{1}{2}u'(r)^2-\frac{\lambda}{2}u(r)^2+\frac{1}{p+1}V(r)u(r)^{p+1},$$
% for $r\in[0,\infty)$ if $\alpha\in P\cup G$ and $r\in[0,z(\alpha)]$ if $\alpha\in N$.
% rewrite \ref{ivp} such that $u'(r)$ in $E(r)$ and
Then, calculate $E'(r)$, where the IVP can be used to simplify the expression:
\begin{align*}
    % &u''(r)+\frac{1}{r}u'(r)-\lambda u(r)+V(r)u(r)^p=0\\
    % &\left[u''(r)-\lambda u(r)+V(r)u(r)^p\right]=-\frac{1}{r}u'(r)\\
    E'(r)&=u''(r)u'(r)-\lambda u(r)u'(r)+V(r)u(r)^pu'(r)+\frac{1}{p+1}V'(r)u(r)^{p+1}\\
    &=\left[u''(r)-\lambda u(r)+V(r)u(r)^p\right]u'(r)+\frac{1}{p+1}V'(r)u(r)^{p+1}\\
    &=-\frac{u'(r)^2}{r}+\frac{1}{p+1}V'(r)u(r)^{p+1}\leq0\text{ for }r>0.\\
    \text{as }&\left[u''(r)-\lambda u(r)+V(r)u(r)^p\right]=-\frac{1}{r}u'(r)
\end{align*}
% {\color{red}Layout?}
% TODO: u(r) is not positive everywhere if $\a\in N$...?
Each of the terms of $E'(r)$ is non-negative: (i) by hypothesis H2, $V'(r)\leq0$; (ii) $u(r)^{p+1}\geq0$ because $u(r)$ is positive on $(0,z(\a))$ for any initial condition; (iii) $r\geq0$ and (iv) $u'(r)^2\geq0$. \underline{Result:} $E(r)$ is non-increasing $(0,z(\a))$.
% \seperate

% TODO: Choose order of argument, for now: first explain consequence of initial condition.
To conclude about behaviour of solutions by type of initial condition, regard $E(r)$ for large $r>0$.
Suppose $\a\in N$, then $u(z(\a))=0$.
Now evaluate:
\begin{empheq}{align*}
  \underset{r\to z(\alpha)}{\lim}E(r)&=
  \underset{r\to z(\alpha)}{\lim}\bigg[
    \frac{1}{2}u'(r)^2-\frac{\lambda}{2}u(r)^2+\frac{1}{p+1}V(r)u(r)^{p+1}
  \bigg] \\
  &=u'(z(\alpha))^2\geq0.
\end{empheq}
% $E(z(\a))=\frac{1}{2}u'(z(\alpha))^2\geq0$.
As $E(r)$ is non-increasing, $E(0)\geq0$.
Alternatively, suppose $\a\in G$.
Then $u(r)\to0$ and $u'(r)\to0$ as $\rtinf$.
% TODO: $\limrtoinf V'(r)$?
Then $E(r)=0$ for $\rtinf$, so again, $E(0)\geq0$.
\underline{Results:} $E(0)\geq0$ and $E(r)$ well-defined on $[0,z(\a)]$ for $\a\in G\cup N$.

For $\a\in P$, require $E(0)<0$.
Now evaluate $E(0)$ and solve for $\a$:
\begin{align*}
  E(0)=&\frac{1}{2}u'(0)^2-\frac{\lambda}{2}u(0)^2+\frac{1}{p+1}V(0)u(0)^{p+1}<0\\
  &\iff -\frac{\lambda}{2}\alpha^2+\frac{1}{p+1}V(0)\alpha^{p+1}<0\\
  &\iff\alpha^{p-1}<\left(\frac{p+1}{2}\right)\frac{\lambda}{V(0)}\\
  &\iff\alpha<\left[\left(\frac{p+1}{2}\right)\frac{\lambda}{V(0)} \right]^{1/(p-1)}
  % \iff\alpha^{p-1}<\left(\frac{p+1}{2}\right)\frac{\lambda}{V(0)}\iff
%\frac{1}{p+1}V(0)\alpha^{p-1}<\frac{\lambda}{2}\\
  % -\frac{\lambda}{2}\alpha^2+\frac{1}{p+1}V(0)\alpha^{p+1}<0\\
  % \implies E(0)=\frac{1}{2}u'(0)^2-\frac{\lambda}{2}u(0)^2+\frac{1}{p+1}V(0)u(0)^{p+1}<0.
\end{align*}
\underline{Conclusion:} $\a\in P$ whenever $0<\a<\left[\left(\frac{p+1}{2}\right)\frac{\lambda}{V(0)}\right]^{1/(p-1)}$.
% \seperate
% To prove $\alpha\in P$, consider the contradictory cases:
% (i) $\alpha\in N$, (ii) $\alpha\in G$.
% Suppose $\alpha\in N$.
% Then $u(z(\alpha))=0$ and $E(z(\alpha))=\frac{1}{2}u'(z(\alpha))^2\geq0$.
% This contradicts $E<0$.
% Suppose $\alpha\in G$.
% Then $u(r)\to0,u'(r)\to0$ as $r\to\infty$.
% Then $E(r)\to0$ as $r\to\infty$, contradicting $E<0$.
% \underline{Conclusion:} $\alpha\in P$.

% By these properties of $E(r)$ the implication will follow.
% What about the initial value of $E(r)$? Evaluate
% Let $0<\alpha<\left[\left(\frac{p+1}{2}\right)\frac{\lambda}{V(0)} \right]^{1/(p-1)}.$
% Rewrite this assumption on $\alpha$ and evaluate $E(0)$:
% \begin{gather*}
%   \alpha<\left[\left(\frac{p+1}{2}\right)\frac{\lambda}{V(0)} \right]^{1/(p-1)}
%   \iff\alpha^{p-1}<\left(\frac{p+1}{2}\right)\frac{\lambda}{V(0)}\iff
% %\frac{1}{p+1}V(0)\alpha^{p-1}<\frac{\lambda}{2}\\
%   -\frac{\lambda}{2}\alpha^2+\frac{1}{p+1}V(0)\alpha^{p+1}<0\\
%   \implies E(0)=\frac{1}{2}u'(0)^2-\frac{\lambda}{2}u(0)^2+\frac{1}{p+1}V(0)u(0)^{p+1}<0.
% \end{gather*}
% Remember $E(r)$ is nonincreasing, so $E(r)<0$ for $r>0$.
% \seperate
% To prove $\alpha\in P$, consider the contradictory cases:
% (i) $\alpha\in N$, (ii) $\alpha\in G$.
% Suppose $\alpha\in N$.
% Then $u(z(\alpha))=0$ and $E(z(\alpha))=\frac{1}{2}u'(z(\alpha))^2\geq0$.
% This contradicts $E<0$.
% Suppose $\alpha\in G$.
% Then $u(r)\to0,u'(r)\to0$ as $r\to\infty$.
% Then $E(r)\to0$ as $r\to\infty$, contradicting $E<0$.
% \underline{Conclusion:} $\alpha\in P$.

\end{proof}
\end{lemma}

% \subsection{Discussion} Consider the Lyapunov function $E(r)$, an analogue to the potential function of classical dynamics. Lyapunov theory treats the stability of a solution near an equilibrium point. For more on Lyapunov theory, see \cite{}.
