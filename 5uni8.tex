\begin{lemma} Let $\alpha\in G$. There exists $\epsilon>0$ such that $(\alpha,\alpha+\epsilon)\subset N$.
\begin{proof}

\begin{outline}
% 	\1 By lemma 5.6, for $\a\in G$, $w$ is unbounded and there exists unique $r_0$ in $(0,\za)$ such that $w(r_0)=0$.
% 	\1 Then by Lemma 6 of Kwong, $r_0\in(d,\infty)$, that is, the unique zero $r_0$ of $w(r)$ on $(0,\za)$ is in the disconjugacy interval $(d,\infty)$ of \ref $w$-IVP. USED WHERE?
%	\1 Let $r_1,r_2$ be such that: $d<r_1<r_0<r_2<\infty$. Since $w(r_1)>0$ and $w(r_2)<0$, there exists $\epsilon>0$ such that for all $\tilde\a\in(\a,\a+\epsilon)$, $$ \tilde u(r_1) > u(r_1) \text{ and } \tilde u(r_2)<u(r_2) $$ where $\tilde u = u(r;\tilde\a)$.
	\1 Hence, for given $\tilde\a\in(\a,\a+\epsilon)$, the graph of $\tilde u$ intersects the graph of $u$ at some point $r_3\in(r_1,r_2)$.
	\1 Claim: there exists $\tilde r\in(r_3,\infty)$ such that $\tilde u(\tilde r)=0$, thus $\tilde\a\in N$.
		\2 Suppose by contradiction that $\tilde u(r)>0$ for all $r>r_3$.
		\2 Then $\tilde u(r)<u(r)$ for all $r>r_3$ by the following argument:
			\3 Suppose by contradiction that $\tilde u(r_4)=u(r_4)$ for some $r_4>r_3$ and $u-\tilde u>0$ on $(r_3,r_4)$.
			\3 Define on $(r_3,r_4)$ function $z\coloneqq u-\tilde u$ that satisfies SHOW: $$ z'' + \frac{1}{r}z' -\lambda z + V(r)\frac{u^p-\tilde u^p}{u-\tilde u}z = 0 $$
% 			\3 Before applying Sturm comparison between $z$ and $w$, let $y$ be a solution to \ref $w$-IVP linearly independent of $w$. Then $y$ must vanish somewhere in $(r_3,r_4)$ BY DISCONJUGACY? WHY? STURM COMPARISON WITH Z? QUESTION: DOES ANY SOLUTION TO w-IVP NEED TO YIELD THE SUPPOSED PROPERTIES OF $u$ AND $\tilde u$? PROBABLY NOT.. PROBABLY BY STURM COMPARISON, AS THE COMPARISON BETWEEN $\frac{u^p-\tilde u^p}{u-\tilde u}$ AND $pu^p$ IS MADE!Thus $\{w,y\}$ is a basis of solutions of \ref $w$-IVP and it is impossible to find a positive solution on $(d,\infty)$, a contradiction with the supposed property of $z$ !! CHECK !!
		\2 Anyway.... The contradiction shows that $z(r)>0$ ($\tilde u(r)<u(r)$) for all $r>r_3$. And $w$-IVP \ref is a Sturm majorant of $z$-IVP \ref.
		\2 Let $\tilde w$ be a solution to $w$-IVP \ref such that $\tilde w(r_3)=0$.
			\3 Note that $\tilde w(r_3)=0$ is a completely different solution than the aforementioned $w(r)$.
		\2 Since $r_3\in(d,\infty)$ by Lemma 6 in Kwong our $\tilde w$ is unbounded and we can assume that $\tilde w(r)\to+\infty$. Why PLUS infinity? Our $w$ goes to $-\infty$... Does this have to do with linear independence? SHOW
		\2 Also $\tilde w>0$ on $(r_3, \infty)$. No solution can vanish more than once in the disconjugacy interval.. SHOW/USE
		\2 Then by the strong version of Sturm comparison: $$ \frac{w'(r)}{w(r)}\leq\frac{z'(r)}{z(r)}\text{ for all }r\in(r_3,r_4)$$
		\2 Which by integration from $r_4$ to $r$ yields: $$ \ln\tilde w(r)\leq\ln\frac{\tilde w(r_4)}{z(r_4)}+\ln z(r) $$
		\2 Now, remember $z\equiv u(r)-\tilde u(r)\to+\infty$ as $r\to\infty$, which is impossible since $0<\tilde u(r)<u(r)$ on $(r_3,\infty)$ and $u(r)\to\infty$! (How could $z\equiv u(r)-\tilde u(r)$ go to infinity if the majorant $u$ goes to zero and the subtrahend is positive everywhere!)
		\2 Then the contradiction is reached and $\tilde u(r)$ has a zero for some $r\in(r_3,\infty)$.
	\1 Conclusion: $\tilde u$ has a zero so $\tilde\a\in N$.
	\1 Since $\tilde\a$ was chosen arbitrarily in $(\alpha,\alpha+\epsilon)$, all $\a'\in(\alpha,\alpha+\epsilon)$ have $\alpha'\in N$.
\end{outline}
By chapter 4, the solution set $G$ is non-empty. Let $\alpha\in G$ and let $u(r;\alpha)$ be the corresponding solution.  The function $w(\alpha,r)=\frac{\partial}{\partial\alpha}u(\alpha;r)$ satisfies \ref.
\\ \\

By lemma \ref{wq}, the function $w$ is unbounded. By lemma \ref{kwong6}, $w$ has a unique zero $r_0\in(d,\infty)$. Kwong discusses the disconjugacy interval $(d,\infty)$ of \ref in more detail. Let $r_1,r_2$ be such that $d<r_1<r_0<r_2$ and note that $w(r_1)>0$ and $w(r_2)<0$. There exists $\epsilon>0$ such that, for all $\tilde\alpha\in(\alpha,\alpha+\epsilon)$,$$\tilde u(r_1)>u(r_1)\text{ and }\tilde u(r_2)<u(r_2),$$ where $\tilde u=u(\tilde\alpha,r)$. Hence there exists $r_3\in(r_1,r_2)$ such that the graphs of $u$ and $\tilde u$ intersect, i.e. $\tilde u(r_3)=u(r_3)$. See also figure \ref{}.
\\ \\

To conclude $\tilde\alpha\in N$ requires existence of a $\tilde r$ such that $\tilde u(\tilde r)=0$. By contradiction, suppose that $\tilde u(r)>0$ for all $r>r_3$. (Note that $\tilde u(r)\geq u(r)>0$ on $r\leq r_3$, since $u(r)>0$ for all $r$.)

Now, for $r>r_3$ and small, $\tilde u(r)<u(r)$. Claim: $u(r)>\tilde u(r)$ for all $r>r_3$. Define the difference between $u$ and $\tilde u$ as $z\coloneqq u-\tilde u$. Note $u(r)>\tilde u(r)\iff z(r)>0$. By contradiction, suppose there exists $r_4>r_3$ such that $\tilde u(r_4)=u(r_4)$ (equivalent to $z(r_4)=0$). On $(r_3,r_4)$ the function $z$ satisfies:
% $$z''+\frac{1}{r}z'+\left[V(r)\frac{u^p-\tilde u^p}{u-\tilde u}\right]z = 0 $$
\begin{align*}
	&z''+\frac{1}{r}z'+\left[V(r)\frac{u^p-\tilde u^p}{u-\tilde u}\right]z = 0\quad\mathrm{because} \\
	(1):\quad &u'' + \frac{1}{r}u' -\lambda u + Vu^p = 0 \\
	(2):\quad &\tilde u'' + \frac{1}{r}\tilde u' -\lambda\tilde u + V\tilde u^p = 0\\
	(1) - (2):\quad &u'' - \tilde u'' + \frac{1}{r}u' - \frac{1}{r}\tilde u' -\lambda u + \lambda\tilde u + Vu^p - V\tilde u^p = 0\\
	 &z'' + \frac{1}{r}z' -\lambda z + \left[Vu^p - V\tilde u^p\right]
	\frac{u-\tilde u}{u-\tilde u} = 0\\
	 &z'' + \frac{1}{r}z' + \left[V(r)\frac{u^p-\tilde u^p}{u-\tilde u}-\lambda\right]z = 0
\end{align*}

%Using Sturm theory, the zeroes of $z$ can be studied.
%and note that proving $z(r)>0$ on $(r_3,\infty)$ implies $\tilde u(r)<u(r)$ on that interval.
% This function satisfies the differential equation $$\label{zivp} BLA $$.
% \\ \\

Also by Sturm comparison of $z$ and $w$, the latter oscillates faster. Let $\tilde w$ be a solution of \eqref{wivp} such that $\tilde w(r_3)=0$.
\\ \\

By integration of the strong version of Sturm, $z(r)\to\infty$ as $r\to\infty$, but this is impossible as $0<\tilde u(r)<u(r)$ on $(r_3,\infty)$ and $u(r)\to0$ as $r\to\infty$. Therefore, $\tilde u$ vanishes at some point $\tilde r\in(r_3,\infty)$ and the proof is complete.
\\ \\

\comm{
for any $\epsilon>0$, all solutions $\tilde u=u(\tilde\alpha,r)$ with $\tilde\alpha\in(\alpha,\alpha+\epsilon)$ are non-vanishing. Then  Remember that $w$ is the derivative of \eqref{ivp} with respect to the initial condition. And by \ref{} $w$ has a unique zero that is contained in the disconjugacy interval. [[]] Let $r_1,r_2$ be such that $d<r_1<r_0<r_2$ and note $w(r_1)=\frac{\partial}{\partial\alpha}u(\alpha;r_1)>0$ and $w(r_2)=\frac{\partial}{\partial\alpha}u(\alpha;r_2)<0$. Let $\epsilon>0$ such that for all $\tilde\alpha\in(\alpha,\alpha+\epsilon)$, $$\tilde u(r_1)>u(r_1)\text{ and }\tilde u(r_2)<u(r_2),$$ where $\tilde u=u(\tilde\alpha,r)$ and $\tilde\alpha\in(\alpha,\alpha+\epsilon)$. The solutions $u$ and $\tilde u$ intersect in some point $r_3$. THEY STURM COMPARE. RELATION Z=U-UT.}

\end{proof}
\end{lemma}
