% !TEX root = main.tex
\newpage
\begin{lemma}Let $\alpha\in (G\cup N),\text{ then }w$ has at least one zero in $(0,z(\alpha))$.
\begin{proof}
  \begin{outline}
    \1 To conclude that $w$ has one zero in $(0,\za)$, use Lagrange identity.
    \1 The proofs for $\a\in G$ and $\a\in N$ will be done seperately.
    \1 First, suppose $\a\in N$.
    \1 Rewrite IVP and \emph{w-d.e.} to the following:
      \2 $(ru')'+r\left[-\lambda u+Vu^p\right]=0$
      \2 $(rw')'+r\left[-\lambda w+pVu^{p-1}w\right]=0$
    \1 Multiply by $w$ and $u$ respectively, then integrate from 0 to $\za$:
      \2 $\int_0^{\za}w(ru')'-u(rw')'dr = \int_0^{\za}r\left[pVu^pw-Vu^pw\right]dr$
    \1 By partial integration for left hand side, one obtains:
      \2 $rwu'\at_0^{\za}-ruw'\at_0^{\za}-\int_0^{\za}\left[ru'w'-ru'w'\right]dr=(p-1)\int_0^{\za}rVu^pwdr$
    \1 Use $u(\za)=0$ to obtain:
      \2 $\za w(\za)u'(\za)-\za u(\za)w'(\za) = (p-1)\int_0^{\za}rVu^pwdr$
    \1 Suppose $w>0$ on $(0,\za)$ then left hand side $\leq0$ as $\za>0$, $w(\za)\geq0$ (?) and $u'(\za)<0$.
    \1 {\color{red} To resolve this, suppose $w>0$ on $(0,\za)$ then $\za>0$, $w(\za)>0$ and $u'(\za)<0\implies\text{ l.h.s. }<0$. That \emph{is} sufficient to show contradiction with $\text{r.h.s.}>0$ as ...}
    \1 {\color{blue} To \emph{actually} resolve this, the initial supposition was correct: $w>0$ on $(0,\za)$ implies $\text{l.h.s.}\leq0$ as $\za>0$, \emph{importantly} $w(\za)>0$ and $u'(\za)<0$. Why can't $w(\za)=0$? By Sturm comparison! Since $pVwu^{p-1}\neq Vu^p$, the zeroes of $w$ and $u$ will not coincide!}
      \2 Note work remains to be done to clarify this part of the argument. From \'Suppose $w>0$...\' to the contradiction.
    \1 By this contradiction then, $w(r)$ has at least one zero on $(0,\za)$.
    \1 To conclude the same for $\a\in G$, again, assume $w>0$ on $(0,\za)$ then still r.h.s. of \ref identity $>0$.
    \1 As for the l.h.s. regard the expression $\frac{u}{w}$...
      \2 Write $\left(\frac{u}{w}\right)'=\frac{wu'-uw'}{w^2}$.
      \2 Note that $u(0)>0$ and $w(0)>0$ implies $\frac{u}{w}(0)>0$.
      \2 Rewrite the \ref identity to read:
        \3 $rwu'-ruw'=(p-1)\int_0^{\za}rVu^pwdr$
        \3 $\frac{wu'}{w^2}-\frac{uw'}{w^2}=\frac{p-1}{rw^2}\int_0^{\za}rVu^pwdr>0$
        \3 $\frac{wu'-uw'}{w^2}>0\implies \left(\frac{u}{w}\right)'>0$
      \2 So $\uowd$ is increasing.
    \1 Now, there is a hole.. Apparantly, this also implies $w>0$ yields contradiction.
      \2 Intuitively, $\uowd$ increasing means that $w$ decays faster than $u$ everywhere.
      \2 Then, as for $\a\in G$ the solution decays to 0, so must $w$.
      \2 But, since $w$ decays faster than $u$, the zero of $w$ must be to the left of $\za=\infty$.
      \2 Hence, $w$ has a zero in $(0,\za)$.
        \3 Hooray!
    \1 Might need to formalise this a bit further.
  \end{outline}

  \seperate

The Lagrange identity for \ref{ivp} and \ref{} will yield information about the zeroes of $w(r)$.
Observe the following identities arising from the differential equations:
\begin{gather*}
  (ru'(r))'+r\left[-\lambda u(r)+V(r)u(r)^p\right]=0\\
  (rw'(r))'+r\left[-\lambda w(r)+pV(r)u(r)^{p-1}w(r)\right]=0.
\end{gather*}
Now multiply by $w(r)$ and $u(r)$ respectively, subtract \_them and integrate from 0 to $z(\alpha)$,
\begin{gather*}
  \int_0^{z(\alpha)}w(r)(ru'(r))'-u(r)(rw'(r))'dr=%
  \int_0^{z(\alpha)}r\left\{pV(r)u(r)^pw(r)-V(r)u(r)^pw(r)\right\}dr,
\end{gather*}
and perform partial integration:
\begin{gather*}
  rw(r)u'(r)\at_0^{z(\alpha)}-ru(r)w'(r)\at_0^{z(\alpha)}-
  \int_0^{z(\alpha)}ru'(r)w'(r)-ru'(r)w'(r)dr\\
  =(p-1)\int_0^{z(\alpha)}rV(r)u(r)^pw(r)dr\\
  z(\alpha)w(z(\alpha))u'(z(\alpha))=(p-1)\int_0^{z(\alpha)}rV(r)u(r)^pw(r)dr.
\end{gather*}
Note that $u(z(\alpha))=0$.

\seperate

In evaluating the integral term, {\color{gray}note $r>0$, $V(r)>0$, and $u(r)^p>0$.
Suppose $w>0$ on $(0,z(\alpha))$.
Then $z(\alpha)w(z(\alpha))u'(z(\alpha))\leq0$ contradicts
$(p-1)\int_0^{z(\alpha)}rV(r)u(r)^pw(r)dr>0$ \Lightning.
Hence $w$ has at least one zero in $(0,z(\alpha))$.}

\seperate

\underline{$\alpha\in G$}
Suppose by contradiction that $w>0$ on $(0,\infty)$. {\color{gray}
Then rewrite ... and note how the integral is still positive by assumption.
Then $\left(\frac{u}{w}\right)'$ is positive.
So $\frac{u}{w}$ is increasing.
By RRR there exists two independent solutions that satisfy TTT.
So TTT for some constants $\alpha_1,\alpha_0$.
Since $w>0$ by hypothesis, $\alpha_1\geq0$.
Suppose $\alpha_1=0$, then $w(r)\to0$ exponentially as $r\to\infty$.
So $w$ changes sign by RRR, a contradiction.
On the other hand, suppose $\alpha_1>0$, then by RRR there exists a $C$ such that LLL.
To see this, note how RRR implies $u(r)\sim r^{-1/2}\exp^{-\sqrt{\lambda}r}$ as $r\to\infty$.
These contradictions \_yield $w$ changes sign at least once on $(0,z(\alpha))$ for $\alpha\in(G\cup N)$.}
the Lagrange identity for and the integral term is positive by the same reasoning as above.
Claim: the function $u/w$ is positive and increasing.
Firstly, it is positive in the origin, because $u(0),w(0)>0$.
Secondly, it is increasing since $r$, $w(r)^2$ are positive, so is the derivative of $u/w$.
Also, since $u(0),w(0)>0$ the function \\ \\
\underline{}\\ \\
\underline{}
\end{proof}
\end{lemma}
