% !TEX root = main.tex
\newpage
\begin{lemma}Let $\alpha\in (G\cup N),\text{ then }w$ has at least one zero in $(0,z(\alpha))$.
\begin{proof}
The Lagrange identity for \ref{ivp} and \ref{wvp} will yield information about the zeroes of $w(r)$.
The cases $\a\in N$ and $\a\in G$ will be considered seperately.
Suppose $\a\in N$.
The differential equations for $u$ and $w$ can be written as:
\begin{gather*}
  (ru'(r))'+r\left[-\lambda u(r)+V(r)u(r)^p\right]=0\\
  (rw'(r))'+r\left[-\lambda w(r)+pV(r)u(r)^{p-1}w(r)\right]=0.
\end{gather*}
Multiply by $w(r)$ and $u(r)$ respectively, subtract the equations and integrate from 0 to $z(\alpha)$,
\begin{gather*}
  \int_0^{z(\alpha)}w(r)(ru'(r))'-u(r)(rw'(r))'dr=%
  \int_0^{z(\alpha)}r\left\{pV(r)u(r)^pw(r)-V(r)u(r)^pw(r)\right\}dr,
\end{gather*}
and perform partial integration for the left hand side: (remember $u(z(\a))=0$)
\begin{align*}
  rw(r)u'(r)\at_0^{z(\alpha)}-ru(r)w'(r)\at_0^{z(\alpha)}&-
  \int_0^{z(\alpha)}\underbrace{\{ru'(r)w'(r)-ru'(r)w'(r)\}}_0dr\\
  &=(p-1)\int_0^{z(\alpha)}rV(r)u(r)^pw(r)dr\\
  z(\alpha)w(z(\alpha))u'(z(\alpha))&=(p-1)\int_0^{z(\alpha)}rV(r)u(r)^pw(r)dr.
\end{align*}

% \seperate

For $\a\in N$, note $r>0,~V>0,~u^p>0$ are finite almost everywhere.
% In evaluating the integral term, note $r>0$, $V(r)>0$, and $u(r)^p>0$.
Suppose $w>0$ on $(0,z(\alpha))$.
Then $z(\alpha)u'(z(\alpha))w(z(\alpha))<0$ contradicts
$(p-1)\int_0^{z(\alpha)}rV(r)u(r)^pw(r)dr>0$.
% Hence $w$ has at least one zero in $(0,z(\alpha))$.
(A similar argument holds for $w<0$.)
% Suppose $w<0$ on $(0,z(\a))$.
% Then $z(\a)u'(z(\a))w(z(\a))\geq0$ contradicts
% $(p-1)\int_0^{z(\alpha)}rV(r)u(r)^pw(r)dr<0$.
Hence, $w$ changes sign at least once on $(0,z(\a))$.
% \seperate

For $\alpha\in G$, suppose by contradiction that $w>0$ on $(0,\infty)$.
Perform integration over $(0,r)$ and rewrite left hand side using the quotient rule:
\begin{align*}
ru'(r)w(r)-rw'(r)u(r)
=r\frac{u'(r)w(r)-w'(r)u(r)}{w(r)^2}
&=rw(r)^2\left(\frac{u(r)}{w(r)}\right)'\\
&=(p-1)\int_0^{z(\alpha)}rV(r)u(r)^pw(r)dr>0.
\end{align*}
% Note $w>0\implies\int_0^r\ldots dr>0$.
\underline{Result:} $\left(\frac{u}{w}\right)'$ is positive, so $\frac{u}{w}(r)$ is increasing.

% \seperate

By Lemma C.1 of \cite{fthes} there exists two independent solutions that satisfy \ref{wvp} as $\rtinf$: $$ \xi_0(r)\sim r^{-\frac{1}{2}}\exp^{-\sqrt{\lambda}r}\quad\text{and}
\quad\xi_1(r)\sim r^{-\frac{1}{2}}\exp^{\sqrt{\lambda}r}$$
So for $\rtinf$,
$w(r)\sim r^{-\frac{1}{2}}\left[\a_0\exp^{-\sqrt{\lambda}r}+\a_1\exp^{-\sqrt{\lambda}r}\right]$
for some constants $\alpha_1,\alpha_0$.
Since $w>0$ by hypothesis, and $\limrtoinf w(r)\sim\a_1$, $\alpha_1\geq0$.
Suppose $\alpha_1=0$, then $w(r)\to0$ exponentially as $r\to\infty$.
So $w$ changes sign by Lemma 1.4.9 of \cite{fthes}, a contradiction.
On the other hand, suppose $\alpha_1>0$.
Then by Lemma 1.2.7 and Lemma C.1 of \cite{fthes}, $u(r)\sim r^{-\frac{1}{2}}\exp^{-\sqrt{\lambda}r}$ as $\rtinf$.
\underline{Result:} for $\a_1>0$, there exists $C$ such that $$
\limrtoinf\frac{u(r)}{w(r)}=\limrtoinf\frac{Cr^{-\frac{1}{2}}\exp^{-\sqrt{\lambda}r}}{\a_1r^{-\frac{1}{2}}\exp^{\sqrt{\lambda}r}}=\limrtoinf\frac{C}{\a_1}\exp^{-2\sqrt{\lambda}r}=0 $$
% exists a $C$ such that LLL.
% To see this, note how RRR implies $u(r)\sim r^{-1/2}\exp^{-\sqrt{\lambda}r}$ as $r\to\infty$.
This contradicts $\frac{u}{w}(r)$ positive and increasing.
\underline{Conclusion:} $w$ changes sign at least once on $(0,z(\a))$ for $\a\in G\cup N$.
% These contradictions \_yield $w$ changes sign at least once on $(0,z(\alpha))$ for $\alpha\in(G\cup N)$.
% the Lagrange identity for and the integral term is positive by the same reasoning as above.
% Claim: the function $u/w$ is positive and increasing.
% Firstly, it is positive in the origin, because $u(0),w(0)>0$.
% Secondly, it is increasing since $r$, $w(r)^2$ are positive, so is the derivative of $u/w$.
% Also, since $u(0),w(0)>0$ the function \\ \\
% \underline{}\\ \\
% \underline{}
\end{proof}
\end{lemma}
