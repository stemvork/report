\newcommand{\Rom}{R_\omega(\bar{x})}
\newcommand{\eiwz}{e^{i\omega z}}

\newcommand{\eikzwt}{e^{i(k_0z-\omega_0t)}}

\newcommand{\ec}{\mathcal{E}}
\newcommand{\ev}{\vv{\mathcal{E}}}
\newcommand{\dc}{\mathcal{D}}
\newcommand{\dv}{\vv{\mathcal{D}}}

\newcommand{\bv}{\vv{\mathcal{B}}}
\newcommand{\hv}{\vv{\mathcal{H}}}

\newcommand{\pc}{\mathcal{P}}
\newcommand{\pl}{\pc_\text{lin}}
\newcommand{\pn}{\pc_\text{nl}}
\newcommand{\pv}{\vv{\mathcal{P}}}

\newcommand{\ca}{\chi^{\text{(1)}}}
\newcommand{\cc}{\chi^{\text{(3)}}}
\newcommand{\ww}{\omega_0}
\newcommand{\kp}{k_\perp}
\newcommand{\overeq}[1]{\overset{\eqref{#1}}{=}}
%\newcommand{\abs}[1]{\left|#1\right|}
\newcommand{\bigo}[1]{\mathcal{O}\left(#1\right)}

\setlength\multicolsep{0pt}

\chapter{Physics of NLS}
\label{physics}

% \begin{abstract}
% Lorem ipsum dolor sit amet, consectetur adipisicing elit, sed do eiusmod tempor
% incididunt ut labore et dolore magna aliqua. Ut enim ad minim veniam, quis
% nostrud exercitation ullamco laboris nisi ut aliquip ex ea commodo consequat.
% Duis aute irure dolor in reprehenderit in voluptate velit esse cillum dolore eu
% fugiat nulla pariatur.
% \end{abstract}


\section{Derive the wave equation from Maxwell}
Any electromagnetic wave is governed by Maxwell's laws. In this work, we work in
absence of external charges or currents. Then Maxwell's laws for the electric
field $\ev$, magnetic field $\hv$, induction electric field $\dv$ and induction
magnetic field $\bv$ are given by:
% \vspace{-0.5cm}
\begin{multicols}{2}
\be\nabla\times\ev = -\frac{\partial \bv}{\partial t},\label{mwerot}\tag{1.1.a}\ee
\be\nabla\times\hv = \frac{\partial \dv}{\partial t},\label{mwhrot}\tag{1.1.b}\ee

\be\nabla\cdot\dv=0,\label{mwddiv}\tag{1.1.c}\ee
\be\nabla\cdot\bv=0.\label{mwbdiv}\tag{1.1.d}\ee
\end{multicols}
% \vspace{-0.25cm}
The fields are in three-dimensional Cartesian coordinates, for example:
$\ev=(\mathcal{E}_1,\mathcal{E}_2,\mathcal{E}_3)$ in ($x,y,z$)
coordinates. Besides considering no external charges or currents, we consider
unitary (relative) permittivities, such that the relation between fields and
induction fields (electric or magnetic) is given as:
% \vspace{-0.5cm}
\begin{multicols}{2}
\be \label{bind} \bv=\mu_0\hv,\tag{1.2.a}\ee

\be \label{dind} \dv=\epsilon_0\ev.\tag{1.2.b}\ee
\end{multicols}
% \vspace{-0.25cm}
The notation used here is from "The Nonlinear Schr\"odinger Equation" by G.
Fibich \cite[p.~3]{fibg}. For more background on electrodynamics
see "Introduction to Electrodynamics" by D.J. Griffiths \cite{grif}. This
reference work also includes an introduction to the necessary vector calculus.

%From these relations and the vector identity for the curl of the curl, a wave
%equation can be derived. We specifically use
%$\nabla\cdot\dv=\nabla\cdot\epsilon_0\ev=0$ and 
%$\nabla\times\bv=\mu_0\frac{\partial\dv}{\partial t}$
%to simplify the equation:
We use vector calculus and Maxwell's laws to rewrite the curl of the curl:
\begin{gather*}
  \nabla\times\left(\nabla\times\ev\right)\overeq{mwerot}
  \nabla\times\left(-\frac{\partial \bv}{\partial t}\right)=
  -\frac{\partial}{\partial t}\left(\nabla\times\bv\right)
  \overset{\substack{(\ref{mwhrot})\\(\ref{bind})}}{=}
  -\mu_0\frac{\partial^2\mathcal{D}}{\partial t^2}\overeq{dind}
  -\mu_0\epsilon_0\frac{\partial^2\mathcal{E}}{\partial t^2},\;\text{and}\\
  \nabla\times\left(\nabla\times\ev\right)=
  \nabla\left(\nabla\cdot\ev\right)-\nabla^2\ev=
  \nabla\left(\nabla\cdot\ev\right)-\Delta\ev
  \overset{\substack{(\ref{mwddiv})\\(\ref{dind})}}{=}
  -\Delta\ev.
\end{gather*}
Combining these and using $\mu_0\epsilon_0=1/c^2$, we arrive at the vector wave equation:
\setcounter{equation}{2}
\begin{equation}
\label{scwe}
  \Delta\ev=\frac{1}{c^2}\frac{\partial^2\ev}{\partial t^2}.
\end{equation}

\section{Validity of plane wave solutions}
Stuyding the left and right hand sides of \cref{scwe}, we see that the vector
wave equation is in fact a system of three scalar wave equations.
\begin{equation*}
  \Delta \ev = \Delta
  \begin{bmatrix}
    \mathcal{E}_x\\
    \mathcal{E}_y\\
    \mathcal{E}_z
  \end{bmatrix} =
  \begin{bmatrix}
    \frac{\partial^2\mathcal{E}_x}{\partial x^2}+
    \frac{\partial^2\mathcal{E}_x}{\partial y^2}+
    \frac{\partial^2\mathcal{E}_x}{\partial z^2}\\
    \frac{\partial^2\mathcal{E}_y}{\partial x^2}+
    \frac{\partial^2\mathcal{E}_y}{\partial y^2}+
    \frac{\partial^2\mathcal{E}_y}{\partial z^2}\\
    \frac{\partial^2\mathcal{E}_z}{\partial x^2}+
    \frac{\partial^2\mathcal{E}_z}{\partial y^2}+
    \frac{\partial^2\mathcal{E}_z}{\partial z^2}
  \end{bmatrix} = \frac{1}{c^2}
  \begin{bmatrix}
    \frac{\partial^2\mathcal{E}_x}{\partial t^2}\\
    \frac{\partial^2\mathcal{E}_y}{\partial t^2}\\
    \frac{\partial^2\mathcal{E}_z}{\partial t^2}
  \end{bmatrix}
\end{equation*}
\begin{equation*}
  \Delta \mathcal{E}_j =
  \sum_{l=1}^3\left[\frac{\partial^2\mathcal{E}_j}{\partial x_l^2}\right] =
  \frac{1}{c^2} \frac{\partial^2 \mathcal{E}_j}{\partial t^2}.
\end{equation*}
This motivates the following ansatz to such a scalar wave equation:
\be \label{pws} \mathcal{E}_j=E_c e^{i(k_0 z-\omega_0 t)},\ee
where $k_0$ is the wavenumber and $\ww$ the frequency. These are so 
called plane wave solutions. The wavefronts have the simple
geometry of an infinite plane at any $z$-value and the electric field is
non-zero in the $x$ and $y$ directions. The wavefronts are spaced by the
wavelength $\lambda$ and the wavenumber $k_0$ is the reciprocal of the
wavelength.

This plane wave travels in the positive $z$-direction for positive wavenumber
$k_0$ and vice versa. Note that the solution does not depend on $x$ or $y$. As a
result, for a fixed $z'$, the electric field $\mathcal{E}$ is constant in the
$(x,y,z')$-plane. 

We substitute \eqref{pws} in \cref{scwe}. Note that only $\Delta_z$ will be non-zero:
% \begin{gather*}
$$ \Delta \ec_j = % \frac{\partial^2}{\partial z^2} E_c \eikzwt = 
k_0^2 \cdot E_c\eikzwt=\frac{1}{c^2} \omega_0^2 \cdot E_c \eikzwt $$
yields the dispersion relation \eqref{disp}:
\be \label{disp} k_0^2=\frac{\omega_0^2}{c^2}.\ee
% \newgroup
% The index of refraction $n$ depends on the wavelength $\lambda$. Equivalently,
% since $$ k = \frac{2\pi}{\lambda} = \frac{\omega}{v_p}, $$
% the index of refraction depends on the wavenumber $k$. \ldots 
% \endgroup
% 
For a general direction in $(x,y,z)$-coordinates, define the wavevector 
$$\vv{k}=(k_x,k_y,k_z),$$ 
where $\left|\vv{k}^2\right|=k_0^2=k_x^2+k_y^2+k_z^2$. 
This satisfies \cref{scwe} when $\vv{k}\perp\ev$ and
\be \label{gpws} 
\ec_j=E_c e^{i(\vv{k}\cdot \vv{r}-\omega_0 t)}.
\ee
%

\section{Derivation of the Helmholtz equation} 
We consider time-harmonic solutions to the scalar wave equation \eqref{scwe} of
the form \be \label{cws} 
\mathcal{E}_j(x,y,z,t)=e^{i\ww t}E(x,y,z) + \text{c.c}, \ee
%
which are continuous wave beam solutions as opposed to pulsed output beams.
The continous beam has (approximately) constant power, whereas pulsed beams can
reach higher peak powers. For more information on the operating principles of
lasers, we refer to \cite{siegm}.

Substituting \eqref{cws} in \cref{scwe} and taking the derivatives leads to the
expression
\begin{gather*}
\Delta\left(e^{-i\ww t} E\right) = \frac{1}{c^2} \frac{\partial^2}{\partial t^2}
\left( e^{-i\ww t} E \right) \\
e^{-i\ww t}\Delta E = \frac{1}{c^2} (-i\ww)^2 E e^{-i\ww t},
\end{gather*}
%
where we can divide by $e^{-i\ww t}\neq0$ and use the dispersion relation
\eqref{disp} to arrive at the scalar linear Helmholtz equation for $E$
%
\be \label{lhh} \Delta E(x,y,z)+k_0^2 E = 0. 
\ee 
%
% \begin{gather*}
% \Delta E = -\frac{\ww^2}{c^2} E \implies \Delta E + k_0^2 E = 0,
% \text{ with }k_0^2 = \frac{\ww^2}{c^2}.
% \end{gather*}
%
%
As an example, \cref{lhh} is solved by the general-direction plane
waves \eqref{gpws}, where
$$E=E_c e^{i(k_x x + k_y y + k_z z)}.$$ 
%

\section{Derivation of the Linear Schr\"odinger equation}
% \revgroup
\label{parax}
We write the incoming field $E_0^\text{inc}(x,y)$ as a sum of plane waves. Then
the electric field $E(x,y,z)$ for non-zero $z$-value follows from propagation.
This is the plane wave spectrum representation of the electromagnetic field and
it is essential to Fourier optics. We have
%
\begin{gather*} 
E_0^\text{inc}(x,y) = \frac{1}{2\pi}\int_D E_c(k_x,k_y)\, e^{i(k_x x+k_y y)}
\mathrm{d}k_x\mathrm{d}k_y,~\text{ such that }\\ 
E(x,y,z) = \frac{1}{2\pi}\int_{\mathbb{R}^2} E_c(k_x,k_y)\,e^{i(k_x x+k_y y+
\sqrt{k_0^2-k_x^2-k_y^2}\,z)}\mathrm{d}k_x\mathrm{d}k_y,
\end{gather*}
%
where $D$ denotes the (circular) laser input beam domain. 
For laser beams oriented in the $z$-direction, most of the plane wave modes 
are nearly parallel to the $z$-axis, which implies $k_z\approx k_0$. We define
$k_\perp^2 = k_x^2 + k_y^2$, such that $k_0^2 = k_\perp^2 + k_z^2$. It is
equivalent to $k_0\approx k_z$ to say that $k_\perp \ll k_z$. 
% These paraxial plane waves satisfy 
% \begin{gather*}
% k_\perp^2 << k_z^2,\quad k_\perp^2 = k_x^2 + k_y^2,\\
% k_0^2=k_x^2 + k_y^2 + k_z^2 =k_\perp^2 + k_z^2,\\
% \implies k_0^2\approx k_z^2.
% \end{gather*}% 04052020B

This motivates studying solutions of the form
\be \label{cwz}
E=e^{ik_0z}\psi(x,y,z)
\ee
%
where $\psi(x,y,z)$ is an envelope (or amplitude) function. The envelope shape
may vary over $z$, in contrast to soliton solutions, see \eqref{psol}. 
%\endgroup

Substituting \eqref{cwz} into the Helmholtz \cref{lhh} yields
%
\be \label{phh}
\psi_{zz}(x,y,z) + 2ik_0\psi_z + \Delta_\perp\psi = 0,
\ee
%
where $\Delta_\perp=\frac{\partial^2}{\partial x^2}
+\frac{\partial^2}{\partial y^2}$ such that $\Delta = \Delta_\perp + 
\frac{\partial^2}{\partial z^2}$. Basically, this is the Helmholtz equation for
the envelope function $\psi(x,y,z)$. Remember that for lasers beams oriented in
the $z$-direction, the wavenumber $k_z$ dominates over $k_\perp$ such that 
$k_0\approx k_z$. The envelope function $\psi(x,y,z)$  will vary slowly 
in $z$ and curve even more slowly. 

Claim: $\abs{\psi_{zz}}\ll k_0\abs{\psi_z}$ and 
$\abs{\psi_{zz}}\ll \Delta_\perp\psi$. 

%\revgroup 
To see this, we first show that $k_0-k_z \ll 1$.  We factor out $k_0^2$, take the
square root on both sides and linearise the square root term of the right hand
side: 
% 04052020E
% 04052020G
% \begin{gather*}
$$ k_z^2=k_0^2+\kp^2=k_0^2\left(1-\frac{\kp^2}{k_0^2}\right)
\implies
k_z=k_0\left(1-\frac{\kp^2}{k_0^2}\right)^\frac{1}{2}
\approx k_0\left(1-\frac{1}{2}\frac{\kp^2}{k_0^2}\right). $$
Finally, we use $k_\perp\ll k_0$ to obtain the intermediate result:
$$ k_0-k_z \approx k_0 - k_0 + \frac{1}{2}\frac{\kp^2}{k_0}=
\frac{1}{2}\frac{\kp^2}{k_0}\ll1. $$

For the first statement of the claim, $\abs{\psi_{zz}}\ll k_0\abs{\psi_z}$, it is equivalent
to show that the ratio of $\abs{\psi_{zz}}$ over $k_0\abs{\psi_z}$ is much smaller than 1. 
We calculate the ratio as follows:
$$
\frac{\left[\psi_{zz}\right]}{\left[k_0\psi_z\right]}=
\frac{\left(k_0-k_z\right)^2 E_c}{k_0\left(k_0-k_z\right)E_c}=
\frac{k_0-k_z}{k_0}=\frac{\kp}{k_0}\approx
\frac{1}{2}\frac{\kp^2}{k_0}\cdot\frac{1}{k_0}\ll1.
$$
For the other statement of the claim, we calculate:
$$
\frac{\left[\psi_{zz}\right]}{\left[\Delta_\perp\psi_z\right]}=
\frac{\left(k_0-k_z\right)^2 E_c}{\kp^2 E_c}=
\frac{\left(k_0-k_z\right)^2}{\kp^2}\approx
\frac{1}{\kp^2}\left(\frac{1}{2}\frac{\kp^2}{k_0}\right)=
\frac{1}{4}\frac{\kp^2}{k_0^4}\ll\frac{1}{4}\frac{\kp^2}{k_0^2}\ll1.
$$
%\endgroup
Using the approxations in \cref{phh} yields the
linear Schr\"odinger equation:
\be \label{lse}
2ik_0\psi_z + \Delta_\perp\psi = 0.
\ee

\section{Polarisation field}
Polarisation describes the influence of an electric field on the centers of the
electrons of the medium. In our consideration, the medium is isotropic and homogenous. The polarisation field $\vv{P}$ contributes to the induction eletric field
$$\dv = \epsilon_0 \ev + \pv.$$ 
%which is then sometimes written as $$\dv = \epsilon\ev.$$
%
In the following, we assume that the electric field is linearly polarised, such
that
$$\ev=(\mathcal{E},0,0),~\pv=(\mathcal{P},0,0),~\dv=(\mathcal{D},0,0),$$
%
Furthermore, we assume that $\mathcal{E}$ is the continuous wave 
electric field from \eqref{cws}.
%$$\mathcal{E}(x,y,z,t)=e^{-i\omega_0t}E(x,y,z,t)+\text{c.c.}$$
We write the Taylor expansion of the polarisation field $\pc=c\ec$ as:
\be \label{ptay} 
%\pc = c_0 + c_1\pc + c_2\pc ^2 + c_3\pc^3 + c_4\pc^4 + c_5\pc ^ 5 +
%\text{h.o.t.},
\pc = c_0 + c_1\pc + c_2\pc ^2 + c_3\pc^3 + c_4\pc^4 + c_5\pc ^ 5 + \bigo{\pc^6}
\ee
where the $c_i$ are real for all $i$. Note that $c_0=0$ except in ferro-electric
materials. The constants $c_i$ are actually a function of the frequency
$\ww$. We rewrite $c_i=\epsilon_0\chi^{\text{(i)}}(\ww)$, where $\chi^\text{(i)}$
is the $i$-th order susceptibility. Then \cref{ptay} reads:
\be \label{ptas}
\pc = \epsilon_0\chi^\text{(1)}\mathcal{E} + \epsilon_0\chi^\text{(2)}\mathcal{E} ^2 + \epsilon_0\chi^\text{(3)}\mathcal{E}^3 + \epsilon_0\chi^\text{(4)}\mathcal{E}^4 + \epsilon_0\chi^\text{(5)}\mathcal{E} ^ 5 + \bigo{\pc^6}
\ee
%
First we consider linear polarisation:
$$\pl = \epsilon_0\ca(\ww)\mathcal{E}.$$
% where $\ca$ is the first-order optical susceptibility, still a function of
% the frequency $\ww$. 
Then the induction electric field $\dc$ is given by:
$$\mathcal{D}=
\epsilon_0\mathcal{E}+\mathcal{P}_{\text{lin}} =
\epsilon_0\ec + \epsilon_0\ca(\ww)\ec = 
\epsilon_0\ec \left( 1 + \ca(\ww) \right) =
\epsilon_0 n_0^2(\ww)\mathcal{E},$$
where $n_0^2(\ww)\coloneqq 1 + \ca(\ww)$ is the linear index of 
refraction (or refractive index) of the medium. 
%
%05052020F
% With linear polarisation $\pl = \epsilon_0 \ca(\ww) \ec$ we get induction
% electric field $$ \dc = \epsilon_0 \left(\ec + \ca(\ww) \ec\right) =
% \epsilon_0 n_0^2(\ww) \ec, $$
% with $n_0^2(\ww) \coloneqq 1 + \ca(\ww).$ This leads to a Helmholtz equation
% that takes linear polarisation into account. See \cref{lhh} with $k_0^2 =
% \frac{\ww^2}{c^2}n_0^2$. 

With this updated induction electric field $\dc=\epsilon_0 n_0^2(\ww)\ec$, we
can update the scalar wave equation and Helmholtz equation. Only the dispersion
relation is affected by considering linear polarisation:
% $\ec(x,y,z,t) = e^{i\ww t} E(x,y,z)$. 
%
\be \label{ldisp} 
k_0^2 = \frac{\ww^2}{c^2}n_0^2(\ww).
\ee
%
We now consider the nonlinear polarisation field $\pn$ 
as the difference between the true polarisation and the linear approximation:
$$\pc=\pl+\pn.$$
%
In an isotropic medium, the relation between $\pc$ and $\ec$ should be
same in all directions. Replacing $\pc$ and $\ec$ by $-\pc$ and $-\ec$
respectively, 
%
\begin{gather*}
-\pn = \epsilon_0\chi^\text{(2)}\left(-\mathcal{E}\right) ^2 +
\epsilon_0\chi^\text{(3)}\left(-\mathcal{E}\right)^3 +
\epsilon_0\chi^\text{(4)}\left(-\mathcal{E}\right)^4 +
\epsilon_0\chi^\text{(5)}\left(-\mathcal{E}\right) ^ 5 + \bigo{\pc^6}\\
-\pn = \epsilon_0\chi^\text{(2)}\mathcal{E} ^2 -
\epsilon_0\chi^\text{(3)}\mathcal{E}^3 +
\epsilon_0\chi^\text{(4)}\mathcal{E}^4 -
\epsilon_0\chi^\text{(5)}\mathcal{E} ^ 5 + \bigo{\pc^6},
\end{gather*}
%
where we see that for the even exponents, the negative signs cancel. Hence, the
even terms cannot contribute to $\pn$ and we have only the odd terms: 
\be \label{pno}
\pn = \epsilon_0\chi^\text{(3)}\mathcal{E}^3 +
\epsilon_0\chi^\text{(5)}\mathcal{E} ^ 5 + \bigo{\pc^7}
\ee
%
The leading-order term is called the Kerr nonlinearity:
% \footnote{Fibich does not give the $\epsilon_0$, but other sources do e.g. 
% \cite{butc}.}
\be
\label{kerr} \pn \approx \epsilon_0\ca(\ww)\ec^3.
\ee
% \todogroup Check in notes if the $\epsilon_0$ is actually there. Continue
% writing about nonlinear (Kerr) polarisation, e.g. about Kerr mediums and about
% the nonlinear Helmholz that follows. And that this Kerr nonlinearity is weak.
% \endgroup

\section{Implications of nonlinear polarisation}
Substituting the continuous wave electric field \eqref{cws} into 
\cref{kerr} yields
$$ \pn \approx \epsilon_0 \cc(\ww) \ec^3 = 
3 \cc(\ww) \left|E\right|^2 E e^{i\ww t} + 
\cc(\ww) E^3 e^{3i\ww t} + \text{c.c.},$$
%04052020(-1)
%
where the second term has a frequency of $3\ww$ (third harmonic). 
This has almost no contribution due to the phase-mismatch with the first harmonic.
Hence, we approximate
%
$$ \pn \approx 3 \epsilon_0 \cc(\ww) \left|E\right|^2 E e^{i\ww t} 
+ \text{c.c.} = 3 \epsilon_0 \cc(\ww) \ec. $$

Then we simplify $\pn$ by defining 
$$ n_2\coloneqq \frac{3\cc}{4\epsilon_0 n_0},$$
so that we obtain the simplified expression
$$\pn = 4\epsilon_0 n_0 n_2 \left|E\right|^2\ec.$$
%
This allows us to write the induction electric field $\dc$ as,
$$ \dc = \epsilon_0 \ec + \pl + \pn = \epsilon_0 n^2 \ec,$$
%
where 
$$n^2 = n_0^2 \left( 1 + \frac{4n_2}{n_0} \left|E\right|^2 \right) =
n_0^2 + 3\cc(\ww)\frac{1}{\epsilon_0} |E|^2.$$

For water, $n_2\sim10^{-22}$ which justifies neglecting nonlinear effects.
With lasers, the nonlinear effect becomes more relevant, but is still weak. For a
typical continuous wave laser with $|E|\sim10^9$, we still have a weak
nonlinearity, as $n_2|E|\sim 10^{-4}\ll n_0\approx 1.33$.
%
% Since the nonlinearity is weak, we can rewrite the expression for $n$ as 
% %$$n = n_0\left(1 + 4\frac{n_2}{n_0} |E|^2\right)$$ 
% $$n = n_0\sqrt{1 + 4\frac{n_2}{n_0} |E|^2} \approx n_0 + 2n_2|E|^2$$ 
% by linearisation of the square root.

We update \cref{lhh} to the scalar nonlinear Helmholtz equation (NLH):
\be \label{nlh}
\Delta E(x,y,z) + k^2 E = 0,\quad\text{where }
k^2 = k_0^2 \left( 1 + \frac{4n_2}{n_0} |E|^2 \right).
\ee
%
We write $E(x,y,z)$ as the product of the $z$-propagation 
and an envelope function $\psi(x,y,z)$:
$$E = e^{ik_0 z}\psi$$ 
%
and substitute in \eqref{nlh} to obtain:
\be \label{nlhcw} 
\psi_{zz} + 2ik_0 \psi_z + \Delta_\perp \psi + 4 k_0^2 \frac{n_2}{n_0}
\abs{\psi}^2\psi = 0.
\ee
%
Just as in \cref{parax}, we apply the paraxial approximation, since
for laser beams oriented in the $z$-direction, we have 
$\abs{\psi_{zz}}\ll k_0\abs{\psi_z}, \abs{\psi_{zz}}\ll \Delta_\perp\psi$.
We finally obtain the nonlinear Schr\"odinger equation (NLS):
\be \label{nls}
2ik_0\psi_z(z,\bar{x})+\Delta_\perp\psi+k_0^2\frac{4n^2}{n_0}\abs{\psi}^2\psi=0.
\ee

\section{Soliton solutions}
The NLS \cref{nls} can be written as a dimensionless equation. Starting from 
\cref{nlhcw}, we apply the rescaling of coordinates
$(x,y,z)\to(\tilde{x},\tilde{y},\tilde{z})$ defined by:
$$ \tilde{x} = \frac{x}{r_0}\quad \tilde{y}=\frac{y}{r_0}
\quad \tilde{z} = \frac{z}{2L_\text{diff}}, $$
%
where $r_0$ is the input beam width and $L_\text{diff}$ is the diffraction
length. We refer to chapter 2 of \cite{fibg} for more information on the
geometrical optics of lasers. There, we also find that 
$L_\text{diff}=k_0\cdot r_0^2$. To rescale $\tilde\psi$, we define:
$$\tilde{\psi}=\frac{\psi}{E_c},\quad\text{where }
E_c\coloneqq\underset{x,y}{\max}\left|\psi_0(x,y)\right|.$$
%
Through the rescaling we obtain the dimensionless NLH for $\tilde\psi$:
$$\frac{f^2}{4}\tilde{\psi}_{\tilde{z}\tilde{z}}(\tilde{z},\tilde{x},\tilde{y})
+ i\tilde{\psi}_{\tilde{z}} +\Delta_\perp\tilde{\psi} 
+ \nu\left|\tilde{\psi}\right|^2\tilde{\psi}=0,$$
that depends on a nonparaxiality parameter $f$
and a nonlinearity parameter $\nu$:
$$f = \frac{1}{r_0 k_0} = \frac{r_0}{L_\text{diff}},\quad
\nu = r_0^2 k_0^2 \frac{4n_2}{n_0}E_c^2.$$
%
% where $\Delta_\perp = \frac{\partial^2}{\partial \tilde{x}^2} +
% \frac{\partial^2}{\partial \tilde{y}^2}$ and $\nu = r_0^2 k_0^2 \frac{4n_2}{n_0}
% E_c^2$ is the nonlinearity parameter and $f = \frac{1}{r_0 k_0} =
% \frac{r_0}{L_\text{diff}}$ is the nonparaxiality parameter. Remember that $2\pi
% f = \frac{\lambda}{r_0}$.
%
Here the approximation of paraxiality is valid for small $f\ll1$ 
and this leads to  
% $f^2\tilde{\psi}_{\tilde{z}\tilde{z}}<<\psi_{\tilde{z}\tilde{z}}$ and we obtain
the dimensionless NLS \cref{dnls}, where the tildes have been dropped for brevity.
%
\be \label{dnls}
i\psi_z(z,x,y) + \Delta_\perp\psi + \nu\left|\psi\right|^2\psi = 0.
\ee
% 
% depending on $\nu$ only, not on $f$. For weak nonlinearity we have $\nu\sim 1$.
% Suppose that the solution does not change shape when propagating. This is called
% a soliton solution. 
Radial solitary-wave solutions to \eqref{dnls} were considered in
\cite{chia} with $\psi$ of the form: 
\be \label{psol} 
\psi_\omega^\text{solitary}(r,z) = e^{i\omega z}R_\omega(r),
\ee
%
where $\omega$ is a real number and $R_\omega$ is the real solution of
$$ -\omega R_\omega + \Delta_\perp R_\omega(r) + R_\omega^3 = 0. $$
%
This can be solved in general by, for example,
$$ R_\omega(r) = \sqrt{\omega}R\left(\sqrt{\omega}r\right). $$
However, taking $\omega=1$ leads to the simplest soliton equation
\be \label{sol}
R''(r) + \frac{1}{r} R' - R + R^3 = 0,\quad 0<r<\infty,
\ee
%
subject to initial condition $R'(0)=0$ and integrability condition
$\underset{r\to\infty}{\lim} R(r) = 0$. The (numerical) solution is known as the
Townes profile, which is positive and monotonically decreasing in $r$.

% \references{dissertation}

