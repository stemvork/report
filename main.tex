\documentclass[11pt,a4paper]{amsart}
\usepackage{amsfonts, amsthm}
\usepackage{hyperref}
\usepackage[english]{babel}
\usepackage{amsmath}
\usepackage{amssymb}
\usepackage{url}
\usepackage{amsxtra}
\usepackage{lineno}
\usepackage{mathrsfs}
\usepackage{fancyhdr}
\usepackage{enumerate}
\usepackage{graphicx}
\usepackage{pdfsync}
\usepackage{psfrag}
\usepackage{tikz}
\usepackage{soul}
\usepackage{listings}
\usepackage{cleveref}
\usepackage{empheq}
\usepackage{marvosym}

\usepackage{markdown}

\tikzstyle{nodino}=[circle,draw,fill,inner sep=0pt,minimum size=0.5mm]
\tikzstyle{infinito}=[circle,inner sep=0pt,minimum size=0mm]
\tikzstyle{nodo}=[circle,draw,fill,inner sep=0pt,minimum size=0.5*\widthof{k}]

\newcommand{\blu}[1]{{\color{blue}#1}}
\newcommand{\ros}[1]{{\color{red}#1}}
\newcommand{\nero}[1]{{\color{black}#1}}
\newcommand{\verde}[1]{{\color{green}#1}}
\newcommand{\magenta}[1]{{\color{magenta}#1}}
\newcommand{\ciano}[1]{{\color{cyan}#1}}

%%%%%%%%%%%%%%%%%%%%%%%%%%%
% DOCUMENT LAYOUT
%%%%%%%%%%%%%%%%%%%%%%%%%%%
\setlength{\textwidth}{16cm}
\setlength{\textheight}{24cm}
\setlength{\oddsidemargin}{0cm}
\setlength{\evensidemargin}{0cm}
\setlength{\marginparwidth}{2cm}
\hoffset=0truecm
\voffset=-1.5truecm
\footskip = 30pt
\marginparsep=-0.2cm

%%%%%%%%%%%%%%%
%theorem's style
%%%%%%%%%%%%%%%	
\newtheorem{theorem}{Theorem}
\newtheorem{prob}{Problem}
\newtheorem{lemma}{Lemma}[section]
\newtheorem{proposition}[lemma]{Proposition}
\newtheorem{corollary}{Corollary}
\newtheorem{definition}[lemma]{Definition}
\newtheorem{assumption}{Assumption}

\theoremstyle{definition}
\newtheorem{remark}[lemma]{Remark}

%%%%%%%%%%%%%%%
%numbering equations
%%%%%%%%%%%%%%%
\numberwithin{equation}{section}

%%%%%%%%%%%%%%%%%%
%TEMPORARY NEWCOMMANDS
%AND PACKAGES
%%%%%%%%%%%%%%%%%%
%\newcommand{\comment}[1]{}
%\def\sidenote#1{\begin{small}\marginpar{\bf #1}\end{small}}
%\usepackage[notref,notcite]{showkeys}
%\usepackage{refcheck}
%\newcommand{\claudio}[1]{{\bf claudio: }{#1}\\ \noindent}
%\newcommand{\diego}[1]{{\bf diego: }{#1}\\ \noindent}


%%%%%%%%%%%%%%%%%%%%%%%%%%%%%%%
%NEWCOMMANDS
%%%%%%%%%%%%%%%%%%%%%%%%%%%%%%%
\newcommand{\pd}[2]{\frac{\partial {#1}}{\partial {#2}}}
\newcommand{\beq}{\begin{equation}}
\newcommand{\eeq}{\end{equation}}
\newcommand{\be}{\begin{equation*}}
\newcommand{\ee}{\end{equation*}}
\newcommand{\n}{\noindent}
\newcommand{\Wbb}{\mathbb{H}^{in}}
%\newcommand{\Tau}{\mathcal{T}}

\newcommand{\vertiii}[1]{{\left\vert\kern-0.25ex\left\vert\kern-0.25ex\left\vert #1 
    \right\vert\kern-0.25ex\right\vert\kern-0.25ex\right\vert}}

\newcommand{\RE}{\mathbb R}
\newcommand{\erre}{\mathbb R}
\newcommand{\CO}{\mathbb C}
\newcommand{\NA}{\mathbb N}
\newcommand{\II}{\mathbb I}
\newcommand{\Obb}{\mathbb O}
\newcommand{\Gbb}{\mathbb G}
\newcommand{\Hbb}{\mathbb{H}}
%\newcommand{\PP}{\mathbb P}
\newcommand{\BB}{\mathscr B}
\newcommand{\DD}{\mathcal D}
\newcommand{\OO}{\mathcal{O}}
%\newcommand{\QQ}{\mathcal{Q}}
\newcommand{\QQ}{Q}
\newcommand{\CC}{\mathcal{C}}
\newcommand{\HH}{\mathbb{H}}
\newcommand{\GG}{\mathcal{G}}
\newcommand{\UU}{\mathcal{U}}
\newcommand{\ulim}{\operatorname{u}-\lim}
\newcommand{\sgn}{\operatorname{sgn}\,}
\newcommand{\supp}{\operatorname{supp}\,}
\newcommand{\Ran}{\operatorname{Ran}\,}
\newcommand{\Ker}{\operatorname{Ker}\,}
\newcommand{\lf}{\left}
\newcommand{\ri}{\right}
\newcommand{\ve}{\varepsilon}
\newcommand{\al}{\alpha}
\newcommand{\ual}{\undeline{\al}}
\newcommand{\bt}{\beta}
\newcommand{\Si}{\Sigma}
\newcommand{\si}{\sigma}
\newcommand{\Ga}{\Gamma}
\newcommand{\ga}{\gamma}
\newcommand{\La}{\Lambda}
\newcommand{\la}{\lambda}
\newcommand{\de}{\delta}
\newcommand{\De}{\Delta}
\newcommand{\ci}{\mathbb{C}}
\newcommand{\wconv}{\xrightarrow{w}}
\newcommand{\ome}{\omega}
\DeclareMathOperator{\sech}{sech}
\DeclareMathOperator{\arctanh}{arctanh}
\providecommand{\ove}[1]{\overline{#1}}
\newcommand{\lan}{\langle}
\newcommand{\ran}{\rangle}
%RENEWCOMMAND
\renewcommand{\Im}{\operatorname{Im}\,}
\renewcommand{\Re}{\operatorname{Re}\,}
\renewcommand{\neg}{\operatorname{Neg}\,}

\renewcommand{\leq}{\leqslant}
\renewcommand{\geq}{\geqslant}

%MORENEWCOMMANDS
\newcommand{\x}{\underline{x}}
\newcommand{\y}{\underline{y}}
\renewcommand{\v}{\underline{v}}
\newcommand{\vd}{\underline{v}_2}
\newcommand{\vu}{\underline{v}_1}
\newcommand{\f}{\frac}
\newcommand{\EE}{\mathcal E}
\newcommand{\C}{\mathbb{C}}
\newcommand{\VV}{R}
\newcommand{\WW}{S}
\newcommand{\ZZ}{Z}
\newcommand{\omestar}{\tilde{\ome}}

\newcommand{\vol}{\operatorname{Vol}}
\newcommand{\one}{{\bf 1}}
\renewcommand{\a}{\underline{a}}
\renewcommand{\b}{\underline{b}}

%%%%%%%%%%%%%%%%%%%%%%%%%%%%%%%%%
%ADDITIONS BY JASPER EENHOORN
%%%%%%%%%%%%%%%%%%%%%%%%%%%%%%%%%
\newcommand{\comm}[1]{}
\setlength\parindent{24pt}
\newcommand{\limrtoinf}{\underset{r\to\infty}{\lim}}
\def\at{
  \left.
  \vphantom{\int}
  \right|
}
%%%%%%%%%%%%%%%%%%%%%%%%%%%%%%%%%
%AUTHOR AND TITLE
%%%%%%%%%%%%%%%%%%%%%%%%%%%%%%%%%
\title[]{On existence and uniqueness of ground state solutions to NONLINEAR SCHR\"ODINGER EQUATION}

\author[]{Jasper Eenhoorn}
\address{Department of Applied Physics, TU Delft, Lorentzweg 1, 2628CJ, Delft, Netherlands, EU}
\email{j.s.eenhoorn@student.tudelft.nl}%

\date{}

%\begin{document}

%\thanks{}

\begin{document}
%\linenumbers
%\begin{abstract} \comm{We consider a  nonlinear Schr\"odinger equation (NLS) posed on a graph or network composed of a generic compact part to which a finite number of half-lines are attached. We call this structure a starlike graph. At the vertices of the graph %structure
%interactions of $\delta$-type can be present and an overall external potential is admitted. Under general assumptions on the potential, we prove that the NLS  is globally well-posed in the energy domain.
%
%We are interested in minimizing the energy of the system on the manifold of constant mass ($L^2$-norm). When existing, the minimizer
%%minimum
%is called ground state and it is the profile of an orbitally stable standing wave for the NLS evolution. We prove that a ground state exists for sufficiently small masses whenever the  quadratic part of the energy admits a simple isolated eigenvalue at the bottom of the spectrum (the linear ground state). This is a wide generalization of a result previously obtained for a star graph with a single vertex.
%The main part of the proof is devoted to prove the concentration compactness principle for starlike structures; this is non trivial
%%obvious 
%due to the lack of translation invariance of the domain. Then we show that a minimizing bounded $H^1$ sequence for the constrained NLS energy with external linear potentials is in fact convergent if its mass is small enough. Examples are provided with discussion of hypotheses on the linear part.}
%\end{abstract}

\begin{abstract}
  In this review, existence and uniqueness of ground state solutions to an NLS-like equation is studied. The nonlinear Schr\"odinger equation (NLS) is the paraxial approximation of the nonlinear Helmholtz equation (NLH). 
\end{abstract}

\maketitle
%\comm{
%\begin{footnotesize}
% \emph{Keywords:} Quantum graphs; non-linear Schr\"odinger equation; concentration-compactness techniques. 
% 
% \emph{MSC 2010:}  35Q55, 81Q35, 35R02.  
% %35Q55  	NLS-like equations (nonlinear Schrödinger) 
% %81Q35  	Quantum mechanics on special spaces: manifolds, fractals, graphs, etc.
%% 35R02  	Partial differential equations on graphs and networks (ramified or polygonal spaces)
% %37K50  	Bifurcation problems
% \end{footnotesize}}

\section{Introduction}

- PDEs and engineering
  - partial differential equations and ordinary differential equations
  - diverse examples of equations solved in engineering context
  - review of types of pde and ode
    - homogenous or inhomogenous
    - linear or nonlinear
    - initial values and boundary values
      - conversion between bv problems and iv problems
      - TODO: is our problem an iv formulation of bv problem?
  - discussion of nls
- Analytical and numerical
  - to determine solutions, often numerical methods
  - some problems have analytical solutions
  - existence and uniqueness are analytical problems
  - analysis of problem as part of solving problem
- Optics and physics
  - principles of optics
    - maxwell equations for em waves
    - light regarded as em wave
    - geometric optics and diffraction optics
    - components of em field and coupling
    - helmholtz: decoupled components of em field
    - helmholtz solving and history
    - principles of bridge to nls
  - derivation of nls from first principles
    - TODO: sketch steps of derivation
  - relation to self-focusing research (history)
  - physical interpretation of our problem
- Solutions and mathematics
  - TODO: what about solutions and existence and mathematics
- Existence and uniqueness
  - definitions of existence and uniqueness
    - existence: there is at least one solution to the problem, possibly infinite
    - uniqueness: there is at most one solution to the problem, if it exists
  - methods of proving existence and uniqueness
  - e. and u. and well-posedness
- Ground states
  - TODO: look into eigenvalues and eigenfunctions, ground state is first eigenfunction, or solution corresponding to first eigenvalue of the operator applied to the problem (iv bv problem)

% ...  This study comprised three parts regarding the existence and uniqueness of ground state solutions to the Nonlinear Schr\"odinger equation (NLS). The report should clarify these results on the level of an interested student with a bachelor in physics or mathematics. {\color{lightgray}In the remainder of this introduction, these results are named and an overview of the report is given. They are named in chronological order of when they were published, which coincides with the order in which they were reviewed for this report.}

%Firstly, the publication of Berestycki, Lions and Peletier [ref] on the existence of ground state solutions to the problem (IVP). In this paper an existence theorem is stated and proved by a shooting argument. It is split into two main lemmata.

%Secondly, the publication of Genoud [ref] (and Kwong [ref]) on the uniqueness of ground state solutions to the same problem. To prove uniqueness, the set of ground state solutions needs to be a singleton (set with one element). This proof is by eight lemmata.

% Lastly, the book on NLS by Fibich discusses both the mathematics and the physics of the derivation and setting of the problem as it arises in nonlinear optics.

% This report discusses the above in a different order. First, an introduction (Chapter \ref{phy}) to the physics of laser beams and the NLS. Then, in Chapter \ref{not} the differential equation is stated and the notation of initial conditions, solutions and solution sets is clarified. These were different among the two papers and the book. For this report, the same notation will be used in all chapters. In Chapter \ref{exi} the existence theorem is stated and a proof is given. The uniqueness theorem is stated in Chapter \ref{uni} and proven through various lemmata. Lastly, Chapter \ref{con} ...

\section{Physics of NLS}
Physics - Chapter 1
NLS as leading-order model for propagation of intense laser beams in isotropic bulk medium. A laser pulse is an electromagnetic wave and is governed by Maxwell’s equations.
Deriving the wave equation from Maxwell is a classic result: Lemma 1.1.
Each component of the electric field satisfies a scalar wave equation and has plane waves as solution with certain dispersion relation. More generally, these plane waves are expressed … with c.c. the complex conjugate.
Linear polarisation of a plane wave. Actually, laser pulse is not linearly polarised, but the leading-order approximation of the laser pulse is linearly polarised.
Now laser beams are (time harmonic) continuous wave solutions to the scalar wave equation. And these yield the scalar linear Helmholtz equation. Plane waves are solutions to a specific scalar linear Helmholtz equation.
Laser beam propagating in $z$-direction, $E_0^inc$ known at $z=0$. Unlike the plane wave, the laser beam electric field decays to zero as the distance from the beam axis goes to infinity. Nevertheless, it can be expressed as a linear superposition of plane waves: and has a corresponding right-propagating solution that is a solution to the Helmholtz equation (each mode is a plane wave, equal to $E_0^inc$ at $z=0$ and propagating in positive $z$-direction. Most of the modes travel parallel to the $z$-axis and are called paraxial plane waves that satisfy .... The favoured solutions to the Helmholtz equation are travelling in the $z$-direction and have electric field envelope that has its own Helmholtz equation with a diffraction term. For example, plane waves can be expressed ... and have an envelope slowly varying as a function of $z$, compared to the carrier oscillations. Lemma 1.2

If the paraxial laser beam mostly consists of paraxial plane waves, suggesting to neglect the ... , and this paraxial approximation yields a linear Schrödinger equation for the envelope. Worth noting: this is a SE arising from classical physics. Also: neglecting the highest-order derivative changes the problem from a boundary problem (Helmholtz) to an initial value problem (Schrödinger). Nevertheless, the results of the approximation agree with experiments. When the electric field is applied to a dielectric medium it induces an additional electric field called polarisation field. There are three mechanisms contributing. To simplify, the electric field is assumed to be linearly polarised and continuous wave, and the dielectric medium is isotropic and homogenous. 

LINEAR, WEAKLY NONLINEAR, KERR NONLINEAR, VECTORIAL KERR NONLINEAR, WEAKLY NONLINEAR HELMHOLTZ, NONLINEAR SCHRODINGER, RELATIVE MAGNITUDE OF KERR, VALIDITY OF NLS.



Chapter 2
Laser beams in a Kerr medium can become narrower with propagation (self-focusing) and is studies through geometric approximation. Start with the dimensional linear Helmholtz equation. Geometric approximation is valid when changes occur over distances much greater than the wavelength, so locally the medium is homogenous. Then create the dimensionless linear Helmholtz. There is the Eikonal equation. This is a nonlinear first-order PDE. The rays (...) are determined by the system of six linear ODEs. Lemma 2.1 states that under the geometric approximation, the rays of the Eikonal equation are perpendicular to the wavefronts of the Helmholtz equation. Fermats principle of least time was refined to the principle of stationary time. Rays connecting two points are extremals of the Traveltime functional. (Geometric approximation -- ray description of light propagation.) Transport equation implies that the power of a ray bundle at a cross-section is constant. This implies that a focused beam has all beam power in a focal point.

\begin{enumerate}
\item PARAGRAPH 2 IS MANY APPLICATIONS, ALSO WAVEGUIDES.
\item PARAGRAPH 3 IS COLLIMATED BEAMS.
\item PARAGRAPH 4 IS FUNDAMENTAL SOLUTION TO HELMHOLTZ.
\item PARAGRAPH 5 IS HELMHOLTZ FOCUSED BEAMS.
\item PARAGRAPH 6 IS SINGULARITY IN HELMHOLTZ.
\item PARAGRAPH 7 IS GLOBAL EXISTENCE IN THE LINEAR HELMHOLTZ.
\item PARAGRAPH 8 IS REPRESENTATION OF INPUT BEAMS.
\item PARAGRAPH 9 IS GEOMETRICAL ANALYSIS OF PARAXIAL PROPAGATION.
\item PARAGRAPH 10 IS ARREST OF LINEAR COLLAPSE BY DIFFRACTION (GAUSSIAN).
\item PARAGRAPH 11 IS DIFFRACTION LENGTH.
\item PARAGRAPH 12 IS DIMENSIONLESS LINEAR SCHRODINGER.
\item PARAGRAPH 13 IS FOURIER TRANSFORM.
\item PARAGRAPH 14 IS NORMS.
\item PARAGRAPH 15 IS LINEAR SCHRODINGER ANALYSIS.
\item PARAGRAPH 16 IS NON-DIRECTIONALITY.
\item PARAGRAPH 17 IS LINEAR AND NONLINEAR SINGULARITIES.
\end{enumerate}

Chapter 3
Early Self-Focusing Research (Include 3.1, 3.2, 3.3, 3.4, Omit 3.5 and 3.6 and Include 3.7)

\section{Solution sets and notation}\label{not}
The previous chapter leads to the following initial value problem (IVP). \underline{What is $u$?} Another important identity is achieved by integrating the IVP over $r$. \underline{The solution sets are disjoint.}
\begin{gather*}-u''(r)-\frac{N-1}{r}u'(r)=g(u(r))\tag{IVP}\label{ivp}\\ u(0)=\alpha,~u'(0)=0\end{gather*}
\begin{equation}\label{ivpint}
\end{equation}

The possible initial values $u(0)=\alpha>0$ can be categorised in three solution sets: solutions that become negative ($N$), solutions that are positive ($P$) for all $r$ and solutions that vanish ($G$, for ground state). In terms of zeroes, solutions with initial condition in $N$ have at least one zero, solutions with initial condition in $P$ have no zeroes and solutions with initial condition in $G$ tend to zero for $r$ to infinity. Formally:
\begin{gather*}
  N = \{\alpha>0: \text{there exists} r>0 \text{such that} u(r;\alpha)=0\} \\
  G = \{\alpha>0: u(r;\alpha)>0 \text{for all} r>0 \text{and} \underset{r\to\infty}{\lim}u(r;\alpha)\}\\
  P = \{\alpha>0: u(r;\alpha)>0 \text{for all} r>0 \text{and} \alpha\notin G\}
\end{gather*}

\newtheorem{thmx}{Theorem}
\renewcommand{\thethmx}{\Alph{thmx}}
\section{Existence of ground state}
<<<<<<< HEAD
The Maxwell equations and ++certain approximations yield the initial value problem ++
=======
The Maxwell equations and ++certain approximations yield the initial value problem ++   
>>>>>>> e03346cd93b5d1395328085326f20581e3361581

... yields the initial value problem (IVP) ++ ... Question: is this model well-posed? Firstly, does there exist at least one solution? Secondly, does the model have one solution at most? Lastly, does the solution change continuously with the initial conditions? Each of these aspects is formalised in a theorem. Rigorous proofs of these theorems are studied and elaborated upon. In ++, existence (of solutions) is proven under certain conditions. In ++, uniqueness (of solutions) is proven for slighty different conditions. For continuous dependence on the initial conditions, refer to ++.

The existence of a ground state solution to \eqref{ivp} is guaranteed under certain conditions on $g$, which is the nonlinear term in the initial value problem. In this chapter the existence theorem and setting for this problem are stated and a proof will be given based on two lemmas. Note that $\mathbb{R}_+=[0,\infty)$.

<<<<<<< HEAD
\emph{Some remarks: note that any solution to (IVP) where $g$ satisfies the hypotheses will be defined on $\rplus$. Any solutions with initial condition $\alpha$ such that $g(\alpha) \geq 0$ will be nonincreasing. A solution that has $u(r_0)=0$ and $u'(r_0)\leq0$ will have vanishing (negative) derivative. Any solution that has both $u(r)=0$ and $u'(r)=0$ will be a trivial solution. What about solutions with initial condition $\alpha<\kappa$? Non-oscillation of solutions because of $g(t)=0$ for $t\leq0$?}

=======
>>>>>>> e03346cd93b5d1395328085326f20581e3361581
\begin{thmx}\label{exithm}
Let $g$ satisfy conditions (A1) to (A7) in section \ref{con}. Then there exists a number $\alpha\in(\alpha_0,\lambda)$ such that the solution $u\in C^2(\mathbb{R}_+)$ of \eqref{ivp} has the properties: \begin{gather*}u(r)>0\text{ for }r\in[0,\infty)\\u'(r)<0\text{ for }r\in(0,\infty)\\ \text{and }\lim_{r\to+\infty}u(r)=0.\end{gather*} If, in addition, we assume $g$ satisfies $$\limsup_{s~\downarrow~0}~\frac{g(s)}{s}<0,$$ then there exist constants $C,\delta>0$ such that $$0<u(r)\leq Ce^{-\delta r} ~~ \textnormal{for} ~ r\in[0,\infty).$$
\end{thmx}

\subsection{Discussion of the theorem}\hfill

\underline{Restricted set of initial conditions} In the initial value problem $\alpha\coloneqq u(0)$ and $u'(0)=0$, and $\alpha\in(0,\infty)$ since the solution depends on $r=|x|$ ($x\in\mathbb{R}^N$). That is, the problem allows any positive initial condition, $\alpha=u(\alpha,0)>0$. However, the properties of the nonlinear term allow for a smaller scope of initial conditions. The theorem restricts the initial conditions to $\alpha\in(\alpha_0,\beta)$. This restriction is a by-product of the proof... {\color{red} ELABORATE}

<<<<<<< HEAD
The dimensions $N\geq2$ are considered. For the case $N=1$, see \cite{}.
=======
The dimensions $N\geq2$ are considered. For the case $N=1$, see \cite{}. 
>>>>>>> e03346cd93b5d1395328085326f20581e3361581

\subsection{Conditions on g}\label{con}\hfill

{\color{red}INTRODUCTION}
\begin{gather}\tag{A1}\label{a1}\text{Let }g\text{ be locally Lipschitz continuous from }\mathbb{R}_+\text{ to }\mathbb{R}\text{ with }g(0)=0.\\
\notag\text{In addition, let }g(u)=0\text{ for }u\leq0.\\
%
<<<<<<< HEAD
\tag{A2}\label{a2}\text{Let }\kappa>0\text{ be finite, where }\kappa\coloneqq\inf(\alpha>0,g(\alpha)\geq0).\\
=======
\tag{A2}\label{a2}\text{Let }\kappa>0\text{ be finite, where }\kappa\coloneqq\inf(\alpha>0,g(\alpha)\geq0).\\ 
>>>>>>> e03346cd93b5d1395328085326f20581e3361581
%
\notag\text{Define }G(t)\coloneqq\int_0^t g(s)ds.\\
%
\tag{A3}\label{a3}\text{Let }\alpha_0>\kappa,\text{ where }\alpha_0\coloneqq\inf(\alpha>0,G(\alpha)>0).\\
%
\tag{A4}\label{a4}\text{Let }\underset{s\downarrow\kappa}{\lim}~\frac{g(s)}{s-\kappa}>0.\\
%
\tag{A5}\label{a5}\text{Let }g(s)>0\text{ for }s\in(\kappa,\alpha_0].\\
%
\tag{A6}\label{a6}\text{Let }\lambda\leq\infty,\text{ where }\lambda\coloneqq\inf(\alpha>\alpha_0,g(\alpha)=0).\\
%
\tag{A7}\label{a7}\text{If }\lambda=\infty,\text{ let }\underset{s\to\infty}{\lim}\frac{g(s)}{s^l}=0,\text{ where }\\ \notag l>0\text{ if }N=2,\text{ and }l<\frac{N+2}{N-2}\text{ for }N>2.
\end{gather}
{\color{gray}
<<<<<<< HEAD
Thirdly, note that $g$ is by assumption negative on $(0,\kappa)$ and hence there exists a number $\alpha>0$ such that $G(\alpha)>0$. Here $$G(t)=\int_0^t g(s)ds.$$ Define $\alpha_0\coloneqq\inf(\alpha>0,G(\alpha)>0)$. Then $\alpha_0$ exists by the above and $\alpha_0>\kappa$.
=======
Thirdly, note that $g$ is by assumption negative on $(0,\kappa)$ and hence there exists a number $\alpha>0$ such that $G(\alpha)>0$. Here $$G(t)=\int_0^t g(s)ds.$$ Define $\alpha_0\coloneqq\inf(\alpha>0,G(\alpha)>0)$. Then $\alpha_0$ exists by the above and $\alpha_0>\kappa$. 
>>>>>>> e03346cd93b5d1395328085326f20581e3361581

Fourthly, $g$ satisfies these two conditions regarding $\kappa$ and $g$ on $(\kappa,\alpha_0]$:
\begin{gather}
\label{6}\underset{s\downarrow\kappa}{\lim}~\frac{g(s)}{s-\kappa}>0;\\
g(s)>0~\text{for}~s\in(\kappa,\alpha_0].
\end{gather}

<<<<<<< HEAD
Fifthly, regarding the behaviour of $g$ after $\alpha_0$, the function may remain positive indefinitely or become negative again. In the former case, let $\lambda=+\infty$ and in the latter case, $\lambda\coloneqq\inf(\alpha>\alpha_0,g(\alpha)=0)$. Then by these assumptions, in any case, $\alpha_0<\lambda\leq\infty$.
=======
Fifthly, regarding the behaviour of $g$ after $\alpha_0$, the function may remain positive indefinitely or become negative again. In the former case, let $\lambda=+\infty$ and in the latter case, $\lambda\coloneqq\inf(\alpha>\alpha_0,g(\alpha)=0)$. Then by these assumptions, in any case, $\alpha_0<\lambda\leq\infty$. 
>>>>>>> e03346cd93b5d1395328085326f20581e3361581

Lastly, if $\lambda=\infty$, then $\underset{s\to\infty}{\lim}\frac{g(s)}{s^l}=0$, with $l<\frac{N+2}{N-2}$. Note that this is not defined for $N=2$, in that case, let $l\in\mathbb{R}$.}

{\color{red}REMARKS} Note that the infimum in \eqref{a2} may be infinite. Then $g(u)<0$ for all $u>0$. {\color{red}WHY NOT?} Hence the explicit requirement.

\newtheorem{example}{Example}
\begin{example} The conditions on $g$ are satisfied for $g(u)=-u+u^3.$ {\color{red}COUPLING TO PHYSICS}
<<<<<<< HEAD
% f(x) = -x + x^3, g(x) = (-1/2) x^2 + 1/4 x^4
=======
>>>>>>> e03346cd93b5d1395328085326f20581e3361581
\end{example}

\subsection{The solutions are defined on the semi-definite interval}\hfill\label{semidef}

Let $\alpha\in(\kappa,\lambda)$ and let $u(\alpha,r)$ be the corresponding solution of \eqref{ivp}. That $g$ is locally Lipschitz continuous implies that the solution $u(\alpha,r)$ is defined on some interval $[0,r_\alpha)$. Remember, the solution may have asymptotes. Boundedness of the solution is a sufficient and necessary condition for $r_\alpha=\infty$. Note that the initial condition is positive, $\alpha>0$. Claim: the solution $u(\alpha,r)$ is bounded above by the initial condition, $u(\alpha,r)\leq u(\alpha,0)=\alpha$ for $r\geq0$. To see this, multiply the \eqref{ivp} by $u'$ and integrate between 0 and $r$ to obtain:

\begin{align}\label{ivpint}
-u''-\frac{N-1}{r}u'&=g(u)\nonumber\\
-\int_0^r\left[u'(s)u''(s)\right]ds-\int_0^r\left[\frac{N-1}{s}[u'(s)]^2\right]ds&=\int_0^r\left[u'(s)g(u(s))\right]ds\nonumber\\
\notag\text{Use }\frac{d}{dr}\left[(u'(r))^2\right]=2u'(r)u''(r)&:\\
-\frac{1}{2}[u'(r)]^2-(N-1)\int_0^r[u'(s)]^2\frac{ds}{s}&=\int_0^r g(u(s))\frac{du}{ds}ds=\int_0^r g(u)du\nonumber\\
-\frac{1}{2}[u'(r)]^2-(N-1)\int_0^r[u'(s)]^2\frac{ds}{s}&=G(u(\alpha,r))-G(\alpha)
\end{align}

Note that $u'(0)=0$ in evaluating the integrand from the first term of the integral. Suppose that $u'>0$ for $r$ small. Then $u(\alpha,r)>u(\alpha,0)$ and since $G$ is nondecreasing on $(\kappa,\lambda)$ the right hand side is positive $G(u(\alpha,r))-G(\alpha)>0$. Then
\begin{align*}
-\frac{1}{2}[u'(r)]^2-(N-1)\int_0^r[u'(s)]^2\frac{ds}{s}&>0\\
\frac{1}{2}[u'(r)]^2+(N-1)\int_0^r[u'(s)]^2\frac{ds}{s}&<0\quad\text{\Lightning},
\end{align*}
<<<<<<< HEAD
which is clearly impossible, both terms are positive. Conclusion: the solution is decreasing at first and bounded above by the initial condition. In fact, suppose $u(\alpha,r)\geq\alpha$ for some $r_0>0$, then again $G(u(\alpha,r_0))-G(\alpha)>0$ and the same contradiction is found.
=======
which is clearly impossible, both terms are positive. Conclusion: the solution is decreasing at first and bounded above by the initial condition. In fact, suppose $u(\alpha,r)\geq\alpha$ for some $r_0>0$, then again $G(u(\alpha,r_0))-G(\alpha)>0$ and the same contradiction is found. 
>>>>>>> e03346cd93b5d1395328085326f20581e3361581
%Define $r_1\coloneqq\inf(r>r_0,u(\alpha,r_1)>0)$. By condition (A1), $g(u(r_0))=0$.   {\color{red} How will the derivative evolve now? The derivative will vanish hyperbolically from there. In other words, let $r_0>0$ such that $u(\alpha,r_0)=0$ and $u'(\alpha,r_0)\leq0$, then for $r\geq r_0$:

It remains to show $u(\alpha,r)$ has a lower bound. Let $r_0\coloneqq\inf(r>0,u(\alpha,r_0)=0).$ Suppose $r_0<\infty$. (Note that $r_0=\infty\implies u(\alpha,r)>0$ for all $r>0$, hence $u(\alpha,r)$ is bounded.) If $u'(\alpha,r_0)=0$, then $u(\alpha,r)\equiv0$. Thus $u'(\alpha,r_0)<0$. Claim: the derivative will decay hyperbolically for $r\geq r_0$ as,  $$u'(\alpha,r)=\big(\frac{r_0}{r}\big)^{N-1}u'(\alpha,r_0)\geq u'(\alpha,r_0)$$.

To see this, use condition (A1) in \eqref{ivp},
<<<<<<< HEAD
%(When $u(\alpha,r)\leq0$, $g(u(r))=0$.) To be safe, let $r_1=\inf(r>r_0,u(\alpha,r_1)\geq0)$. Then $u(\alpha,r)\leq0$ on $[r_0,r_1]$ and
$$-u''-\frac{N-1}{r}u'=0,$$%,\text{ for }r\in[r_0,r_1].$$
=======
%(When $u(\alpha,r)\leq0$, $g(u(r))=0$.) To be safe, let $r_1=\inf(r>r_0,u(\alpha,r_1)\geq0)$. Then $u(\alpha,r)\leq0$ on $[r_0,r_1]$ and 
$$-u''-\frac{N-1}{r}u'=0,$$%,\text{ for }r\in[r_0,r_1].$$ 
>>>>>>> e03346cd93b5d1395328085326f20581e3361581
which is valid for $u(\alpha,r)\leq0.$ To be safe, let $r_1=\inf(r>r_0,u(\alpha,r_1)=0)$ and suppose $r_1<\infty$. Then $u(\alpha,r)\leq0$ on $[r_0,r_1]$. Now solve for $u'=u'(\alpha,r)$ on $[r_0,r_1]$:

\begin{align*}
-\frac{d}{dr}u'-\frac{N-1}{r}u'&=0\\
\frac{du'}{dr}&=-\frac{N-1}{r}u'\\
\frac{du'}{u'}&=-\frac{N-1}{r}dr\\
\left.\ln{u'}\right\rvert_{r_0}^r&=\left[(N-1)\ln{r}\right]_r^{r_0}\\
\ln{u'(r)}-\ln{u'(r_0)}&=(N-1)\left[\ln{r_0}-\ln{r}\right]\\
\frac{u'(r)}{u'(r_0)}&=\left(\frac{r_0}{r}\right)^{N-1}\\
u'(\alpha,r)=\big(\frac{r_0}{r}\big)^{N-1}u'(\alpha,r_0)&\geq u'(\alpha,r_0).%\text{ on }[r_0,r_1].
\end{align*}

<<<<<<< HEAD
%This expression is valid for $u(\alpha,r)\leq0$. Let $r_1=\inf(r>r_0,u(\alpha,r_1)=0)$. Suppose $r_1<\infty$. Then $u'(\alpha,r)\leq0$ on $[r_0,r_1]$. But $u(\alpha,r_1)>u(\alpha,r)$ for all $r\in(r_0,r_1)$. A contradiction, hence $u(\alpha,r)<0$ on $[r_0,\infty)$ and the hyperbolic decay holds on $[r_0,\infty)$.
=======
%This expression is valid for $u(\alpha,r)\leq0$. Let $r_1=\inf(r>r_0,u(\alpha,r_1)=0)$. Suppose $r_1<\infty$. Then $u'(\alpha,r)\leq0$ on $[r_0,r_1]$. But $u(\alpha,r_1)>u(\alpha,r)$ for all $r\in(r_0,r_1)$. A contradiction, hence $u(\alpha,r)<0$ on $[r_0,\infty)$ and the hyperbolic decay holds on $[r_0,\infty)$. 
>>>>>>> e03346cd93b5d1395328085326f20581e3361581
It follows that $u'(\alpha,r)\leq0$ on $[r_0,r_1]$. Hence $u(\alpha,r)<0$ on $(r_0,r_1]$, which contradicts the assumption on $r_1$. Thus $r_1=\infty$ and $u(\alpha,r)<0$ for $r>r_0$. Note how $u'(\alpha,r)\uparrow0$ for $r\to\infty$. Then $u(\alpha,r)$ has some lower bound. Since the solution is bounded, it is defined on $(0,\infty)$.

\subsection{The shooting method}\hfill

Now any $\alpha\in I$ is defined on $(0,\infty)$. Also, $g(\alpha)>0$ and therefore $u''(\alpha,0)<0$ by the \eqref{ivp}: $u'(0)=0$ and $-u''(\alpha,0)=g(u(\alpha,0))=g(\alpha)>0$. Then for $r>0$ and small: $u'(\alpha,r)<0$ and $u(\alpha,r)>0$. To analyse the behaviour of the solutions for larger $r$, distinguish two sets of initial conditions: solutions that have a vanishing derivative in some point and are positive up to and including that point, and solutions and vanish in some point, but have negative derivative up to and including that point. {\color{red}These sets are defined below.} If these sets are open, nonempty and disjoint, then there exist elements $\alpha^*\in I$ such that $u(\alpha^*,r)>0$ for all $r\geq0$ and $u'(\alpha^*,r)<0$ for all $r>0$. By \cref{lem} and its proof, the sets have these properties and hence such elements exist. Intuitively, a solution that is positive everywhere and has negative derivative everywhere has a limit for $r$ tending to infinity, and is possibly ground state. However, that also requires that this limit is zero.

<<<<<<< HEAD
\begin{lemma}\label{llemma}
=======
\begin{lemma}\label{llemma} 
>>>>>>> e03346cd93b5d1395328085326f20581e3361581
Let $g$ be locally Lipschitz continuous on $\mathbb{R_+}$ such that $g(0)=0$. Let $\alpha_1\in(0,\infty)$ be such that $u(\alpha_1,r)>0$ for all $r\geq0$ and $u'(\alpha_1,r)<0$ for all $r>0$. Then the number $l=\underset{r\to\infty}{\lim}u(\alpha_1,r)$ satisfies $g(l)=0$. Furthermore, if $g$ satisfies \eqref{6} and $g(\kappa)=0$, then $l\neq\kappa$.
\end{lemma}
\begin{proof}
Let $\alpha_1$ be as assumed in the lemma and let $r$ tend to infinity in the \eqref{ivp}:
\begin{equation}\label{ivplim}\limrtoinf\Big[u''(\alpha_1,r)+\frac{N-1}{r}u'(\alpha_1,r)+g(u(\alpha_1,r))\Big]=0,\end{equation} and note that $l=\limrtoinf u(r)$. To prove the first statement of the lemma: the number $l=\limrtoinf u(r)$ satisfies $g(l)=0$, information about the limits of $u'$ and $u''$ is required. In fact, they both need to converge to zero. Then $g(l)=0$ and it remains to show that $g\neq\kappa$. \underline{Claim:} both $u''\to0$ and $u'\to0$ as $r\to\infty.$
<<<<<<< HEAD
\begin{proof}[Proof of the claim] Remember expression \eqref{ivpint}, where the \eqref{ivp} was multiplied by $u'$ and integrated from 0 to $r$. Now evaluate the limit for $r$ tending to infinity:
=======
\begin{proof}[Proof of the claim] Remember expression \eqref{ivpint}, where the \eqref{ivp} was multiplied by $u'$ and integrated from 0 to $r$. Now evaluate the limit for $r$ tending to infinity: 
>>>>>>> e03346cd93b5d1395328085326f20581e3361581
\begin{align*}
	\underset{r\to\infty}{\lim}\left[-\frac{1}{2}[u'(\alpha_1,r)]^2-(N-1)\int_0^r[u'(\alpha_1,s)]^2\frac{ds}{s}\right]
        &=\limrtoinf\left[G(u(\alpha,r))-G(\alpha)\right] \\
    \underset{r\to\infty}{\lim}~\frac{1}{2}[u'(\alpha_1,r)]^2+(N-1)\underset{r\to\infty}{\lim}\int_0^r[u'(\alpha_1,s)]^2\frac{ds}{s}
        &=G(\alpha_1)-\underset{r\to\infty}{\lim}G(u(\alpha,r))\\
    \underset{r\to\infty}{\lim}~\frac{1}{2}[u'(\alpha_1,r)]^2+(N-1)\int_0^\infty[u'(\alpha_1,s)]^2\frac{ds}{s}
        &=G(\alpha_1)-G(l)\\
    \underset{r\to\infty}{\lim}~\frac{1}{2}[u'(\alpha_1,r)]^2+(N-1)\int_0^\infty[u'(\alpha_1,s)]^2\frac{ds}{s}
        &<\infty
%G(\alpha_1)=G(l)+\underset{r\to\infty}{\lim}\frac{1}{2}\big[u'(\alpha_1,r)\big]^2+(N-1)\int_0^{\infty}\big[u'(\alpha_1,s)\big]^2\frac{ds}{s} \\
%      G(\alpha_1)-G(l)=\underset{r\to\infty}{\lim}\frac{1}{2}\big[u'(\alpha_1,r)\big]^2+(N-1)\int_0^{\infty}\big[u'(\alpha_1,s)\big]^2\frac{ds}{s} \\
%     \text{noting that }G(l)<G(\alpha_1)<\infty,\text{ hence }0<G(\alpha_1)-G(l)<\infty \\
%\comm{\implies\int_0^{\infty}\left[u'(\alpha_1,s)\right]^2\frac{ds}{s}<\infty,~&\underset{r\to\infty}{\lim}[u'(\alpha_1,r)]^2<\infty}
<<<<<<< HEAD
\end{align*}
and thus both terms of the left hand side should be finite, so $u'(\alpha_1,r)$ converges as $r\to\infty.$ Remember now that $u(\alpha_1,r)$ is bounded, so the derivative must converge to 0: $$\underset{r\to\infty}{\lim}u'(\alpha_1,r)=0.$$ Now return to \eqref{ivplim} and use the acquired information:
\begin{align*}
	\limrtoinf\Big[u''(\alpha_1,r)+\frac{N-1}{r}u'(\alpha_1,r)+g(u(\alpha_1,r))\Big]&=0\\
=======
\end{align*} 
and thus both terms of the left hand side should be finite, so $u'(\alpha_1,r)$ converges as $r\to\infty.$ Remember now that $u(\alpha_1,r)$ is bounded, so the derivative must converge to 0: $$\underset{r\to\infty}{\lim}u'(\alpha_1,r)=0.$$ Now return to \eqref{ivplim} and use the acquired information: 
\begin{align*}
	\limrtoinf\Big[u''(\alpha_1,r)+\frac{N-1}{r}u'(\alpha_1,r)+g(u(\alpha_1,r))\Big]&=0\\ 
>>>>>>> e03346cd93b5d1395328085326f20581e3361581
	-\limrtoinf\left[u''(\alpha_1,r)\right]
    -\limrtoinf\left[\frac{N-1}{r}u'(\alpha_1,r)\right]&=\limrtoinf g(u(\alpha_1,r))\\
	-\limrtoinf\left[u''(\alpha_1,r)\right]&=g(l)
\end{align*}

<<<<<<< HEAD
But $g(l)$ is finite since $u$ is bounded and thus $u''$ converges as $r$ tends to infinity. By similar argument the limit is zero. Note that $u'$ is bounded, so $u''$ has to converge to zero. $$\limrtoinf u''(\alpha_1,r)=0.$$ So the claim is valid and $g(l)=0$.\end{proof}

It remains to be shown that if $g$ satisfies \eqref{6} then $l\neq\kappa.$ Suppose to the contrary $l=\kappa$. Then introduce the following substitution:
$$v(r)=r^{(1/2)(N-1)}\left[u(r)-\kappa\right],$$

where $u(r)=u(\alpha_1,r)$. Combining this function, its derivatives and the \eqref{ivp}, one can obtain a differential equation in $v$. This will then be used to argue that $l=\kappa$ can only lead to contradictions when $g$ satisfies \eqref{6} and $g(\kappa)=0$.
=======
But $g(l)$ is finite since $u$ is bounded and thus $u''$ converges as $r$ tends to infinity. By similar argument the limit is zero. Note that $u'$ is bounded, so $u''$ has to converge to zero. $$\limrtoinf u''(\alpha_1,r)=0.$$ So the claim is valid and $g(l)=0$.\end{proof}  

It remains to be shown that if $g$ satisfies \eqref{6} then $l\neq\kappa.$ Suppose to the contrary $l=\kappa$. Then introduce the following substitution: 
$$v(r)=r^{(1/2)(N-1)}\left[u(r)-\kappa\right],$$ 

where $u(r)=u(\alpha_1,r)$. Combining this function, its derivatives and the \eqref{ivp}, one can obtain a differential equation in $v$. This will then be used to argue that $l=\kappa$ can only lead to contradictions when $g$ satisfies \eqref{6} and $g(\kappa)=0$. 
>>>>>>> e03346cd93b5d1395328085326f20581e3361581

Calculate the derivatives of $v$:
\begin{align*}v(r)&=r^{(N-1)/2}\left[u(r)-\kappa\right] \\
v'(r)&=\frac{1}{2}(N-1)r^{(N-3)/2}\left[u(r)-\kappa\right]+r^{(N-1)/2}u'(r)\\
\begin{split}v''(r)&=\frac{1}{4}(N-1)(N-3)r^{(N-5)/2}\left[u(r)-\kappa\right]\\&+\frac{1}{2}(N-1)r^{(N-3)/2}u'(r)+\frac{1}{2}(N-1)r^{(N-3)/2}u'(r)\\&+r^{(N-1)/2}u''(r)\end{split}\\
v''(r)&=\frac{1}{4}(N-1)(N-3)r^{(N-5)/2}\left[u(r)-\kappa\right]+(N-1)r^{(N-3)/2}u'(r)+r^{(N-1)/2}u''(r)
\end{align*} and take out any integer powers of $r$: (then all terms carry $r^{(N-1)/2}$)
\begin{align*}
v(r)&=r^{(N-1)/2}\left[u(r)-\kappa\right] \\
v'(r)&=\frac{1}{2}(N-1)r^{(N-1)/2}r^{-1}\left[u(r)-\kappa\right]+r^{(N-1)/2}u'(r)\\
v''(r)=&\frac{1}{4}(N-1)(N-3)r^{(N-1)/2}r^{-2}\left[u(r)-\kappa\right]+\underline{(N-1)r^{(N-1)/2}r^{-1}u'(r)+r^{(N-1)/2}u''(r)}
\end{align*}
and multiply the \eqref{ivp} by $r^{(N-1)/2}$:
\begin{align*} -u''(r)-(N-1)r^{-1}u'(r)&=g(u(r))\\
-r^{(N-1)/2}u''(r)-(N-1)r^{(N-1)/2}r^{-1}u'(r)&=g(u(r))r^{(N-1)/2}\tag{*}
\end{align*}
to see that the last two (underlined) terms of $v''(r)$ are equal (up to a minus sign) to the left hand side of (*). That means we can write $-g(u(r))r^{(N-1)/2}$ in the expression for $v''(r)$: $$v''(r)=\frac{1}{4}(N-1)(N-3)r^{(N-1)/2}r^{-2}\left[u(r)-\kappa\right]-g(u(r))r^{(N-1)/2}.$$ Now take out a factor $v(r)=r^{(N-1)/2}\left[u(r)-\kappa\right]$ and multiply by $-1$ to obtain: \begin{align*}
v''(r)&=r^{(N-1)/2}\left[u(r)-\kappa\right]\left\{
		\frac{1}{4}(N-1)(N-3)r^{-2}-\frac{g(u)}{u(r)-\kappa}\right\}\\
v''(r)&=v\left\{\frac{(N-1)(N-3)}{4r^2}-\frac{g(u)}{u(r)-\kappa}\right\}\\
-v''(r)&=\left\{\frac{g(u)}{u(r)-\kappa}-\frac{(N-1)(N-3)}{4r^2}\right\}v
\end{align*}
<<<<<<< HEAD
Also $v(r)>0$ for $r\geq0$ by definition of $v(r)$. {\color{red} Now $v(r)>0$ as $r>0$ and $u(r)>\kappa$ for $r\in(0,\infty)$. What about the term in brackets? Claim: there exist ...} The $r$-term is positive and increasing and the term in brackets is positive and decreasing. Thus $v(r)$ is positive. As mentioned, this differential equation in $v$ is what will be used to show that $l=\kappa$ is impossible under the assumptions on $g$. Before diving into the cases, a lower bound for the term in brackets will be calculated. Remember that by assumption $u(r)\downarrow\kappa$ as $r\uparrow\infty$ and $g$ satisfies \eqref{6}. Claim: there exist positive numbers $\omega$ and $R_1$ such that: $$\frac{g(u)}{u(r)-\kappa}-\frac{(N-1)(N-3)}{4r^2}\geq\omega\quad\text{for all}~r\geq R_1$$
\begin{proof}[Proof of the claim] By assumption, $g(\kappa)=0$ and using the definition of the derivative: $$\underset{u(r)\downarrow\kappa}{\lim}~\frac{g(u(r))}{u(r)-\kappa}=\underset{u(r)\downarrow\kappa}{\lim}~\frac{g(u(r))-g(\kappa)}{u(r)-\kappa}~\overset{(u(r)-\kappa=h)}{=}~\underset{h\downarrow0}{\lim}~\frac{g(\kappa+h)-g(\kappa)}{h}=g'(\kappa^+)$$ and $g'(\kappa^+)>0$ by conditions on $g$. Let $\epsilon>0$. Then there exists $R_\epsilon>0$ such that \begin{gather*}r\geq R_\epsilon\implies\left|\frac{g(u(r))}{u(r)-\kappa}-g'(\kappa)\right|\leq\frac{\epsilon}{2}\\ %and by definition of the absolute value $$-\frac{\epsilon}{2}\leq\frac{g(u(r)}{u(r)-\kappa}-g'(\kappa)\leq\frac{\epsilon}{2}$$ of which the first inequality will be used: \begin{gather*}-\frac{\epsilon}{2}\leq\frac{g(u(r)}{u(r)-\kappa}-g'(\kappa)\\ \frac{g(u(r)}{u(r)-\kappa}-g'(\kappa)\geq-\frac{\epsilon}{2}\\
\implies\frac{g(u(r))}{u(r)-\kappa}\geq g'(\kappa)-\frac{\epsilon}{2}\tag{A} \end{gather*}Note also that there exists $R_\theta>0$ such that \begin{gather*}r\geq R_\theta\implies\left|\frac{(N-1)(N-3)}{4r^2}\right|\leq\frac{\epsilon}{2}\\% and again using the definition of the absolute value: \begin{gather*}-\frac{\epsilon}{2}\leq\frac{(N-1)(N-3)}{4r^2}\leq\frac{\epsilon}{2}\\
=======
Also $v(r)>0$ for $r\geq0$ by definition of $v(r)$. The $r$-term is positive and increasing and the term in brackets is positive and decreasing. Thus $v(r)$ is positive. As mentioned, this differential equation in $v$ is what will be used to show that $l=\kappa$ is impossible under the assumptions on $g$. Before diving into the cases, a lower bound for the term in brackets will be calculated. Remember that by assumption $u(r)\downarrow\kappa$ as $r\uparrow\infty$ and $g$ satisfies \eqref{6}. Claim: there exist positive numbers $\omega$ and $R_1$ such that: $$\frac{g(u)}{u(r)-\kappa}-\frac{(N-1)(N-3)}{4r^2}\geq\omega\quad\text{for all}~r\geq R_1$$
\begin{proof}[Proof of the claim] By assumption, $g(\kappa)=0$ and using the definition of the derivative: $$\underset{u(r)\downarrow\kappa}{\lim}~\frac{g(u(r))}{u(r)-\kappa}=\underset{u(r)\downarrow\kappa}{\lim}~\frac{g(u(r))-g(\kappa)}{u(r)-\kappa}~\overset{(u(r)-\kappa=h)}{=}~\underset{h\downarrow0}{\lim}~\frac{g(\kappa+h)-g(\kappa)}{h}=g'(\kappa^+)$$ and $g'(\kappa^+)>0$ by conditions on $g$. Let $\epsilon>0$. Then there exists $R_\epsilon>0$ such that \begin{gather*}r\geq R_\epsilon\implies\left|\frac{g(u(r))}{u(r)-\kappa}-g'(\kappa)\right|\leq\frac{\epsilon}{2}\\ %and by definition of the absolute value $$-\frac{\epsilon}{2}\leq\frac{g(u(r)}{u(r)-\kappa}-g'(\kappa)\leq\frac{\epsilon}{2}$$ of which the first inequality will be used: \begin{gather*}-\frac{\epsilon}{2}\leq\frac{g(u(r)}{u(r)-\kappa}-g'(\kappa)\\ \frac{g(u(r)}{u(r)-\kappa}-g'(\kappa)\geq-\frac{\epsilon}{2}\\ 
\implies\frac{g(u(r))}{u(r)-\kappa}\geq g'(\kappa)-\frac{\epsilon}{2}\tag{A} \end{gather*}Note also that there exists $R_\theta>0$ such that \begin{gather*}r\geq R_\theta\implies\left|\frac{(N-1)(N-3)}{4r^2}\right|\leq\frac{\epsilon}{2}\\% and again using the definition of the absolute value: \begin{gather*}-\frac{\epsilon}{2}\leq\frac{(N-1)(N-3)}{4r^2}\leq\frac{\epsilon}{2}\\ 
>>>>>>> e03346cd93b5d1395328085326f20581e3361581
\implies\frac{(N-1)(N-3)}{4r^2}\geq-\frac{\epsilon}{2}.\tag{B}\\ \text{Addition yields: }\frac{g(u(r))}{u(r)-\kappa}-\frac{(N-1)(N-3)}{4r^2}\geq g'(\kappa)-\epsilon\tag{A+B}\end{gather*} And the claim is valid, let $\omega=g'(\kappa)-\epsilon$ with $\epsilon>0$ small enough and $R_1=\max(R_\epsilon,R_\theta)$. \end{proof}

From this, $-v''(r)>0$ for $r\geq R_1$ and thus $v''(r)<0$ for $r\geq R_1$, which implies $v'(r)\downarrow L\geq-\infty$ as $r\uparrow\infty$. To see this, remember that $v''(r)<0$ implies $v(r)$ will be concave down, that is, the tangent line will lie above the function. Even if the derivative $v'(r)$ would be positive for $r$ slightly larger than $R_1$, since $v''(r)<0$ indefinitely, the function will remain concave down and the derivative will become negative and stay negative. Consider the following possible limits $L$: $L<0$ and $L\geq0$. In the first case, $L<0$, then $v(r)\to-\infty$ as $r\to\infty$ which is impossible, since $v(r)>0$ for $r\geq0$ \Lightning. Then consider $L\geq0$. Note that $v'(R_1)\geq0$. Indeed, suppose $v'(R_1)<0$. Then since $v''(r)<0$, the function is concave down and the derivative will only decrease. Then the limit of the derivative can not be $L\geq0$. Thus $v'(R_1)\geq0$. This, by the same argument, implies the derivative will be positive for $r\geq R_1$. Suppose the derivative is negative for some $R_2>R_1$ then the derivative will remain negative for $r\geq R_2$ as $v''(r)<0$. Clearly, $v(r)\geq v(R_1)>0$. This implies that $-v''(r)\geq\omega v(R_1)>0$ and therefore $v'(r)\downarrow-\infty$ as $r\to\infty$. Why? Because if $v''(r)<0$ indefinitely, the limit of $v'(r)$ can not be $L\geq0$, as the derivative will decrease while $v''(r)<0$. Since $v''(r)<0$ indefinitely, the limit of $v'(r)$ will be minus infinity. Again, this contradicts $v(r)>0$ for $r\geq0$. \Lightning Conclusion: in any case $v''(r)<0$ for $r\geq R_1$ which implies $v'(r)\downarrow-\infty$, which contradicts $v(r)>0$ for $r\geq0$ and hence $l=\kappa$ is an impossible assumption under the assumptions on $g$ and hence $l\neq\kappa$. This concludes the proof of the lemma.
\end{proof}

\subsection{P and N are nonempty and open}

\begin{lemma}\label{lem} Under the assumptions of \cref{exithm}, the sets $P$ and $N$ are nonempty, disjoint and open.
\end{lemma}
\begin{proof}
The sets $P$ and $N$ are disjoint by definition. The order in which the statements of the lemma will be proven is as follows: $P$ is nonempty, $P$ is open, $N$ is open. That $N$ is nonempty is outside the scope of this report. See \cite{ber} for the proof.

<<<<<<< HEAD
\subsection*{P is non-empty}
To prove that $P$ is non-empty, let $\alpha_p\in(\kappa,\alpha_0]$.
=======
\subsection*{P is non-empty} 
To prove that $P$ is non-empty, let $\alpha_p\in(\kappa,\alpha_0]$. 
>>>>>>> e03346cd93b5d1395328085326f20581e3361581
Consider the following cases: (i) $\alpha_p\in N$, (ii) $\alpha_p\notin P$, (iii) $\alpha_p\in P$ and note they are mutually exclusive. If the initial condition is in $N$, then the solution vanishes in some point $r_0$. Then, if the initial condition is not in $N$, the solution does not vanish anywhere: $u(\alpha_p,r)>0$ for $r\geq0$. If the initial condition is \underline{not} in $P$, then the derivative is negative everywhere. Disproving these two cases yields the properties: the solution is positive everywhere, but the derivative vanishes in some point $r_0$, which is exactly the definition of $P$. %The sets $P$ and $N$ are disjoint and the initial condition cannot be in $P$ and not in $P$ and $P$.
Hence disproving case (i) and (ii) implies case (iii) applies.

First, suppose by contradiction that $\alpha_p\in N$. Then by definition, there exists a $r_0$ such that $u(\alpha_p,r_0)=0$ and $u'(\alpha_p,r_0)<0$. Now evaluate \eqref{ivpint} in $r_0$, using $u(\alpha_p,r_0)=0$, $G(0)=0$ and that on $(0,r_0]$ the derivative is strictly negative:
\begin{align*}-\frac{1}{2}[u'(r_0)]^2-(N-1)&\int_0^{r_0}[u'(s)]^2\frac{ds}{s}=G(u(\alpha_p,r_0))-G(\alpha_p)\\
G(\alpha_p)-G(u(\alpha_p,r_0))&=\frac{1}{2}[u'(r_0)]^2+(N-1)\int_0^{r_0}[u'(s)]^2\frac{ds}{s}\\
G(\alpha_p)-G(0)&=\frac{1}{2}[u'(r_0)]^2+(N-1)\int_0^{r_0}[u'(s)]^2\frac{ds}{s}\\
G(\alpha_p)&=\frac{1}{2}[u'(r_0)]^2+(N-1)\int_0^{r_0}[u'(s)]^2\frac{ds}{s}\\
    \implies &G(\alpha_p)>0,
\end{align*}
but the assumption $\alpha_p\in(\kappa,\alpha_0]$ implies $G(\alpha_p)<0$, a contradiction, hence $\alpha_p\notin N$.

Next, suppose $\alpha_p\notin P$, then $u(\alpha_p,r)>0$ for $r\geq0$ and $u'(\alpha_p,r)<0$ for $r>0$. That implies $u(\alpha_p,r)\downarrow l\geq0$ as $r\uparrow\infty$ and by \cref{llemma} the limit is zero, $l=0$. Similar to the previous argument and the proof of \cref{llemma}, observe \eqref{ivpint} for $r$ tending to infinity, and note the following: $l=\limrtoinf u(\alpha_p,r)$, $G(l)=G(0)=0$, $\limrtoinf\left[u'(\alpha_p,r)\right]^2=0$ as well as the lower bound on the integral term $\int_0^\infty\left[u'(\alpha_p,s)\right]^2\frac{ds}{s}\geq0$, to conclude that:
\begin{align*}
\limrtoinf\left[G(\alpha_p)-G(u(\alpha_p,r))\right]&=\limrtoinf\left[\frac{1}{2}[u'(r)]^2+(N-1)\int_0^{r}[u'(s)]^2\frac{ds}{s}\right]\\
G(\alpha_p)-G(l)&=0+(N-1)\int_0^{\infty}[u'(s)]^2\frac{ds}{s}\\
G(\alpha_p)-G(0)&\geq0\\
&\implies G(\alpha_p)\geq0,
\end{align*}
to reach the same contradiction, $\alpha_p\in(\kappa,\alpha_0]\implies G(\alpha_p)<0$. Hence $\alpha_p\in P$, and any $\alpha\in(\kappa,\alpha_0]$ is in $P$, that is, $(\kappa,\alpha_0]\subset P$ and $P$ is nonempty.\hfill\ensuremath{\square}\vspace{1em}

The proofs for openness of both $P$ and $N$ invoke continuous dependence on the initial data as explained in detail in for example \cite{codlev}.

\subsection*{P is open} To prove that $P$ is open, note that for $\alpha\in P$, $$r_0=\inf\{r>0,~u'(\alpha,r)=0,~u(\alpha,r)>0\}>0$$ and\begin{empheq}[left = \empheqlbrace]{align*}
	&u(\alpha,r)>0\quad\text{for all }r\in[0,r_0] \\
    &u'(\alpha,r)<0\quad\text{for all }r\in(0,r_0).
\end{empheq}
<<<<<<< HEAD
From \eqref{ivp} follows $u''(\alpha,r_0)=-g(u(\alpha,r_0))$. Consider $u''(\alpha,r_0)=0$. Then $g(u(\alpha,r_0))=0$. The only zero of $g$ where $0<u(\alpha,r_0)<\alpha$ is $\kappa$, i.e. $u(\alpha,r_0)=\kappa$. But since $u'(\alpha,r_0)=0$ and $u''(\alpha,r_0)=0$, $u(\alpha,r)\equiv\kappa$, which is impossible.
=======
From \eqref{ivp} follows $u''(\alpha,r_0)=-g(u(\alpha,r_0))$. Consider $u''(\alpha,r_0)=0$. Then $g(u(\alpha,r_0))=0$. The only zero of $g$ where $0<u(\alpha,r_0)<\alpha$ is $\kappa$, i.e. $u(\alpha,r_0)=\kappa$. But since $u'(\alpha,r_0)=0$ and $u''(\alpha,r_0)=0$, $u(\alpha,r)\equiv\kappa$, which is impossible. 
>>>>>>> e03346cd93b5d1395328085326f20581e3361581

Consider $u''(\alpha,r_0)\neq0$. Then because the derivative was negative up to $r_0$ and has now vanished, $u''(\alpha,r_0)>0$. This implies the derivative is positive to the right of $r_0$ and there exists a $r_1>r_0$ such that $u(\alpha,r)>u(\alpha,r_0)$ for all $r\in(r_0,r_1]$. {\color{red} FIGURE}. By continuous dependence on the initial condition, let $\alpha^*$ be sufficiently close to $\alpha$ then for all $r\in(0,r_1]$ the following hold:\begin{empheq}[left=\empheqlbrace]{align*}
	&u(\alpha^*,r_1)>u(\alpha^*,r_0) \\
    &\alpha^*>u(\alpha^*,r)>0
\end{empheq} which can be interpreted as: for $\alpha^*$ sufficiently small, $u(\alpha^*,r_1)$ is still above $u(\alpha^*,r_0)$, the derivative vanishes in some point $r_0^*\in(0,r_1]$ and the solution does not vanish on $(0,r_1]$. Then these properties together imply $\alpha^*\in P$ and hence $P$ is open. ($P$ is open if for any initial condition $\alpha$ in $P$, there exists a real number $\epsilon_\alpha>0$ such that points whose distance from $\alpha$ is less than $\epsilon_\alpha$ are also in $P$.)\hfill\ensuremath{\square}

\subsection*{N is open} To prove that $N$ is open, note that for $\alpha\in N$, $$r_0=\inf\{r>0,~u(\alpha,r)=0,~u'(\alpha,r)<0\}>0$$ and\begin{empheq}[left = \empheqlbrace]{align*}
	&u(\alpha,r)>0\quad\text{for all }r\in[0,r_0) \\
    &u'(\alpha,r)<0\quad\text{for all }r\in(0,r_0].
\end{empheq}
In any case, since $u(\alpha,r_0)=0$, $g(u(\alpha,r_0))=g(0)=0$. In fact, remember $g(u)=0$ for $u\leq0$. Note that $u'(\alpha,r_0)<0$. Then there exists $r_1>r_0$ such that $u(\alpha,r)<u(\alpha,r_0)=0$ for $r\in(r_0,r_1]$. The \eqref{ivp} yields for $r\in(r_0,r_1]$:\begin{gather*}u''(\alpha,r)=-\frac{N-1}{r}u'(\alpha,r)\\ u'(\alpha,r)=\left(\frac{r_0}{r}\right)^{N-1}u'(\alpha,r_0)\geq u'(\alpha,r_0)\end{gather*} Thus $u'(\alpha,r)\uparrow0$ for $r\to\infty$. Note that the function does not become positive, $r_1$ could be extended to infinity. By continuous dependence on the initial condition, let $\alpha^*$ be sufficiently close to $\alpha$ that for all $r\in(0,r_1]$ the following hold:\begin{empheq}[left=\empheqlbrace]{align*}
	&u(\alpha^*,r_1)<u(\alpha^*,r_0) \\
    &u'(\alpha^*,r)=\left(\frac{r_0^*}{r}\right)^{N-1}u'(\alpha,r_0^*)
\end{empheq} where $r_0^*$ uniquely satisfies $u(\alpha^*,r_0^*)=0$ on $(0,r_1]$. This can be interpreted as: for $\alpha^*$ sufficiently small, $u(\alpha^*,r_1)$ is still below $u(\alpha^*,r_0)$ (now not necessarily zero) and the derivative decays hyperbolically after the solution vanishes in some point $r_0^*\in(0,r_1]$. Then these properties together imply $\alpha^*\in N$ and hence $N$ is open. ($N$ is open if for any initial condition $\alpha$ in $N$, there exists a real number $\epsilon_\alpha>0$ such that points whose distance from $\alpha$ is less than $\epsilon_\alpha$ are also in $N$.)
\end{proof}

\subsection{Conclusions on existence}\hfill

Under the conditions on $g$ as specified after the theorem, there exists a ground state solution to \eqref{ivp} as proven by ODE methods. Remember that the interval of definition of solutions with initial condition $\alpha\in(\kappa,\lambda)$ is $(0,\infty)$ and that any solution that is positive everywhere and has negative derivative everywhere is convergent. By the properties of $g$ and the \eqref{ivp}, this limit is a zero of $g$ and if $\kappa$ is a zero of $g$, then $l\neq\kappa$ by \cref{llemma}. Lastly, the shooting method requires that the sets of initial conditions $P$ and $N$ are disjoint, nonempty and open, as proven by \cref{lem}. Then there exist elements with intial condition $\alpha$ not in $P$ or $N$ that have $u(\alpha,r)>0$ for $r\geq0$ and $u'(\alpha,r)<0$ for $r>0$. Such a solution by \cref{llemma} tends to zero for $r$ to infinity and hence is a ground state solution.









\comm{
\subsection{Previously}
\vspace{3cm}
In \cite{ber}, the existence of a ground state solution to \eqref{ivp} is formalised in a theorem of which a proof is given. This chapter studies that paper.

<<<<<<< HEAD
The existence of a ground state solution to \eqref{ivp} corresponds to the existence of an initial condition $\alpha_0$ belonging to solution set $G$. This is not guaranteed, the set G could be empty. This is the case if none of the solutions exhibit ground state behaviour.
=======
The existence of a ground state solution to \eqref{ivp} corresponds to the existence of an initial condition $\alpha_0$ belonging to solution set $G$. This is not guaranteed, the set G could be empty. This is the case if none of the solutions exhibit ground state behaviour. 
>>>>>>> e03346cd93b5d1395328085326f20581e3361581

As illustrated in chapter \ref{not}, the solution sets are disjoint. Hence, the set $G$ is non-empty if both $P$ and $N$ are non-empty and open, and additionally, a solution not belonging to $P$ or $N$ satisfies the ground state conditions. Summarising, by proving the following statements, there exist solutions that do not belong to either $P$ or $N$. Then it remains to show such a solution belongs to $G$.

[[A ground state solution will be found in $(\zeta_0,\beta)$, because...]]

A proof of the existence theorem thus requires the following statements to be true:
\begin{enumerate}
\item The set $P$ is non-empty,
\item The set $P$ is open,
\item The set $N$ is non-empty,
\item The set $N$ is open,
\item For $\alpha_1 \not\in (P\cup N)$, $\underset{r\to\infty}{\lim}u(\alpha_1,r)\neq\kappa$.
\end{enumerate}
\comm{
The third statement is hard to prove, that is, it requires knowledge outside the scope of a bachelor's thesis. For last statement, admit the following lemma.

<<<<<<< HEAD
\begin{lemma}\label{llemma}
=======
\begin{lemma}\label{llemma} 
>>>>>>> e03346cd93b5d1395328085326f20581e3361581
Let $g$ be locally Lipschitz continuous on $\mathbb{R_+}$ such that $g(0)=0$. Let $\alpha_1\in(0,\infty)$ be such that $u(\alpha_1,r)>0$ for all $r\geq0$ and $u'(\alpha_1,r)<0$ for all $r>0$. Then the number $l=\underset{r\to\infty}{\lim}u(\alpha_1,r)$ satisfies $g(l)=0$. Furthermore, if $g$ satisfies \eqref{6} and $g(\kappa)=0$, then $l\neq\kappa$.
\end{lemma}

What does this lemma yield? [Given the \eqref{ivp} with nonlinear function $g$ and certain conditions on $g$, for \emph{any initial condition} $\alpha_1\in(0,\infty)$ not belonging to $P$ or $N$, the solution $u$ has a limit $l$ for $r$ tending to infinity and the function $g$ vanishes for this limit value.] Also, if $g$ vanishes in $\kappa$, then the limit value $l$ is different from $\kappa$. In fact, it will turn out $l=0$, if $\alpha_1\in(0,\lambda)$, since a solution is decreasing and $\lambda$ is the next zero of $g$ upwards of $\kappa$. Conclusion: such a solution is a ground state solution.

%\text{ and both terms of right hand side are finite:}
\begin{proof} To prove $g(l)=0$, let $r$ tend to infinity in \eqref{ivp}: \begin{equation}\limrtoinf\Big[u''(r)+\frac{N-1}{r}u'(r)+g(u(r))=0\Big],\end{equation} and note that $l=\limrtoinf u(r)$. \underline{Claim:} both $u''$ and $u'$ converge to zero for $r$ tending to infinity. Then $g(l)=0$ and it remains to show that $g\neq\kappa$.
\begin{proof}[Proof of the claim] Observe identity \eqref{ivpint} for $r$ tending to infinity,
\begin{empheq}{align*}
	\underset{r\to\infty}{\lim}\Big[&G(\alpha_p)=G(u(\alpha_1,r))+\frac{1}{2}\big[u'(\alpha_1,r)\big]^2+(N-1)\int_0^{r}\big[u'(\alpha_1,s)\big]^2\frac{ds}{s}\Big] \\
    &G(\alpha_1)=G(l)+\underset{r\to\infty}{\lim}\frac{1}{2}\big[u'(\alpha_1,r)\big]^2+(N-1)\int_0^{\infty}\big[u'(\alpha_1,s)\big]^2\frac{ds}{s} \\
     &G(\alpha_1)-G(l)=\underset{r\to\infty}{\lim}\frac{1}{2}\big[u'(\alpha_1,r)\big]^2+(N-1)\int_0^{\infty}\big[u'(\alpha_1,s)\big]^2\frac{ds}{s} \\
    &\text{noting that }G(l)<G(\alpha_1)<\infty,\text{ hence }0<G(\alpha_1)-G(l)<\infty \\
    \implies &\int_0^{\infty}u'(\alpha_1,s)\big]^2\frac{ds}{s}<\infty,~\underset{r\to\infty}{\lim}[u'(\alpha_1,r)]^2<\infty
\end{empheq}
<<<<<<< HEAD
But this integral can only be finite if $u'(\alpha_1,r)$ converges. [[the limit is 0]]

Observe \eqref{ivp} for $r\to\infty$, $$\limrtoinf\Big[u''(r)+\frac{N-1}{r}u'(r)+g(u(r))=0\Big],$$ then since $u'$ and $g$ are bounded, $u''$ must converge. Since $u'$ is bounded, $u''\to0$ as $r\to\infty$.
=======
But this integral can only be finite if $u'(\alpha_1,r)$ converges. [[the limit is 0]] 

Observe \eqref{ivp} for $r\to\infty$, $$\limrtoinf\Big[u''(r)+\frac{N-1}{r}u'(r)+g(u(r))=0\Big],$$ then since $u'$ and $g$ are bounded, $u''$ must converge. Since $u'$ is bounded, $u''\to0$ as $r\to\infty$. 
>>>>>>> e03346cd93b5d1395328085326f20581e3361581
\end{proof}

Now to show $g\neq\kappa$...
\end{proof}

Returning to the other statements,...}
}\comm{
<<<<<<< HEAD
\subsection*{$P$ is non-empty}
To prove that P is non-empty, let $\alpha_p\in(\gamma,\kappa]$.
Consider the following cases: (i) $\alpha_p\in N$, (ii) $\alpha_p\notin P$, (iii) $\alpha_p\in P$ and note they are mutually exclusive [[figure]]. %The sets $P$ and $N$ are disjoint and the initial condition cannot be in $P$ and not in $P$ and $P$.
=======
\subsection*{$P$ is non-empty} 
To prove that P is non-empty, let $\alpha_p\in(\gamma,\kappa]$. 
Consider the following cases: (i) $\alpha_p\in N$, (ii) $\alpha_p\notin P$, (iii) $\alpha_p\in P$ and note they are mutually exclusive [[figure]]. %The sets $P$ and $N$ are disjoint and the initial condition cannot be in $P$ and not in $P$ and $P$. 
>>>>>>> e03346cd93b5d1395328085326f20581e3361581
Hence disproving case (i) and (ii) implies case (iii) applies.

First, suppose by contradiction that $\alpha_p\in N$. Then by definition, there exists a $r_0$ such that $u(\alpha_p,r_0)=0$ and $u'(\alpha_p,r_0)<0$. Now evaluate \eqref{ivpint} in $r_0$,
\begin{empheq}{align*}
	&G(\alpha_p)=G(u(\alpha_p,r_0))+\frac{1}{2}\big[u'(\alpha_p,r_0)\big]^2+(N-1)\int_0^{r_0}\big[u'(\alpha_p,s)\big]^2\frac{ds}{s} \\
    &\text{noting that }G(u(\alpha_p,r_0)=G(0)=0,~[u'(\alpha_p,r_0)]^2>0\text{ and } \int_0^{r_0}\big[u'(\alpha_p,s)\big]^2\frac{ds}{s}>0 \\
    \implies &G(\alpha_p)>0,
\end{empheq}
but this contradicts $\alpha_p\in(\gamma,\kappa]\implies G(\alpha_p)<0$, hence $\alpha_p\notin N$.

<<<<<<< HEAD
Next, suppose $\alpha_p\notin P$, then $u(\alpha_p,r)>0$ and $u'(\alpha_p,r)<0$ for $r>0$. That implies $u(\alpha_p,r)\downarrow l\geq0$ as $r\uparrow\infty$ and by \ref{llemma} $l=0$. Observe \eqref{ivpint} for $r$ tending to infinity,
=======
Next, suppose $\alpha_p\notin P$, then $u(\alpha_p,r)>0$ and $u'(\alpha_p,r)<0$ for $r>0$. That implies $u(\alpha_p,r)\downarrow l\geq0$ as $r\uparrow\infty$ and by \ref{llemma} $l=0$. Observe \eqref{ivpint} for $r$ tending to infinity, 
>>>>>>> e03346cd93b5d1395328085326f20581e3361581
\begin{empheq}{align*}
	\underset{r\to\infty}{\lim}\big[&G(\alpha_p)=G(u(\alpha_p,r))+\frac{1}{2}\big[u'(\alpha_p,r)\big]^2+(N-1)\int_0^{r}\big[u'(\alpha_p,s)\big]^2\frac{ds}{s}\big] \\
    &G(\alpha_p)=G(l)+\underset{r\to\infty}{\lim}\frac{1}{2}\big[u'(\alpha_p,r)\big]^2+(N-1)\int_0^{\infty}\big[u'(\alpha_p,s)\big]^2\frac{ds}{s}\big] \\
    &\text{noting that }G(l)=G(0)=0,~\underset{r\to\infty}{\lim}[u'(\alpha_p,r)]^2=0\text{ (see proof of \ref{llemma})} \\
    &G(\alpha_p)=(N-1)\int_0^{\infty}\big[u'(\alpha_p,s)\big]^2\frac{ds}{s}\big] \\
    \implies &G(\alpha_p)>0,
<<<<<<< HEAD
\end{empheq}and again, since $\alpha_p\in(\gamma,\kappa]\implies G(\alpha_p)<0$, a contradiction is reached and hence $\alpha_p\in P$.
=======
\end{empheq}and again, since $\alpha_p\in(\gamma,\kappa]\implies G(\alpha_p)<0$, a contradiction is reached and hence $\alpha_p\in P$. 
>>>>>>> e03346cd93b5d1395328085326f20581e3361581

Conclusion: $(\gamma,\kappa]\subset P$ and $P$ is non-empty.

\subsection*{$P$ is open} To prove that $P$ is open, note that for $\alpha\in P$, $$r_0=\inf\{r>0,~u'(\alpha,r)=0,~u(\alpha,r)>0\}>0$$ and\begin{empheq}[left = \empheqlbrace]{align*}
	&u(\alpha,r)>0\quad\text{for all }r\in[0,r_0) \\
    &u'(\alpha,r)<0\quad\text{for all }r\in(0,r_0).
\end{empheq}
From the \eqref{ivp} follows $u''(\alpha,r_0)=-g(u(\alpha,r_0))$ and considering $u''(\alpha,r_0)=0$, since $0<u(\alpha,r_0),\alpha$: $u(\alpha,r_0)=\gamma,~u'(\alpha,r_0)=0$ and by [[ uniqueness argument ]] we conclude $u(\alpha,r)\equiv\gamma$, an impossible solution [[because]]. Now suppose $u''(\alpha,r_0)\neq0$ and because $\alpha\in P$, $u''(\alpha,r_0)>0$. This implies there exists a $r_1>r_0$ such that $u(\alpha,r)>u(\alpha,r_0)$ for all $r\in(r_0,r_1]$ and [[because ...]] for $\alpha^*$ near $\alpha$ we have for all $r\in(0,r_1]$\begin{empheq}[left=\empheqlbrace]{align*}
	&u(\alpha^*,r_1)>u(\alpha^*,r_0) \\
    &\alpha^*>u(\alpha^*,r)>0
\end{empheq} which implies $\alpha^*\in P$ and hence $P$ is open. Similarly, }

\newpage \section{Uniqueness of ground state}
\comm{Uniqueness is proven by showing solution set $G$ contains at most one point: since $G$ is non-empty by chapter \ref{}, then $G$ contains precisely one point. The first lemma states that any points near a ground state initial condition ($\alpha\in G$) are not ground state. [[how about P?]]}
\comm{
\newpage
\begin{lemma}\label{kwong6} (Lemma 6 in \cite{kwong}) Let $(c,\infty)$ be the disconjugacy interval of \eqref{}. Every solution of \eqref{} with a zero in $(c,\infty)$ is unbounded.

Conversely, if the last zero of an unbounded solution of \eqref{} is $\rho$, then $\rho$ is an interior point of the disconjugacy interval. In the other words, $\rho>c$.
\begin{proof} \end{proof}
\end{lemma}}
% !TEX root = main.tex
\newpage
\begin{lemma}\label{lya}
  Suppose that $V(0):=\underset{r\to0}{\lim}~V(r)$ exists and is finite.
  Then $$0<\alpha<\left[\left(\frac{p+1}{2}\right)\frac{\lambda}{V(0)} \right]^{1/(p-1)}\implies \alpha\in P.$$

% \emph{\color{teal}Some remarks: as mentioned, the function $V(r)$ possibly has a singularity in $r=0$. In this lemma, we suppose the limit of $V(r)$ for $r\to0$ exists and is finite. Then a set of initial conditions is implied for which the solution is positive everywhere. This set is bound below by 0 and the upper bound is given by analysis of the Lyapunov function for this problem. Then, as the Lyapunov function is nonincreasing, for negative Lyapunov initial value ($E(0)<0$) the Lyapunov function remains negative. However, evaluating the Lyapunov function in $z(\alpha)$ or for $r\to\infty$ respectively, it follows that certain initial conditions yield nonnegative Lyapunov function values! From these contradictions, the set of initial conditions for which the solution belongs to $P$ is concluded.}

% \emph{\color{red}I would like to derive the Lyapunov function and explain/show how it is analogous to the classical dynamical potential. PHYSICS :). The argument that these initial conditions lead to negative $E(r)$ is given, but needs restructuring. Now, are these all initial conditions yielding solutions of $P$? Is it possible for $V(0)$ to be infinite and still yield $u\in P$?}

\begin{proof}
% TODO: Actually, don't need this r_0 business.. $\a\in P\implies\text{ PD for all }r>0$.
% Remember solutions $u(r;\a)$ to \ref{ivp} with $u(0)=\alpha\in P$ are positive definite on $(0,r_0)$ for some $r_0$ that satisfies $u'(r_0)=0$. Do such initial conditions exist? What interval of initial conditions belongs to $P$?
Remember solutions $u(r;\a)$ to \ref{ivp} with $u(0)=\alpha\in P$ are positive everywhere. Do such initial conditions exist? (See Chapter 4: $P$ is nonempty.) What interval of initial conditions belongs to $P$?
% TODO: define _the_ Lyapunov function $E(r)$ as
% Is it the Lyapunov function or some Lyapunov function?
% When is a function a Lyapunov function? Does this function satisfy those conditions?
To determine such an interval (of initial conditions), define (the Lyapunov or the energy) function $E(r)$ on $(0,z(\a))$ as:
$$E(r)\coloneqq\frac{1}{2}u'(r)^2-\frac{\lambda}{2}u(r)^2+\frac{1}{p+1}V(r)u(r)^{p+1},$$
% for $r\in[0,\infty)$ if $\alpha\in P\cup G$ and $r\in[0,z(\alpha)]$ if $\alpha\in N$.
% rewrite \ref{ivp} such that $u'(r)$ in $E(r)$ and
Then, calculate $E'(r)$, where the IVP can be used to simplify the expression:
\begin{align*}
    % &u''(r)+\frac{1}{r}u'(r)-\lambda u(r)+V(r)u(r)^p=0\\
    % &\left[u''(r)-\lambda u(r)+V(r)u(r)^p\right]=-\frac{1}{r}u'(r)\\
    E'(r)&=u''(r)u'(r)-\lambda u(r)u'(r)+V(r)u(r)^pu'(r)+\frac{1}{p+1}V'(r)u(r)^{p+1}\\
    &=\left[u''(r)-\lambda u(r)+V(r)u(r)^p\right]u'(r)+\frac{1}{p+1}V'(r)u(r)^{p+1}\\
    &=-\frac{u'(r)^2}{r}+\frac{1}{p+1}V'(r)u(r)^{p+1}\leq0\text{ for }r>0.\\
    \text{as }&\left[u''(r)-\lambda u(r)+V(r)u(r)^p\right]=-\frac{1}{r}u'(r)
\end{align*}
% {\color{red}Layout?}
% TODO: u(r) is not positive everywhere if $\a\in N$...?
Each of the terms of $E'(r)$ is non-negative: (i) by hypothesis H2, $V'(r)\leq0$; (ii) $u(r)^{p+1}\geq0$ because $u(r)$ is positive on $(0,z(\a))$ for any initial condition; (iii) $r\geq0$ and (iv) $u'(r)^2\geq0$. \underline{Result:} $E(r)$ is non-increasing $(0,z(\a))$.
% \seperate

% TODO: Choose order of argument, for now: first explain consequence of initial condition.
To conclude about behaviour of solutions by type of initial condition, regard $E(r)$ for large $r>0$.
Suppose $\a\in N$, then $u(z(\a))=0$.
Now evaluate:
\begin{empheq}{align*}
  \underset{r\to z(\alpha)}{\lim}E(r)&=
  \underset{r\to z(\alpha)}{\lim}\bigg[
    \frac{1}{2}u'(r)^2-\frac{\lambda}{2}u(r)^2+\frac{1}{p+1}V(r)u(r)^{p+1}
  \bigg] \\
  &=u'(z(\alpha))^2\geq0.
\end{empheq}
% $E(z(\a))=\frac{1}{2}u'(z(\alpha))^2\geq0$.
As $E(r)$ is non-increasing, $E(0)\geq0$.
Alternatively, suppose $\a\in G$.
Then $u(r)\to0$ and $u'(r)\to0$ as $\rtinf$.
% TODO: $\limrtoinf V'(r)$?
Then $E(r)=0$ for $\rtinf$, so again, $E(0)\geq0$.
\underline{Results:} $E(0)\geq0$ and $E(r)$ well-defined on $[0,z(\a)]$ for $\a\in G\cup N$.

For $\a\in P$, require $E(0)<0$.
Now evaluate $E(0)$ and solve for $\a$:
\begin{align*}
  E(0)=&\frac{1}{2}u'(0)^2-\frac{\lambda}{2}u(0)^2+\frac{1}{p+1}V(0)u(0)^{p+1}<0\\
  &\iff -\frac{\lambda}{2}\alpha^2+\frac{1}{p+1}V(0)\alpha^{p+1}<0\\
  &\iff\alpha^{p-1}<\left(\frac{p+1}{2}\right)\frac{\lambda}{V(0)}\\
  &\iff\alpha<\left[\left(\frac{p+1}{2}\right)\frac{\lambda}{V(0)} \right]^{1/(p-1)}
  % \iff\alpha^{p-1}<\left(\frac{p+1}{2}\right)\frac{\lambda}{V(0)}\iff
%\frac{1}{p+1}V(0)\alpha^{p-1}<\frac{\lambda}{2}\\
  % -\frac{\lambda}{2}\alpha^2+\frac{1}{p+1}V(0)\alpha^{p+1}<0\\
  % \implies E(0)=\frac{1}{2}u'(0)^2-\frac{\lambda}{2}u(0)^2+\frac{1}{p+1}V(0)u(0)^{p+1}<0.
\end{align*}
\underline{Conclusion:} $\a\in P$ whenever $0<\a<\left[\left(\frac{p+1}{2}\right)\frac{\lambda}{V(0)}\right]^{1/(p-1)}$.
% \seperate
% To prove $\alpha\in P$, consider the contradictory cases:
% (i) $\alpha\in N$, (ii) $\alpha\in G$.
% Suppose $\alpha\in N$.
% Then $u(z(\alpha))=0$ and $E(z(\alpha))=\frac{1}{2}u'(z(\alpha))^2\geq0$.
% This contradicts $E<0$.
% Suppose $\alpha\in G$.
% Then $u(r)\to0,u'(r)\to0$ as $r\to\infty$.
% Then $E(r)\to0$ as $r\to\infty$, contradicting $E<0$.
% \underline{Conclusion:} $\alpha\in P$.

% By these properties of $E(r)$ the implication will follow.
% What about the initial value of $E(r)$? Evaluate
% Let $0<\alpha<\left[\left(\frac{p+1}{2}\right)\frac{\lambda}{V(0)} \right]^{1/(p-1)}.$
% Rewrite this assumption on $\alpha$ and evaluate $E(0)$:
% \begin{gather*}
%   \alpha<\left[\left(\frac{p+1}{2}\right)\frac{\lambda}{V(0)} \right]^{1/(p-1)}
%   \iff\alpha^{p-1}<\left(\frac{p+1}{2}\right)\frac{\lambda}{V(0)}\iff
% %\frac{1}{p+1}V(0)\alpha^{p-1}<\frac{\lambda}{2}\\
%   -\frac{\lambda}{2}\alpha^2+\frac{1}{p+1}V(0)\alpha^{p+1}<0\\
%   \implies E(0)=\frac{1}{2}u'(0)^2-\frac{\lambda}{2}u(0)^2+\frac{1}{p+1}V(0)u(0)^{p+1}<0.
% \end{gather*}
% Remember $E(r)$ is nonincreasing, so $E(r)<0$ for $r>0$.
% \seperate
% To prove $\alpha\in P$, consider the contradictory cases:
% (i) $\alpha\in N$, (ii) $\alpha\in G$.
% Suppose $\alpha\in N$.
% Then $u(z(\alpha))=0$ and $E(z(\alpha))=\frac{1}{2}u'(z(\alpha))^2\geq0$.
% This contradicts $E<0$.
% Suppose $\alpha\in G$.
% Then $u(r)\to0,u'(r)\to0$ as $r\to\infty$.
% Then $E(r)\to0$ as $r\to\infty$, contradicting $E<0$.
% \underline{Conclusion:} $\alpha\in P$.

\end{proof}
\end{lemma}

% \subsection{Discussion} Consider the Lyapunov function $E(r)$, an analogue to the potential function of classical dynamics. Lyapunov theory treats the stability of a solution near an equilibrium point. For more on Lyapunov theory, see \cite{}.


\newpage
\begin{lemma}Let $\alpha\in G\cup N$, and $u=u(\alpha,r)$. Then $u'(r)<0$ for all $r\in(0,z(\alpha))$ and $u'(z(\alpha))<0$ if $\alpha\in N$.
\begin{proof} 
Write $z(\alpha)=\infty$ when $\alpha\in G$, since: $u(\alpha,r)\to0$ as $r\to\infty$. %Thus $z(\alpha)=\infty$. 
Let $\alpha\in G\cup N$. By lemma \ref{lya} then, $E(r)$ is well-defined for $r\in[0,z(\alpha))$ at least and $\lim_{r\to z(\alpha)}E(r)\geq0.$ Evaluate: \begin{empheq}{align*}\underset{r\to z(\alpha)}{\lim}E(r)&=\underset{r\to z(\alpha)}{\lim}\bigg[\frac{1}{2}u'(r)^2-\frac{\lambda}{2}u(r)^2+\frac{1}{p+1}V(r)u(r)^{p+1}\bigg] \\ &=u'(z(\alpha))^2\geq0.\end{empheq} 
Since $E(r)$ is non-increasing, $E(r)\geq0$ for all $r\in[0,z(\alpha)]$.

Suppose $u''(0)=0$ and note $u'(0)=0$, so $u\equiv\alpha$ \Lightning. Suppose $u''(0)>0$ then $u'(r)>0$ and $u(r)>u(0)=\alpha$ for $r>0$ and small. Now $\alpha\in(G\cup N)\iff -\frac{\lambda}{2}\alpha^2+\frac{1}{p+1}V(0)\alpha^{p+1}\geq0$, hence $$E(r)-E(0)=\frac{1}{2}\left[u'(r)^2-u'(0)^2\right]-\frac{\lambda}{2}\left[u(r)^2-u(0)^2\right]+\frac{1}{p+1}\left[V(r)u(r)^{p+1}-V(0)u(0)^{p+1}\right]\geq0.$$ This contradicts $E(r)$ nonincreasing. Hence $u''(0)<0$ and $u'(r)<0$ for $r>0$ and small.

Claim: $u'<0$ on $(0,z(\alpha))$. Suppose by contradiction $r_0=\inf(r>0,u'(r)=0)<z(\alpha)$. Note how $u''(r_0)<0\implies u'(r)>0$ on $(0,r_0)$ \Lightning. Hence $u''(r_0)\geq0$. Invoke \ref{ivp}:
%Note $u'(r_0)=0$ and 
\begin{gather*}u''(r_0)=\lambda u(r_0)-V(r_0)u(r_0)^p\geq0\\u(r_0)\leq\left[\frac{\lambda}{V(r_0)}\right]^{1/(p-1)}<\left[\left(\frac{p+1}{2}\right)\frac{\lambda}{V(r_0)}\right]^{1/(p-1)}\\
\implies-\frac{\lambda}{2}u(r_0)^2+\frac{1}{p+1}V(r_0)u(r_0)^{p+1}<0,\end{gather*}Then using $u'(r_0)=0$, evaluate $E(r_0)$: $$E(r_0)=-\frac{\lambda}{2}u(r_0)^2+\frac{1}{p+1}V(r_0)u(r_0)^{p+1}<0.$$ %\begin{empheq}{align*} \eqref{ivp}:\quad&u''(r)+\frac{N-1}{r}u'(r)+g(u(r))=0 \\ r=r_0, u'(r_0)=0:\quad&u''(r_0)+g(u(r))=0 \\ &-g(u(r_0))=u''(r_0) \\ &\lambda u(r_0)-V(r_0)u(r_0)^p=u''(r_0)\geq 0 \\ &\lambda u(r_0)-V(r_0)u(r_0)^p\geq0\quad[[]]\\ \implies &u(r_0)\leq\Bigg[\frac{\lambda}{V(r_0)}\Big]^{1/(p-1)}<\Bigg[\Bigg(\frac{p+1}{2}\Bigg)\frac{\lambda}{V(r_0)}\Big]^{1/(p-1)}\Bigg]\end{empheq}
But $E(r_0)<0$ contradicts $E(r)\geq0$, so $u'<0$ on $(0,z(\alpha))$.

It remains to show $u'(z(\alpha))<0$ whenever $\alpha\in N$. Suppose $u'(z(\alpha))=0$. Then $u\equiv0$, since $u(z(\alpha))=0$. Hence $u'(z(\alpha))<0$.
%The following claim contradicts $\alpha\in(G\cup N)$. \underline{Claim:} $E(r_0)<0$. To see this, evaluate $E(r)$ in $r_0$ as follows: \begin{empheq}{align*} E(r_0)=~&\frac{1}{2}u'(r_0)^2-\frac{\lambda}{2}u(r_0)^2+\frac{1}{p+1}V(r_0)u(r_0)^{p+1} \\ =&-\frac{\lambda}{2}u(r_0)^2+\frac{1}{p+1}V(r_0)u(r_0)^{p+1} \end{empheq}

%\underline{$u'(z(\alpha))<0$} Now if $\alpha\in N$, note that by the above $u'(r)<0$ on $(0,z(\alpha))$. If $u'(z(\alpha))=0$ then since $u(z(\alpha))=0$, the solution would be trivial, $u\equiv0$. Hence $u'(z(\alpha))<0$ and the proof is complete.
\end{proof}
\end{lemma}
\newpage
\begin{lemma}Let $\alpha\in (G\cup N),\text{ then }w$ has at least one zero in $(0,z(\alpha))$.
\begin{proof}
The Lagrange identity for \ref{ivp} and \ref{} will yield information about the zeroes of $w(r)$. Observe the following identities arising from the differential equations:\begin{gather*}
(ru'(r))'+r\left[-\lambda u(r)+V(r)u(r)^p\right]=0\\
(rw'(r))'+r\left[-\lambda w(r)+pV(r)u(r)^{p-1}w(r)\right]=0.
\end{gather*} Now multiply by $w(r)$ and $u(r)$ respectively, subtract them and integrate from 0 to $z(\alpha)$, \begin{gather*}
\int_0^{z(\alpha)}w(r)(ru'(r))'-u(r)(rw'(r))'dr=%
\int_0^{z(\alpha)}r\left\{pV(r)u(r)^pw(r)-V(r)u(r)^pw(r)\right\}dr,
\end{gather*} and perform partial integration:\begin{gather*}
rw(r)u'(r)\at_0^{z(\alpha)}-ru(r)w'(r)\at_0^{z(\alpha)}-\int_0^{z(\alpha)}ru'(r)w'(r)-ru'(r)w'(r)dr\\=(p-1)\int_0^{z(\alpha)}rV(r)u(r)^pw(r)dr\\z(\alpha)w(z(\alpha))u'(z(\alpha))=(p-1)\int_0^{z(\alpha)}rV(r)u(r)^pw(r)dr.
\end{gather*} Note that $u(z(\alpha))=0$.   In evaluating the integral term, {\color{gray}note $r>0$, $V(r)>0$, and $u(r)^p>0$. Suppose $w>0$ on $(0,z(\alpha))$. Then $z(\alpha)w(z(\alpha))u'(z(\alpha))\leq0$ contradicts $(p-1)\int_0^{z(\alpha)}rV(r)u(r)^pw(r)dr>0$ \Lightning. Hence $w$ has at least one zero in $(0,z(\alpha))$.}

\underline{$\alpha\in G$} Suppose by contradiction that $w>0$ on $(0,\infty)$. {\color{gray} Then rewrite ... and note how the integral is still positive by assumption. Then $\left(\frac{u}{w}\right)'$ is positive. So $\frac{u}{w}$ is increasing. By RRR there exists two independent solutions that satisfy TTT. So TTT for some constants $\alpha_1,\alpha_0$. Since $w>0$ by hypothesis, $\alpha_1\geq0$. Suppose $\alpha_1=0$, then $w(r)\to0$ exponentially as $r\to\infty$. So $w$ changes sign by RRR, a contradiction. On the other hand, suppose $\alpha_1>0$, then by RRR there exists a $C$ such that LLL. To see this, note how RRR implies $u(r)\sim r^{-1/2}\exp^{-\sqrt{\lambda}r}$ as $r\to\infty$. These contradictions yield $w$ changes sign at least once on $(0,z(\alpha))$ for $\alpha\in(G\cup N)$.} the Lagrange identity for and the integral term is positive by the same reasoning as above. Claim: the function $u/w$ is positive and increasing. Firstly, it is positive in the origin, because $u(0),w(0)>0$. Secondly, it is increasing since $r$, $w(r)^2$ are positive, so is the derivative of $u/w$. Also, since $u(0),w(0)>0$ the function \\ \\
\underline{}\\ \\
\underline{}
\end{proof}
\end{lemma}
{\color{gray} Intermezzo: some definitions. Let $\theta(r)\coloneqq-ru'(r)/u(r)$ for $r\in[0,z(\alpha)$ and $\rho\coloneqq\theta^{-1}$. Then PPProperties. Many auxiliary functions will be introduced now. These functions construct the information needed to prove $w$ has a unique zero. Bear with me as the following functions and variables are introduced: $\theta(r),\beta,\rho(\beta),\phi_{\beta}(r),\nu_{\beta}(r)$. In the lemma that follows, even more functions and variables are introduced: $\bar\beta,\sigma(\beta),\xi(r), \Xi(r)\coloneqq\sigma(\beta)^{-1},\beta_0$.}

{\color{teal}Define $\theta(r)\coloneqq-ru'(r)/u(r)$ on $r\in[0,z(\alpha))$. Note $\theta(r)$ has the following properties: \begin{enumerate}[(i)]\item $\theta(0)=0$, \item $\theta'(r)>0$ for all $r\in(0,z(\alpha))$, and \item $\underset{r\to z(\alpha)}{\lim}\theta(r)=\infty$. \end{enumerate} SHOW THETA(R) HAS THESE PROPERTIES.

Define $\rho\coloneqq\theta^{-1}$. Since $\theta(r)$ is continuous and increasing SHOW.., there exists a unique $r=\rho(\beta)>0$ such that $\theta(r)=\beta$. SHOW. Then $\rho(r)$ is continuous and increasing, with $\rho(0)=0$ and $\underset{\beta\to\infty}{\lim}\rho(\beta)=z(\alpha).$ SHOW.

Define $\nu_{\beta}(r)\coloneqq ru'(r)+\beta u(r)=-u(r){\theta(r)-\beta}.$ Let $\beta>0,$ then $\nu_{\beta}(r)>0$ if $r<\rho(\beta)$ and $\nu_{\beta}(r)<0$ if $r>\rho(\beta)$. SHOW.

Define $\phi_{\beta}(r)\coloneqq\left[\beta(p-1)-2\right]V(r)u(r)^p-rV'(r)u(r)^p+2\lambda u(r).$ Now observe $\nu_{\beta}(r)$ satisfies the differential equation $$\nu_{\beta}''(r)+\frac{1}{r}\nu_{\beta}(r)'-\lambda\nu_{\beta}+pV(r)u(r)^{p-1}\nu_{\beta}=\phi_{\beta}(r)$$} SHOW.

\begin{outline}
  \1 After these three lemmata, the following results are obtained:
    \2 $\a\in P$ for $\a\in(0,\a_0)$ with $\a_0=
      \left[\frac{\lambda}{V(0)}\frac{p+1}{2}\right]^{\frac{1}{p-1}}$, we have $\a\in P$.
    \2 For $\a\in G$ we have $u'(r)<0$ on $(0,\za)$ and for $\a\in N$ we have $u'(r)<0$ on $(0,\za]$.
    \2 For $\a\in G\cup N$ at least one zero for $w(r)$ in $(0,\za)$. In other words, $w$ oscillates faster.
  \1 To obtain results about the zeroes of $u$ we will need more information about the zeroes of $w$.
  \1 In fact, it will be shown that $w$ has a unique zero in $(0,\za)$.
  \1 Then, by the difference function $z\coloneqq u-\tilde u$ and Sturm theory:
    \2 $\a\in N\implies \tilde\a\in N\text{ for }\tilde\a\in(\a,\a+\epsilon)$.
  \1 Lastly, $\a\in N\implies\tilde\a\in N\text{ for }\tilde\a\in[\a,\infty)$.
    \2 So all initial conditions greater than or equal to initial conditions with solution in $N$ are also in $N$.
    \2 Furthermore, the difference between $u$ and $\tilde u$ is decreasing with respect to increasing initial condition.
  \1 This is sufficient to show that any initial condition below $\a_0\in G$ cannot be in $N$ as that would contradict $\a_0\in G$ by this property of initial conditions in $N$. As $G$ is nonempty by chapter 4 and $G$ and $N$ are now disjoint... also noting $P$ is disjoint with $G$ and $N$ and nonempty by lemma 5.1... $\a_0\in G$ is unique!
    \2 Most work remains in explaining using lemmata 7 and 8 that $G$ contains at most one point.
\end{outline}
\noindent\hrulefill
\begin{outline}
  \1 In the lemmata following, many related properties are used. This section is dedicated to properly introducing those functions and their properties.
  \1 First, let $\theta(r)\coloneqq-r\frac{u'(r)}{u(r)}$ for $r\in[0,\za]$.
    \2 Note that by IVP: $\theta(0)=\underset{r\downarrow0}{\lim}\theta(r)=\underset{r\downarrow0}{\lim}-r\frac{u'(r)}{u(r)}=\underset{r\downarrow0}{\lim}\frac{-ru'(r)}{\a}=0$
    \2 By properties $r>0$, $u'(r)<0$ on $(0,\za)$ and $u(r)>0$ on $(0,\za)$: $\theta(r)>0$ on $(0,\za)$.
    \2 Also, $\theta(r)$ is increasing, that is $\theta(r)'>0$.
      \3 To see this... SHOW
    \2 The limit $\underset{r\to\za}{\lim}\theta(r)=\infty$.
      \3 To see this for $\a\in N$ SHOW
      \3 To see this for $\a\in G$ SHOW
  \1 As $\theta(r)$ is continuous and increasing, there exists $\rho\coloneqq\theta^{-1}$ as there exists a unique $r=\rho(\beta)>0\text{ s.t. } \theta(r)=\beta$.
    \2 Note that $\rho(0)=0$ and $\underset{\beta\to\infty}{\lim}\rho(\beta)=\za$.
    \2 Note this $\rho(\beta)$ is continuous and increasing with respect to $\beta$.
    \2 Note that $\rho(r)>0$ on $(0,\infty)$.
  \1 Now define $\nub\coloneqq ru'(r)+\beta u(r)=-u(r)\left[\theta(r)-\beta\right]$
  \1 Let $\beta>0$ then $\nub(r)>0$ if $r<\rho(\beta)$ and $\nub<0$ if $r>\rho(\beta)$.
    \2 Note that $u(r)>0$ so $\nub>0$ if $\theta(r)-\beta<0$. That is, $0>\theta(r)-\beta\iff\beta>\theta(r)\iff\rho(\beta)>r$. Similarly, $\nub<0$ if $\beta<\theta(r)\iff\rho(\beta)<r$.
  \1 All of these properties can also be illustrated by beautiful figures. SHOW
  \1 \emph{Complicated:} $\nub(r)$ satisfies the following differential equation:
    \2 $\nub''(r)+\frac{1}{r}\nub'(r)-\lambda\nub(r)+pV(r)u(r)^{p-1}\nub(r)=\phib(r)$
    \2 Where $\phib(r)=\left[\beta(p-1)-2\right]V(r)u(r)^p-V'(r)u(r)^p+2\lambda u(r)$.
    \2 SHOWWWWWWW
  \1 In lemma 4 this will be used to show that the sign of $\phib$ is related to a continuous decreasing function $\sigma(\beta)$ and this function crosses $\rho(\beta)$ in a unique $\beta_0$ (lemma 5.5). The specific $\nu_{\beta_0}$ and $\rho_0=\rho(\beta_0)$ then are used in strong version of Sturm comparison theorem in lemma 6 to show that $w$ has a unique zero on $(0,\za)$! COOL! Hooray! Useful.
\end{outline}

\newpage
\begin{lemma}Let $\alpha\in G\cup N$. There exist $\beta_0>0$ and a unique function $\sigma:[0,\bar\beta]\to[0,\infty)$ with the following properties: \begin{enumerate}[(a)]
	\item  $\sigma$ is continuous and decreasing, $\sigma(0)>0$ and $\sigma(\bar\beta)=0$;
    \item for all $\beta>0$ we have: $\phi_\beta(r)<0$ if $r<\sigma(\beta)$, and $\phi_\beta>0$ if $r>\sigma(\beta)$.
\end{enumerate}
\begin{proof} 
Let $\beta>0$ and $r\in[0,z(\alpha))$. Then \begin{align*}
\phi_{\beta}(r)&=\left[\beta(p-1)-2\right]V(r)u(r)^p-rV'(r)u(r)^p+2\lambda u(r) %\\ &=V(r)u(r)^p\left[\beta(p-1)-2-\frac{rV'(r)u(r)^p}{V(r)u(r)^p}+\frac{2\lambda u(r)}{V(r)u(r)^p}\right] 
\\ &=V(r)u(r)^p\left[\beta(p-1)-2-r\frac{V'(r)}{V(r)}+\frac{2\lambda}{V(r)u(r)^{p-1}}\right] 
\\ &= V(r)u(r)^p\left[\beta(p-1)-2-\xi(r)\right]
\\\text{where }\xi(r)&=r\frac{V'(r)}{V(r)}-\frac{2\lambda}{V(r)u(r)^{p-1}}.
\end{align*}

To conclude about the sign of $\phi(r)$, note that $V(r)\geq0$ and $u(r)\geq0$. Hence the sign of $\phi(r)$ will vary with $\beta$ and $r$ as dictated by the term in brackets. Write $\beta(p-1)-2-\xi(r)>0\iff\beta>\frac{2+\xi(r)}{p-1}\coloneqq\Xi(r)$.

Note that $\xi(r)\leq0$ and strictly decreasing on $(0,z(\alpha))$ with $\lim_{r\to z(\alpha)}\xi(r)=-\infty$. To see this, note $h(r)=r\frac{V'(r)}{V(r)}$ is nonincreasing, $V'(r)\leq0$ and $u'(r)<0$ hence the second term of $\xi(r)$ is strictly decreasing. Thus $\xi(r)$ is strictly decreasing.

It remains to show $\Xi(r)=\left[2+\xi(r)\right]/(p-1)$ satisfies $\Xi(0)>0$. Remember that  Indeed, let $\Xi(0)=\bar\beta$, LLLLLL, for all $\beta\in[0,\bar\beta]$. Note $\Xi(r)$ is continuous and decreasing, so $\sigma(\beta)\coloneqq\left.\Xi(r)^{-1}\right|_{[0,\bar\beta]}$ has properties (a) and (b). SHOW. \\

Also, note $\Xi(0)>0\iff\xi(0)>-2$. SHOW.\\

Consider the following cases ......$V(0)<\infty$ .... $V(0)=\infty$ ... hence $\xi(0)>-2$ for $\alpha\in(G\cup N)$.

\end{proof}
\end{lemma}
% !TEX root = main.tex

\begin{lemma}
  Let $\alpha\in G\cup N$.
  There exists a unique $\beta_0>0$ such that $\rho(\beta_0)=\sigma(\beta_0)$.
  This follows immediately from the aforementioned properties of $\rho(\beta)$ and $\sigma(\beta)$.
  Sketching the two graphs, there is a unique intersection.
  % TODO: actually make the graph

%\begin{proof} \end{proof}
\end{lemma}

\begin{lemma}\label{wq}For $\alpha\in G\cup N,w$ has a unique zero $r_0\in(0,z(\alpha)).$ Furthermore, $w(z(\alpha))<0$ if $\alpha\in N$ and $\underset{r\to\infty}{\lim}w(r)=-\infty$ if $\alpha\in G$.

\emph{}\\[11pt]\emph{Some remarks: the functions $u(r)$ and $w(r)$ will be compared `'in the rest of the uniqueness proof'. This lemma derives that $w(r)$ has a \textbf{unique} zero $r_0$ to the left of $z(\alpha)$. Remember $z(\alpha)$ is the zero of $u(r)$. On the other hand, if $u(r)$ is positive and decreasing everywhere--has no finite zero $z(\alpha)$ such that $u(z(\alpha))=0$--then still $w(r)$ has a finite zero $r_0$ and $w(r)$remains negative to the right of the zero. Even more, the function $w(r)$ tends to negative infinity for $r\to\infty$, $\underset{r\to\infty}{\lim}w(r)=-\infty$.}\\[11pt]
\begin{proof}
The solution $\nu(r)$ changes sign in $\rho_0\in[0,z(\alpha)]$. As $\nu(z(\alpha))z(\alpha)u'(z(\alpha))<0$, the solution $\nu(r)$ changes sign once. Let $\tau$ be the first zero of $w(r)$. Then by Sturm comparison, $\rho_0\in(0,\tau)$, i.e. $\nu$ oscillates faster than $w$.

Also by Sturm comparison, $w$ can have no further zero and $\tau$ is the unique zero of $w$.

\underline{Sign of $v$ changes in $\rho_0$} \\ \\

\underline{$v$ changes sign only once} \\ \\

\underline{$w$ has a first zero} \\ \\

\underline{Comparing $w$ and $v$, conclusion: $w$ has a unique zero} \\ \\

\underline{Also if $\alpha\in G$}


\end{proof}
\end{lemma}

% !TEX root = main.tex

\begin{lemma}
  Let $\alpha^*\in N$.
  Then $[\alpha^*,\infty)\subset N\text{ and }z:[\alpha^*,\infty)\to(0,\infty)$ is monotone decreasing.
\begin{proof}
\begin{outline}
  \1 First of all, $N$ is open.
    \2 To see this, let $\hat\a\in N$.
    \2 There exists $r_0$ such that $u(r_0;\hat\a)=0$.
    \2 Let $\hat r> r_0$ then $u(\hat r;\hat\a)<0$.
    \2 By continuous dependence of $u$ on $\a$, for all $\a$ sufficiently close to $\hat\a$ one has $u(\hat r;\a)<0$.
    \2 Thus $N$ is open.
  \1 Secondly, $z:N\to(0,\infty)$ is continuous.
    \2 For all $\epsilon>0$ there exists a $\delta>0$ such that
\end{outline}

\seperate

\underline{$N$ is open}
Let $\hat\alpha\in N$.
Then there exists a $\hat r>0$ such that $u(\hat\alpha,\hat r)<0$.
By continuous dependence on the initial data [[Cod. Lev.]],
$u(\alpha,\hat r)<0$ for all $\alpha$ sufficiently close to $\hat\alpha$.

\underline{$z$ is continuous}
None of these solutions can be tangent to the $r$-axis, \_SHOW
hence the function $z:N\to(0,\infty)$ is continuous.
That is...

\seperate

\underline{$z$ is decreasing}
Let $\alpha^*\in N$.
Then by \ref{}, $w(z(\alpha^*))<0$ and for
$\epsilon>0$ sufficiently small,
$(\alpha^*,\alpha^*+\epsilon)\subset N$
and $u(z(\alpha^*),\alpha)<0$
for all $\alpha\in(\alpha^*,\alpha^*+\epsilon)$.

Remember that $w$ is the derivative of $u$ with respect to the initial condition.
Since $w(z(\alpha^*))<0$, for initial conditions upward of $\alpha^*$ ($\alpha\in(\alpha^*,\alpha^*+\epsilon)$):
$u(\alpha,z(\alpha^*))<u(\alpha^*,z(\alpha^*))=0$.
By the intermediate value theorem [[Cod. Lev.]], there exists a $r\in(0,z(\alpha^*))$ such that $u(\alpha,r)=0$.
Then $z(\alpha)\leq r\leq z(\alpha^*)$ for all $\alpha\in(\alpha^*,\alpha^*+\epsilon)$.
Conclusion: $z$ is decreasing on $(\alpha^*,\alpha^*+\epsilon)$.

\seperate

\underline{Domain of $z$ extends to infinity}
In fact, $z$ is decreasing on $[\alpha^*,\infty)$.
That is, let $$\bar\alpha\coloneqq\sup\{\alpha>\alpha^*\subset N\text{
and }z:[\alpha^*,\alpha)\to(0,\infty)\text{ is decreasing}\}.$$
Then the lemma requires $\bar\alpha=\infty$.
Suppose by contradiction $\bar\alpha<\infty$.
Then there exists $z(\bar\alpha)\coloneqq\lim_{\alpha\to\bar\alpha}z(\alpha)\in[0,\infty).$
Clearly, $\bar\alpha\in N$, since $u(\bar\alpha,z(\bar\alpha)=0$ by continuity of $z$.
But then $[\bar\alpha,\bar\alpha+\epsilon)\in N$ for $\epsilon>0$ sufficiently small.
This contradicts the definition of $\bar\alpha$ as the supremum.
Then $\bar\alpha=\infty$.
Conclusion: for $\alpha^*\in N$, $[\alpha^*,\infty)\subset N$
and $z:[\alpha^*,\infty)\to(0,\infty)$ is decreasing.

\end{proof}
\end{lemma}

\begin{lemma} Let $\alpha\in G$. There exists $\epsilon>0$ such that $(\alpha,\alpha+\epsilon)\subset N$.
\begin{proof} By chapter 4, the solution set $G$ is non-empty. Let $\alpha\in G$ and let $u(\alpha,r)$ be the corresponding solution.  The function $w(\alpha,r)=\frac{\partial}{\partial\alpha}u(\alpha;r)$ satisfies

By lemma \ref{wq}, the function $w$ is unbounded and by lemma \ref{kwong6}, $w$ has a unique zero $r_0\in(d,\infty)$. Let $r_1,r_2$ be such that $d<r_1<r_0<r_2$ and note that $w$ is positive in $r_1$ and negative in $r_2$. There exists $\epsilon>0$ such that, for all $\tilde\alpha\in(\alpha,\alpha+\epsilon)$,$$\tilde u(r_1)>u(r_1)\text{ and }\tilde u(r_2)<u(r_2),$$ where $\tilde u=u(\tilde\alpha,r)$. Hence there exists $r_3\in(r_1,r_2)$ such that the graphs of $u$ and $\tilde u$ intersect, i.e. $\tilde u(r_3)=u(r_3)$. See also figure \ref{}.

To conclude $\tilde\alpha\in N$ requires existence of a $\tilde r$ such that $\tilde u(\tilde r)=0$. By contradiction, suppose that $\tilde u(r)>0$ for all $r>r_3$. (Note that $\tilde u(r)\geq u(r)>0$ on $r\leq r_3$, since $u(r)>0$ for all $r$.)

Define the difference $z\coloneqq u-\tilde u$ and note that proving $z(r)>0$ on $(r_3,\infty)$ implies $\tilde u(r)<u(r)$ on that interval. This function satisfies the differential equation $$\label{zivp} BLA $$.
\\
\\
Also by Sturm comparison of $z$ and $w$, the latter oscillates faster. Let $\tilde w$ be a solution of \eqref{wivp} such that $\tilde w(r_3)=0$. 
\\
\\
By integration of the strong version of Sturm, $z(r)\to\infty$ as $r\to\infty$, but this is impossible as $0<\tilde u(r)<u(r)$ on $(r_3,\infty)$ and $u(r)\to0$ as $r\to\infty$. Therefore, $\tilde u$ vanishes at some point $\tilde r\in(r_3,\infty)$ and the proof is complete.
\\
\\ \comm{
for any $\epsilon>0$, all solutions $\tilde u=u(\tilde\alpha,r)$ with $\tilde\alpha\in(\alpha,\alpha+\epsilon)$ are non-vanishing. Then  Remember that $w$ is the derivative of \eqref{ivp} with respect to the initial condition. And by \ref{} $w$ has a unique zero that is contained in the disconjugacy interval. [[]] Let $r_1,r_2$ be such that $d<r_1<r_0<r_2$ and note $w(r_1)=\frac{\partial}{\partial\alpha}u(\alpha;r_1)>0$ and $w(r_2)=\frac{\partial}{\partial\alpha}u(\alpha;r_2)<0$. Let $\epsilon>0$ such that for all $\tilde\alpha\in(\alpha,\alpha+\epsilon)$, $$\tilde u(r_1)>u(r_1)\text{ and }\tilde u(r_2)<u(r_2),$$ where $\tilde u=u(\tilde\alpha,r)$ and $\tilde\alpha\in(\alpha,\alpha+\epsilon)$. The solutions $u$ and $\tilde u$ intersect in some point $r_3$. THEY STURM COMPARE. RELATION Z=U-UT.} \end{proof}
\end{lemma}
\comm{\section{Main theorem\label{s:mainth}}
\begin{proof}[Proof of Theorem \ref{t:prob1}]
We prove first that $mE_0<\nu<\infty$. The lower bound $\nu>mE_0$ is a direct consequence of the fact that $E[\Psi] < E^{lin}[\Psi]$, for all $\Psi\in \EE$, and that, by the definition of $E_0$  and $E^{lin}$, one has
\[\inf \{ E^{lin}[\Psi] \text{ s.t. } \Psi\in \EE , \, M[\Psi]=m  \} = -mE_0.\]\\
To prove that $\nu<+\infty$ we first note that, by using  H\"older and   Gagliardo-Nirenberg inequalities, one can prove the bounds:  
\begin{equation}\label{4.1a}
 \|\Psi\|_{{2\mu+2}}^{2\mu+2} \leq c \|\Psi\|_{H^1}^{\mu} \|\Psi\|^{2+\mu}; \end{equation}
\begin{equation}\label{W-1}
(\Psi,W_- \Psi)  \leq \|W_-\|_r \|\Psi\|^2_{2r/(r-1)} \leq c \|W_-\|_r  \|\Psi\|_{H^1}^{2\alpha}\|\Psi\|_q^{2(1-\alpha)} \end{equation}
for all $q\in[2,2r/(r-1)]$ and with $ \alpha= \frac{2}{2+q}\left(1-\frac{q(r-1)}{2r}\right)$; and 
\begin{equation}\label{4.1b}
 |\Psi(\v)|^2\leq \|\Psi\|_{\infty}^2\leq c  \|\Psi\|_{H^1}\|\Psi\| \quad  \forall \v \in V \,.
\end{equation}
We remark that the inequalities \eqref{4.1a} - \eqref{4.1b} hold true for any connected finite graph. 
 If $M[\Psi] = m$, by \eqref{4.1a} - \eqref{4.1b} we have 
\[
 E[\Psi] + m
 \geq \| \Psi \|_{H^1}^2 -C \frac{m^{\frac{2+\mu}{2}}}{\mu+1}\|\Psi\|_{H^1}^{\mu}
  - C \sqrt m \sum_{\underline v\in V_v}|\alpha(\underline v)|   \|\Psi\|_{H^1} -  Cm^{1-1/(2r)} \|W_-\|_r  \|\Psi\|_{H^1}^{1/r}.
\]
We notice that for any $a,b,c,d>0$, $r\geq 1$, and   $0< \mu < 2$ there exist $\de,\beta >0$ such that $a x^2 - bx^\mu  -cx -d x^{1/r} > \de x^2 - \beta $, for any $x\geq0$, then  
\beq
\label{e:dec}
 E[\Psi] + m
 \geq \de \| \Psi\|_{H^1}^2 -\beta \,,
\eeq
which implies $\nu \leq \beta + m$.\\

In the  remaining part of the proof we shall prove that we can choose $m^*$ such that for $m<m^*$ minimizing sequences have a  convergent subsequence.

Let $\{\Psi_n\}_{n\in\NA}$ be a minimizing sequence, i.e., $\Psi_n \in\EE$, $M[\Psi_n]=m$, and $\lim_{n\to\infty} E[\Psi_n] = -\nu$. Concerning the mass constraint, we remark that it is enough to assume $M[\Psi_n]\to m$ as $n\to \infty$, in such a case one can define $\widetilde \Psi_n = \sqrt{m} \Psi_n /\|\Psi_n\|$ and note that $\lim_{n\to\infty} E[\widetilde\Psi_n] = \lim_{n\to\infty} E[\Psi_n] $. 

We shall prove that there exists $\hat \Psi \in H^1 (\GG)$ such that $M[\hat \Psi] = m $, $E[\hat \Psi] =-\nu$ and $\Psi_n \to \hat \Psi$ in $ H^1 (\GG)$.

We can assume that 
$E[\Psi_n] \leq -\nu/2$ then by inequality \eqref{e:dec}, up to taking a subsequence,  we can assume that 
\[
\sup_{n\in\NA}\|\Psi_n\|_{H^1}\leq \infty,\]
moreover the following lower bound holds true
\beq
\label{e:floor}
 \frac{1}{\mu+1} \| \Psi_n\|_{2\mu+2}^{2\mu+2} +(\Psi,W_-\Psi) +  \sum_{\v \in V_- } |\al(v)| |\Psi_{n}(\v)|^2 \geq
\frac{\nu}2 \,.
\eeq

Next we use Lem. \ref{l:cc} and  prove that  vanishing and dichotomy
cannot occur for $\{\Psi_n\}_{n\in\NA}$. Set $\tau = \lim_{t\to\infty}\liminf_{n\to\infty}
\rho(\Psi_n,t)$. First we prove that  vanishing cannot occur. If
$\tau=0$,  then by Lem. \ref{l:cc} there would exist a subsequence $\Psi_{n_k}$
such that $\| \Psi_{n_k}\|_{p} \to 0 $ for all $2<p\leq \infty$ but
this, together with Eqs. \eqref{W-1} and \eqref{4.1b},   would contradict 
\eqref{e:floor}. \\
To prove that dichotomy cannot occur, suppose $0<\tau<m$, then there
would exist $\VV_k$ and $\WW_k$ satisfying \eqref{dic1}-\eqref{dic7}.
In particular we know that
\[
 \liminf_{k\to \infty} \left(\|\Psi_{n_k}' \|^2 - \| \VV_k' \|^2
- \| \WW_k' \|^2 \right)  \geq 0
\]
\[
\lim_{k\to \infty} \lf( \| \Psi_{n_k}\|_p^p - \|\VV_k \|_p^p -
\| \WW_k \|_p^p \ri)=0 \qquad 2\leq p < \infty 
\]
and 
\be
\lim_{k\to \infty}\left||\Psi_{n_k}(\v)|^2-  |\VV_{k}(\v)|^2 -| \WW_{k}(\v)|^2\right| = 0\,.
\ee
Moreover we claim that 
\begin{equation}\label{claim1}
\lim_{k\to\infty}  (\Psi_{n_k}, W\Psi_{n_k}) - (\VV_k,W \VV_k) - (\WW_k, W\WW_k) \geq  0  ,
\end{equation}
we postpone the proof of this claim to the end of the discussion.  Summing up, we arrive at
\be
\liminf_{k\to\infty} \lf(
E[\Psi_{n_k} ] - E[ \VV_k] - E[\WW_k]
\ri) \geq 0 \,,
\ee
which implies
\beq
\label{e:black-1}
\limsup_{k\to\infty} \lf(
 E[ \VV_k] + E[\WW_k]
\ri) \leq -\nu \,.
\eeq
Notice that, given $\Psi\in\EE$ and $\de >0$, then
\[
E[\Psi] = \frac{1}{\de^2} E[\de \Psi] + \frac{\de^{2\mu} -1}{\mu+1} \| \Psi
\|_{2\mu+2}^{2\mu+2}.
\]
We remark that $\VV_k, \WW_k \in \EE$, since $\Psi_{n_k}$ satisfies   the continuity condition at
the vertices  and the multiplication with the cut-off functions
preserves that. Let $\de_k= \sqrt {m / M[\VV_k]}$ and $\ga_k = \sqrt {m / M[\WW_k]}$  such that $M[\de_k \VV_k] ,\,
M[\ga_k \WW_k] =m$. Then, using the above equality and the fact that
$E[\de_k \VV_k], E[\ga_k \WW_k] \geq -\nu$,  one has
\[
E[\VV_k] \geq - \frac{\nu}{\de^2_k} + \frac{\de^{2\mu}_k -1}{\mu+1} \| \VV_k
\|_{2\mu+2}^{2\mu+2}
\]
\[
E[\WW_k] \geq - \frac{\nu}{\ga^2_k} + \frac{\ga^{2\mu}_k -1}{\mu+1} \| \WW_k
\|_{2\mu+2}^{2\mu+2}
\]
from which 
\[
E[\VV_k]+E[\WW_k] \geq -\nu \lf( \frac{1}{\de^2_k} + \frac{1}{\ga^2_k} \ri) +
\frac{\de^{2\mu}_k -1}{\mu+1} \| \VV_k \|_{2\mu+2}^{2\mu+2} +
\frac{\ga^{2\mu}_k -1}{\mu+1} \| \WW_k \|_{2\mu+2}^{2\mu+2}\,.
\]
Notice that  by \eqref{dic4}
\[
\frac{1}{\de^2_k} \to \frac{\tau}{m} \qquad \qquad \frac{1}{\ga^2_k} \to 1-\frac{\tau}{m}\,.
\]
Let $\theta = \min \{ (\tau/m)^{-\mu} , (1-\tau/m)^{-\mu} \}$ and notice that $\theta
>1$ since $0<\tau/m <1$. Therefore
\begin{align}
\label{e:black-2}
\liminf_{k\to\infty} \lf(
 E[ \VV_k] + E[\WW_k]
\ri) 
&\geq -\nu + \frac{\theta -1}{\mu+1} \liminf_{k\to\infty} \| \Psi_{n_k}
\|_{2\mu+2}^{2\mu+2} > -\nu,
\end{align}
where we used the fact that $\liminf_{k\to\infty} \| \Psi_{n_k}
\|_{2\mu+2}^{2\mu+2} \neq 0$. The latter claim is proved by noticing
that $\liminf_{k\to\infty} \| \Psi_{n_k} 
\|_{2\mu+2}^{2\mu+2} = 0$,  together with $\| \Psi_{n_k}
\|_{H^1}$ bounded and Eqs. \eqref{W-1} and \eqref{4.1b},  would imply $\liminf_{k\to\infty} (\Psi_{n_k},W_-\Psi_{n_k}) =0$
 and $\liminf_{k\to\infty} \|\Psi_{n_k}\|_\infty =0$. Hence, there would be a contradiction with   inequality \eqref{e:floor}. We conclude that if $0<\tau<m$  we get a contradiction, cfr. inequalities \eqref{e:black-1} and \eqref{e:black-2}. To end the analysis of the case $0<\tau<m$ we are left to prove the claim \eqref{claim1}.  We rewrite $W = W_+- W_-$ and consider first the term with $W_+$. We have that 
\[\begin{aligned}
 &(\Psi_{n_k}, W_+\Psi_{n_k}) - (\VV_k,W_+ \VV_k) - (\WW_k, W_+\WW_k)     \\ 
=&   \sum_e \int_{I_e}(W_+)_e \left[1 - (\Theta_k)_e^2 - (\Phi_k)_e^2\right] |(\Psi_{n_k})_e|^2 dx \geq 0. 
\end{aligned}\]
Since $\VV_k$ and $\WW_k$ have disjoint supports, we have that 
\begin{equation*}\begin{aligned}
&\left| (\Psi_{n_k}, W_-\Psi_{n_k}) - (\VV_k,W_- \VV_k) - (\WW_k, W_-\WW_k) \right|  \\ 
\leq &  |(\ZZ_k, W_-\ZZ_k)| +2|(\VV_k,W_- \ZZ_k)| +2 |(\WW_k, W_-\ZZ_k) | \\  
\leq & | (\ZZ_k, W_-\ZZ_k)| +2(\VV_k,W_-\VV_k)^{1/2 }(\ZZ_k,W_- \ZZ_k)^{1/2} +2(\WW_kW_-\WW_k)^{1/2 }(\ZZ_k,W_- \ZZ_k)^{1/2} .
\end{aligned}\end{equation*}
The terms containing $\VV_k$ and $\WW_k$ are bounded by Lemma \ref{l:cc} and inequality \eqref{W-1}. The terms containing $\ZZ_k$, go to zero by  inequality \eqref{W-1}  and because $\|\ZZ_k\|\to 0$ by Eq. \eqref{ZZk}. From which the claim \eqref{claim1} follows. 

Since  $0\leq\tau<m$ leads us to a contradiction, it must be $\tau=m$. 

Now we prove that for $m<m^\ast$ the minimizing sequence is not {\em runaway}. Here the limitation on the mass plays a role for the first time.
By absurd suppose that $\{\Psi_n\}_{n\in\NA} $ is {\em runaway}, then we have that 
\begin{equation}\label{limit}
\lim_{n\to\infty} \Psi_{n} (\v) =0\quad \forall \underline{v}\in V\qquad \text{and}\qquad \lim_{n\to\infty}(\Psi_n, W_- \Psi_n)= 0.
\end{equation} The first limit  is a direct consequence of Lem. \ref{l:cc}, Eq. \eqref{e:run-1}. To prove the second one, assume that $\Psi_n$ escapes at infinity on the external  edge $e^*$ (this can always be done up to taking a subsequence). We note that
\begin{equation*}
\lim_{n\to\infty}\int_{I_e} (W_-)_e |(\Psi_n)_e|^2 dx = 0 \qquad \forall e \neq e^* ,
\end{equation*}
this is a  direct consequence of Lemma \ref{l:cc} and inequality  \eqref{W-1} applied to the edge $I_e$. We are left to prove that 
\begin{equation}\label{holiday}
\lim_{n\to \infty}\int_{0}^{+\infty} (W_-)_{e^*} |(\Psi_n)_{e^*}|^2 dx = 0 . 
\end{equation}
 We start by noticing that  $\|\Psi_n\|_{H^1}$ is uniformly bounded, hence, so is $\|\Psi_n\|_p$ for all $p\in [2,+\infty]$, by \eqref{gajardo3} (with $q=2$). As a consequence, we have that  for any $\ve>0$ there exists $R>0$ (independent of $n$) such that
 \begin{equation*}
\int_{R}^{+\infty} (W_-)_{e^*} |(\Psi_n)_{e^*}|^2 dx \leq \|(W_-)_{e^*}\|_{L^r(R,\infty)} \|\Psi_n\|_{2r'}^2 \leq \ve,
\end{equation*}
with $r'$ such that $r^{-1}+{r'}^{-1} =1 $.  For such $R$, there exists $n_0$ such that for all $n>n_0$ one has 
\begin{equation*}
\int_{0}^{R} (W_-)_{e^*} |(\Psi_n)_{e^*}|^2 dx \leq  \|W_-\|_r \|(\Psi_n)_{e^*}\|_{L^{2r'}(0,R)}^2 \leq \ve  
\end{equation*}
by \eqref{e:run-1} (see also Rem. \ref{r:2.8}), from which the second limit in \eqref{limit}. 

Recalling that, by Lem. \ref{l:cc} - Eq. \eqref{e:run-1}, one has $\lim_{n\to\infty}\|(\Psi_n)_e\|_{L^{2\mu+2}(I_e)} =0$ for all $e\neq e^*$, and by Eq. \eqref{limit}, we infer 
 \begin{equation}
 \label{little}
 \lim_{n\to \infty} E[\Psi_n]  \geq \lim_{n\to\infty }  \int_0^\infty |(\Psi_n)_{e^*}'|^2 dx -\frac{1}{\mu+1} \int_0^\infty  |(\Psi_n)_{e^*}|^{2\mu+2} dx.
 \end{equation}
 Let  $\chi:\RE_+ \to [0,1]$ be a  function such that $\chi \in C^\infty(\RE_+)$, $\chi(0) = 0$ and $\chi(x)=1$ for all $x\geq 1$.  Define 
\[\psi_n^*(x) :=  \chi(x)(\Psi_n)_{e^*}(x) , \]
so that $\psi_n^*(0)= 0$, and ${\|\psi_n^*}'\|_{L^2(\RE_+)}^2 \leq c$. By Lem. \ref{l:cc} - Eq. \eqref{e:run-1}, for all $p\geq 2$,
\begin{equation}\label{psinstar1}
\lim_{n\to\infty} \|\Psi_n\|_p^p = \lim_{n\to\infty} \|(\Psi_n)_{e^*}\|_{L^p((0,\infty))}^p = \lim_{n\to\infty} \|\psi_n^*\|_{L^p((0,\infty))}^p,
\end{equation}
where we used the fact that $\lim_{n\to\infty} \|(\Psi_n)_{e^*}\|_{L^p((0,1))} = 0  $, and the trivial bound $ \|\psi_n^*\|_{L^p((0,1))} \leq  \|\chi\|_{L^\infty((0,1))} \|(\Psi_n)_{e^*}\|_{L^p((0,1))}$.  In particular, $ \lim_{n\to\infty} \|(\Psi_n)_{e^*}\|_{L^2((0,\infty))}^2 = \lim_{n\to\infty} \|\psi_n^*\|_{L^2((0,\infty))}^2 = m$. 
Moreover we have that 
 \begin{equation}\label{psinstar2}
  \lim_{n\to\infty } \frac12 \int_0^\infty |(\Psi_n)_{e^*}'|^2 dx \geq   \lim_{n\to\infty } \frac12 \int_0^\infty |{\psi_n^*}'|^2 dx. 
\end{equation}
 To prove the latter inequality, we note that 
\[\begin{aligned}
 \lim_{n\to\infty } \int_0^\infty |(\Psi_n)_{e^*}'|^2 - |{\psi_n^*}'|^2 dx 
 =&   
 \lim_{n\to\infty } \int_0^\infty |(\Psi_n)_{e^*}'|^2\left(1-\chi^{2}\right)  dx  \\ 
 & +
 \lim_{n\to\infty } \int_0^1 |(\Psi_n)_{e^*}|^2 {\chi'}^{2} + 2\chi {\chi'} \Re \overline {(\Psi_n)_{e^*}'}(\Psi_n)_{e^*} dx \\
 = &   \lim_{n\to\infty } \int_0^1 |(\Psi_n)_{e^*}'|^2\left(1- \chi^{2}\right)  dx  \geq 0 ,
\end{aligned}\]
where we used again Lem. \ref{l:cc} - Eq. \eqref{e:run-1} and the bounds $\|\chi\|_\infty, \|\chi'\|_\infty\leq c $. 

We have the following chain of inequalities/identities  
\begin{align}
&\lim_{n\to \infty} E[\Psi_n] \nonumber \\
& \geq 
\lim_{n\to\infty } \int_0^\infty |{\psi_n^*}'(x)|^2 dx -\frac{1}{\mu+1} \int_0^\infty  |\psi_n^*(x)|^{2\mu+2} dx \nonumber\\
&\text{(we used Eqs. \eqref{little}, \eqref{psinstar1} and \eqref{psinstar2})} \nonumber \\
&   \geq \inf \Big\{ \int_0^\infty |\psi ' (x)|^{2} dx -\frac{1}{\mu+1} \int_0^\infty |\psi (x)|^{2\mu+2} \, dx  \text{ s.t. } \psi\in H^1(\RE^+), \, \psi(0)=0\, ,\| \psi\|_{L^2(\RE^+)}^2 =m \Big\} \nonumber \\ 
&\text{(we used the fact that $\psi_n^*\in H^1(\RE_+)$,  $\psi_n^*(0)= 0$, and $\|\psi_n^*\|_{L^2(\RE_+)}^2 \to  m$ as $n\to\infty$)} \nonumber
\\ 
&  =  \inf \Big\{  \int_\RE |{\psi}' (x)|^{2} dx-\frac{1}{\mu+1} \int_\RE|{\psi} (x)|^{2\mu+2} \, dx   \text{ s.t. } \psi\in H^1(\RE), \, \psi(x)=0 \; \forall x\leq 0 ,\| \psi\|_{L^2(\RE)}^2 =m  \Big\}\nonumber \\
&\text{(where we used the fact that $\psi\in H^1(\RE_+)$ and $\psi(0)= 0$ if and only if its zero extension} \nonumber\\
&\text{belongs to $H^1(\RE)$, see, e.g., \cite[Th. 5.29]{AF03}})\nonumber
\\ 
& \geq \inf \left\{  \int_\RE |\psi  ' (x)|^{2} dx-\frac{1}{\mu+1} \int_\RE|\psi (x)|^{2\mu+2} \, dx  \text{ s.t. } \psi\in H^1(\RE), \, \| \psi\|_{L^2(\RE)}^2 =m  \right\} \label{infsol}\\
&\text{(we enlarged the set on which the $\inf$ is taken).} \nonumber
\end{align}

It is well known that the infimum in the latter minimization problem is indeed  attained and that the minimizing function (up to translations and phase multiplications) is given by the soliton profile 
\begin{equation*}%\label{soliton}
\phi(x) = [ (\mu + 1) \ome_{\RE}]^{\frac{1}{2\mu}} \sech^{\frac{1}{\mu}} (\mu \sqrt{\ome_{\RE}} x).
\end{equation*}
The frequency   $\omega_\RE$ is fixed by the mass constraint through the relation  
\begin{equation*}%\label{solmass}
m =\|\phi\|^2_{L^2(\RE)} =   2\f{(\mu+1)^{\f 1 \mu }  }{\mu} \ome_\RE^{ \f 1 \mu - \f 1 2} \int_0^1 (1-t^2)^{\f 1 \mu -1} dt,
\end{equation*}
which gives 
\begin{equation*}%\label{solmass2}
\omega_\RE = \left( 2\f{(\mu+1)^{\f 1 \mu }  }{\mu}  \int_0^1 (1-t^2)^{\f 1 \mu -1} dt\right)^{-\frac{2\mu}{2-\mu}} m^{\frac{2\mu}{2-\mu}}.
\end{equation*}
The infimum in the minimization problem \eqref{infsol} is given by  the nonlinear energy  of the soliton 
\begin{equation*}%\label{solen}
  	  \int_\RE |\phi'(x) |^{2} dx-\frac{1}{\mu+1} \int_\RE|\phi(x)|^{2\mu+2} \, dx  =
- \f{2-\mu}{2+\mu}  \; \ome_\erre \, m = -\gamma_\mu m^{1+\frac{2\mu}{2-\mu}}, 
\end{equation*}
with $\gamma_\mu= \f{2-\mu}{2+\mu}\left( 2\f{(\mu+1)^{\f 1 \mu }  }{\mu}  \int_0^1 (1-t^2)^{\f 1 \mu -1} dt\right)^{-\frac{2\mu}{2-\mu}}$. So that by the inequality \eqref{infsol}, we conclude that  if $\Psi_n$ is a runaway sequence  it must be 
\begin{equation}\label{lower}
\lim_{n\to \infty} E[\Psi_n] \geq  -\gamma_\mu m^{1+\frac{2\mu}{2-\mu}} .
\end{equation}


To show that for $m$ small enough a minimizing sequence cannot be runaway we  compute the energy   on a trial function.  As trial function we choose the  function $\Phi(\ome)$, with $\omega = \omega(m)$,  given in Th. \ref{t:bif}. By the same theorem we have that the energy  $ E[\Phi(\ome)] = -E_0 m + o(m)$, and by a simple continuity argument we infer that there exists $m^*$  such that $ E[\Phi(\ome)]  < -\gamma_\mu m^{1+\frac{2\mu}{2-\mu}}$ for all  $0<m<m^*$. This, together with the lower bound \eqref{lower},   imply that a minimizing sequence cannot be runaway. 

By Lem. \ref{l:cc} we conclude that for all $0<m<m^*$ there exists a state $\hat\Psi \in\EE$ such that  minimizing sequences converge, up to taking  subsequences, to $\hat\Psi $ in $L^p$ for $p  \geq 2$. In particular, $M[\hat \Psi] = m$, and the potential, vertices, and nonlinear terms in $E[\Psi_n]$ converge to the corresponding ones in $E[\hat\Psi]$. Taking into account also the weak lower continuity of the $H^1$ norm we have
\[
E[ \hat \Psi] \leq \lim_{n\to \infty} E[ \Psi_n] = -\nu
\]
which implies that $E[ \hat \Psi] = -\nu$. Since $E[ \hat \Psi] = \lim_{n\to \infty}  E[ \Psi_n]$ then $\| \hat \Psi ' \| = \lim_{n\to \infty} \|  \Psi_n ' \|$ and we
have proved that $\Phi_n \to \hat \Psi$ in $H^1$.
\end{proof}}	

\subsection*{Acknowledgments.}  
\comm{The authors are grateful to Gregory Berkolaiko, Pavel Exner and Delio Mugnolo for useful discussions. 
D.F. and D.N.  acknowledge the support of FIRB 2012 project ``Dispersive dynamics: Fourier Analysis and Variational Methods'', Ministry of University and
Research of Italian Republic  (code RBFR12MXPO).
C.C. acknowledges the support of the FIR 2013 project ``Condensed Matter in Mathematical Physics'', Ministry of University and
Research of Italian Republic  (code RBFR13WAET)}

\begin{thebibliography}{99}
\comm{
\bibitem{[ACFN1]}
R.~Adami, C.~Cacciapuoti, D.~Finco, and D.~Noja, \emph{Fast solitons on star
  graphs}, Rev. Math. Phys \textbf{23} (2011), no.~4, 409--451.

\bibitem{[ACFN2]}
R.~Adami, C.~Cacciapuoti, D.~Finco, and D.~Noja, \emph{On the structure of
 critical energy levels for the cubic focusing {NLS} on star graphs}, J. Phys.
 A: Math. Theor. \textbf{45} (2012), 192001, 7pp.
 
\bibitem{[ACFN4]}
R.~Adami, C.~Cacciapuoti, D.~Finco, and D.~Noja, \emph{Stationary states of
{NLS} on star graphs}, EPL \textbf{100} (2012), 10003, 6pp.

\bibitem{[ACFN3]}
R.~Adami, C.~Cacciapuoti, D.~Finco, and D.~Noja, \emph{Variational properties
  and orbital stability of standing waves for {NLS} equation on a star graph},
J. Differ.  Equations {\bf 257} (2014), 3738--3777.

\bibitem{acfn-aihp} R. Adami, C. Cacciapuoti, D. Finco, D. Noja,
{\em Constrained energy minimization and orbital stability for the NLS equation on a star graph},  
Ann. Inst. Poincar\'e, An. Non Lin. \textbf{31} (2014), no. 6, 1289--1310. 
 
\bibitem{ACFN16} R.~Adami, C.~Cacciapuoti, D.~Finco, and D.~Noja, \emph{Stable standing waves for a NLS on star graphs as local minimizers of the constrained energy},
J. Differ. Equations \textbf{260} (2016) 7397--7415.

\bibitem{ANcmp} R. Adami, and D.   Noja,
 \emph{Stability and Symmetry-Breaking Bifurcation for the Ground States of a NLS
 with a $\delta'$ Interaction}, Comm. Math. Phys. {\bf 318} (2013), 247--289. 
 
\bibitem{ANV12}
R.~Adami, D.~Noja, and N.~Visciglia \emph{Constrained energy minimization and ground states for NLS with point defects}, Discrete and Continuous Dynamical Systems B
{\bf 18} (2013), 1155--1188. 

\bibitem{AST1}  R. Adami, E. Serra, P.  Tilli, \emph{NLS ground states on graphs}, Calc. Var. and PDEs \textbf{54} (2015), no. 1, 743--761.

\bibitem{AST2} R. Adami, E. Serra, P.  Tilli, \emph{Threshold phenomena and existence results for NLS ground states on metric graphs}, J. Func. An. \textbf{271} (2016), no. 1, 201--223.

\bibitem{AF03} R.~A. Adams,  and J.~J.~F.~ Fournier, Sobolev spaces, Pure and Applied Mathematics Series Vol. 140, Academic press, 2003.

\bibitem{AMN15} F. Ali Mehmeti, K. Ammari, and S. Nicaise, \emph{Dispersive effects for the Schr\"odinger equation on a tadpole graph}, arXiv:1512.05269 [math-ph] (2015).

\bibitem{[BI1]}V. Banica, and L. I. Ignat, \emph{Dispersion for the Schr\"odinger equation on networks},  J. Math. Phys. \textbf{52} (2011), no. 8, 083703, 14pp.

\bibitem{[BI2]}V. Banica, and  L. I. Ignat, \emph{Dispersion for the Schr\"odinger equation on the line with multiple Dirac delta potentials and on delta trees},  Analysis \& P.D.E. {\bf 7} (2014), no. 4, 903--927. 

\bibitem{BerKu} G. Berkolaiko, P. Kuchment, Introduction to Quantum Graphs, Mathematical Surveys and Monographs 186,  AMS (2013).
    
\bibitem{Berko16} G. Berkolaiko, and  W. Liu, \emph{Simplicity of eigenvalues and non-vanishing of eigenfunctions of a quantum graph}, arXiv:1601.06225v2 [math-ph] (2016), to appear on J. Math. Anal. Appl.. 

\bibitem{cfn15} C. Cacciapuoti, D. Finco, D. Noja, \emph{Topology induced bifurcations for the NLS on the tadpole graph}, Phys. Rev. E
\textbf{91} (2015), no. 1,  013206, 8 pp.

\bibitem{Caz03}
T.~Cazenave, {S}emilinear {S}chr\"{o}dinger {E}quations, Courant Lecture Notes in Mathematics, AMS, vol. 10, Providence, 2003.

\bibitem{Caz06}
T.~Cazenave, An introduction to semilinear elliptic equations, Editora
  do IM-UFRJ, Rio de Janeiro, 2006.

\bibitem{[CL]}
T.~Cazenave, and P.-L. Lions, \emph{Orbital stability of standing waves for some
  nonlinear {S}chr\"odinger equations}, Commun. Math. Phys. \textbf{85} (1982),
  549--561.
  
\bibitem{EJ} P. Exner, and M. Jex, \emph{On the ground state of quantum graphs with attractive $\delta$-coupling}, Physics Letters A \textbf{376} (2012), 713--717.

\bibitem{H} S. Haeseler, \emph{Heat kernel estimates and related inequalities on metric graphs}, arXiv:1101.3010v1 [math-ph] (2011).

\bibitem{Kelleretal} M. Keller, D. Lenz, and  R. Wojciechowski, \emph{Note on basic features of large time behaviour of heat kernels}, J. reine angew. Math. \textbf{708} (2015), 73--95.

\bibitem{kirr-kevrekidis-pelinovsky:11} E. Kirr, P. G. Kevrekidis, and D. E. Pelinovsky, \emph{Symmetry-Breaking Bifurcation in the Nonlinear Schr\"odinger Equation with Symmetric Potentials}, Comm. Math. Phys, {\bf 308} (2011), 795--844. 

\bibitem {KPS} V. Kostrykin, J. Potthoff, and R. Schrader, \emph{Contraction Semigroups on Metric Graphs}, 
Proceedings of Symposia in Pure Mathematics \textbf{77} (2008), 423--458.

\bibitem{[KS99]} V.~Kostrykin and R.~Schrader, \emph{{K}irchhoff's rule for quantum wires}, J.Phys. A: Math. Gen. \textbf{32} (1999), no.~4, 595--630.

\bibitem{LL01}
E.~H. Lieb and M.~Loss, Analysis, second ed., Graduate Studies in
  Mathematics, vol.~14, American Mathematical Society, Providence, RI, 2001.

\bibitem{[MP16]}  J. Marzuola, and D. E. Pelinovsky, \emph{Ground states on the dumbbell graph}, Applied Mathematics Research Express 2016, 98--145 (2016).
  
\bibitem{Mugnolo} D. Mugnolo, Semigroup Methods for Evolution Equations on Networks, Springer (2014).

\bibitem{Noja14} D. Noja, \emph{Nonlinear Schr\"odinger equations on graphs:
recent results and open problems}, Phil. Trans. Roy Soc. A {\bf 372} (2014), 20130002,
20 pages.

\bibitem{[NPS15]} D. Noja, D. Pelinovsky, and G. Shaikhova, \emph{Bifurcation and stability of standing waves in the nonlinear Schr\"odinger
equation on the tadpole graph}, Nonlinearity {\bf 28} (2015), 2343--2378.

\bibitem{[PS16]} D. E. Pelinovsky, and G. Schneider, \emph{Bifurcations of standing localized waves on periodic graphs},  arXiv:1603.05463v1 [math.DS] (2016).

\bibitem{RSIV} M. Reed, and B. Simon, Methods of modern mathematical physics IV, Analysis of operators, Academic Press, London (1978).}
\end{thebibliography}








\end{document}



\begin{figure}[h]
	\begin{center}
		\begin{tikzpicture}
			\node at (0,0)[nodo](5){};
			\node at (-2,0.5)[nodo](1){};
			\node at (0.5,1.5)[nodo](2){};
			\node at (1.5,-0.5)[nodo](3){};
			\node at (-1.5,-1)[nodo](4){};
			\node at (-5,0.5)[infinito](6){};
			\draw [-,black] (-5.5,0.5) circle (0.5cm) ;			
			\node at (-3.5,3)[infinito](7){};
			\node at (1.5,3.5)[infinito](8){};
			\node at (3.5,-2.5)[infinito](9){};
			\node at (-2.5,-2.5)[infinito](10){};
			\draw[-] (1)--(2);
			\draw[-] (2)--(3);
			\draw[-] (3)--(4);
			\draw[-] (4)--(1);
			\draw[-] (5)--(1);
			\draw[-] (5)--(2);
			\draw[-] (5)--(4);
			\draw[-] (1)--(6);
			\draw[-] (1)--(7);
			\draw[-] (2)--(8);
			\draw[-] (3)--(9);
			\draw[-] (4)--(10);
			\node at (-2.2,0.25){$v_1$};
			\node at (0.25,1.7){$v_2$};
			\node at (1.7,-0.3){$v_3$};
			\node at (-1.25,-1.25){$v_4$};
			\node at (0.25,-0.25){$v_5$};
			\node at (-4.80,0.7){$v_6$};
			\node at (-3.6,3.15){$\infty$};
			\node at (1.6,3.65){$\infty$};
			\node at (3.65,-2.65){$\infty$};
			\node at (-2.65,-2.65){$\infty$};
			%\node at (-3.5,0.75){$b_1$};
			%\node at (-2.55,1.9){$b_2$};
			%\node at (0.8,2.77){$b_3$};
			%\node at (2.7,-1.3){$b_4$};
			%\node at (-2.25,-1.65){$b_5$};
			%\node at (-0.8,1.32){$b_6$};
			%\node at (1.2,0.7){$b_7$};
			%\node at (0.2,-1.1){$b_8$};
			%\node at (-1.93,-0.3){$b_9$};
			%\node at (-0.8,0.47){$b_{10}$};
			%\node at (0.49,0.55){$b_{11}$};
			%\node at (-0.9,-0.27){$b_{12}$};
		\end{tikzpicture}
		\caption{13 edges (9 interior, 4 exterior); 6 vertices; one tadpole.}
	\end{center}
\end{figure}
